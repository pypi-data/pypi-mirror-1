\documentclass{howto}

\title{The Python Replybot}
\author{Barry Warsaw}
\authoraddress{\email{barry(at)python.org}}

\date{\today}
\release{3.1}                   % software release, not documentation
\setreleaseinfo{}               % empty for final release
\setshortversion{3.1}           % major.minor only for software

\begin{document}

\maketitle

% This makes the Abstract go on a separate page in the HTML version;
% if a copyright notice is used, it should go immediately after this.
%
\ifhtml
\chapter*{Front Matter\label{front}}
\fi

\begin{abstract}
\noindent
This document describes the Python Replybot, software to send auto-replies to
email messages.  For example, let's just say that you're being overwhelmed by
email to a particular address.  You could set up the replybot so that you send
a helpful message to anyone that emails you.  You could also use the replybot
to send replies for emails to your webmaster address, etc.

\noindent
The Python Replybot website is at \url{http://barry.warsaw.us/software}
\end{abstract}

% The ugly "%begin{latexonly}" pseudo-environment supresses the table
% of contents for HTML generation.
%
%begin{latexonly}
\tableofcontents
%end{latexonly}

\section{Introduction}

This document describes the Python Replybot, software to send auto-replies to
email messages.  For example, let's just say that you're being overwhelmed by
email to a particular address.  You could set up the replybot so that you send
a helpful message to anyone that emails you.  You could also use the replybot
to send replies for emails to your webmaster address, etc.

Once a reply is sent to an address, it is given a grace period during which no
more replies are sent.  This prevents senders from getting bombarded with
auto-replies, or for the replybot being used as a third party mailbomb agent.

Addresses can be whitelisted so that they never get auto-replies.  Responses
are never sent to email messages that have a null Return-Path or that have a
Precedence bulk, junk, or list.  Individual senders can control whether they
get an auto-reply by use of the X-Ack header.  A value of 'no' prevents a
response, while a value of 'yes' forces a response (if the other inhibitions
are passed, and that the --testing flag is given).

The messages used in the replies can come from any URL, including from http
URLs.  For efficiency, the replybot caches these files on the file system.

\section{Installation}

Installing the replybot is easy, as it's a standard Python distutils-based
distribution.  Simply unpack the source tarball and run this command:

\begin{verbatim}
python setup.py install
\end{verbatim}

This will install the \program{replybot} program in \file{/usr/local/bin} (by
default) and the \module{botlib} module in Python's \file{site-packages}.

Connecting the replybot up to your mail system depends on which mail server
you use.  For \ulink{Postfix}{http://www.postfix.org}, simply pipe the output
from the alias to the replybot command.  For example, say you wanted to add a
an auto-response to \email{webmaster@example.com}.  Here's what you would add
to your \file{/etc/aliases} file\footnote{or whatever file contains your
aliases; see the Postfix documentation for details.}:

\begin{verbatim}
webmaster-replybot: |"/usr/local/bin/replybot -r reply-url"
real-webmaster:     :include:/etc/webmasters.txt
webmaster:          real-webmaster, webmaster-replybot
\end{verbatim}

A lot of details are omitted here; for specifics of setting up your aliases,
see your mail server documentation.  The command line switches for the
\program{replybot} program are described in the \ref{invoking} section.

Before enabling the replybot, you should review and edit the configuration
file.  The configuration details are described in section \ref{config}.

\section{Invoking \program{replybot}\label{invoking}}

The \program{replybot} program takes a number of command line switch, some
optional, some required.  Also, any additional arguments provided on the
command line are used as key/value pairs for a substitution mapping.  See
below for details.

\subsection{Command line options}

Here is the list of command line options:

\begin{itemize}
\item \programopt{--C} \var{file}
      \longprogramopt{configuration}=\var{file}

      Specify the replybot configuration file.  If this option is not given,
      otherwise search for the file in this order: \file{replybot.cfg} in the
      directory containing the replybot script, \file{replybot.cfg} in the
      current working directory, \file{~/.replybot}.  If no configuration file
      is found, an error occurs.  See the \ref{config} section for details of
      the configuration file format.

\item \programopt{-r}, \longprogramopt{reply-url} -- Provide the URL to find
      the reply message at.  This will be downloaded and cached for a certain
      amount of time to avoid network traffic.

\item \programopt{-p}, \longprogramopt{purge-cache} -- This option purges
      certain information in the replybot's database.  You can have multiple
      purge options on the command line.  After a purge, replybot exits.  This
      option takes one of the following arguments:

      Here
      are the options: `notices' purges the cache of reply messages; `replies'
      purges the last reply dates for all recipients; `whitelist' purges all
      whitelist flags; `all' combines all the previous purge options.

\end{itemize}

\end{document}
