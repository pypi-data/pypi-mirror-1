% Complete documentation on the extended LaTeX markup used for Python
% documentation is available in ``Documenting Python'', which is part
% of the standard documentation for Python.  It may be found online
% at:
%
%     http://www.python.org/doc/current/doc/doc.html

\documentclass[letterpaper,hyperref,titlepage]{manual}

% latex2html doesn't know [T1]{fontenc}, so we cannot use that:(

\usepackage{amsmath}
\usepackage[latin1]{inputenc}
\usepackage{textcomp}
\usepackage{fullpage}
\usepackage{graphicx}
\graphicspath{{./graphics/}}
\usepackage{url}
\usepackage{index}
\usepackage{chicago}
\usepackage{supertabular}
%\setlongtables
%\usepackage{epsfig}

% The commands of this document do not reset module names at section level
% (nor at chapter level).
% --> You have to do that manually when a new module starts!
%     (use \py@reset)
%begin{latexonly}
\makeatletter
\renewcommand{\section}{\@startsection{section}{1}{\z@}%
   {-3.5ex \@plus -1ex \@minus -.2ex}%
   {2.3ex \@plus.2ex}%
   {\reset@font\Large\py@HeaderFamily}}
\makeatother
%end{latexonly}

% additional mathematical functions
\DeclareMathOperator{\abs}{abs}

% provide a cross-linking command for the index
%begin{latexonly}
%\newcommand*\see[2]{\protect\seename #1}
%\newcommand*{\seename}{$\to$}
%end{latexonly}

% some convenience declarations
\newcommand{\pypedal}{PyPedal}
\newcommand{\PyPedal}{PyPedal}  % Only beginning of sentence, otherwise use \pypedal{}
\newcommand{\PYPEDAL}{PyPedal}
\newcommand{\python}{Python}

% mark internal comments
% for any published version switch to the second (empty) definition of the macro!
% \newcommand{\remark}[1]{(\textbf{Note to authors: #1})}
\newcommand{\remark}[1]{}

\title{A Manual for use of PyPedal\\A software package for pedigree analysis}
\author{John B. Cole}
\authoraddress{Animal Improvement Programs Laboratory, Agricultural Research Service, United States Department of Agriculture, Room 306 Bldg 005 BARC-West, 10300 Baltimore Avenue, Beltsville, MD 20705-2350}

% I use date to indicate the manual-updates,
% release below gives the matching software version.
%\date{December 01, 2005}        % update before release!
\date{November 29, 2005 \\ Revised \today}                   % update before release!
                                % Use an explicit date so that reformatting
                                % doesn't cause a new date to be used.  Setting
                                % the date to \today can be used during draft
                                % stages to make it easier to handle versions.

\release{2.0.0b4}               % (software) release version;
\setshortversion{2.0}           % this is used to define the \version macro
\makeindex                      % tell \index to actually write the .idx file
\newindex{func}{fdx}{fnd}{Function Index}

\begin{document}

\maketitle

\begin{abstract}
Cole, J.B.  2005.  A Manual for use of PyPedal: A software package for pedigree analysis.  Animal Improvement Programs Laboratory, Agricultural Research Service, United States Department of Agriculture.

This manual in eleven chapters describes PyPedal (v 2.0), a software package for pedigree analysis, report generation, and
data visualization.  Metrics include coefficients of inbreeding and relationship, effective founder and ancestor numbers, and
founder genome equivalents.  Tools are provided for identifying ancestors and descendants, computing coefficients of
inbreeding from potential matings, quantifying pedigree completeness, and visualizing pedigrees.  Scripting support is
provided by the Python programming language; this language may be used to easily automate analyses and implement new features.
Input and output files utilize plain-text formats.  The program has been used for the analysis of dairy cattle and working dog
pedigrees.  PyPedal runs on the GNU/Linux and Microsoft Windows operating systems.  The program, documentation, and examples
of usage are available at \url{http://pypedal.sourceforge.net/}.

Mention of trade names or commercial products in this manual is solely for the purpose of providing specific information and
does not imply recommendation or endorsement by the U.S. Department of Agriculture.

All programs and services of the U.S. Department of Agriculture are offered on a nondiscriminatory basis without regard to
race, color, national origin, religion, sex, age, marital status, or handicap.

Revised \today
\end{abstract}

% This makes the contents more accessible from the front page of the HTML.
\ifhtml
\part*{General}
\chapter*{Front Matter}
\label{front}
\fi

\section*{Legal Notice}
\label{sec:legal-notice}

Copyright (c) 2002, 2003, 2004, 2005.  John B. Cole.  All rights reserved.

Permission to use, copy, modify, and distribute this software for any purpose
without fee is hereby granted, provided that this entire notice is included in
all copies of any software which is or includes a copy or modification of this
software and in all copies of the supporting documentation for such software.

\subsection*{Disclaimer}

The author of this software does not make any warranty, express or implied, or
assume any liability or responsibility for the accuracy, completeness, or
usefulness of any information, apparatus, product, or process disclosed, or
represent that its use would not infringe privately-owned rights. Reference
herein to any specific commercial products, process, or service by trade name,
trademark, manufacturer, or otherwise, does not necessarily constitute or imply
its endorsement, recommendation, or favoring by the United States Government or
the author. The views and opinions of authors expressed herein do not necessarily
state or reflect those of the United States Government and shall not be used for
advertising or product endorsement purposes.

%% Local Variables:
%% mode: LaTeX
%% mode: auto-fill
%% fill-column: 79
%% indent-tabs-mode: nil
%% ispell-dictionary: "american"
%% reftex-fref-is-default: nil
%% TeX-auto-save: t
%% TeX-command-default: "pdfeLaTeX"
%% TeX-master: "numarray"
%% TeX-parse-self: t
%% End:

\tableofcontents
\listoftables
\listoffigures

\declaremodule{extension}{pypedal}
\moduleauthor{John B. Cole}{jcole@aipl.arsusda.gov}
\modulesynopsis{Py{P}edal}
\chapter{License}
\label{cha:license}
\index{license}

Py{P}edal -- a Python package for pedigree analysis.
Copyright (C) 2005  John B. Cole

This library is free software; you can redistribute it and/or modify it
under the terms of the GNU Lesser General Public License as published by
the Free Software Foundation; either version 2.1 of the License, or (at
your option) any later version.

This library is distributed in the hope that it will be useful, but
WITHOUT ANY WARRANTY; without even the implied warranty of
MERCHANTABILITY or FITNESS FOR A PARTICULAR PURPOSE.  See the GNU
Lesser General Public License for more details.

You should have received a copy of the GNU Lesser General Public
License along with this library; if not, write to the Free Software
Foundation, Inc., 51 Franklin St, Fifth Floor, Boston, MA 02110-1301,
USA.
\chapter{Introduction}
\label{cha:introduction}

\begin{quote}
This chapter introduces the \PyPedal{} module for Python 2.4, provides an overview of key features of the software, and describes the contents of this manual.
\end{quote}

\PyPedal{} (\textbf{P}ython \textbf{Ped}igree An\textbf{al}ysis) is a tool for analyzing pedigree files.  It calculates several quantitative measures of genetic diversity from pedigrees, including average coefficients of inbreeding and relationship, effective founder numbers, and effective ancestor numbers.  Checks are performed catch common mistakes in pedigree files, such as parents with more recent birthdates or smaller ID numbers than their offspring and animals appearing as both sires and dams in the pedigree.  Tools for pedigree visualization and report generation are also provided.  \PyPedal{} only makes use of information on pedigree structure, not individual genotypes.  Allelotypes can be assigned to founders for use in gene-dropping simulations to calculate the effective number of founder genomes, but no other measures of alleic diversity are currently supported.

\PyPedal{} is a Python (\url{http://www.python.org/}) language module that may be called by programs or used interactively from the interpreter.  You must have Python 2.4 (or later) installed in order to use \PyPedal{} as \PyPedal{} makes use of features found only in that version.  The Numarray module must also be installed in order for you to use \PyPedal{}, and may be found at \url{http://www.stsci.edu/resources/software_hardware/numarray}.  In addition, there are a number of third-party packages used by \PyPedal{}; they are discussed in Chapter \ref{cha:installation}.

This manual is the official documentation for \PyPedal{}. It includes a tutorial and is the most authoritative source of information about \PyPedal{} with the exception of the source code. The tutorial material will walk you through a set of manipulations of a simple pedigree.  All users of \PyPedal{} are encouraged to follow the tutorial with a working \PyPedal{} installation. The best way to learn is by doing --- the aim of this tutorial is to guide you along this doing.

This content of this manual is broken down as follows:
\begin{description}
\item[License] Chapter \ref{cha:license} describes the license under which \PyPedal{} is distributed.  It is important that you review the license before using the program.
\item[Installing PyPedal] Chapter \ref{cha:installation} provides information
   on testing Python and installing PyPedal.
\item[High-Level Overview] Chapter \ref{cha:high-level-overview} gives a
   high-level overview of the components of the \PyPedal{} system as a whole.
\item[Methodology] Chapter \ref{cha:methodology} provides a brief overview of the methodology used to calculate measures of genetic diversity.
\item[HOWTOs] Chapter \ref{cha:howtos} provides demonstrations of how to perform common tasks.
\item[Graphics] Chapter \ref{cha:graphics} provides details on producing graphics with \PyPedal{}.
\item[Reports] Chapter \ref{cha:reports} provides details about the report generation tools available in \PyPedal{}.
\item[Implementing New Features] Chapter \ref{cha:newfeatures} introduces the idea of extensibility and walks the reader through the development of a new \PyPedal{} routine.
\item[Applications Programming Interface] Chapter \ref{cha:api} includes a complete reference, including useage notes, for all functions in all \PyPedal{} modules.
% \item[PyPedal Tutorial] Chapter \ref{cha:tutorial} provides brief tutorial for new users of \PyPedal{}.
\item[Glossary] Chapter \ref{cha:glossary} provides a glossary of terms.
\item[References and Indices] are provided at the end of the manual.
\end{description}

\section{Implemented Features}
A full list of features, including notes on useage and computational details, is provided in Chapter \ref{cha:api}.  Some of the notable features of \PyPedal{} include:
\begin{itemize}
\item Reading pedigree files in user-defined formats;
\item Checking pedigree integrity (duplicate IDs, parents younger than offspring, etc.);
\item Generating summary information such as frequency of appearance in the pedigree file;
\item Reordering and renumbering of pedigree files.
\item Computation of the numerator relationship matrix ($A$) from a pedigree file using the tabular method;
\item Inbreeding calculations for large pedigrees;
\item Computation of average total and average individual coefficients of inbreeding and relationship;
\item Decomposition of $A$ into $T$ and $D$ such that $A=TDT'$;
\item Computation of the direct inverse of $A$ (not accounting for inbreeding) using the method of Henderson \citeyear{ref143};
\item Computation of the direct inverse of $A$ (accounting for inbreeding) using the method of Quaas \citeyear{ref235};
\item Storage of $A$ and its inverse between user sessions as persistent Python objects using the \textbf{pickle} module to avoid unnecessary calculations;
\item Calculation of theoretical and actual effective population sizes;
\item Computation of effective founder number using the exact algorithm of Lacy \citeyear{ref640};
\item Computation of effective founder number using the approximate algorithm of Boichard et al. \citeyear{ref352};
\item Computation of effective ancestor number using the algorithms of Boichard et al. \citeyear{ref352};
\item Selection of subpedigrees containing all ancestors of an animal;
\item Identification of the common relatives of two animals;
\item Output to ASCII text files, including matrices, coefficients of inbreeding and relationship, and summary information;
\end{itemize}
\PyPedal{} has been used to perform calculations on pedigrees as large as 600,000 animals and has been used in scientific research \cite{Cole2004a}.

\section{Where to get information and code}
\PyPedal{} and its documentation are available at: \url{http://pypedal.sourceforge.net/}.  The Sourceforge site, \url{http://sourceforge.net/projects/pypedal/}, provides tools for reporting bugs (\url{https://sourceforge.net/tracker/?func=add&group_id=106679&atid=645233}, making feature requests (\url{https://sourceforge.net/tracker/?func=add&group_id=106679&atid=645236}), and discussing \PyPedal{} (\url{https://sourceforge.net/forum/?group_id=106679}).

\section{Acknowledgments}
\PyPedal{} was initially written to support the author's dissertation research while at Louisiana State University, Baton Rouge (\url{http://www.lsu.edu/}).  The initial development was supported in part by a grant from The Seeing Eye, Inc., Morristown, NJ, USA.  It lay fallow for some time but has recently come under active development again.  This is due in part to a request from colleagues at the University of Minnesota that led to the inclusion of new functionality in \PyPedal{}.  The author wishes to thank Paul Van{R}aden for very helpful suggestions for improving the ability of \PyPedal{} to handle certain computations in very large pedigrees.  Additional feedback in the form of bug reports, feature requests, and discussion of computing strategies was provided by Bradley J. Heins (University of Minnesota-Twin Cities), Edward H. Hagen (Institute for Theoretical Biology, Humboldt-Universit\"{a}t zu Berlin), Kathy Hanford (University of Nebraska, Lincoln), Thomas von Hassell, and Gianluca Saba.
\chapter{Installing PyPedal}
\label{cha:installation}

\begin{quote}
   This chapter explains how to install and test \PYPEDAL{} from either the source distribution or from the binary distribution.
\end{quote}

Before we can begin the tutorial, we need to make sure that you can install and test Python, the Numeric or Numarray extension, and the \PYPEDAL{} extension.

\section{Testing the Python installation}

The first step is to install Python if you haven't already. Python is available from the Python project page at \url{http://sourceforge.net/projects/python/}.  Click on the link corresponding to your platform, and follow the instructions
described there. \PYPEDAL{} requires version 2.3 as a minimum.  When installed, starting Python by typing python at the shell or double-clicking on the Python interpreter should give a prompt such as:
\begin{verbatim}
Python 2.3.3 (#2, Feb 17 2004, 11:45:40)
[GCC 3.3.2 (Mandrake Linux 10.0 3.3.2-6mdk)] on linux2
Type "help", "copyright", "credits" or "license" for more information.
\end{verbatim}
If you have problems getting Python to work, consider contacting your local support person or e-mailing \ulink{python-help@python.org}{mailto:python-help@python.org} for help. If neither solution works, consider posting on the
\ulink{comp.lang.python}{news:comp.lang.python} newsgroup (details on the newsgroup/mailing list are available at
\url{http://www.python.org/psa/MailingLists.html\#clp}).


\section{Testing the Numarray Python Extension Installation}

The standard Python distribution does not come, as of this writing, with the
numarray Python extensions installed, but your system administrator may have
installed them already. To find out if your Python interpreter has numarray
installed, type \samp{import numarray} at the Python prompt. You'll see one of
two behaviors (throughout this document user input and python interpreter
output will be emphasized as shown in the block below):
\begin{verbatim}
>>> import numarray
Traceback (innermost last):
File "<stdin>", line 1, in ?
ImportError: No module named numarray
\end{verbatim}
indicating that you don't have numarray installed, or:
\begin{verbatim}
>>> import numarray
>>> numarray.__version__
'0.9'
\end{verbatim}
indicating that numarray is installed. If it is installed, you can skip the next section and go ahead to section \ref{sec:installing-pypedal}.  If you don't, you have to get and install the numarray extensions as described on the Numarray website at \url{http://www.stsci.edu/resources/software_hardware/numarray}.

\section{Installing PyPedal}
\label{sec:installing-pypedal}

In order to get \PYPEDAL{}, visit the official website at \url{http://sourceforge.net/projects/pypedal}.  Click on the "PyPedal" release and you will be presented with a list of the available files. The files whose names end in ".tar.gz" are source code releases. The other files are binaries for a given platform (if any are available).

It is possible to get the latest sources directly from our CVS repository using the facilities described at SourceForge. Note that while every effort is made to ensure that the repository is always ``good'', direct use of the repository is subject to more errors than using a standard release.

\subsection{Installing on Unix, Linux, and Mac OSX}
\label{sec:installing-unix}

The source distribution should be uncompressed and unpacked as follows (for
example):
\begin{verbatim}
gunzip pypedal-2.0.0a12.tar.gz
tar xf pypedal-2.0.0a12.tar.gz
\end{verbatim}
Follow the instructions in the top-level directory for compilation and installation. Note that there are options you must consider before beginning.  Installation is usually as simple as:
\begin{verbatim}
python setup.py install
\end{verbatim}
or:
\begin{verbatim}
python setupall.py install
\end{verbatim}
There are currently no extra packages for \PYPEDAL{}.

\paragraph*{Important Tip} \label{sec:tip:from-pypedal-import} Just like all Python modules and packages, the \PYPEDAL{} module can be invoked using either the \samp{import PyPedal} form, or the \samp{from PyPedal import ...} form.  Because most of the functions we'll talk about are in the numarray module, in this document, all of the code samples will assume that they have been preceded
by a statement:
\begin{verbatim}
>>> from numarray PyPedal *
\end{verbatim}


\subsection{Installing on Windows}
\label{sec:installing-windows}

To install numarray, you need to be in an account with Administrator privileges.  As a general rule, always remove (or hide) any old version of \PYPEDAL{} before installing the next version.

Please note that we have \textbf{NOT} tested \PYPEDAL{} on any Win-32 platforms!  However, \PYPEDAL{} should install and run properly on Win-32 as long as the dependencies mentioned above are satisfied.


\subsubsection{Installation from source}

\begin{enumerate}
\item Unpack the distribution: (NOTE: You may have to download an "unzipping" utility)
\begin{verbatim}
C:\> unzip PyPedal.zip 
C:\> cd PyPedal
\end{verbatim}
\item Build it using the distutils defaults:
\begin{verbatim}
C:\pyPedal> python setup.py install
\end{verbatim}
This installs \PYPEDAL{} in \texttt{C:\textbackslash{}pythonXX} where XX is the version number of your python installation, e.g. 20, 21, etc.
\end{enumerate}

\subsubsection{Installation from self-installing executable}

\begin{enumerate}
\item Click on the executable's icon to run the installer.
\item Click "next" several times.  I have not experimented with customizing the installation directory and don't recommend changing any of the installation defaults.  If you do and have problems, let us know.
\item Assuming everything else goes smoothly, click "finish".
\end{enumerate}


\subsubsection{Installation on Cygwin}

No information on installing \PYPEDAL{} on Cygwin is available.  If you manage to get it working, let us know.


\section{Testing the PyPedal Python Extension Installation}

To find out if you have correctly installed \PYPEDAL{}, type \samp{import PyPedal} at the Python prompt. You'll see one of
two behaviors (throughout this document user input and Python interpreter output will be emphasized as shown in the block below):
\begin{verbatim}
>>> import PyPedal
Traceback (innermost last):
File "<stdin>", line 1, in ?
ImportError: No module named PyPedal
\end{verbatim}
indicating that you don't have \PYPEDAL{} installed, or:
\begin{verbatim}
>>> import PyPedal
>>> PyPedal.__version__
'2.0.0a1'
\end{verbatim}
indicating that \PYPEDAL{} is installed.


\section{At the SourceForge...}
\label{sec:at-sourceforge}

The SourceForge project page for numarray is at
\url{http://sourceforge.net/projects/pyedal}. On this project page you will find
links to:
\begin{description}
\item[The PyPedal Discussion List] You can subscribe to a discussion list about \PYPEDAL{} using the project page at SourceForge. The list is a good place to ask questions and get help. Send mail to pyedal-discussion@lists.sourceforge.net.  There is also a pypedal-discussion group that you may join.
\item[The Web Site] Click on "home page" to get to the \PYPEDAL{} Home Page, which has links to documentation and other resources.
\item[Bugs and Patches] Bug tracking and patch-management facilities is provided on the SourceForge project page.
\item[FTP Site] The FTP Site contains this documentation in several formats, plus maybe some other goodies we have lying around.
\end{description}
\chapter{High-Level Overview}
\label{cha:high-level-overview}

\begin{quote}
In this chapter, a high-level overview of \PYPEDAL{} is provided, giving
the reader the definitions of the key components of the system. This section
defines the concepts used by the remaining sections.
\end{quote}

\section{Interacting with PyPedal}
\label{sec:interacting}
\index{interacting with PyPedal}
There are two ways to interact with \PyPedal{}: interactively\index{interacting with PyPedal!interactively} from a Python command line, and programmatically\index{interacting with PyPedal!programmatically} using a script that is run using the Python interpreter.  The latter is preferred to the former for any but trivial examples, although it is useful to work with the command line while learning how to use \PyPedal{}.  A number of sample programs are included with the \PyPedal{} distribution.  Examples of both styles of interaction may be found in the tutorial (Chapter \ref{tutorial}).

\section{The PyPedal Object Model}
\label{sec:pypedal-objects}
\index{objects}

At the heart of \PyPedal{} are four different types of objects.  These objects
combine data and the code that operate on those data into one convenient package.
Although most \PyPedal{} users will only work directly with one or two of these
objects it is worthwhile to know a little about all of them.  An instance of the
\textbf{NewPedigree} class stores a pedigree read from an input file, as well as
metadata about that pedigree.  The pedigree is a Python list of \textbf{NewAnimal}
objects.  Information about the pedigree, such as the number and identity of founders,
is contained in an instance of the \textbf{PedigreeMetadata} class.

The fourth \PyPedal{} class, \textbf{New{AM}atrix}, is used to manipulate numerator
relationship matrices (NRM).  When working with large pedigrees it can take a long
time to compute the elements of a NRM, and having an easy way to save and restore
them is quite convenient.

Here is an example of Python code using the NewPedigree object (\texttt{examples/new_lacy.py}):
\begin{verbatim}
import pyp_newclasses, pyp_nrm. pyp_metrics
from pyp_utils import pyp_nice_time

options = {}
options['messages'] = 'verbose'
options['renumber'] = 0
options['counter'] = 5

if __name__ == '__main__':
    print 'Starting pypedal.py at %s' % (pyp_nice_time())
    # Example taken from Lacy (1989), Appendix A.
    options['pedfile'] = 'new_lacy.ped'
    options['pedformat'] = 'asd'
    options['pedname'] = 'Lacy Pedigree'
    example = pyp_newclasses.NewPedigree(options)
    example.load()
    if example.kw['messages'] == 'verbose':
        print '[INFO]: Calling pyp_metrics.effective_founders_lacy at %s' % (pyp_nice_time())
    pyp_metrics.effective_founders_lacy(example)
\end{verbatim}
See section \ref{sec:tip:from-pypedal-import}.

\section{Pedigree Files}
\label{sec:pedigree-files}
\index{pedigree files}
Pedigree files consist of plain-text files (also known as ASCII or flatfiles) whose rows contain
records on individual animals and whose columns contain different variables.  The columns are
delimited (separated from one another) by some character such as a space or a tab (\\t).  Pedigree
files may also contain comments (notes) about the pedigree that are ignored by \PyPedal{}; comments
always begin with an octothorpe (\#).  For example, the following pedigree contains records for 13
animals, and each record contains three variables (animal ID, sire ID, and dam ID):
\begin{verbatim}
# This pedigree is taken from Boichard et al. (1997).
# Each records contains an animal ID, a sire ID, and
# a dam ID.
1 0 0
2 0 0
3 0 0
4 0 0
5 2 3
6 0 0
7 5 6
8 0 0
9 1 2
10 4 5
11 7 8
12 7 8
13 7 8
\end{verbatim}
When this pedigree is processed by \PyPedal{} the comments are ignored.  If you need to change the
default column separator, which is a space (' '), set the \texttt{sepchar} option to the desired
value.  For example, if your columns are tab-delimited you would set the option as:
\begin{verbatim}
options['sepchar'] = '\t'
\end{verbatim}
Options are discussed at length in section \ref{sec:pypedal-options}.

\subsection{Pedigree Format Codes}
\label{sec:pedigree-format-codes}
\index{pedigree format codes}
Pedigree format codes consisting of a string of characters are used to describe
the contents of a pedigree file.  The simplest pedigree file that can be read by \PyPedal{}
is shown above; the pedigree format for this file is \texttt{asd}.  A pedigree format is required
for reading a pedigree; there is no default code used, and \PyPedal{} wil halt with an error if you
do not specify one.  You specify the format using an option statement at the start of your program:
\begin{verbatim}
options['pedformat'] = 'asd'
\end{verbatim}
Please note that the format codes are case-sensitive, which means that 'a' is considered to be a different character than 'A'.  The codes currently recognized by \PyPedal{} are:
\begin{itemize}
\item a = animal (REQUIRED)
\item s = sire (REQUIRED)
\item d = dam (REQUIRED)
\item g = generation
\item x = sex
\item b = birthyear (YYYY)
\item f = inbreeding
\item r = breed
\item n = name
\item y = birthdate in "MMDDYYYY" format
\item l = alive (1) or dead (0)
\item e = age
\item A = animal ID as a string (cannot contain sepchar)
\item S = sire ID as a string (cannot contain sepchar)
\item D = dam ID as a string (cannot contain sepchar)
\item L = alleles (two alleles separated by a non-null character)
\end{itemize}
As noted, all pedigrees must contain columns corresponding to animals, sires, and dams.  Pedigree codes should be entered in the same order in which the columns occur in the pedigee file.  The character that separates alleles when the 'L' format code is used cannot be the same character used to separate columns in the pedigree file.  If you do use the same character, \PyPedal{} will write an error message to the log file and screen and halt.

If you used an earlier version of \PyPedal{} you may have added a pedigree format string, e.g. \texttt{\% asd}, to your pedigree file(s).  You no longer need to include that string in your pedigrees, and if \PyPedal{} sees one while reading a pedigree file it will ignore that line.
\subsection{Options}
\label{sec:pypedal-options}
\index{options}
Many aspects of \PyPedal{}'s operation can be controlled using a series of options.  A complete list of these options, their defaults, and a brief desription of their purpose is presented in Table \ref{tbl:options}.  Options are stored in a Python dictionary that you must create in your programs.  You must specify values for the \texttt{pedfile} and \texttt{pedformat} options; all others are optional.  \texttt{pedfile} is a string containing the name of the file from which your pedigree will be read.  \texttt{pedformat} is a string containing a pedigree format code (see section \ref{sec:pedigree-format-codes}) for each column in the datafile in the order in which those columns occur.  The following code fragement demonstrates how options are specified.
\begin{verbatim}
options = {}
options['messages'] = 'verbose'
options['renumber'] = 0
options['counter'] = 5
options['pedfile'] = 'new_lacy.ped'
options['pedformat'] = 'asd'
options['pedname'] = 'Lacy Pedigree'
example = pyp_newclasses.NewPedigree(options)
\end{verbatim}
First, a dictionary named 'options' is created; you may use any name you like as long as it is a valid Python variable name.  Next, values are assigned to several options.  Finally, 'options' is passed to pyp_newclasses.NewPedigree(), which requires that you pass it a dictionary of options.  If you do not provide any options, \PyPedal{} will halt with an error.
\begin{center}
    \begin{table}
        \caption{Options for controlling PyPedal.}
        \label{tbl:options}
        \centerline{
        \begin{tabular}{llp{4in}}
            \hline
            Option & Default & Note(s) \\
            \hline
            alleles\_sepchar  & '/'          & The character separating the two alleles in an animal's allelotype. 'alleles\_sepchar' must NOT be the same as 'sepchar'! \\
            counter          & 1000         & How often should PyPedal write a note to the screen when reading large pedigree files. \\
            database\_name   & 'pypedal'    & The name of the database to be used when using the pyp\_reports nodule. \\
            dbtable_name     & filetag      & The name of the database table to which the current pedigree will be written when using the pyp_reports module. \\
            debug\_messages  & 0            & Indicates whether or not PyPedal should print debugging information. \\
            f_computed       & 0            & Indicates whether or not CoI have been computed for animals in the current pedigree.  If the pedigree format string includes 'f' this will be set to 1; it is also set to 1 on a successful return from pyp_nrm/inbreeding(). \\
            file\_io         & 1            & When true, routines that can write results to output files will do so and put messages in the program log to that effect. \\
            filetag          & pedfile      & A filetag is a descriptive label attached to output files created when processing a pedigree.  By default the filetag is based on 'pedfile', minus its file extension. \\
            form\_nrm        & 0            & Indicates whether or not to form a NRM and bind it to the pedigree as an instance of a NewAMatrix object. \\
            log_long_filenames & 0            & When nonzero long logfile names will be used, which means that logfilenames will include datestamps. \\
            log_ped_lines    & 0            & When \> 0 indicates how many lines read from the pedigree file should be printed in the log file for debugging purposes. \\
            logfile          & filetag.log  & The name of the file to which PyPedal should write messages about its progress. \\
            messages         & 'verbose'    & How chatty should be PyPedal be with respect to messages to the user.  'verbose' indicates that all status messages will be written to STDOUT, while 'quiet' suppresses all output to STDOUT. \\
            missing\_parent  & '0'          & Indicates what code is used to identify missing/unknown parentsin the pedigree file. \\
            nrm\_method      & 'nrm'        & Specifies that an NRM formed from the current pedigree as an instance of a NewAMatrix object should ('frm') or should not ('nrm') be corrected for parental inbreeding. \\
            pedfile          & None         & File from which pedigree is read; must provide. \\
            pedformat        & 'asd'        & See PEDIGREE\_FORMAT\_CODES for details. \\
            pedname          & 'Untitled'   & A name/title for your pedigree. \\
            pedgree\_is\_renumbered & 0     & Indicates whether or not the pedigree has been renumbered. \\
            renumber         & 0            & Renumber the pedigree after reading from file (0/1). \\
            sepchar          & ' '          & The character separating columns of input in the pedfile. \\
            set\_ancestors   & 0            & Iterate over the pedigree to assign ancestors lists to parents in the pedigree (0/1). \\
            set\_alleles     & 0            & Assign alleles for use in gene-drop simulations (0/1). \\
            set\_generations & 0            & Iterate over the pedigree to infer generations (0/1). \\
            slow\_reorder    & 1            & Option to override the slow, but more correct, reordering routine used by PyPedal by default (0/1).  ONLY CHANGE THIS IF YOU REALLY UNDERSTAND WHAT IT DOES!  Careless use of this option can lead to erroneous results. \\
            \hline
        \end{tabular}}
    \end{table}
\end{center}
A single \PyPedal{} program may be used to read one or more pedigrees.  Each pedigree that you read must be passed its own dictionary of options.  The easiest way to do this is by creating a dictionary with global options.  You can then customize the dictionary for each pedigree you want to read.  Once you have created a \PyPedal{} pedigree by calling pyp_newclasses.NewPedigree(options) you can change the options dictionary without affecting that pedigree because it has a separate copy of those options stored in its 'kw' attribute.  The following code fragment demonstrates how to read two pedigree files using the same dictionary of options.
\begin{verbatim}
options = {}
options['messages'] = 'verbose'
options['renumber'] = 0
options['counter'] = 5

if __name__ == '__main__':
#   Read the first pedigree
    options['pedfile'] = 'new_lacy.ped'
    options['pedformat'] = 'asd'
    options['pedname'] = 'Lacy Pedigree'
    example1 = pyp_newclasses.NewPedigree(options)
    example1.load()
#   Read the second pedigree
    options['pedfile'] = 'new_boichard.ped'
    options['pedformat'] = 'asdg'
    options['pedname'] = 'Boichard Pedigree'
    example2 = pyp_newclasses.NewPedigree(options)
    example2.load()
\end{verbatim}
Note that \texttt{pedformat} only needs to be changed if the two pedigrees have different formats.  Only \texttt{pedfile} \textbf{has} to be changed at all.

All pedigree options other than \texttt{pedfile} and \texttt{pedformat} have default values.  If you provide a value that is invalid the option will revert to the default.  In most cases, a message to that effect will also be placed in the log file.

\section{Logging}
\label{sec:logging}
\index{logging}
\PyPedal{} uses the \texttt{logging} module that is part of the Python standard library to record events during pedigree processing.  Informative messages, as well as warnings and errors, are written to the logfile, which can be found in the directory from which you ran \PyPedal{}.  An example of a log from a successful (error-free) run of a program is presented below:
\begin{verbatim}
Fri, 06 May 2005 10:27:22 INFO     Logfile boichard2.log instantiated.
Fri, 06 May 2005 10:27:22 INFO     Preprocessing boichard2.ped
Fri, 06 May 2005 10:27:22 INFO     Opening pedigree file
Fri, 06 May 2005 10:27:22 INFO     Pedigree comment (line 1): # This pedigree is taken from Boicherd et al. (1997).
Fri, 06 May 2005 10:27:22 INFO     Pedigree comment (line 2): # It contains two unrelated families.
Fri, 06 May 2005 10:27:22 WARNING  Encountered deprecated pedigree format string (% asdg
) on line 3 of the pedigree file.
Fri, 06 May 2005 10:27:22 WARNING  Reached end-of-line in boichard2.ped after reading 23 lines.
Fri, 06 May 2005 10:27:22 INFO     Closing pedigree file
Fri, 06 May 2005 10:27:22 INFO     Assigning offspring
Fri, 06 May 2005 10:27:22 INFO     Creating pedigree metadata object
Fri, 06 May 2005 10:27:22 INFO     Forming A-matrix from pedigree
Fri, 06 May 2005 10:27:22 INFO     Formed A-matrix from pedigree
\end{verbatim}
The \texttt{WARNING}s let you know when something unexpected or unusual has happened, although you might argue that coming to the end of an input file is neither.  If you get unexpected results from your program make sure that you check the logfile for details -- some subroutines return default values such as -999 when a problem occurs but do not halt the program.  Note that comments found in the pedigree file are written to the log, as are deprecated pedigree format strings used by earlier versions of \PyPedal{}.  When an error from which \PyPedal{} cannot recover occurs a message is written to both the screen and the logfile.  We can see from the following log that the number of columns in the pedigree file did not match the number of columns in the pedigree format option.
\begin{verbatim}
Thu, 04 Aug 2005 15:36:18 INFO     Logfile hartlandclark.log instantiated.
Thu, 04 Aug 2005 15:36:18 INFO     Preprocessing hartlandclark.ped
Thu, 04 Aug 2005 15:36:18 INFO     Opening pedigree file
Thu, 04 Aug 2005 15:36:18 INFO     Pedigree comment (line 1): # Pedigree from van Noordwijck and Scharloo (1981) as presented
Thu, 04 Aug 2005 15:36:18 INFO     Pedigree comment (line 2): # in Hartl and Clark (1989), p. 242.
Thu, 04 Aug 2005 15:36:18 ERROR    The record on line 3 of file hartlandclark.ped does not have the same number of columns (4) as the pedigree format string (asd) says that it should (3). Please check your pedigree file and the pedigree format string for errors.
\end{verbatim}
There is no sensible "best guess" that \PyPedal{} can make about handling this situation, so it halts.  There are some cases where \PyPedal{} does "guess" how it should proceed in the face of ambiguity, which is why it is always a good idea to check for \texttt{WARNING}s in your logfiles.
\chapter{Methodology}
\label{cha:methodology}
\begin{quote}
In this chapter, a high-level overview of \PYPEDAL{} is provided, giving
the reader the definitions of the key components of the system. This section
defines the concepts used by the remaining sections.
\end{quote}
\section{Reordering and Renumbering}
\label{sec:methodology-reordering-and-renumbering}
\index{reordering and renumbering}
Many computations on pedigrees require that the pedigree be renumbered such that animal IDs are consecutive from 1 to \character{n}, where \character{n} is the total number of animalsin the pedigree.  The renumbering process requires that the pedigree be reordered such that parents always precede their offspring in the list of animal IDs.  The actual ID assigned to an animal is of no particular importance, and it is even possible for parents to have larger IDs than their ofspring.  \PyPedal{} can reorder any pedigree unless there is an error in it that would prevent unambiguously placing parents before offspring.  For example, a pedigree containing a keypunch error such that an animal is one of its own grandparents cannot be reordered because there is no way to unambiguously order the animals.  The \module{pyp_utils} module provides two routines for pedigree reordering, \function{reorder()} and \function{fast\_reorder()}.  By default, \function{reorder()} is used to reorder pedigrees in place.  It does this by maintaining a list of animal IDs that have been processed; whenever a parent that is not in the  list of encountered animals the offspring of that parent are moved to the end of the pedigree.  This ensures the pedigree is properly sorted such that all parents precede their offspring.  This procedure will always correctly reorder a pedigree but it can be quite inefficient as it is similar to an insertion sort, which has a worst-case runtime proportional to $n^{2}$ \cite{Cormen2003}.

\function{fast\_reorder()} provides a much faster means of reordering a pedigree, but can incorrectly reorder a pedigree in some cases.  When an instance of a \class{NewAnimal} object is created the \method{pad_id()} method is called.  \method{pad_id()} uses the animal ID and birth year to form an ID used by \function{by pyp_utils/fast_reorder()} for quick sorting; if your pedigree file is numbered such that offspring always have larger IDs than their parents and your birth years (if provided) are correct (that is, parents always born BEFORE offspring) then \function{pyp_utils.fast_reorder()} works as expected.  If you do not provide birth years in your pedigree file but your parent IDs are always smaller than your animal IDs, the reordering will be correct.  If you do not provide birth years, all animals in the pedigree will be assigned a default value of `1900'.  In that case, if parents have IDs larger than that of one or more of their offspring, the pedigree will be incorrecrly reordered by \function{fast\_reorder()}.  If your pedigree file contains birth years, or you know that parents always have smaller IDs than their offspring, then \function{fast\_reorder()} will correctly reorder your pedigree in linear time.

The performance difference between the two reordering routines is not very noticeable on pedigrees of a few hundred to a few thousand animals, but is quite dramatic for very large pedigrees.  If your pedigree file is already reordered then there is essentially no performance difference between the two.  When creating a pedigree file from data stored in a relational database, let the database perform the sort for you by using an \samp{ORDER BY} statement.
\section{Measures of Genetic Variation}
\label{sec:methodology-genetic-variation}
\index{measures of genetic variation}
Coefficients of inbreeding and relationship \cite{Wright1922} have been commonly used to describe the genetic diversity in
livestock populations \cite{ref351}.  Inbreeding coefficients represent an individual's expected genetic homozygosity due to
the relatedness of its parents. Coefficients of relationship describe the expected proportion of genes two individuals share
due to their relatedness. These are relative measures that depend on such factors as the completeness and depth of pedigrees.
Over time, these coefficients change in response to breeding and culling decisions, and they may be used as indicators of the
genetic variability of a population. Rapid methods for calculating coefficients of inbreeding and relationship for large
populations have been implemented \cite{ref337}.

Populations under study rarely conform to the theory established for the use of coefficients of inbreeding \cite{Wright1931}.
Lacy \citeyear{ref640} and Boichard et al. \citeyear{ref352} proposed measures of genetic variation based on ideas from
conservation genetics. Lacy \citeyear{ref640} proposed the idea of the number of founder equivalents in assessing
populations. A founder is an ancestor whose parents are unknown. If all founders contribute to the population equally, then the
founder equivalent is equal to the number of founders. When founders contribute unequally to the population, the number of
founder equivalents decreases. Boichard et al. \citeyear{ref352} developed the idea of founder ancestor equivalents, which is
the minimum number of ancestors necessary to explain the genetic diversity of the current population. Founder ancestor
equivalents account for bottlenecks, unlike founder equivalents, and are more accurate in populations undergoing intense
selection.  Caballero and Toro \citeyear{ref817} discussed the relationships among these and other measures of diversity in
small populations, and demonstrate their use \cite{ref435}.

Roughsedge et al. \citeyear{ref641} used average coefficients of inbreeding, average coefficients of relationship, founder
equivalent numbers, and founder ancestor numbers to document the decrease in genetic diversity in the British dairy cattle
population over the last 25 years. Changes in founder equivalent number and founder ancestor number reflected the use of a
small number of influential individuals to improve the genetic merit of that population. Accompanying changes in average
inbreeding and relationship did not accurately reflect that loss of diversity. Such results highlight the need for additional
tools when assessing complex populations.
\section{Computational Details}
\label{sec:methodology-computational-details}
\index{computational details}
\subsection{Inbreeding and Related Measures}
\label{sec:methodology-computation-inbreeding}
\index{computational details!inbreeding and related measures}
Coefficients of relationship and inbreeding are calculated using the method of Wiggans et al. \citeyear{ref337}.  An empty dictionary is created to store animal IDs and coefficients of inbreeding.  For each animal in the pedigree, working from youngest to oldest, the dictionary is queried for the animal ID.  If the animal does not have an entry in the dictonary, a subpedigree containing only relatives of that animal is extracted and the coefficients of inbreeding are calculated and stored in the dictionary.  A second dictionary keeps track of sire-dam combinations seen in the pedigree.  If a full-sib of an animal whose pedigree has already been processed is encountered the full-sib receives a COI identical to that of the animal already processed.  This approach allows for computation of COI for arbitrarily large populations because it does not require allocation of a single NRM of order $n^{2}$, where $n$ is the size of the pedigreed population.  In most cases, the NRM for a subpedigree is on the order of 200, although this can vary with species and population data structure.

Average and maximum coefficients of inbreeding are computed for the entire population and for all individuals with non-zero inbreeding.  The average relationship among all individuals is also computed.  Theoretical and realized effective population
sizes, $N_{e(t)}$, and $N_{e(r)}$, were estimated as \cite{ref91}:

\[ N_{e(t)} = \dfrac{ 4 N_m N_f } { N_m + N_f } \]

and

\[ N_{e(t)} = \dfrac{1}{2 \Delta f} \]

where $N_m$ and $N_f$ are the number of sires and dams in the population, respectively, and $\Delta f$ is the change in
population average inbreeding between generations \textit{t} and \textit{t+1}.  Interpretation of $N_{e(t)}$ can be
problematic when $\Delta f$ is calculated from incomplete or error-prone pedigrees.
\subsection{Generation Coefficients}
\label{sec:methodology-computation-generation-coefficients}
\index{computational details!generation coefficients}
Generation coefficients are assigned using the method of \cite{Pattie1965}.  Founders, defined as individuals with unknown parents, are assigned generation codes of 0. All other animals are assigned generation codes as:

\[ GC_o = \dfrac{ ( GC_s + GC_d ) } { 2 } + 1 \]

where $GC_o$, $GC_s$, $GC_d$ represent offspring, sire, and dam codes, respectively.
\subsection{Effective Founder Number}
\label{sec:methodology-computation-effective-founder-number}
\index{computational details!effective founder number}
The effective founder number ($f_e$) was calculated as:

\[ f_e = \dfrac{ 1 } { \sum{ p_i^2 } } \]

where $p_i$ is the proportion of genes contributed by ancestor i to the current population \cite{ref640}. If all founders
had contributed equally to the population, then $f_e$ would be the same as the actual number of founders.  When founders
contribute to the population unequally, $f_e$ is smaller than the actual number of founders. The greater the inequity in
founder contributions, the smaller the effective founder number.

A subpedigree approach, similar to that used for calculation of inbreeding (see \ref{sec:methodology-computation-inbreeding} for details), is also used for calculating $f_e$.
\subsection{Founder Genome Equivalents}
\label{sec:methodology-computation-founder-genome-equivalents}
\index{computational details!founder genome equivalents}
Lacy \citeyear{ref640} also defined the number of founder genome equivalents ($f_g$) as a measure of genetic diversity.  A
founder genome equivalent is the number of founders that would produce a population with the same diversity of founder alleles
as the pedigree population assuming all founders contributed equally to each generation of descendants. Founder genome
equivalents are calculated as:

\[ f_g = \dfrac{ 1 } { \sum{ \dfrac{p_i} {r_i} } } \]

where $p_i$ is the proportion of genes contributed by ancestor \textit{i} to the current population and $r_i$ is the
proportion of founder \textit{i}'s genes that are retained in the current population.  Like $f_e$, $f_g$ accounts for
unequal founder contributions.  Unlike $f_e$, $f_g$ also accounts for the fraction of founder genomes lost from the
pedigree through drift during bottlenecks. Although $f_g$ is the more accurate description of the amount of founder variation
present in a population, it can only be calculated directly for simple pedigrees. For large or complex pedigrees, the number of
founder genome equivalents must be approximated based on computer simulation of a large number of segregations through the
pedigree. This is done by assigning each founder a unique pair of alleles and randomly transmitting those alleles through the
pedigree \cite{ref1719}. The number of founder genome equivalents is similar to the effective founder number, but the former
has been devalued based on the proportion of its genome that has probably been lost to drift \cite{ref640}.
\subsection{Effective Ancestor Number}
\label{sec:methodology-computation-effective-ancestor-number}
\index{computational details!effective ancestor number}
In populations that have undergone a bottleneck the effective number of founders computed using Lacy's \citeyear{ref640} approach is overestimated. Large contributions made by recent ancestors are more important to the population with respect to the loss of genetic diversity than equal contributions made long ago. Boichard et al. \citeyear{ref352} proposed a second measure of diversity to deal with such situations, the effective number of ancestors ($f_a$), which considers the genetic contribution of all ancestors in the population, not just founders. The effective number of ancestors treats all ancestors in the population the same way, and is computed as:

\[f_a = \dfrac{1}{\sum{q_i^2}}\]

where $q_i$ is the genetic contribution of the \textit{i}th ancestor not explained by the previous \textit{i}-1 ancestors.
The ancestors with the greatest contributions are selected iteratively.  The number of ancestors with a positive genetic
contribution is less than or equal to the actual number of founders.
\subsection{Pedigree Completeness}
\label{sec:methodology-pedigre-completeness}
\index{computational details!pedigree completeness}
Pedigree completeness \cite{Cassell2003a}, the proportion of known pedigree information for an arbitrary number of generations, is computed as:

\[ c_p = \dfrac{a_k}{\sum_{i=1}^g{2^i}} \]

where $c_p$ is pedigree completeness and $a_k$ is the number of known ancestors in \textit{g} generations.  The default (which may be overridden) is to compute four-generation pedigree completeness.  Low $c_p$ indicates that there is little pedigree information available for an individual, which may result in biased estimates of inbreeding and other measures of diversity.
\chapter{HOWTOs}
\label{cha:howtos}
\begin{quote}
In this chapter, examples of common operations are presented.
\end{quote}
\section{Basic Tasks}
\label{sec:howto-basic-operations}
\index{how do I!basic tasks}
\subsection{How do I load a pedigree from a file?}
\label{sec:howto-load-pedigree}
\index{how do I!basic tasks!load a pedigree}
Each pedigree that you read must be passed its own dictionary of options that must have at least a pedigree file name (\var{pedfile}) and a pedigree format string (\var{pedformat}).  You then call \method{pyp_newclasses.NewPedigree()} and pass the options dictionary as an argument.  The following code fragment demonstrates how to read a pedigree file:
\begin{verbatim}
options = {}
options['pedfile'] = 'new_lacy.ped'
options['pedformat'] = 'asd'

example1 = pyp_newclasses.NewPedigree(options)
example1.load()
\end{verbatim}
The options dictionary may be named anything you like.  In this manual, and in the example programs distributed with \PyPedal{}, \var{options} is the name used.
\subsection{How do I load multiple pedigrees in one program?}
\label{sec:howto-load-multiple-pedigrees}
\index{how do I!basic tasks!load multiple pedigrees}
A \PyPedal{} program can load more than one pedigree at a time.  Each pedigree must be passed its own options dictionary, and the pedigrees must have different names.  This is easily done by creating a dictionary with global options and customizing it for each pedigree.  Once you have created a pedigree by calling \method{pyp\_newclasses.NewPedigree('options')} you can change the options dictionary without affecting that pedigree (a pedigree stores a copy of the options dictionary in its \member{kw} attribute).  The following code fragment demonstrates how to read two pedigree files in a single program:
\begin{verbatim}
#   Create the empty options dictionary
options = {}

#   Read the first pedigree
options['pedfile'] = 'new_lacy.ped'
options['pedformat'] = 'asd'
options['pedname'] = 'Lacy Pedigree'
example1 = pyp_newclasses.NewPedigree(options)
example1.load()

#   Read the second pedigree
options['pedfile'] = 'new_boichard.ped'
options['pedformat'] = 'asdg'
options['pedname'] = 'Boichard Pedigree'
example2 = pyp_newclasses.NewPedigree(options)
example2.load()
\end{verbatim}
Note that \var{pedformat} only needs to be changed if the two pedigrees have different formats.  You do not even have to change \var{pedfile}.
\subsection{How do I renumber a pedigree?}
\label{sec:howto-renumber-pedigree}
\index{how do I!basic tasks!renumber a pedigree}
Set the \member{renumber} option to \samp{1} before you load the pedigree.
\begin{verbatim}
options = {}
options['renumber'] = 1
options['pedfile'] = 'new_lacy.ped'
options['pedformat'] = 'asd'
if __name__ == '__main__':
    example1 = pyp_newclasses.NewPedigree(options)
    example1.load()
\end{verbatim}
If you do not renumber a pedigree at load time and choose to renumber it later you must set the \member{renumber} option and call the pedigree's \method{renumber()} method:
\begin{verbatim}
example.kw['renumber'] = 1
example.renumber()
\end{verbatim}
For more details on pedigree renumbering see Section \ref{sec:renumbering}.
\subsection{How do I turn off output messages?}
\label{sec:howto-turn-off-messages}
\index{how do I!basic tasks!turn off output}
You may want to suppress the output that is normally written to STDOUT by scripts.  You do this by setting the \member{messages} option:
\begin{verbatim}
options['messages'] = 'quiet'
\end{verbatim}
The default setting for \member{messages} is \samp{verbose}, which produces lots of messages.
\section{Calculating Measures of Genetic Variation}
\label{sec:howto-genetic-variation}
\index{how do I!calculate genetic variation}
\subsection{How do I calculate coefficients of inbreeding?}
\label{sec:howto-calculate-inbreeding}
\index{how do I!calculate genetic variation!coefficients of inbreeding}
This requires that you have a renumbered pedigree (HOWTO \ref{sec:howto-renumber-pedigree}).
\begin{verbatim}
options = {}
options['renumber'] = 1
options['pedfile'] = 'new_lacy.ped'
options['pedformat'] = 'asd'
example1 = pyp_newclasses.NewPedigree(options)
example1.load()
example_inbreeding = pyp_nrm.inbreeding(example)
print example_inbreeding
\end{verbatim}
The dictionary returned by \function{pyp_nrm.inbreeding(example)}, \var{example_inbreeding}, contains two dictionaries: \var{fx} contains coefficients of inbreeding (COI) keyed to renumbered animal IDs and \var{metadata} contains summary statistics.  \var{metadata} also contains two dictionaries: \var{all} contains summary statistics for all animals, while \var{nonzero} contains summary statistics for only animals with non-zero coefficients of inbreeding.  If you print \var{example_inbreeding} you'll get the following:
\begin{verbatim}
{'fx': {1: 0.0, 2: 0.0, 3: 0.0, 4: 0.0, 5: 0.0, 6: 0.0, 7: 0.0, 8: 0.0, 9: 0.0,
10: 0.0, 11: 0.0, 12: 0.0, 13: 0.0, 14: 0.0, 15: 0.0, 16: 0.0, 17: 0.0, 18: 0.0,
19: 0.0, 20: 0.0, 21: 0.0, 22: 0.0, 23: 0.0, 24: 0.0, 25: 0.0, 26: 0.0, 27: 0.0,
28: 0.25, 29: 0.0, 30: 0.0, 31: 0.25, 32: 0.0, 33: 0.0, 34: 0.0, 35: 0.0, 36: 0.0,
37: 0.0, 38: 0.21875, 39: 0.0, 40: 0.0625, 41: 0.0, 42: 0.0, 43: 0.03125, 44: 0.0,
45: 0.0, 46: 0.0, 47: 0.0},
'metadata': {'nonzero': {'f_max': 0.25, 'f_avg': 0.16250000000000001,
'f_rng': 0.21875, 'f_sum': 0.8125, 'f_min': 0.03125, 'f_count': 5},
'all': {'f_max': 0.25, 'f_avg': 0.017287234042553192, 'f_rng': 0.25,
'f_sum': 0.8125, 'f_min': 0.0, 'f_count': 47}}}
\end{verbatim}
Obtaining the COI for a given animal, say 28, is simple:
\begin{verbatim}
>>> print example_inbreeding['fx'][28]
'0.25'
\end{verbatim}
To print the mean COI for the pedigree:
\begin{verbatim}
>>> print example_inbreeding['metadata']['all']['f_avg']
'0.017287234042553192'
\end{verbatim}
\section{Databases and Report Generation}
\label{sec:howto-databases-and-reports}
\index{how do I!databases and reports}
\subsection{How do I load a pedigree into a database?}
\label{sec:howto-load-pedigree-db}
\index{how do I!databases and reports!load a pedigree}
The \module{pyp\_reports} module (\ref{sec:pyp-reports}) uses the \module{pyp\_db} module (Section \ref{sec:pyp-db})
to store and manipulate a pedigree in an SQLite database.  In order to use these tools you must first load your pedigree into
the database.  This is done with a call to \function{pyp\_db.loadPedigreeTable()}:
\begin{verbatim}
options = {}
options['pedfile'] = 'hartlandclark.ped'
options['pedname'] = 'Pedigree from van Noordwijck and Scharloo (1981)'
options['pedformat'] = 'asdb'

example = pyp_newclasses.NewPedigree(options)
example.load()

pyp_nrm.inbreeding(example)
pyp_db.loadPedigreeTable(example)
\end{verbatim}
The routines in \module{pyp\_reports} will check to see if your pedigree has already been loaded; if it
has not, a table will be created and populated for you.
\subsection{How do I update a pedigree in the database?}
\label{sec:howto-pedigree-db-update-table}
\index{how do I!databases and reports!update pedigree table}
Changes to a \PyPedal{} pedigree object are not automatically saved to the database.  If you have changed
your pedigree, such as by calculating coefficients of inbreeding, and you want those changes visible to the
database you have to call \function{pyp\_db.loadPedigreeTable()} again.  \textbf{IMPORTANT NOTE:} If you call
\function{pyp\_db.loadPedigreeTable()} after you have already loaded your pedigree into the database it will
drop the existing table and reload it; all data in the existing table will be lost!  In the following
example, the pedigree is written to table \textbf{hartlandclark} in the database \textbf{pypedal}:
\begin{verbatim}
options = {}
options['pedfile'] = 'hartlandclark.ped'
options['pedname'] = 'Pedigree from van Noordwijck and Scharloo (1981)'
options['pedformat'] = 'asdb'

example = pyp_newclasses.NewPedigree(options)
example.load()

pyp_db.loadPedigreeTable(example)
\end{verbatim}
\member{pypedal} is the default database name used by \PyPedal{}, and can be changed using a pedigree's \member{database_name} option.  By default, table names are formed from the pedigree file name.  A table name can be specified using a pedigree's \member{dbtable_name} option.  Continuing the above example, suppose that I calculated coefficients of inbreeding on my pedigree and want to store the resulting pedigree in a new table named \var{noordwijck_and_scharloo_inbreeding}:
\begin{verbatim}
options['dbtable_name'] = 'noordwijck_and_scharloo_inbreeding'
pyp_nrm.inbreeding(example)
pyp_db.loadPedigreeTable(example)
\end{verbatim}
You should see messages in the log telling you that the table has been created and populated:
\begin{verbatim}
Tue, 29 Nov 2005 11:24:22 WARNING  Table noordwijck_and_scharloo_inbreeding does
                                   not exist in database pypedal!
Tue, 29 Nov 2005 11:24:22 INFO     Table noordwijck_and_scharloo_inbreeding
                                   created in database pypedal!
\end{verbatim}
\section{Contribute a HOWTO}
\label{sec:howto-contribute}
\index{how do I!contribute a HOWTO}
Users are invited to contribute HOWTOs demonstrating how to solve problems they've found interesting.  In order for such HOWTOs to be considered for inclusion in this manual they must be licensed under the GNU Free Documentation License version 1.2 or later (\url{http://www.gnu.org/copyleft/fdl.html}).  Authorship will be acknowledged, and copyright will remain with the author of the HOWTO.
\chapter{Graphics}
\label{cha:graphics}
\begin{quote}
This chapter presents an overview of using the graphics routines \PyPedal{}.
\end{quote}
\section{PyPedal Graphics}
\label{sec:graphics-overview}
\index{graphics}
\PyPedal{} is capable of producing graphics from information contained in a pedigree, including pedigree drawings, line graphs of changes in genetic diversity over time, and visualizations of numerator relationship matrices.  These graphics are non-interactive: output images are created and written to output files.  A separate program must be used to view and/or print the image; web browsers make reasonably good viewers for a small number of images.  If you are creating and viewing large numbers of images you may want to obtain an image management package for your platform.  Default and supported file formats for each of the graphics routines are presented in Table \ref{tbl:pypedal-graphics-formats}.
\begin{center}
    \begin{table}
        \caption{Default graphics formats.}
        \label{tbl:pypedal-graphics-formats}
        \centerline{
        \begin{tabular}{llp{2in}}
            \hline
            Routine & Default Format & Supported Formats \\
            \hline
            draw\_pedigree & JPG & JPG, PNG, PS \\
            pcolor\_matrix\_pylab & PNG & PNG only \\
            plot\_founders\_by\_year & PNG & PNG only \\
            plot\_founders\_pct\_by\_year & PNG & PNG only \\
            plot\_line\_xy & PNG & PNG only \\
            rmuller\_pcolor\_matrix\_pil & PNG & PNG only \\
            rmuller\_spy\_matrix\_pil & PNG & PNG only \\
            spy\_matrix\_pylab & PNG & PNG only \\
            \hline
        \end{tabular}}
    \end{table}
\end{center}
\subsection{Drawing Pedigrees}
\label{sec:graphics-drawing-pedigrees}
\index{graphics!drawing pedigrees}
The pedigree from Figure 2 in Boichard et al. \citeyear{ref352} is shown in Figure \ref{fig:boichard2-pedigree}, and shows males enclosed in rectangles and females in ovals.  Figure \ref{fig:new-ids2-pedigree-basic} shows a pedigree in which strings are used for animal IDs; animal are enclosed in ovals because sexes were not specified in the pedigree file and the \member{set\_sexes} option was not specified.  A more complex German Shepherd pedigree is presented in Figure \ref{fig:doug-pedigree-basic}; the code used to create this pedigree is:
\begin{verbatim}
pyp_graphics.draw_pedigree(example, gfilename='doug_p_rl_notitle', gname=1,
    gdirec='RL', gfontsize=12)
\end{verbatim}
\begin{figure}
  \begin{center}
    \includegraphics[width=4in]{boichard2Pedigree.eps}
    \caption{Pedigree 2 from Boichard et al. (1997)}
    \label{fig:boichard2-pedigree}
  \end{center}
\end{figure}
\begin{figure}
  \begin{center}
    \includegraphics[width=3in]{BoichardPedigreeBasic.eps}
    \caption{A pedigree with strings as animal IDs}
    \label{fig:new-ids2-pedigree-basic}
  \end{center}
\end{figure}
\begin{figure}
  \begin{center}
    \includegraphics[width=4in]{dougPRlNotitle.eps}
    \caption{German Shepherd pedigree}
    \label{fig:doug-pedigree-basic}
  \end{center}
\end{figure}
The resulting graphic is written to doug\_p\_rl\_notitle.jpg; note from Table \ref{tbl:pypedal-graphics-formats} that the default file format for \function{draw\_pedigree()} is \textbf{JPG} rather than \textbf{PNG}, as is the case for the other graphics routines.  To get a PNG simply pass the argument \var{gformat='png'} to \function{draw\_pedigree()}.  For details on the options taken by \function{draw\_pedigree()} please refer to the API documentation (Section \ref{sec:pyp-graphics-draw-pedigree}).
\subsection{Drawing Line Graphs}
\label{sec:graphics-drawing-line-graphs}
\index{graphics!drawing line graphs}
The \texttt{plot\_line\_xy()} routine provides a convenient tool for creating two-dimensional line graphs.  Figure \ref{fig:ayrshire-coi-graph} shows the plot of inbreeding by birth year for the US Ayrshire cattle population.  The plot is produced by the call:
\begin{verbatim}
pyp_db.loadPedigreeTable(ay)
coi_by_year = pyp_reports.meanMetricBy(ay,metric='fa',byvar='by')
cby = coi_by_year
del(cby[1900])
pyp_graphics.plot_line_xy(coi_by_year, gfilename='ay_coi_by_year',
    gtitle='Inbreeding coefficients for Ayrshire cows', gxlabel='Birth year',
    gylabel='Coefficient of inbreeding')
\end{verbatim}
\begin{figure}
  \begin{center}
    \includegraphics[width=4in]{ayCoiByYear.eps}
    \caption{Average inbreeding by birth year for the US Ayrshire cattle population}
    \label{fig:ayrshire-coi-graph}
  \end{center}
\end{figure}
The code above uses \function{pyp_reports.meanMetricBy()} (see \ref{sec:pyp-reports-mean-metric-by}) to populate \var{coi_by_year}; the keys in \var{coi\_by\_year} are plotted in the x-axis, and the values are plotted on the y-axis.  The default birth year, 1900, was deleted from the dictionary before the plot was drawn because leaving the default birthyear in the plot was distracting and somewhat misleading.  The only restriction that you have to observe is that the value plotted on the y-ais has to be a numeric quantity.

If you need more complicated plots than are produced by \function{plot\_line\_xy()} you can write a new plotting function (Chapter \ref{cha:newfeatures}) that uses the tools in matplotlib (\url{http://matplotlib.sourceforge.net/}).  For complete details on the options taken by \function{plot\_line\_xy{}} please refer to the API documentation (\ref{sec:pyp-graphics-plot-line-xy}).
\subsection{Visualizing Numerator Relationship Matrices}
\label{sec:graphics-visualizing-nrm}
\index{graphics!visualizing relationship matrices}
Two routines are provided for visualization of numerator relationship matrices (NRM), \function{rmuller_pcolor_matrix_pil()} and \function{rmuller_spy_matrix_pil()}.

As an example, we will consider the NRM for the pedigree in Figure \ref{fig:boichard2-pedigree}.  The matrix is square and symmetric; the diagonal values correspond to $1+f_a$, where $f_a$ is an animal's coefficient of inbreeding; animals with a diagonal element $>1$ are inbred.
\[
    \scriptsize
    \left[ \begin{array}{llllllllllllllllllll}
        1. & 0. & 0. & 0. & 0.5 & 0. & 0.25 & 0.25 & 0.25 & 0.25 & 0.25 & 0.25 & 0.25 & 0.25 & 0. & 0. & 0. & 0. & 0. & 0. \\
        0. & 1. & 0. & 0. & 0.5 & 0. & 0.25 & 0.25 & 0.25 & 0.25 & 0.25 & 0.25 & 0.25 & 0.25 & 0. & 0. & 0. & 0. & 0. & 0. \\
        0. & 0. & 1. & 0. & 0.  & 0.5 & 0.25 & 0.25 & 0.25 & 0.25 & 0.25 & 0.25 & 0.25 & 0.25 & 0. & 0. & 0. & 0. & 0. & 0. \\
        0. & 0. & 0. & 1. & 0. & 0.5 & 0.25 & 0.25 & 0.25 & 0.25 & 0.25 & 0.25 & 0.25 & 0.25 & 0. & 0. & 0. & 0. & 0. & 0. \\
        0.5 & 0.5 & 0. & 0. & 1. & 0. & 0.5 & 0.5 & 0.5 & 0.5 & 0.5 & 0.5 & 0.5 & 0.5 & 0. & 0. & 0. & 0. & 0. & 0. \\
        0. & 0. & 0.5 & 0.5 & 0. & 1. & 0.5 & 0.5 & 0.5 & 0.5 & 0.5 & 0.5 & 0.5 & 0.5 & 0. & 0. & 0. & 0. & 0. & 0. \\
        0.25 & 0.25 & 0.25 & 0.25 & 0.5 & 0.5 & 1. & 0.5 & 0.5 & 0.5 & 0.5 & 0.5 & 0.5 & 0.5 & 0. & 0. & 0. & 0. & 0. & 0. \\
        0.25 & 0.25 & 0.25 & 0.25 & 0.5 & 0.5 & 0.5 & 1. & 0.5 & 0.5 & 0.5 & 0.5 & 0.5 & 0.5 & 0. & 0. & 0. & 0. & 0. & 0. \\
        0.25 & 0.25 & 0.25 & 0.25 & 0.5 & 0.5 & 0.5 & 0.5 & 1. & 0.5 & 0.5 & 0.5 & 0.5 & 0.5 & 0. & 0. & 0. & 0. & 0. & 0. \\
        0.25 & 0.25 & 0.25 & 0.25 & 0.5 & 0.5 & 0.5 & 0.5 & 0.5 & 1. & 0.5 & 0.5 & 0.5 & 0.5 & 0. & 0. & 0. & 0. & 0. & 0. \\
        0.25 & 0.25 & 0.25 & 0.25 & 0.5 & 0.5 & 0.5 & 0.5 & 0.5 & 0.5 & 1. & 0.5 & 0.5 & 0.5 & 0. & 0. & 0. & 0. & 0. & 0. \\
        0.25 & 0.25 & 0.25 & 0.25 & 0.5 & 0.5 & 0.5 & 0.5 & 0.5 & 0.5 & 0.5 & 1. & 0.5 & 0.5 & 0. & 0. & 0. & 0. & 0. & 0. \\
        0.25 & 0.25 & 0.25 & 0.25 & 0.5 & 0.5 & 0.5 & 0.5 & 0.5 & 0.5 & 0.5 & 0.5 & 1. & 0.5 & 0. & 0. & 0. & 0. & 0. & 0. \\
        0.25 & 0.25 & 0.25 & 0.25 & 0.5 & 0.5 & 0.5 & 0.5 & 0.5 & 0.5 & 0.5 & 0.5 & 0.5 & 1. & 0. & 0. & 0. & 0. & 0. & 0. \\
        0. & 0. & 0. & 0. & 0. & 0. & 0. & 0. & 0. & 0. & 0. & 0. & 0. & 0. & 1. & 0. & 0.5 & 0.5 & 0.5 & 0.5 \\
        0. & 0. & 0. & 0. & 0. & 0. & 0. & 0. & 0. & 0. & 0. & 0. & 0. & 0. & 0. & 1. & 0.5 & 0.5 & 0.5 & 0.5 \\
        0. & 0. & 0. & 0. & 0. & 0. & 0. & 0. & 0. & 0. & 0. & 0. & 0. & 0. & 0.5 & 0.5 & 1. & 0.5 & 0.75 & 0.75 \\
        0. & 0. & 0. & 0. & 0. & 0. & 0. & 0. & 0. & 0. & 0. & 0. & 0. & 0. & 0.5 & 0.5 & 0.5 & 1. & 0.75 & 0.75 \\
        0. & 0. & 0. & 0. & 0. & 0. & 0. & 0. & 0. & 0. & 0. & 0. & 0. & 0. & 0.5 & 0.5 & 0.75 & 0.75 & 1.25 & 0.75 \\
        0. & 0. & 0. & 0. & 0. & 0. & 0. & 0. & 0. & 0. & 0. & 0. & 0. & 0. & 0.5 & 0.5 & 0.75 & 0.75 & 0.75 & 1.25
    \end{array} \right]
    \normalsize
\]
Note that the array only contains six distinct values: 0., 0.25, 0.5, 0.75, 1.0, and 1.25.  These six values will be used to create the color map used by \function{rmuller_pcolor_matrix_pil()}.

\function{rmuller_pcolor_matrix_pil()} produces pseudocolor plots from NRM.  A pseudocolor plot is an array of cells that are colored based on the values the corresponding cells in the NRM. The minimum and maximum values in the NRM are assigned the first and last colors in the colormap; other cells are colored by mapping their values to colormap elements.  In the example above, the minimum value is 0.0 and the maximum value is 1.0 (Figure \ref{fig:boichard2-pseudocolor}).  The two inbred animals in the population are easily identified as the yellow diagonal elements in the bottom-left corner of the matrix.
\begin{figure}[tb]
  \begin{center}
    \includegraphics[width=3in]{boichard2Pcolor.eps}
    \caption{Pseudocolored NRM from the Boichard et al. (1997) pedigree}
    \label{fig:boichard2-pseudocolor}
  \end{center}
\end{figure}
\function{rmuller_spy_matrix_pil()} is similar to \function{rmuller_pcolor_matrix_pil()}, but it is used to visualize the sparsity of a matrix.  Cells are either filled, indicating that the value is non-zero, or not filled, indicating that the cell's value is zero.  In Figure \ref{fig:boichard2-sparsity} it is easy to see the two separate families in the pedigree.
\begin{figure}[tb]
  \begin{center}
    \includegraphics[width=3in]{boichard2Spy.eps}
    \caption{Sparsity of the NRM from the Boichard et al. (1997) pedigree}
    \label{fig:boichard2-sparsity}
  \end{center}
\end{figure}
\chapter{Report Generation}
\label{cha:reports}
\begin{quote}
An overview of the report generation tools in \PyPedal{} is provided in this chapter.  The creation of a new, custom report
is demonstrated.
\end{quote}
\section{Overview}
\label{sec:reports-overview}
\index{report generation}
\PyPedal{} has a framework in place to support basic report generation.  This franework consists of two components: a database access module, \module{pyp\_db} (Section \ref{sec:pyp-db}), and a reporting module, \module{pyp\_reports} (Section \ref{sec:pyp-reports}).  The SQLite 3 database engine (\url{http://www.sqlite.org/}) is used to store data and generate reports.  The ReportLab extension to Python (\url{http://www.reportlab.org/}) allows users to create reports in the Adobe Portable Document Format (PDF).  As a result, there are two types of reports that can be produced: internal summaries that can be fed to other \PyPedal{} routines (e.g. the report produced by \function{pyp\_reports.meanMetricBy()} can be passed to \function{pyp\_graphics.plot\_line\_xy()} to produce a plot) and printed reports in PDF format.  When referencing the \module{pyp\_reports} API note that the convention used in \PyPedal{} is that procedures which produce PDFs are prepended with 'pdf'.  Sections \ref{sec:reports-custom-internal-reports} and \ref{sec:reports-custom-printed-reports} demonstrate how to create new or custom reports.  \module{pyp\_reports} was added to \PyPedal{} with the intention that end-users develop their own custom reports using \function{pyp\_reports.meanMetricBy()} as a template.  More material on adding new functionality to \PyPedal{} can be found in Chapter \ref{cha:newfeatures}.

Column names, data types, and descriptions of contents for pedigree tables are presented in Table
\ref{tbl:reports-db-column-names}.  The \constant{metric\_to\_column} and \constant{byvar\_to\_column} dictionaries in
\module{pyp\_db} are used to convert between convenient mnemonics and database column names.  You may need to refer
to Table \ref{tbl:reports-db-column-names} for unmapped column names when writing custom reports.  If you happen
to view a table scheme using the \textbf{sqlite3} command-line utility you will notice that the columns are ordered
differently in the database than they are in the table; the table has been alphabetized for easy reference.
\begin{center}
    \begin{table}
        \caption{Columns in pedigree database tables.}
        \label{tbl:reports-db-column-names}
        \centerline{
        \begin{tabular}{llp{2.5in}}
            \hline
            Name          &   Type          & Note(s) \\
            \hline
            age           &   real          &  Age of animal \\
            alive         &   char(1)       &  Animal's mortality status \\
            ancestor      &   char(1)       &  Ancestor status \\
            animalID      &   integer       &  \textbf{Must be unique!} \\
            animalName    &   varchar(128)  &  Animal name \\
            birthyear     &   integer       &  Birth year \\
            breed         &   text          &  Breed \\
            coi           &   real          &  Coefficient of inbreeding \\
            damID         &   integer       &  Dam's ID \\
            founder       &   char(1)       &  Founder status \\
            gencoeff      &   real          &  Pattie's generation coefficient \\
            generation    &   real          &  Generation \\
            herd          &   integer       &  Herd ID \\
            infGeneration &   real          &  Inferred generation \\
            num\_daus     &   integer       &  Number of daughters \\
            num\_sons     &   integer       &  Number of sons \\
            num\_unk      &   integer       &  Offspring of unknown sex \\
            originalHerd  &   varchar(128)  &  Original herd ID \\
            originalID    &   text          &  Animal's original ID \\
            pedgreeComp   &   real          &  Pedigree completeness \\
            renumberedID  &   integer       &  Animal's renumbered ID \\
            sex           &   char(1)       &  Sex of animal \\
            sireID        &   integer       &  Sire's ID \\
            \hline
        \end{tabular}}
    \end{table}
\end{center}
\section{Creating a Custom Internal Report}
\label{sec:reports-custom-internal-reports}
\index{report generation!creating custom internal reports}
Internal reports \index{internal reports} typically aggregate data such that the result can be handed off to another \PyPedal{} routine for further processing.   To do this, the pedigree is loaded into a table in an SQLite database against which queries are made.  This is faster and more flexible than writing reporting routines that loop over the pedigree to construct reports, but it does require some knowledge of the Structured Query Language (SQL; \url{http://www.sql.org/}).  The canonical example of this kind of report is the passing of the dictionary returned by \function{pyp\_reports.meanMetricBy()} to \function{pyp\_graphics.plot\_line\_xy()} (see \ref{sec:graphics-drawing-pedigrees}).  That approach is outlined in code below.
\begin{verbatim}
def inbreedingByYear(pedobj):
    curs = pyp_db.getCursor(pedobj.kw['database_name'])

    # Check and see if the pedigree has already been loaded.  If not, do it.
    if not pyp_db.tableExists(pedobj.kw['database_name'], pedobj.kw['dbtable_name']):
        pyp_db.loadPedigreeTable(pedobj)

    MYQUERY = "SELECT birthyear, pyp_mean(coi) FROM %s GROUP BY birthyear \
        ORDER BY birthyear ASC" % (pedobj.kw['dbtable_name'])
    curs.execute(MYQUERY)
    myresult = curs.fetchall()
    result_dict = {}
    for _mr in myresult:
        _level, _mean = _mr
        result_dict[_level] = _mean
    return result_dict
\end{verbatim}
You should always check to see if your pedigree has been loaded into the database before you try and make queries against the pedigree table or your program may crash.  \function{inbreedingByYear()} returns a dictionary containing average coefficients of inbreeding keyed to birth years.  The query result, \var{myresult}, is a list of tuples; each tuple in the list
corresponds to one row in an SQL resultset. The tuples in \var{myresult} are unpacked into temporary variables that are then stored in the dictionary, \var{result\_dict} (for information on tuples see the Python Tutorial (\url{http://www.python.org/doc/tut/node7.html#SECTION007300000000000000000}).  If the resultset is empty, \var{result\_dict} will also be empty.  As long as you can write a valid SQL query for the report you'd like to assemble, there is no limitation on the reports that can be prepared by \PyPedal{}.
\section{Creating a Custom Printed Report}
\label{sec:reports-custom-printed-reports}
\index{report generation!creating custom printed reports}
If you are interested in custom printed reports you should begin by opening the file \texttt{pyp\_reports.py} and reading
through the code for the \function{pdfPedigreeMetadata()} report.  It has been heavily commented so that it can be used as
a template for developing other reports.  ReportLab provides fairly low-level tools that you can use to assemble
documents.  The basic idea is that you create a canvas on which your image will be drawn.  You then create text objects and
draw them on the canvas.  When your report is assembled you save the canvas on which it's drawn to a file.  \PyPedal{} provides
a few convenience functions for such commonly-used layouts as title pages and page "frames".  In the following sections of code I will discuss the creation of a \function{pdfInbreedingByYear()} printed report to accompany the
\function{inbreedingByYear()} internal report written in Section \ref{sec:reports-custom-internal-reports}.  First, we import ReportLab and check to see if the user provided an output file name.  If they didn't, revert to a default.
\begin{verbatim}
def pdfInbreedingByYear(pedobj,results,titlepage=0,reporttitle='',reportauthor='', \
    reportfile=''):
    import reportlab
    if reportfile == '':
        _pdfOutfile = '%s_inbreeding_by_year.pdf' % ( pedobj.kw['default_report'] )
    else:
        _pdfOutfile = reportfile
\end{verbatim}
Next call \function{\_pdfInitialize()}, which returns a dictionary of settings, mostly related to page size and
margin locations, that is used throughout the routine.  \function{\_pdfInitialize()} uses the \member{paper\_size} keyword
in the pedigree's options dictionary, which is either `letter' or `A4', and the \member{default\_unit}, which is either
`inch' or `cm' to populate the returned structure.  This should allow users to move between paper sizes without little
or no work.  Once the PDF settings have been computed we instantiate a canvas object on which to draw.
\begin{verbatim}
_pdfSettings = _pdfInitialize(pedobj)
canv = canvas.Canvas(_pdfOutfile, invariant=1)
canv.setPageCompression(1)
\end{verbatim}
There is a hook in the code to toggle cover pages on and off.  It is arguably rather pointless to put a cover page on a one-page document, but all TPS reports require new coversheets.  The call to \function{\_pdfDrawPageFrame()} frames the page with a header and footer that includes the pedigree name, date and time the report was created, and the page number.
\begin{verbatim}
if titlepage:
    if reporttitle == '':
        reporttitle = 'meanMetricBy Report for Pedigree\n%s' \
            % (pedobj.kw['pedname'])
    _pdfCreateTitlePage(canv, _pdfSettings, reporttitle, reportauthor)
_pdfDrawPageFrame(canv, _pdfSettings)
\end{verbatim}
The largest chunk of code in \function{pdfInbreedingByYear()} is dedicated to looping over the input dictionary, \var{results}, and writing its contents to text objects.  If you want to change the typeface for the rendered text, you need to make the appropriate changes to all calls to \texttt{canv.setFont("Times-Bold", 12)}.  The ReportLab documentation includes a discussion of available typefaces.
\begin{verbatim}
canv.setFont("Times-Bold", 12)
tx = canv.beginText( _pdfSettings['_pdfCalcs']['_left_margin'],
    _pdfSettings['_pdfCalcs']['_top_margin'] - 0.5 * \
        _pdfSettings['_pdfCalcs']['_unit'] )
\end{verbatim}
Every printed report will have a section of code in which the input is processed and written to text objects. In this case, the code loops over the key-and-value pairs in \var{results}, determines the width of the key, and creates a string with the proper spacing between the key and its value.  That string is then written to a \method{tx.textLine()} object.
\begin{verbatim}
# This is where the actual content is written to a text object that
# will be displayed on a canvas.
for _k, _v in results.iteritems():
    if len(str(_k)) <= 14:
        _line = '\t%s:\t\t%s' % (_k, _v)
    else:
        _line = '\t%s:\t%s' % (_k, _v)
    tx.textLine(_line)
\end{verbatim}
ReportLab's text objects do not automatically paginate themselves.  If you write, say, ten pages of material to a text object and render it without manually paginating the object you're going to get a single page of chopped-off text.  The following section of code is where the actual pagination occurs, so careful cutting-and-pasting should make pagination seamless.
\begin{verbatim}
    # Paginate the document if the contents of a textLine are longer than one page.
    if tx.getY() < _pdfSettings['_pdfCalcs']['_bottom_margin'] + \
        0.5 * _pdfSettings['_pdfCalcs']['_unit']:
        canv.drawText(tx)
        canv.showPage()
        _pdfDrawPageFrame(canv, _pdfSettings)
        canv.setFont('Times-Roman', 12)
        tx = canv.beginText( _pdfSettings['_pdfCalcs']['_left_margin'],
            _pdfSettings['_pdfCalcs']['_top_margin'] -
            0.5 * _pdfSettings['_pdfCalcs']['_unit'] )
\end{verbatim}
Once we're done writing our text to text objects we need to draw the text object on the canvas and make the canvas visible.  If you omit this step, perhaps because of the kind of horrible cutting-and-pasting accident to which I am prone, your PDF will not be written to a file.
\begin{verbatim}
if tx:
    canv.drawText(tx)
    canv.showPage()
canv.save()
\end{verbatim}
While \PyPedal{} does not yet have any standard reports that include graphics, ReportLab does support adding graphics, such as a pedigree drawing, to a canvas.  Interested readers should refer to the ReportLab documentation.
\chapter{Implementing New Features}
\label{cha:newfeatures}
\begin{quote}
In this chapter, an example of wil be provided of how to extend \PyPedal{} by creating a user-defined routine.  New routines may implement a new measure of genetic diversity, extend the graphics module, add a new report, or group a series of actions into a single convenient routine.
\end{quote}
\section{Overview}
\label{sec:newfeatures-overview}
\index{new features}
One of the appealing features of \PyPedal{} is its easy extensibility.  In this section, we will demonstrate how to add a
user-written module to \PyPedal{}.  The file \texttt{pyp\_template.py} that is distributed with \PyPedal{} is a
skeleton that can be used to help you get started writing your custom module(s).  You should also look at the source
code of the standard modules, particularly if there is already a routine that does something similar to what you would
like to do, to see if you can jump-start your project by reusing code.
\subsection{Defining the Problem}
\label{sec:newfeatures-overview-problem}
\index{new features!defining the problem}
Before you open your editor and begin writing code you need to clearly define your problem.  Answering a few questions can
help you do this:
\begin{itemize}
\item What output do I want from my routine?
\item What calculations do I need to perform?
\item What input do I need to give my routine in order to perform those calculations?
\item Are there any \PyPedal{} routines that already do something similar?
\end{itemize}
The last question is as important as the others --- if there is already a \PyPedal{} routine that does similar calculations
you can use it as a starting point.  Code reuse is a great idea.

The problem that will motivate the rest of this section sounds very tricky, but is not really so bad because we are going to
reuse a lot of code.  I want to create a routine for drawing pedigrees that color nodes (animals) based on their importance as
measured by their connectedness to other animals in the pedigree.  After a brief review of the contents of the Module Template
in Section \ref{sec:newfeatures-template}, I will present a detailed solution to this problem in Section
\ref{sec:newfeatures-solving-the-problem}.
\section{Module Template}
\label{sec:newfeatures-template}
\index{new features!module template}
The file \file{pyp\_template.py} is a skeleton that can be used to get started writing a custom module.  The first thing you should do is save a copy of \file{pyp\_template.py} with your working module name; we will use the filename \file{pyp\_jbc.py} for the following example.  You should also fill-in the module header so that it contains your name, e-mail address, etc.  The version number of your module does not have to match that of the main \PyPedal{} distribution, and is only used as an aid to the programmer.
\index{new features!module template!header}
\begin{verbatim}
###############################################################################
# NAME: pyp_jbc.py
# VERSION: 1.0.0 (16NOVEMBER2005)
# AUTHOR: John B. Cole, PhD (jcole@aipl.arsusda.gov)
# LICENSE: LGPL
###############################################################################
# FUNCTIONS:
#     get_color_32()
#     color_pedigree()
#     draw_colored_pedigree()
###############################################################################
\end{verbatim}
The imports section of the template includes \code{import} statements for all of the standard \PyPedal{} modules.  There's
no harm in including all of them in your module, but it's good practice to include only the modules you need. You should
always include the \module{logging} module because it's needed for communicating with the log file.  For \module{pyp\_jbc}
I am including only the \module{pyp\_graphics}, \module{pyp\_network}, and \module{pyp\_utils} modules.
\index{new features!module template!imports}
\begin{verbatim}
##
# pyp_jbc provides tools for enhanced pedigree drawing.
##
import logging
from PyPedal import pyp_graphics
from PyPedal import pyp_network
from PyPedal import pyp_utils
\end{verbatim}
There is a very sketchy function prototype included in the template.  It is probably enough for you to get started if you have
a little experience programming in Python.  If you don't have any experience programming in Python you should be able to get
up-and-running with a little trial-and-error and some study of \PyPedal{} source.  You should always write a comment block similar to that attached to \function{yourFunctionName()} for each of your functions.  This comment block is recognized by PythonDoc, a tool for automatically generating program documentation.  Parameters are the inputs that you send to a function, return is a description of the function's output, and defreturn is the type of output that is returned, such as a list, dictionary, integer, or tuple.
\index{new features!module template!function prototype}
\begin{verbatim}
##
# yourFunctionName() <description of what function does>
# @param <parameter_name> <parameter description>
# @return <description of returned value(s)
# @defreturn <type of returned data, e.g., 'dictionary' or 'list'>
def yourFunctionName(pedobj):
    try:
        # Do something here
        logging.info('pyp_template/yourFunctionName() did something.')
        # return a value/dictionary/etc.
    except:
        logging.error('pyp_template/yourFunctionName() encountered a problem.')
        return 0
\end{verbatim}
\section{Solving the Problem}
\label{sec:newfeatures-solving-the-problem}
\index{new features!solving the problem}
The measure of connectedness I am going to use for coloring the pedigree is the proportion of animals in the pedigree that are descended from each animal in the pedigree.  In order to do this we need to do the following:
\begin{enumerate}
\item Compute the proportion of animals in the pedigree that are descended from each animal in the pedigree; the values will
be stored in a dictionary keyed by animal IDs.
\item Map the proportion of descendants from decimal values on the interval (0,1) to RGB triples.
\item Use the RGB triples to set the fill color for nodes.
\end{enumerate}
There is not an existing function for the first item, but there is a function in the \module{pyp\_network} module, \function{find_descendants()}, for identifying all of the descendants of an animal.  We can use the length of the list of descendants and the number of animals in the pedigree to calculate the proportion of animals in the pedigree descended from that animal.  The \function{color\_pedigree()} function creates a dictionary and loops over the pedigree to compute the proporions.  It also calls \function{draw\_colored\_pedigree()}, which is a modified version of \function{pyp\_graphics.draw\_pedigree()}, to draw the pedigree with colored nodes.
\begin{verbatim}
##
# color_pedigree() forms a graph object from a pedigree object and
# determines the proportion of animals in a pedigree that are
# descendants of each animal in the pedigree.  The results are used
# to feed draw_colored_pedigree().
# @param pedobj A PyPedal pedigree object.
# @return A 1 for success and a 0 for failure.
# @defreturn integer
def color_pedigree(pedobj):
    _pedgraph = pyp_network.ped_to_graph(pedobj)
    _dprop = {}
    # Walk the pedigree and compute proportion of animals in the
    # pedigree that are descended from each animal.
    for _p in pedobj.pedigree:
        _dcount = pyp_network.find_descendants(_pedgraph,_p.animalID,[])
        if len(_dcount) < 1:
            _dprop[_p.animalID] = 0.0
        else:
            _dprop[_p.animalID] = float(len(_dcount)) / \
                float(pedobj.metadata.num_records)
    del(_pedgraph)
    _gfilename = '%s_colored' % \
        (pyp_utils.string_to_table_name(pedobj.metadata.name))
    draw_colored_pedigree(pedobj, _dprop, gfilename=_gfilename,
        gtitle='Colored Pedigree', gorient='p', gname=1, gdirec='',
        gfontsize=12, garrow=0, gtitloc='b')
\end{verbatim}
\function{pyp\_graphics.draw\_pedigree()} was copied into \module{pyp\_jbc}, renamed to \function{draw\_colored\_pedigree()}, and modified to draw colored nodes.  Two basic changes were made to accomplish that: the function was altered to accept a dictionary of weights to be used for coloring, and code for actually coloring the nodes was written.  The first change was simply the addition of a new required parameter, \var{shading}, to the function header.  The second step required a little more work.  For each animal in the pedigree, the descendant proportion is looked-up in the shading dictionary, the proportion is passed to \function{get\_color\_32()} and converted into an RGB triple, and the \member{filled} and \member{color} attributes for the node representing that animal are set.  The hardest part of creating this routine was determining where changes should be made when modifying \function{pyp\_graphics.draw\_pedigree()}.
\begin{verbatim}
##
# draw_colored_pedigree() uses the pydot bindings to the graphviz library
# to produce a directed graph of your pedigree with paths of inheritance
# as edges and animals as nodes.  If there is more than one generation in
# the pedigree as determind by the 'gen' attributes of the animals in the
# pedigree, draw_pedigree() will use subgraphs to try and group animals in
# the same generation together in the drawing.  Nodes will be colored
# based on the number of outgoing connections (number of offspring).
# @param pedobj A PyPedal pedigree object.
# @param shading A dictionary mapping animal IDs to levels that will be
#                used to color nodes.
# ...
# @return A 1 for success and a 0 for failure.
# @defreturn integer
def draw_colored_pedigree(pedobj, shading, gfilename='pedigree', \
    gtitle='My_Pedigree', gformat='jpg', gsize='f', gdot='1', gorient='l', \
    gdirec='', gname=0, gfontsize=10, garrow=1, gtitloc='b', gtitjust='c'):

    from pyp_utils import string_to_table_name
    _gtitle = string_to_table_name(gtitle)
    ...
    # If we do not have any generations, we have to draw a less-nice graph.
    if len(gens) <= 1:
        for _m in pedobj.pedigree:
            ...
            _an_node = pydot.Node(_node_name)
            ...
            _color = get_color_32(shading[_m.animalID],0.0,1.0)
            _an_node.set_style('filled')
            _an_node.set_color(_color)
            ...
    # Otherwise we can draw a nice graph.
    ...
        ...
            for _m in pedobj.pedigree:
                ...
                _an_node = pydot.Node(_node_name)
                ...
                _color = get_color_32(shading[_m.animalID])
                _an_node.set_style('filled')
                _an_node.set_color(_color)
                ...
\end{verbatim}
The \function{get\_color\_32()} function is a modified version of \function{pyp\_graphics.rmuller\_get\_color()} that returns RGB triplets of the form \samp{\#1a2b3c}, which are required by the program that renders the graphs.  This is another example of how code reuse can reduce development time.
\begin{verbatim}
##
# get_color_32() Converts a float value to one of a continuous range of colors
# using recipe 9.10 from the Python Cookbook.
# @param a Float value to convert to a color.
# @param cmin Minimum value in array (0.0 by default).
# @param cmax Maximum value in array (1.0 by default).
# @return An RGB triplet.
# @defreturn integer
def get_color_32(a,cmin=0.0,cmax=1.0):
    try:
        a = float(a-cmin)/(cmax-cmin)
    except ZeroDivisionError:
        a=0.5 # cmax == cmin
    blue = min((max((4*(0.75-a),0.)),1.))
    red = min((max((4*(a-0.25),0.)),1.))
    green = min((max((4*math.fabs(a-0.5)-1.,0)),1.))
    _r = '%2x' % int(255*red)
    if _r[0] == ' ':
        _r = '0%s' % _r[1]
    _g = '%2x' % int(255*green)
    if _g[0] == ' ':
        _g = '0%s' % _g[1]
    _b = '%2x' % int(255*blue)
    if _b[0] == ' ':
        _b = '0%s' % _b[1]
    _triple = '#%s%s%s' % (_r,_g,_b)
    return _triple
\end{verbatim}
This change will probably be to rolled into \function{rmuller\_get\_color()} so that the form of the return triplet is user-selectable.

The program \file{new_jbc.py} demonstrates use of the new \function{pyp\_jbc.color\_pedigree()} routine:
\begin{verbatim}
options = {}
options['renumber'] = 1
options['sepchar'] = '\t'
options['missing_parent'] = 'animal0'

if __name__=='__main__':
    options['pedfile'] = 'new_ids2.ped'
    options['pedformat'] = 'ASD'
    options['pedname'] = 'Boichard Pedigree'

    example = pyp_newclasses.NewPedigree(options)
    example.load()
    pyp_jbc.color_pedigree(example)
\end{verbatim}
The resulting colorized pedigree can be seen in Figure \ref{fig:boichard2-pedigree-colorized}.  Each of the nodes is colored
according to the proportion of animals in the complete pedigree descended from a given animal.  Clearly there is still room
for improvement; for example, there is no key provided in the image so that you can see how colors map to proportions.  Implementation
of a key is left as an exercise for the reader.
\begin{figure}
  \begin{center}
    \includegraphics[width=6in]{BoichardPedigreeColored.eps}
    \caption{Colorized version of the pedigree in Figure \ref{fig:new-ids2-pedigree-basic}}
    \label{fig:boichard2-pedigree-colorized}
  \end{center}
\end{figure}
\section{Contributing Code to PyPedal}
\label{sec:newfeatures-overview-contributing-code}
\index{new features!contributing code}
If you would like to contribute your code back to \PyPedal{} please note that it must be licensed under version 2.1 or any
later version of the GNU Lesser General Public License.  The GNU LGPL has all of the restrictions of the GPL except that you
may use the code at compile time without the derivative work becoming a GPL work. This allows the use of the code in
proprietary works.  You must also complete and return the joint copyright assignment form distributed as
\texttt{pypedal\_copyright\_assignment.pdf} before any contributions can be accepted and merged into the development tree.

Contributors are asked to document their code using the documentation comments recognized by PythonDoc 2.0 or later
(\url{http://effbot.org/zone/pythondoc.htm}).  PythonDoc is used to generates API documentation in HTML and other formats
based on descriptions in Python source files.  You are also strongly encouraged to provide example programs abd datasets
with any code submissions.
\chapter{API}
\label{cha:api}
\begin{quote}
This chapter provides an overview of the \PYPEDAL{} Application Programming Interface (API).  More simply, it is a reference to the various classes, methods, and procedures that make up the \PYPEDAL{} module.
\end{quote}
\section{Some Background}
A complete list of the core \PyPedal() modules is presented in Table \ref{tbl:pypedal-modules}.  Using the \PyPedal{} API is quite simple.  The following discussion assumes that you have imported each of the Python modules using, e.g.,  \samp{from PyPedal import pyp_utils} rather than \samp{from PyPedal.pyp_utils import *}.  The latter is poor style and can result in namespace pollution; this is not known to be a problem with \PyPedal{}, but I offer no guarantees that this will remain the case.  In order to access a function in the \module{pyp_utils} module, such as \function{pyp_nice_time()}, you use a dotted notation with a '.' separating the module name and the function name.  For example:
\begin{verbatim}
[jcole@aipl440 jcole]$ python
Python 2.4 (#1, Feb 25 2005, 12:30:11)
[GCC 3.3.3] on linux2
Type "help", "copyright", "credits" or "license" for more information.
>>> from PyPedal import pyp_utils
>>> pyp_utils.pyp_nice_time()
'Mon Aug 15 16:27:38 2005'
\end{verbatim}
\begin{center}
    \begin{table}
        \caption{PyPedal modules.}
        \label{tbl:pypedal-modules}
        \centerline{
        \begin{tabular}{llp{4in}}
            \hline
            Module Name & Description \\
            \hline
            pyp\_db & \parbox[t]{4in}{Working with SQLite relational databases: create databases, add/drop tables, load PyPedal pedigrees into tables.} \\
            pyp\_demog & \parbox[t]{4in}{Generate demographic reports, age distributions, for the pedigreed population.} \\
            pyp\_graphics & \parbox[t]{4in}{Visualize pedigrees and numerator relationship matrices (NRM).} \\
            pyp\_io & \parbox[t]{4in}{Save and load NRM and inverses of NRM; write pedigrees to formats used by other packages.} \\
            pyp\_metrics & \parbox[t]{4in}{Compute metrics on pedigrees: effective founder and ancestor numbers, effective number of founder genomes, pedigree completeness.  Tools for identifying related animals, calculating coefficients of inbreeding and relationship, and computing expected offspring inbreeding from matings.} \\
            pyp\_network & \parbox[t]{4in}{Convert pedigrees directed graphs.} \\
            pyp\_newclasses & \parbox[t]{4in}{Pedigree, animal, and metadata classes used by PyPedal.} \\
            pyp\_nrm & \parbox[t]{4in}{Creating, decompose, and inverting NRM, and recurse through pedigrees.} \\
            pyp\_reports & \parbox[t]{4in}{Create reports from pedigree database (loaded in pyp_db).} \\
            pyp\_template & \parbox[t]{4in}{Template for developers to use when adding new features to \PyPedal{} (Chapter \ref{cha:newfeatures}).} \\
            pyp\_utils & \parbox[t]{4in}{Load, reorder and renumber pedigrees; set flags in individual animal records; string and date-time tools.} \\
            \hline
        \end{tabular}}
    \end{table}
\end{center}
\section{pyp\_db}
pyp\_db\index[func]{pyp_db} contains a set of procedures for ...

\subsection*{Module Contents}

\begin{description}
\item[\textbf{createPedigreeDatabase(dbname='pypedal')} $\Rightarrow$ integer]\index[func]{pyp_db!createPedigreeDatabase()}
createPedigreeDatabase() creates a new database in SQLite.
\begin{description}
\item[\emph{dbname}] The name of the database to create.
\item[Returns:] A 1 on successful database creation, a 0 otherwise.
\end{description}

\item[\textbf{createPedigreeTable(curs, tablename='example')} $\Rightarrow$ integer]\index[func]{pyp_db!createPedigreeTable()}
createPedigreeDatabase() creates a new pedigree table in a SQLite database.
\begin{description}
\item[\emph{tablename}] The name of the table to create.
\item[Returns:] A 1 on successful table creation, a 0 otherwise.
\end{description}

\item[\textbf{databaseQuery(sql, curs=0, dbname='pypedal')} $\Rightarrow$ string]\index[func]{pyp_db!databaseQuery()}
databaseQuery() executes an SQLite query. This is a wrapper function used by the reporting functions that need to fetch data from SQLite. I wrote it so that any changes that need to be made in the way PyPedal talks to SQLite will only need to be changed in one place.
\begin{description}
\item[\emph{sql}] A string containing an SQL query.
\item[\emph{\_curs}] An [optional] SQLite cursor.
\item[\emph{dbname}] The database into which the pedigree will be loaded.
\item[Returns:] The results of the query, or 0 if no resultset.
\end{description}

\item[\textbf{getCursor(dbname='pypedal')} $\Rightarrow$ cursor]\index[func]{pyp_db!getCursor()}
getCursor() creates a database connection and returns a cursor on success or a 0 on failure. It isvery useful for non-trivial queries because it creates SQLite aggrefates before returning the cursor. The reporting routines in pyp\_reports make heavy use of getCursor().
\begin{description}
\item[\emph{dbname}] The database into which the pedigree will be loaded.
\item[Returns:] An SQLite cursor if the database exists, a 0 otherwise.
\end{description}

\item[\textbf{loadPedigreeTable(pedobj)} $\Rightarrow$ integer]\index[func]{pyp_db!loadPedigreeTable()}
loadPedigreeDatabase() takes a PyPedal pedigree object and loads the animal records in that pedigree into an SQLite table.
\begin{description}
\item[\emph{pedobj}] A PyPedal pedigree object.
\item[\emph{dbname}] The database into which the pedigree will be loaded.
\item[\emph{tablename}] The table into which the pedigree will be loaded.
\item[Returns:] A 1 on successful table load, a 0 otherwise.
\end{description}

\item[\textbf{tableCountRows(dbname='pypedal', tablename='example')} $\Rightarrow$ integer]\index[func]{pyp_db!tableCountRows()}
tableCountRows() returns the number of rows in a table.
\begin{description}
\item[\emph{dbname}] The database into which the pedigree will be loaded.
\item[\emph{tablename}] The table into which the pedigree will be loaded.
\item[Returns:] The number of rows in the table 1 or 0.
\end{description}

\item[\textbf{tableDropRows(dbname='pypedal', tablename='example')} $\Rightarrow$ integer]\index[func]{pyp_db!tableDropRows()}
tableDropRows() drops all of the data from an existing table.
\begin{description}
\item[\emph{dbname}] The database into which the pedigree will be loaded.
\item[\emph{tablename}] The table into which the pedigree will be loaded.
\item[Returns:] A 1 if the data were dropped, a 0 otherwise.
\end{description}

\item[\textbf{tableExists(dbname='pypedal', tablename='example')} $\Rightarrow$ integer]\index[func]{pyp_db!tableExists()}
tableExists() queries the sqlite\_master view in an SQLite database to determine whether or not a table exists.
\begin{description}
\item[\emph{dbname}] The database into which the pedigree will be loaded.
\item[\emph{tablename}] The table into which the pedigree will be loaded.
\item[Returns:] A 1 if the table exists, a 0 otherwise.
\end{description}
\end{description}

\subsection*{The PypMean Class}
\begin{description}
\item[\textbf{PypMean()} (class)]\index[func]{pyp_db!PypMean}
PypMean is a user-defined aggregate for SQLite for returning means from queries.
\end{description}

\subsection*{The PypSSD Class}\index[func]{pyp_db!PypSSD}
\begin{description}
\item[\textbf{PypSSD()} (class)]
PypSSD is a user-defined aggregate for SQLite for returning sample standard deviations from queries.
\end{description}

\subsection*{The PypSum Class}
\begin{description}
\item[\textbf{PypSum()} (class)]\index[func]{pyp_db!PypSum}
PypSum is a user-defined aggregate for SQLite for returning sums from queries.
\end{description}

\subsection*{The PypSVar Class}
\begin{description}
\item[\textbf{PypSVar()} (class)]\index[func]{pyp_db!PypSVar}
PypSVar is a user-defined aggregate for SQLite for returning sample variances from queries.
\end{description}
\section{pyp\_demog}
pyp\_demog\index[func]{pyp_demog} contains a set of procedures for demographic calculations on the population describe in a pedigree.
\subsection*{Module Contents}

\begin{description}
\item[\textbf{age\_distribution(pedobj, sex=1)} $\Rightarrow$ None]\index[func]{pyp_demog!age\_distribution()}
age\_distribution() computes histograms of the age distribution of males and females in the population. You can also stratify by sex to get individual histograms.
\begin{description}
\item[\emph{myped}] An instance of a PyPedal NewPedigree object.
\item[\emph{sex}] A flag which determines whether or not to stratify by sex.
\end{description}

\item[\textbf{founders\_by\_year(pedobj)} $\Rightarrow$ dictionary]\index[func]{pyp_demog!founders\_by\_year()}
founders\_by\_year() returns a dictionary containing the number of founders in each birthyear.
\begin{description}
\item[\emph{pedobj}] A PyPedal pedigree object.
\item[Returns:] dict A dictionary containing entries for each sex/gender code defined in the global SEX\_CODE\_MAP.
\end{description}

\item[\textbf{set\_age\_units(units='year')} $\Rightarrow$ None]\index[func]{pyp_demog!set\_age\_units()}
set\_age\_units() defines a global variable, BASE\_DEMOGRAPHIC\_UNIT.
\begin{description}
\item[\emph{units}] The base unit for age computations ('year'|'month'|'day').
\item[Returns:] None
\end{description}

\item[\textbf{set\_base\_year(year=1950)} $\Rightarrow$ None]\index[func]{pyp_demog!age\_distribution()}
set\_base\_year() defines a global variable, BASE\_DEMOGRAPHIC\_YEAR.
\begin{description}
\item[\emph{year}] The year to be used as a base for computing ages.
\item[Returns:] None
\end{description}

\item[\textbf{sex\_ratio(pedobj)} $\Rightarrow$ dictionary]\index[func]{pyp_demog!sex\_ratio()}
sex\_ratio() returns a dictionary containing the proportion of males and females in the population.
\begin{description}
\item[\emph{myped}] An instance of a PyPedal NewPedigree object.
\item[Returns:] dict A dictionary containing entries for each sex/gender code defined in the global SEX\_CODE\_MAP.
\end{description}

\end{description}
\section{pyp\_graphics}
pyp\_graphics\index[func]{pyp_graphics} contains routines for working with graphics in PyPedal, such as creating directed graphs from pedigrees using PyDot and visualizing relationship matrices using Rick Muller's spy and pcolor routines (\url{http://aspn.activestate.com/ASPN/Cookbook/Python/}). The Python Imaging Library (\url{http://www.pythonware.com/products/pil/}), matplotlib (\url{http://matplotlib.sourceforge.net/}), Graphviz (\url{http://www.graphviz.org/}), and pydot (\url{http://dkbza.org/pydot.html}) are required by one or more routines in this module. They ARE NOT distributed with PyPedal and must be installed by the end-user! Note that the matplotlib functionality in PyPedal requires only the Agg backend, which means that you do not have to install GTK/PyGTK or WxWidgets/PyWxWidgets just to use PyPedal. Please consult the sites above for licensing and installation information.

\subsection*{Module Contents}

\begin{description}
\item[\textbf{draw\_pedigree(pedobj, gfilename='pedigree', gtitle='My\_Pedigree', gformat='jpg', gsize='f', gdot='1')} $\Rightarrow$ integer]\index[func]{pyp_graphics!draw\_pedigree()}
draw\_pedigree() uses the pydot bindings to the graphviz library -- if they are available on your system -- to produce a directed graph of your pedigree with paths of inheritance as edges and animals as nodes. If there is more than one generation in the pedigree as determind by the ``gen'' attributes of the anumals in the pedigree, draw\_pedigree() will use subgraphs to try and group animals in the same generation together in the drawing.
\begin{description}
\item[\emph{pedobj}] A PyPedal pedigree object.
\item[\emph{gfilename}] The name of the file to which the pedigree should be drawn
\item[\emph{gtitle}] The title of the graph.
\item[\emph{gsize}] The size of the graph: 'f': full-size, 'l': letter-sized page.
\item[\emph{gdot}] Whether or not to write the dot code for the pedigree graph to a file (can produce large files).
\item[Returns:] A 1 for success and a 0 for failure.
\end{description}

\item[\textbf{pcolor\_matrix\_pylab(A, fname='pcolor\_matrix\_matplotlib')} $\Rightarrow$ lists]\index[func]{pyp_graphics!pcolor\_matrix\_pylab()}
pcolor\_matrix\_pylab() implements a matlab-like 'pcolor' function to display the large elements of a matrix in pseudocolor using the Python Imaging Library.
\begin{description}
\item[\emph{A}] Input Numpy matrix (such as a numerator relationship matrix).
\item[\emph{fname}] Output filename to which to dump the graphics (default 'tmp.png')
\item[\emph{do\_outline}] Whether or not to print an outline around the block (default 0)
\item[\emph{height}] The height of the image (default 300)
\item[\emph{width}] The width of the image (default 300)
\item[Returns:] A list of Animal() objects; a pedigree metadata object.
\end{description}

\item[\textbf{plot\_founders\_by\_year(pedobj, gfilename='founders\_by\_year', gtitle='Founders by Birthyear')} $\Rightarrow$ integer]\index[func]{pyp_graphics!plot\_founders\_by\_year()}
founders\_by\_year() uses matplotlib -- if available on your system -- to produce a bar graph of the number (count) of founders in each birthyear.
\begin{description}
\item[\emph{pedobj}] A PyPedal pedigree object.
\item[\emph{gfilename}] The name of the file to which the pedigree should be drawn
\item[\emph{gtitle}] The title of the graph.
\item[Returns:] A 1 for success and a 0 for failure.
\end{description}

\item[\textbf{plot\_founders\_pct\_by\_year(pedobj, gfilename='founders\_pct\_by\_year', gtitle='Founders by Birthyear')} $\Rightarrow$ integer]\index[func]{pyp_graphics!plot\_founders\_pct\_by\_year()}
founders\_pct\_by\_year() uses matplotlib -- if available on your system -- to produce a line graph of the frequency (percentage) of founders in each birthyear.
\begin{description}
\item[\emph{pedobj}] A PyPedal pedigree object.
\item[\emph{gfilename}] The name of the file to which the pedigree should be drawn
\item[\emph{gtitle}] The title of the graph.
\item[Returns:] A 1 for success and a 0 for failure.
\end{description}

\item[\textbf{rmuller\_get\_color(a, cmin, cmax)} $\Rightarrow$ integer]\index[func]{pyp_graphics!rmuller\_get\_color()}
rmuller\_get\_color() Converts a float value to one of a continuous range of colors using recipe 9.10 from the Python Cookbook.
\begin{description}
\item[\emph{a}] Float value to convert to a color.
\item[\emph{cmin}] Minimum value in array (?).
\item[\emph{cmax}] Maximum value in array (?).
\item[Returns:] An RGB triplet.
\end{description}

\item[\textbf{rmuller\_pcolor\_matrix\_pil(A, fname='tmp.png', do\_outline=0, height=300, width=300)} $\Rightarrow$ lists]\index[func]{pyp_graphics!rmuller\_pcolor\_matrix\_pil()}
rmuller\_pcolor\_matrix\_pil() implements a matlab-like 'pcolor' function to display the large elements of a matrix in pseudocolor using the Python Imaging Library.
\begin{description}
\item[\emph{A}] Input Numpy matrix (such as a numerator relationship matrix).
\item[\emph{fname}] Output filename to which to dump the graphics (default 'tmp.png')
\item[\emph{do\_outline}] Whether or not to print an outline around the block (default 0)
\item[\emph{height}] The height of the image (default 300)
\item[\emph{width}] The width of the image (default 300)
\item[Returns:] A list of Animal() objects; a pedigree metadata object.
\end{description}

\item[\textbf{rmuller\_spy\_matrix\_pil(A, fname='tmp.png', cutoff=0.1, do\_outline=0, height=300, width=300)} $\Rightarrow$ lists]\index[func]{pyp_graphics!rmuller\_spy\_matrix\_pil()}
rmuller\_spy\_matrix\_pil() implements a matlab-like 'spy' function to display the sparsity of a matrix using the Python Imaging Library.
\begin{description}
\item[\emph{A}] Input Numpy matrix (such as a numerator relationship matrix).
\item[\emph{fname}] Output filename to which to dump the graphics (default 'tmp.png')
\item[\emph{cutoff}] Threshold value for printing an element (default 0.1)
\item[\emph{do\_outline}] Whether or not to print an outline around the block (default 0)
\item[\emph{height}] The height of the image (default 300)
\item[\emph{width}] The width of the image (default 300)
\item[Returns:] A list of Animal() objects; a pedigree metadata object.
\end{description}

\item[\textbf{spy\_matrix\_pylab(A, fname='spy\_matrix\_matplotlib')} $\Rightarrow$ lists]\index[func]{pyp_graphics!spy\_matrix\_pylab()}
spy\_matrix\_pylab() implements a matlab-like 'pcolor' function to display the large elements of a matrix in pseudocolor using the Python Imaging Library.
\begin{description}
\item[\emph{A}] Input Numpy matrix (such as a numerator relationship matrix).
\item[\emph{fname}] Output filename to which to dump the graphics (default 'tmp.png')
\item[\emph{do\_outline}] Whether or not to print an outline around the block (default 0)
\item[\emph{height}] The height of the image (default 300)
\item[\emph{width}] The width of the image (default 300)
\item[Returns:] A list of Animal() objects; a pedigree metadata object.
\end{description}

\end{description}
\section{pyp\_io}
pyp\_io\index[func]{pyp_io} contains several procedures for writing structures to and reading them from disc (e.g. using pickle() to store and retrieve A and A-inverse). It also includes a set of functions used to render strings as HTML or plaintext for use in generating output files.

\subsection*{Module Contents}

\begin{description}
\item[\textbf{a\_inverse\_from\_file(inputfile)} $\Rightarrow$ matrix]\index[func]{pyp_io!a\_inverse\_from\_file()}
a\_inverse\_from\_file() uses the Python pickle system for persistent objects to read the inverse of a relationship matrix from a file.
\begin{description}
\item[\emph{inputfile}] The name of the input file.
\item[Returns:] The inverse of a numerator relationship matrix.
\end{description}

\item[\textbf{a\_inverse\_to\_file(pedobj, ainv='')}]\index[func]{pyp_io!a\_inverse\_to\_file()}
a\_inverse\_to\_file() uses the Python pickle system for persistent objects to write the inverse of a relationship matrix to a file.
\begin{description}
\item[\emph{pedobj}] A PyPedal pedigree object.
\item[\emph{filetag}] A descriptor prepended to output file names.
\end{description}

\item[\textbf{dissertation\_pedigree\_to\_file(pedobj)}]\index[func]{pyp_io!dissertation\_pedigree\_to\_file()}
dissertation\_pedigree\_to\_file() takes a pedigree in 'asdxfg' format and writes is to a file.
\begin{description}
\item[\emph{pedobj}] A PyPedal pedigree object.
\end{description}

\item[\textbf{dissertation\_pedigree\_to\_pedig\_format(pedobj)}]\index[func]{pyp_io!dissertation\_pedigree\_to\_pedig\_format()}
dissertation\_pedigree\_to\_pedig\_format() takes a pedigree in 'asdbxfg' format, formats it into the form used by Didier Boichard's 'pedig' suite of programs, and writes it to a file.
\begin{description}
\item[\emph{pedobj}] A PyPedal pedigree object.
\end{description}

\item[\textbf{dissertation\_pedigree\_to\_pedig\_format\_mask(pedobj)}]\index[func]{pyp_io!dissertation\_pedigree\_to\_pedig\_format\_mask()}
dissertation\_pedigree\_to\_pedig\_format\_mask() Takes a pedigree in 'asdbxfg' format, formats it into the form used by Didier Boichard's 'pedig' suite of programs, and writes it to a file. THIS FUNCTION MASKS THE GENERATION ID WITH A FAKE BIRTH YEAR AND WRITES THE FAKE BIRTH YEAR TO THE FILE INSTEAD OF THE TRUE BIRTH YEAR. THIS IS AN ATTEMPT TO FOOL PEDIG TO GET f\_e, f\_a et al. BY GENERATION.
\begin{description}
\item[\emph{pedobj}] A PyPedal pedigree object.
\end{description}

\item[\textbf{dissertation\_pedigree\_to\_pedig\_interest\_format(pedobj)}]\index[func]{pyp_io!dissertation\_pedigree\_to\_pedig\_interest\_format()}
dissertation\_pedigree\_to\_pedig\_interest\_format() takes a pedigree in 'asdbxfg' format, formats it into the form used by Didier Boichard's parente program for the studied individuals file.
\begin{description}
\item[\emph{pedobj}] A PyPedal pedigree object.
\end{description}

\item[\textbf{pickle\_pedigree(pedobj, filename='')} $\Rightarrow$ integer]\index[func]{pyp_io!pickle\_pedigree()}
pickle\_pedigree() pickles a pedigree.
\begin{description}
\item[\emph{pedobj}] An instance of a PyPedal pedigree object.
\item[\emph{filename}] The name of the file to which the pedigree object should be pickled (optional).
\item[Returns:] A 1 on success, a 0 otherwise.
\end{description}

\item[\textbf{pyp\_file\_footer(ofhandle, caller=''Unknown PyPedal routine'')} $\Rightarrow$ None]\index[func]{pyp_io!pyp\_file\_footer()}
pyp\_file\_footer()
\begin{description}
\item[\emph{ofhandle}] A Python file handle.
\item[\emph{caller}] A string indicating the name of the calling routine.
\item[Returns:] None
\end{description}

\item[\textbf{pyp\_file\_header(ofhandle, caller=''Unknown PyPedal routine'')} $\Rightarrow$ integer]
pyp\_file\_header()\index[func]{pyp_io!pyp\_file\_header()}
\begin{description}
\item[\emph{ofhandle}] A Python file handle.
\item[\emph{caller}] A string indicating the name of the calling routine.
\item[Returns:] None
\end{description}

\item[\textbf{renderTitle(title\_string, title\_level=''1'')} $\Rightarrow$ integer]\index[func]{pyp_io!renderTitle()}
renderTitle() ... Produced HTML output by default.

\item[\textbf{unpickle\_pedigree(filename='')} $\Rightarrow$ object]\index[func]{pyp_io!unpickle\_pedigree()}
unpickle\_pedigree() reads a pickled pedigree in from a file and returns the unpacked pedigree object.
\begin{description}
\item[\emph{filename}] The name of the pickle file.
\item[Returns:] An instance of a NewPedigree object on success, a 0 otherwise.
\end{description}

\end{description}
\section{pyp\_metrics}
pyp\_metrics\index[func]{pyp_metrics} contains a set of procedures for calculating metrics on PyPedal pedigree objects. These metrics include coefficients of inbreeding and relationship as well as effective founder number, effective population size, and effective ancestor number.

\subsection*{Module Contents}
\begin{description}
\item[\textbf{a\_coefficients(pedobj, a='', method='nrm')} $\Rightarrow$ dictionary]\index[func]{pyp_metrics!a_coefficients()}
a\_coefficients() writes population average coefficients of inbreeding and relationship to a file, as well as individual animal IDs and coefficients of inbreeding. Some pedigrees are too large for fast\_a\_matrix() or fast\_a\_matrix\_r() -- an array that large cannot be allocated due to memory restrictions -- and will result in a value of -999.9 for all outputs.
\begin{description}
\item[\emph{pedobj}] A PyPedal pedigree object.
\item[\emph{a}] A numerator relationship matrix (optional).
\item[\emph{method}] If no relationship matrix is passed, determines which procedure should be called to build one (nrm|frm).
\item[Returns:] A dictionary of non-zero individual inbreeding coefficients.
\end{description}

\item[\textbf{a\_effective\_ancestors\_definite(pedobj, a='', gen='')} $\Rightarrow$ float]\index[func]{pyp_metrics!a_effective_ancestors()}
a\_effective\_ancestors\_definite() uses the algorithm in Appendix B of \citeN{ref352} to compute the effective ancestor number for a myped pedigree. NOTE: One problem here is that if you pass a pedigree WITHOUT generations and error is not thrown. You simply end up wth a list of generations that contains the default value for Animal() objects, 0. Boichard's algorithm requires information about the generation of animals. If you do not provide an input pedigree with generations things may not work. By default the most recent generation -- the generation with the largest generation ID -- will be used as the reference population.
\begin{description}
\item[\emph{pedobj}] A PyPedal pedigree object.
\item[\emph{a}] A numerator relationship matrix (optional).
\item[\emph{gen}] Generation of interest.
\item[Returns:] The effective founder number.
\end{description}

\item[\textbf{a\_effective\_ancestors\_indefinite(pedobj, a='', gen='', n=25)} $\Rightarrow$ float]\index[func]{pyp_metrics!a_effective_ancestors_indefinite()}
a\_effective\_ancestors\_indefinite() uses the approach outlined on pages 9 and 10 of Boichard et al. \cite{ref352} to compute approximate upper and lower bounds for f\_a. This is much more tractable for large pedigrees than the exact computation provided in a\_effective\_ancestors\_definite(). NOTE: One problem here is that if you pass a pedigree WITHOUT generations and error is not thrown. You simply end up wth a list of generations that contains the default value for Animal() objects, 0. NOTE: If you pass a value of n that is greater than the actual number of ancestors in the pedigree then strange things happen. As a stop-gap, a\_effective\_ancestors\_indefinite() will detect that case and replace n with the number of founders - 1. Boichard's algorithm requires information about the GENERATION of animals. If you do not provide an input pedigree with generations things may not work. By default the most recent generation -- the generation with the largest generation ID -- will be used as the reference population.
\begin{description}
\item[\emph{pedobj}] A PyPedal pedigree object.
\item[\emph{a}] A numerator relationship matrix (optional).
\item[\emph{gen}] Generation of interest.
\item[Returns:] The effective founder number.
\end{description}

\item[\textbf{a\_effective\_founders\_boichard(pedobj, a='', gen='')} $\Rightarrow$ float]\index[func]{pyp_metrics!a_effective_founders_boichard()}
a\_effective\_founders\_boichard() uses the algorithm in Appendix A of \citeN{ref352} to compute the effective founder number for pedobj. Note that results from this function will not necessarily match those from a\_effective\_founders\_lacy(). Boichard's algorithm requires information about the GENERATION of animals. If you do not provide an input pedigree with generations things may not work. By default the most recent generation -- the generation with the largest generation ID -- will be used as the reference population.
\begin{description}
\item[\emph{pedobj}] A PyPedal pedigree object.
\item[\emph{a}] A numerator relationship matrix (optional).
\item[\emph{gen}] Generation of interest.
\item[Returns:] The effective founder number.
\end{description}

\item[\textbf{a\_effective\_founders\_lacy(pedobj, a='')} $\Rightarrow$ float]\index[func]{pyp_metrics!a_effective_founders_lacy()}
a\_effective\_founders\_lacy() calculates the number of effective founders in a pedigree using the exact method of \citeN{ref640}.
\begin{description}
\item[\emph{pedobj}] A PyPedal pedigree object.
\item[\emph{a}] A numerator relationship matrix (optional).
\item[Returns:] The effective founder number.
\end{description}

\item[\textbf{common\_ancestors(anim\_a, anim\_b, pedobj)} $\Rightarrow$ list]\index[func]{pyp_metrics!common_ancestors()}
common\_ancestors() returns a list of the ancestors that two animals share in common.
\begin{description}
\item[\emph{anim\_a}] The renumbered ID of the first animal, a.
\item[\emph{anim\_b}] The renumbered ID of the second animal, b.
\item[\emph{pedobj}] A PyPedal pedigree object.
\item[Returns:] A list of animals related to anim\_a AND anim\_b
\end{description}

\item[\textbf{descendants(anid, pedobj, \_desc)} $\Rightarrow$ list]\index[func]{pyp_metrics!descendants()}
descendants() uses pedigree metadata to walk a pedigree and return a list of all of the descendants of a given animal.
\begin{description}
\item[\emph{anid}] An animal ID
\item[\emph{pedobj}] A Python list of PyPedal Animal() objects.
\item[\emph{\_desc}] A Python dictionary of descendants of animal anid.
\item[Returns:] A list of descendants of anid.
\end{description}

\item[\textbf{effective\_founder\_genomes(pedobj, rounds=10)} $\Rightarrow$ float]\index[func]{pyp_metrics!effective_founder_genomes()}
effective\_founder\_genomes() simulates the random segregation of founder alleles through a pedigree after the method of \citeN{ref1719}. At present only two alleles are simulated for each founder. Summary statistics are computed on the most recent generation.
\begin{description}
\item[\emph{pedobj}] A PyPedal pedigree object.
\item[\emph{rounds}] The number of times to simulate segregation through the entire pedigree.
\item[Returns:] The effective number of founder genomes over based on 'rounds' gene-drop simulations.
\end{description}

\item[\textbf{effective\_founders\_lacy(pedobj)} $\Rightarrow$ float]\index[func]{pyp_metrics!effective_founders_lacy()}
effective\_founders\_lacy() calculates the number of effective founders in a pedigree using the exact method of  \citeN{ref640}. This version of the routine a\_effective\_founders\_lacy() is designed to work with larger pedigrees as it forms ``familywise'' relationship matrices rather than a ``populationwise'' relationship matrix.
\begin{description}
\item[\emph{pedobj}] A PyPedal pedigree object.
\item[Returns:] The effective founder number.
\end{description}

\item[\textbf{fast\_a\_coefficients(pedobj, a='', method='nrm', debug=0)} $\Rightarrow$ dictionary]\index[func]{pyp_metrics!fast_a_coefficients()}
a\_fast\_coefficients() writes population average coefficients of inbreeding and relationship to a file, as well as individual animal IDs and coefficients of inbreeding. It returns a list of non-zero individual CoI.
\begin{description}
\item[\emph{pedobj}] A PyPedal pedigree object.
\item[\emph{a}] A numerator relationship matrix (optional).
\item[\emph{method}] If no relationship matrix is passed, determines which procedure should be called to build one (nrm|frm).
\item[Returns:] A dictionary of non-zero individual inbreeding coefficients.
\end{description}

\item[\textbf{founder\_descendants(pedobj)} $\Rightarrow$ dictionary [\#]]\index[func]{pyp_metrics!founder_descendants()}
founder\_descendants() returns a dictionary containing a list of descendants of each founder in the pedigree.
\begin{description}
\item[\emph{pedojb}] An instance of a PyPedal NewPedigree object.
\end{description}

\item[\textbf{generation\_lengths(pedobj, units='y')} $\Rightarrow$ dictionary]\index[func]{pyp_metrics!generation_lengths()}
generation\_lengths() computes the average age of parents at the time of birth of their first offspring. This is implies that selection decisions are made at the time of birth of of the first offspring. Average ages are computed for each of four paths: sire-son, sire-daughter, dam-son, and dam-daughter. An overall mean is computed, as well. IT IS IMPORTANT to note that if you DO NOT provide birthyears in your pedigree file that the returned dictionary will contain only zeroes! This is because when no birthyer is provided a default value (1900) is assigned to all animals in the pedigree.
\begin{description}
\item[\emph{pedobj}] A PyPedal pedigree object.
\item[\emph{units}] A character indicating the units in which the generation lengths should be returned.
\item[Returns:] A dictionary containing the five average ages.
\end{description}

\item[\textbf{generation\_lengths\_all(pedobj, units='y')} $\Rightarrow$ dictionary]\index[func]{pyp_metrics!generation_lengths_all()}
generation\_lengths\_all() computes the average age of parents at the time of birth of their offspring. The computation is made using birth years for all known offspring of sires and dams, which implies discrete generations. Average ages are computed for each of four paths: sire-son, sire-daughter, dam-son, and dam-daughter. An overall mean is computed, as well. IT IS IMPORTANT to note that if you DO NOT provide birthyears in your pedigree file that the returned dictionary will contain only zeroes! This is because when no birthyear is provided a default value (1900) is assigned to all animals in the pedigree.
\begin{description}
\item[\emph{pedobj}] A PyPedal pedigree object.
\item[\emph{units}] A character indicating the units in which the generation lengths should be returned.
\item[Returns:] A dictionary containing the five average ages.
\end{description}

\item[\textbf{mating\_coi(anim\_a, anim\_b, pedobj)} $\Rightarrow$ float]\index[func]{pyp_metrics!mating_coi()}
mating\_coi() returns the coefficient of inbreeding of offspring of a mating between two animals, anim\_a and anim\_b.
\begin{description}
\item[\emph{anim\_a}] The renumbered ID of an animal, a.
\item[\emph{anim\_b}] The renumbered ID of an animal, b.
\item[\emph{pedobj}] A PyPedal pedigree object.
\item[Returns:] The coefficient of relationship of anim\_a and anim\_b
\end{description}

\item[\textbf{min\_max\_f(pedobj, a='', n=10)} $\Rightarrow$ list]\index[func]{pyp_metrics!min_max_f()}
min\_max\_f() takes a pedigree and returns a list of the individuals with the n largest and n smallest coefficients of inbreeding. Individuals with CoI of zero are not included.
\begin{description}
\item[\emph{pedobj}] A PyPedal pedigree object.
\item[\emph{a}] A numerator relationship matrix (optional).
\item[\emph{n}] An integer (optional, default is 10).
\item[Returns:] Lists of the individuals with the n largest and the n smallest CoI in the pedigree as (ID, CoI) tuples.
\end{description}

\item[\textbf{num\_equiv\_gens(pedobj)} $\Rightarrow$ dictionary]\index[func]{pyp_metrics!num_equiv_gens()}
num\_equiv\_gens() computes the number of equivalent generations as the sum of (1/2)\^{}n, where n is the number of generations separating an individual and each of its known ancestors.
\begin{description}
\item[\emph{pedobj}] A PyPedal pedigree object.
\item[Returns:] A dictionary containing the five average ages.
\end{description}

\item[\textbf{num\_traced\_gens(pedobj)} $\Rightarrow$ dictionary]\index[func]{pyp_metrics!num_traced_gens()}
num\_traced\_gens() is computed as the number of generations separating offspring from the oldest known ancestor in in each selection path. Ancestors with unknown parents are assigned to generation 0. See Valera at al. \cite{Valera2005a} for details.
\begin{description}
\item[\emph{pedobj}] A PyPedal pedigree object.
\item[Returns:] A dictionary containing the five average ages.
\end{description}

\item[\textbf{partial\_inbreeding(pedobj)} $\Rightarrow$ dictionary]\index[func]{pyp_metrics!partial_inbreeding()}
partial\_inbreeding() computes the number of equivalent generations as the sum of $\frac{1}{2}^{n}$, where n is the number of generations separating an individual and each of its known ancestors.
\begin{description}
\item[\emph{pedobj}] A PyPedal pedigree object.
\item[Returns:] A dictionary containing the five average ages.
\end{description}

\item[\textbf{pedigree\_completeness(pedobj, gens=4)}]\index[func]{pyp_metrics!pedigree_completeness()}
pedigree\_completeness() computes the proportion of known ancestors in the pedigree of each animal in the population for a user-determined number of generations. Also, the mean pedcomps for all animals and for all animals that are not founders are computed as summary statistics.  This is similar to pedigree completeness as computed by \citeN{ref615}, but with some of the modifications of VanRaden (2003) (\url{http://www.aipl.arsusda.gov/reference/changes/eval0311.html}).
\begin{description}
\item[\emph{pedobj}] A PyPedal pedigree object.
\item[\emph{gens}] The number of generations the pedigree should be traced for completeness.
\end{description}

\item[\textbf{related\_animals(anim\_a, pedobj)} $\Rightarrow$ list]\index[func]{pyp_metrics!related_animals()}
related\_animals() returns a list of the ancestors of an animal.
\begin{description}
\item[\emph{anim\_a}] The renumbered ID of an animal, a.
\item[\emph{pedobj}] A PyPedal pedigree object.
\item[Returns:] A list of animals related to anim\_a
\end{description}

\item[\textbf{relationship(anim\_a, anim\_b, pedobj)} $\Rightarrow$ float]\index[func]{pyp_metrics!relationship()}
relationship() returns the coefficient of relationship for two animals, anim\_a and anim\_b.
\begin{description}
\item[\emph{anim\_a}] The renumbered ID of an animal, a.
\item[\emph{anim\_b}] The renumbered ID of an animal, b.
\item[\emph{pedobj}] A PyPedal pedigree object.
\item[Returns:] The coefficient of relationship of anim\_a and anim\_b
\end{description}

\item[\textbf{theoretical\_ne\_from\_metadata(pedobj)} $\Rightarrow$ None]\index[func]{pyp_metrics!theoretical_ne_from_metadata()}
theoretical\_ne\_from\_metadata() computes the theoretical effective population size based on the number of sires and dams contained in a pedigree metadata object. Writes results to an output file.
\begin{description}
\item[\emph{pedobj}] A PyPedal pedigree object.
\end{description}
\end{description}
\section{pyp\_newclasses}
pyp\_newclasses\index[func]{pyp_newclasses} contains the new class structure that will be a part of PyPedal 2.0.0Final. It includes a master class to which most of the computational routines will be bound as methods, a NewAnimal() class, and a PedigreeMetadata() class.

\subsection*{Module Contents}

\begin{description}
\item[\textbf{NewAMatrix(kw)} (class)]\index[func]{pyp_newclasses!NewAMatrix}
NewAMatrix provides an instance of a numerator relationship matrix as a Numarray array of floats with some convenience methods.
For more information about this class, see \emph{The NewAMatrix Class}

\item[\textbf{NewAnimal(locations, data, mykw)} (class)]\index[func]{pyp_newclasses!NewAnimal}
The NewAnimal() class is holds animals records read from a pedigree file.
For more information about this class, see \emph{The NewAnimal Class}

\item[\textbf{NewPedigree(kw)} (class)]\index[func]{pyp_newclasses!NewPedigree}
The NewPedigree class is the main data structure for PyP 2.0.0Final.
For more information about this class, see \emph{The NewPedigree Class}

\item[\textbf{PedigreeMetadata(myped, kw)} (class)]\index[func]{pyp_newclasses!PedigreeMetadata}
The PedigreeMetadata() class stores metadata about pedigrees.
For more information about this class, see \emph{The PedigreeMetadata Class}
\end{description}

\subsection*{The NewAMatrix Class}
\begin{description}
\item[\textbf{NewAMatrix(kw)} (class)]
NewAMatrix provides an instance of a numerator relationship matrix as a Numarray array of floats with some convenience methods. The idea here is to provide a wrapper around a NRM so that it is easier to work with. For large pedigrees it can take a long time to compute the elements of A, so there is real value in providing an easy way to save and retrieve a NRM once it has been formed.

\item[\textbf{form\_a\_matrix(pedigree)} $\Rightarrow$ integer]\index[func]{pyp_newclasses!NewAMatrix!form_a_matrix()}
form\_a\_matrix() calls pyp\_nrm/fast\_a\_matrix() or pyp\_nrm/fast\_a\_matrix\_r() to form a NRM from a pedigree.
\begin{description}
\item[\emph{pedigree}] The pedigree used to form the NRM.
\item[Returns:] A NRM on success, 0 on failure.
\end{description}

\item[\textbf{info()} $\Rightarrow$ None]\index[func]{pyp_newclasses!NewAMatrix!info()}
info() uses the info() method of Numarray arrays to dump some information about the NRM. This is of use predominantly for debugging.
\begin{description}
\item[\emph{None}]
\item[Returns:] None
\end{description}

\item[\textbf{load(nrm\_filename)} $\Rightarrow$ integer]\index[func]{pyp_newclasses!NewAMatrix!load()}
load() uses the Numarray Array Function ``fromfile()'' to load an array from a binary file. If the load is successful, self.nrm contains the matrix.
\begin{description}
\item[\emph{nrm\_filename}] The file from which the matrix should be read.
\item[Returns:] A load status indicator (0: failed, 1: success).
\end{description}

\item[\textbf{save(nrm\_filename)} $\Rightarrow$ integer]\index[func]{pyp_newclasses!NewAMatrix!save()}
save() uses the Numarray method ``tofile()'' to save an array to a binary file.
\begin{description}
\item[\emph{nrm\_filename}] The file to which the matrix should be written.
\item[Returns:] A save status indicator (0: failed, 1: success).
\end{description}

\end{description}

\subsection*{The NewAnimal Class}
\begin{description}
\item[\textbf{NewAnimal(locations, data, mykw)} (class)]
The NewAnimal() class is holds animals records read from a pedigree file.

\item[\textbf{\_\_init\_\_(locations, data, mykw)} $\Rightarrow$ object]\index[func]{pyp_newclasses!NewAnimal!\_\_init\_\_()}
\_\_init\_\_() initializes a NewAnimal() object.
\begin{description}
\item[\emph{locations}] A dictionary containing the locations of variables in the input line.
\item[\emph{data}] The line of input read from the pedigree file.
\item[Returns:] An instance of a NewAnimal() object populated with data
\end{description}

\item[\textbf{pad\_id()} $\Rightarrow$ integer]\index[func]{pyp_newclasses!NewAnimal!pad_id()}
pad\_id() takes an Animal ID, pads it to fifteen digits, and prepends the birthyear (or 1950 if the birth year is unknown). The order of elements is: birthyear, animalID, count of zeros, zeros.
\begin{description}
\item[\emph{self}] Reference to the current Animal() object
\item[Returns:] A padded ID number that is supposed to be unique across animals
\end{description}

\item[\textbf{printme()} $\Rightarrow$ None]\index[func]{pyp_newclasses!NewAnimal!printme()}
printme() prints a summary of the data stored in the Animal() object.
\begin{description}
\item[\emph{self}] Reference to the current Animal() object
\end{description}


\item[\textbf{string\_to\_int(idstring)} $\Rightarrow$ None]\index[func]{pyp_newclasses!NewAnimal!string_to_int()}
string\_to\_int() takes an Animal/Sire/Dam ID as a string and returns a string that can be represented as an integer by replacing each character in the string with its corresponding ASCII table value.

\item[\textbf{stringme()} $\Rightarrow$ None]\index[func]{pyp_newclasses!NewAnimal!stringme()}
stringme() returns a summary of the data stored in the Animal() object as a string.
\begin{description}
\item[\emph{self}] Reference to the current Animal() object
\end{description}


\item[\textbf{trap()} $\Rightarrow$ None]\index[func]{pyp_newclasses!NewAnimal!trap()}
trap() checks for common errors in Animal() objects
\begin{description}
\item[\emph{self}] Reference to the current Animal() object
\end{description}

\end{description}

\subsection*{The NewPedigree Class}
\begin{description}
\item[\textbf{NewPedigree(kw)} (class)]
The NewPedigree class is the main data structure for PyP 2.0.0Final.

\item[\textbf{load(pedsource='file')} $\Rightarrow$ None]\index[func]{pyp_newclasses!NewPedigree!load()}
load() wraps several processes useful for loading and preparing a pedigree for use in an analysis, including reading the animals into a list of animal objects, forming lists of sires and dams, checking for common errors, setting ancestor flags, and renumbering the pedigree.
\begin{description}
\item[\emph{renum}] Flag to indicate whether or not the pedigree is to be renumbered.
\item[\emph{alleles}] Flag to indicate whether or not pyp\_metrics/effective\_founder\_genomes() should be called for a single round to assign alleles.
\item[Returns:] None
\end{description}

\item[\textbf{preprocess()} $\Rightarrow$ None]\index[func]{pyp_newclasses!NewPedigree!preprocess()}
preprocess() processes a pedigree file, which includes reading the animals into a list of animal objects, forming lists of sires and dams, and checking for common errors.
\begin{description}
\item[\emph{None}]
\item[Returns:] None
\end{description}

\item[\textbf{renumber()} $\Rightarrow$ None]\index[func]{pyp_newclasses!NewPedigree!renumber()}
renumber() updates the ID map after a pedigree has been renumbered so that all references are to renumbered rather than original IDs.
\begin{description}
\item[\emph{None}]
\item[Returns:] None
\end{description}

\item[\textbf{save(filename='', outformat='o', idformat='o')} $\Rightarrow$ integer]\index[func]{pyp_newclasses!NewPedigree!save()}
save() writes a PyPedal pedigree to a user-specified file. The saved pedigree includes all fields recognized by PyPedal, not just the original fields read from the input pedigree file.
\begin{description}
\item[\emph{filename}] The file to which the pedigree should be written.
\item[\emph{outformat}] The format in which the pedigree should be written: 'o' for original (as read) and 'l' for long version (all available variables).
\item[\emph{idformat}] Write 'o' (original) or 'r' (renumbered) animal, sire, and dam IDs.
\item[Returns:] A save status indicator (0: failed, 1: success)
\end{description}

\item[\textbf{updateidmap()} $\Rightarrow$ None]\index[func]{pyp_newclasses!NewPedigree!updateidmap()}
updateidmap() updates the ID map after a pedigree has been renumbered so that all references are to renumbered rather than original IDs.
\begin{description}
\item[\emph{None}]
\item[Returns:] None
\end{description}

\end{description}

\subsection*{The PedigreeMetadata Class}
\begin{description}
\item[\textbf{PedigreeMetadata(myped, kw)} (class)]
The PedigreeMetadata() class stores metadata about pedigrees. Hopefully this will help improve performance in some procedures, as well as provide some useful summary data.

\item[\textbf{\_\_init\_\_(myped, kw)} $\Rightarrow$ object]\index[func]{pyp_newclasses!PedigreeMetadata!\_\_init\_\_()}
\_\_init\_\_() initializes a PedigreeMetadata object.
\begin{description}
\item[\emph{self}] Reference to the current Pedigree() object
\item[\emph{myped}] A PyPedal pedigree.
\item[\emph{kw}] A dictionary of options.
\item[Returns:] An instance of a Pedigree() object populated with data
\end{description}

\item[\textbf{fileme()} $\Rightarrow$ None]\index[func]{pyp_newclasses!PedigreeMetadata!fileme()}
fileme() writes the metada stored in the Pedigree() object to disc.
\begin{description}
\item[\emph{self}] Reference to the current Pedigree() object
\end{description}

\item[\textbf{nud()} $\Rightarrow$ integer-and-list]\index[func]{pyp_newclasses!PedigreeMetadata!nud()}
nud() returns the number of unique dams in the pedigree along with a list of the dams
\begin{description}
\item[\emph{self}] Reference to the current Pedigree() object
\item[Returns:] The number of unique dams in the pedigree and a list of those dams
\end{description}

\item[\textbf{nuf()} $\Rightarrow$ integer-and-list]\index[func]{pyp_newclasses!PedigreeMetadata!nuf()}
nuf() returns the number of unique founders in the pedigree along with a list of the founders
\begin{description}
\item[\emph{self}] Reference to the current Pedigree() object
\item[Returns:] The number of unique founders in the pedigree and a list of those founders
\end{description}

\item[\textbf{nug()} $\Rightarrow$ integer-and-list]\index[func]{pyp_newclasses!PedigreeMetadata!nug()}
nug() returns the number of unique generations in the pedigree along with a list of the generations
\begin{description}
\item[\emph{self}] Reference to the current Pedigree() object
\item[Returns:] The number of unique generations in the pedigree and a list of those generations
\end{description}

\item[\textbf{nus()} $\Rightarrow$ integer-and-list]\index[func]{pyp_newclasses!PedigreeMetadata!nus()}
nus() returns the number of unique sires in the pedigree along with a list of the sires
\begin{description}
\item[\emph{self}] Reference to the current Pedigree() object
\item[Returns:] The number of unique sires in the pedigree and a list of those sires
\end{description}

\item[\textbf{nuy()} $\Rightarrow$ integer-and-list]\index[func]{pyp_newclasses!PedigreeMetadata!nuy()}
nuy() returns the number of unique birthyears in the pedigree along with a list of the birthyears
\begin{description}
\item[\emph{self}] Reference to the current Pedigree() object
\item[Returns:] The number of unique birthyears in the pedigree and a list of those birthyears
\end{description}

\item[\textbf{printme()} $\Rightarrow$ None]\index[func]{pyp_newclasses!PedigreeMetadata!printme()}
printme() prints a summary of the metadata stored in the Pedigree() object.
\begin{description}
\item[\emph{self}] Reference to the current Pedigree() object
\end{description}

\item[\textbf{stringme()} $\Rightarrow$ None]\index[func]{pyp_newclasses!PedigreeMetadata!stringme()}
stringme() returns a summary of the metadata stored in the pedigree as a string.
\begin{description}
\item[\emph{self}] Reference to the current Pedigree() object
\end{description}

\end{description}
\section{pyp\_nrm}
pyp\_nrm\index[func]{pyp_nrm} contains several procedures for computing numerator relationship matrices and for performing operations on those matrices. It also contains routines for computing CoI on large pedigrees using the recursive method of VanRaden \cite{VanRaden1992a}.

\subsection*{Module Contents}

\begin{description}
\item[\textbf{a\_decompose(pedobj)} $\Rightarrow$ matrices] \index[func]{pyp_nrm!a\_decompose()}
Form the decomposed form of A, TDT', directly from a pedigree (after Henderson \cite{ref143}, Mrode \cite{ref224}). Return D, a diagonal matrix, and T, a lower triagular matrix such that A = TDT'.
\begin{description}
\item[\emph{pedobj}] A PyPedal pedigree object.
\item[Returns:] A diagonal matrix, D, and a lower triangular matrix, T.
\end{description}

\item[\textbf{a\_inverse\_df(pedobj)} $\Rightarrow$ matrix] \index[func]{pyp_nrm!a\_inverse\_df()}
Directly form the inverse of A from the pedigree file - accounts for inbreeding - using the method of Quaas \cite{ref235}.
\begin{description}
\item[\emph{pedobj}] A PyPedal pedigree object.
\item[Returns:] The inverse of the NRM, A, accounting for inbreeding.
\end{description}

\item[\textbf{a\_inverse\_dnf(pedobj, filetag='\_a\_inverse\_dnf\_')} $\Rightarrow$ matrix] \index[func]{pyp_nrm!a\_inverse\_dnf()}
Form the inverse of A directly using the method of Henderson \cite{ref143} which does not account for inbreeding.
\begin{description}
\item[\emph{pedobj}] A PyPedal pedigree object.
\item[Returns:] The inverse of the NRM, A, not accounting for inbreeding.
\end{description}

\item[\textbf{a\_matrix(pedobj, save=0)} $\Rightarrow$ array] \index[func]{pyp_nrm!a\_matrix()}
a\_matrix() is used to form a numerator relationship matrix from a pedigree. DEPRECATED. use fast\_a\_matrix() instead.
\begin{description}
\item[\emph{pedobj}] A PyPedal pedigree object.
\item[\emph{save}] Flag to indicate whether or not the relationship matrix is written to a file.
\item[Returns:] The NRM as a numarray matrix.
\end{description}

\item[\textbf{fast\_a\_matrix(pedigree, pedopts, save=0)} $\Rightarrow$ matrix] \index[func]{pyp_nrm!fast\_a\_matrix()}
Form a numerator relationship matrix from a pedigree. fast\_a\_matrix() is a hacked version of a\_matrix() modified to try and improve performance. Lists of animal, sire, and dam IDs are formed and accessed rather than myped as it is much faster to access a member of a simple list rather than an attribute of an object in a list. Further note that only the diagonal and upper off-diagonal of A are populated. This is done to save n(n+1) / 2 matrix writes. For a 1000-element array, this saves 500,500 writes.
\begin{description}
\item[\emph{pedigree}] A PyPedal pedigree.
\item[\emph{pedopts}] PyPedal options.
\item[\emph{save}] Flag to indicate whether or not the relationship matrix is written to a file.
\item[Returns:] The NRM as Numarray matrix.
\end{description}

\item[\textbf{fast\_a\_matrix\_r(pedigree, pedopts, save=0)} $\Rightarrow$ matrix] \index[func]{pyp_nrm!fast\_a\_matrix\_r()}
Form a relationship matrix from a pedigree. fast\_a\_matrix\_r() differs from fast\_a\_matrix() in that the coefficients of relationship are corrected for the inbreeding of the parents.
\begin{description}
\item[\emph{pedobj}] A PyPedal pedigree object.
\item[\emph{save}] Flag to indicate whether or not the relationship matrix is written to a file.
\item[Returns:] A relationship as Numarray matrix.
\end{description}

\item[\textbf{form\_d\_nof(pedobj)} $\Rightarrow$ matrix] \index[func]{pyp_nrm!form\_d\_nof()}
Form the diagonal matrix, D, used in decomposing A and forming the direct inverse of A. This function does not write output to a file - if you need D in a file, use the a\_decompose() function. form\_d() is a convenience function used by other functions. Note that inbreeding is not considered in the formation of D.
\begin{description}
\item[\emph{pedobj}] A PyPedal pedigree object.
\item[Returns:] A diagonal matrix, D.
\end{description}

\item[\textbf{inbreeding(pedobj, method='tabular')} $\Rightarrow$ dictionary] \index[func]{pyp_nrm!inbreeding()}
inbreeding() is a proxy function used to dispatch pedigrees to the appropriate function for computing CoI. By default, small pedigrees $<$ 10,000 animals) are processed with the tabular method directly. For larger pedigrees, or if requested, the recursive method of VanRaden \cite{VanRaden1992a} is used.
\begin{description}
\item[\emph{pedobj}] A PyPedal pedigree object.
\item[\emph{method}] Keyword indicating which method of computing CoI should be used (tabular|vanraden).
\item[Returns:] A dictionary of CoI keyed to renumbered animal IDs.
\end{description}

\item[\textbf{inbreeding\_tabular(pedobj)} $\Rightarrow$ dictionary] \index[func]{pyp_nrm!inbreeding\_tabular()}
inbreeding\_tabular() computes CoI using the tabular method by calling fast\_a\_matrix() to form the NRM directly. In order for this routine to return successfully requires that you are able to allocate a matrix of floats of dimension len(myped)**2.
\begin{description}
\item[\emph{pedobj}] A PyPedal pedigree object.
\item[Returns:] A dictionary of CoI keyed to renumbered animal IDs
\end{description}

\item[\textbf{inbreeding\_vanraden(pedobj, cleanmaps=1)} $\Rightarrow$ dictionary] \index[func]{pyp_nrm!inbreeding\_vanraden()}
inbreeding\_vanraden() uses VanRaden's \cite{VanRaden1992a} method for computing coefficients of inbreeding in a large pedigree. The method works as follows: 1. Take a large pedigree and order it from youngest animal to oldest (n, n-1, ..., 1); 2. Recurse through the pedigree to find all of the ancestors of that animal n; 3. Reorder and renumber that ``subpedigree''; 4. Compute coefficients of inbreeding for that ``subpedigree'' using the tabular method (Emik and Terrill \cite{Emik1949a}); 5. Put the coefficients of inbreeding in a dictionary; 6. Repeat 2 - 5 for animals n-1 through 1; the process is slowest for the early pedigrees and fastest for the later pedigrees.
\begin{description}
\item[\emph{pedobj}] A PyPedal pedigree object.
\item[\emph{cleanmaps}] Flag to denote whether or not subpedigree ID maps should be delete after they are used (0|1)
\item[Returns:] A dictionary of CoI keyed to renumbered animal IDs
\end{description}

\item[\textbf{recurse\_pedigree(pedobj, anid, \_ped)} $\Rightarrow$ list] \index[func]{pyp_nrm!recurse\_pedigree()}
recurse\_pedigree() performs the recursion needed to build the subpedigrees used by inbreeding\_vanraden(). For the animal with animalID anid recurse\_pedigree() will recurse through the pedigree myped and add references to the relatives of anid to the temporary pedigree, \_ped.
\begin{description}
\item[\emph{pedobj}] A PyPedal pedigree.
\item[\emph{anid}] The ID of the animal whose relatives are being located.
\item[\emph{\_ped}] A temporary PyPedal pedigree that stores references to relatives of anid.
\item[Returns:] A list of references to the relatives of anid contained in myped.
\end{description}

\item[\textbf{recurse\_pedigree\_idonly(pedobj, anid, \_ped)} $\Rightarrow$ list] \index[func]{pyp_nrm!recurse\_pedigree\_idonly()}
recurse\_pedigree\_idonly() performs the recursion needed to build subpedigrees.
\begin{description}
\item[\emph{pedobj}] A PyPedal pedigree.
\item[\emph{anid}] The ID of the animal whose relatives are being located.
\item[\emph{\_ped}] A PyPedal list that stores the animalIDs of relatives of anid.
\item[Returns:] A list of animalIDs of the relatives of anid contained in myped.
\end{description}

\item[\textbf{recurse\_pedigree\_n(pedobj, anid, \_ped, depth=3)} $\Rightarrow$ list] \index[func]{pyp_nrm!recurse\_pedigree\_n()}
recurse\_pedigree\_n() recurses to build a pedigree of depth n. A depth less than 1 returns the animal whose relatives were to be identified.
\begin{description}
\item[\emph{pedobj}] A PyPedal pedigree.
\item[\emph{anid}] The ID of the animal whose relatives are being located.
\item[\emph{\_ped}] A temporary PyPedal pedigree that stores references to relatives of anid.
\item[\emph{depth}] The depth of the pedigree to return.
\item[Returns:] A list of references to the relatives of anid contained in myped.
\end{description}

\item[\textbf{recurse\_pedigree\_onesided(pedobj, anid, \_ped, side)} $\Rightarrow$ list] \index[func]{pyp_nrm!recurse\_pedigree\_onesided()}
recurse\_pedigree\_onsided() recurses to build a subpedigree from either the sire or dam side of a pedigree.
\begin{description}
\item[\emph{pedobj}] A PyPedal pedigree.
\item[\emph{side}] The side to build: 's' for sire and 'd' for dam.
\item[\emph{anid}] The ID of the animal whose relatives are being located.
\item[\emph{\_ped}] A temporary PyPedal pedigree that stores references to relatives of anid.
\item[Returns:] A list of references to the relatives of anid contained in myped.
\end{description}

\end{description}
\section{pyp\_reports}
pyp\_reports contains a set of procedures for generating reports
\index[func]{pyp_reports}
\label{sec:pyp-reports}
\subsection*{Module Contents}
\begin{description}
\item[\textbf{\_pdfCreateTitlePage(canv, \_pdfSettings, reporttitle='', reportauthor='')} $\Rightarrow$ None]
\index[func]{pyp_reports!_pdfCreateTitlePage()}
\label{sec:pyp-reports-pdf-create-title-page}
\_pdfCreateTitlePage() adds a title page to a ReportLab canvas object.
\begin{description}
\item[\emph{canv}] An instance of a ReportLab Canvas object.
\item[\emph{\_pdfSettings}] An options dictionary created by \_pdfInitialize().
\index[func]{pyp_reports!_pdfSettings()}
\label{sec:pyp-reports-pdf-settings}
\item[Returns:] None
\end{description}
\item[\textbf{\_pdfDrawPageFrame(canv, \_pdfSettings)} $\Rightarrow$ None]
\_pdfDrawPageFrame() nicely frames page contents and includes the document title in a header and the page number in a footer.
\index[func]{pyp_reports!_pdfDrawPageFrame()}
\label{sec:pyp-reports-pdf-draw-page-frame}
\begin{description}
\item[\emph{canv}] An instance of a ReportLab Canvas object.
\item[\emph{\_pdfSettings}] An options dictionary created by \_pdfInitialize().
\item[Returns:] None
\end{description}
\item[\textbf{\_pdfInitialize(pedobj)} $\Rightarrow$ dictionary]
\_pdfInitialize() returns a dictionary of metadata that is used for report generation.
\index[func]{pyp_reports!_pdfInitialize()}
\label{sec:pyp-reports-pdf-initialize}
\begin{description}
\item[\emph{pedobj}] A PyPedal pedigree object.
\item[Returns:] A dictionary of metadata that is used for report generation.
\end{description}
\item[\textbf{meanMetricBy(pedobj, metric='fa', byvar='by')} $\Rightarrow$ dictionary]
meanMetricBy() returns a dictionary of means keyed by levels of the 'byvar' that can be used to draw graphs or prepare reports of summary statistics.
\index[func]{pyp_reports!meanMetricBy()}
\label{sec:pyp-reports-mean-metric-by}
\begin{description}
\item[\emph{pedobj}] A PyPedal pedigree object.
\item[\emph{metric}] The variable to summarize on a BY variable.
\item[\emph{byvar}] The variable on which to group the metric.
\item[Returns:] A dictionary containing means for the metric variable keyed to levels of the byvar.
\end{description}
\item[\textbf{pdfPedigreeMetadata(pedobj, titlepage=0, reporttitle='', reportauthor='', reportfile='')} $\Rightarrow$ integer ]
pdfPedigreeMetadata() produces a report, in PDF format, of the metadata from the input pedigree. It is intended for use as a template for custom printed reports.
\index[func]{pyp_reports!pdfPedigreeMetadata()}
\label{sec:pyp-reports-pdf-pedigree-metadata}
\begin{description}
\item[\emph{pedobj}] A PyPedal pedigree object.
\item[\emph{titlepage}] Show (1) or hide (0) the title page.
\item[\emph{reporttitle}] Title of report; if '', \_pdfTitle is used.
\item[\emph{reportauthor}] Author/preparer of report.
\item[\emph{reportfile}] Optional name of file to which the report should be written.
\item[Returns:] A 1 on success, 0 otherwise.
\end{description}
\end{description}
\section{pyp\_utils}
pyp\_utils\index[func]{pyp_utils} contains a set of procedures for creating and operating on PyPedal pedigrees. This includes routines for reordering and renumbering pedigrees, as well as for modifying pedigrees.

\subsection*{Module Contents}

\begin{description}
\item[\textbf{assign\_offspring(pedobj)} $\Rightarrow$ integer]\index[func]{pyp_utils!assign_offspring()}
assign\_offspring() assigns offspring to their parent(s)'s unknown sex offspring list (well, dictionary).
\begin{description}
\item[\emph{myped}] An instance of a NewPedigree object.
\item[Returns:] 0 for failure and 1 for success.
\end{description}

\item[\textbf{assign\_sexes(pedobj)} $\Rightarrow$ integer]\index[func]{pyp_utils!assign_sexes()}
assign\_sexes() assigns a sex to every animal in the pedigree using sire and daughter lists for improved accuracy.
\begin{description}
\item[\emph{pedobj}] A renumbered and reordered PyPedal pedigree object.
\item[Returns:] 0 for failure and 1 for success.
\end{description}

\item[\textbf{delete\_id\_map(filetag='\_renumbered\_')} $\Rightarrow$ integer]\index[func]{pyp_utils!delete_id_map()}
delete\_id\_map() checks to see if an ID map for the given filetag exists. If the file exists, it is deleted.
\begin{description}
\item[\emph{filetag}] A descriptor prepended to output file names that is used to determine name of the file to delete.
\item[Returns:] A flag indicating whether or not the file was successfully deleted (0|1)
\end{description}

\item[\textbf{fast\_reorder(myped, filetag='\_new\_reordered\_', io='no', debug=0)} $\Rightarrow$ list]\index[func]{pyp_utils!fast_reorder()}
fast\_reorder() renumbers a pedigree such that parents precede their offspring in the pedigree. In order to minimize overhead as much as is reasonably possible, a list of animal IDs that have already been seen is kept. Whenever a parent that is not in the seen list is encountered, the offspring of that parent is moved to the end of the pedigree. This should ensure that the pedigree is properly sorted such that all parents precede their offspring. myped is reordered in place. fast\_reorder() uses dictionaries to renumber the pedigree based on paddedIDs.
\begin{description}
\item[\emph{myped}] A PyPedal pedigree object.
\item[\emph{filetag}] A descriptor prepended to output file names.
\item[\emph{io}] Indicates whether or not to write the reordered pedigree to a file (yes|no).
\item[\emph{debug}] Flag to indicate whether or not debugging messages are written to STDOUT.
\item[Returns:] A reordered PyPedal pedigree.
\end{description}

\item[\textbf{load\_id\_map(filetag='\_renumbered\_')} $\Rightarrow$ dictionary]\index[func]{pyp_utils!load_id_map()}
load\_id\_map() reads an ID map from the file generated by pyp\_utils/renumber() into a dictionary. There is a VERY similar function, pyp\_io/id\_map\_from\_file(), that is deprecated because it is much more fragile that this procedure.
\begin{description}
\item[\emph{filetag}] A descriptor prepended to output file names that is used to determine the input file name.
\item[Returns:] A dictionary whose keys are renumbered IDs and whose values are original IDs or an empty dictionary (on failure).
\end{description}

\item[\textbf{pedigree\_range(pedobj, n)} $\Rightarrow$ list]\index[func]{pyp_utils!pedigree_range()}
pedigree\_range() takes a renumbered pedigree and removes all individuals with a renumbered ID $>$ n. The reduced pedigree is returned. Assumes that the input pedigree is sorted on animal key in ascending order.
\begin{description}
\item[\emph{myped}] A PyPedal pedigree object.
\item[\emph{n}] A renumbered animalID.
\item[Returns:] A pedigree containing only animals born in the given birthyear or an empty list (on failure).
\end{description}

\item[\textbf{pyp\_nice\_time()} $\Rightarrow$ string]\index[func]{pyp_utils!pyp_nice_time()}
pyp\_nice\_time() returns the current date and time formatted as, e.g., Wed Mar 30 10:26:31 2005.
\begin{description}
\item[\emph{None}]
\item[Returns:] A string containing the formatted date and time.
\end{description}

\item[\textbf{renumber(myped, filetag='\_renumbered\_', io='no', outformat='0', debug=0)} $\Rightarrow$ list]\index[func]{pyp_utils!renumber()}
renumber() takes a pedigree as input and renumbers it such that the oldest animal in the pedigree has an ID of '1' and the n-th animal has an ID of 'n'. If the pedigree is not ordered from oldest to youngest such that all offspring precede their offspring, the pedigree will be reordered. The renumbered pedigree is written to disc in 'asd' format and a map file that associates sequential IDs with original IDs is also written.
\begin{description}
\item[\emph{myped}] A PyPedal pedigree object.
\item[\emph{filetag}] A descriptor prepended to output file names.
\item[\emph{io}] Indicates whether or not to write the renumbered pedigree to a file (yes|no).
\item[\emph{outformat}] Flag to indicate whether or not to write an asd pedigree (0) or a full pedigree (1).
\item[\emph{debug}] Flag to indicate whether or not progress messages are written to stdout.
\item[Returns:] A reordered PyPedal pedigree.
\end{description}

\item[\textbf{reorder(myped, filetag='\_reordered\_', io='no')} $\Rightarrow$ list]\index[func]{pyp_utils!reorder()}
reorder() renumbers a pedigree such that parents precede their offspring in the pedigree. In order to minimize overhead as much as is reasonably possible, a list of animal IDs that have already been seen is kept. Whenever a parent that is not in the seen list is encountered, the offspring of that parent is moved to the end of the pedigree. This should ensure that the pedigree is properly sorted such that all parents precede their offspring. myped is reordered in place. reorder() is VERY slow, but I am pretty sure that it works correctly.
\begin{description}
\item[\emph{myped}] A PyPedal pedigree object.
\item[\emph{filetag}] A descriptor prepended to output file names.
\item[\emph{io}] Indicates whether or not to write the reordered pedigree to a file (yes|no).
\item[Returns:] A reordered PyPedal pedigree.
\end{description}

\item[\textbf{reverse\_string(mystring)} $\Rightarrow$ string]\index[func]{pyp_utils!reverse_string()}
reverse\_string() reverses the input string and returns the reversed version.
\begin{description}
\item[\emph{mystring}] A non-empty Python string.
\item[Returns:] The input string with the order of its characters reversed.
\end{description}

\item[\textbf{set\_age(pedobj)} $\Rightarrow$ integer]\index[func]{pyp_utils!set_age()}
set\_age() Computes ages for all animals in a pedigree based on the global BASE\_DEMOGRAPHIC\_YEAR defined in pyp\_demog.py. If the by is unknown, the inferred generation is used. If the inferred generation is unknown, the age is set to -999.
\begin{description}
\item[\emph{pedobj}] A PyPedal pedigree object.
\item[Returns:] 0 for failure and 1 for success.
\end{description}

\item[\textbf{set\_ancestor\_flag(pedobj)} $\Rightarrow$ integer]\index[func]{pyp_utils!set_ancestor_flag()}
set\_ancestor\_flag() loops through a pedigree to build a dictionary of all of the parents in the pedigree. It then sets the ancestor flags for the parents. set\_ancestor\_flag() expects a reordered and renumbered pedigree as input!
\begin{description}
\item[\emph{pedobj}] A PyPedal NewPedigree object.
\item[Returns:] 0 for failure and 1 for success.
\end{description}

\item[\textbf{set\_generation(pedobj)} $\Rightarrow$ integer]\index[func]{pyp_utils!set_generation()}
set\_generation() Works through a pedigree to infer the generation to which an animal belongs based on founders belonging to generation 1. The igen assigned to an animal as the larger of sire.igen+1 and dam.igen+1. This routine assumes that myped is reordered and renumbered.
\begin{description}
\item[\emph{pedobj}] A PyPedal NewPedigree object.
\item[Returns:] 0 for failure and 1 for success.
\end{description}

\item[\textbf{set\_species(pedobj, species='u')} $\Rightarrow$ integer]\index[func]{pyp_utils!set_species()}
set\_species() assigns a specie to every animal in the pedigree.
\begin{description}
\item[\emph{pedobj}] A PyPedal pedigree object.
\item[\emph{species}] A PyPedal string.
\item[Returns:] 0 for failure and 1 for success.
\end{description}

\item[\textbf{simple\_histogram\_dictionary(mydict, histchar='*', histstep=5)} $\Rightarrow$ dictionary]\index[func]{pyp_utils!simple_histogram_dictionary()}
simple\_histogram\_dictionary() returns a dictionary containing a simple, text histogram. The input dictionary is assumed to contain keys which are distinct levels and values that are counts.
\begin{description}
\item[\emph{mydict}] A non-empty Python dictionary.
\item[\emph{histchar}] The character used to draw the histogram (default is '*').
\item[\emph{histstep}] Used to determine the number of bins (stars) in the diagram.
\item[Returns:] A dictionary containing the histogram by level or an empty dictionary (on failure).
\end{description}

\item[\textbf{sort\_dict\_by\_keys(mydict)} $\Rightarrow$ dictionary]\index[func]{pyp_utils!sort_dict_by_keys()}
sort\_dict\_by\_keys() returns a dictionary where the values in the dictionary in the order obtained by sorting the keys. Taken from the routine sortedDictValues3 in the ``Python Cookbook'', p. 39.
\begin{description}
\item[\emph{mydict}] A non-empty Python dictionary.
\item[Returns:] The input dictionary with keys sorted in ascending order or an empty dictionary (on failure).
\end{description}

\item[\textbf{sort\_dict\_by\_values(first, second)} $\Rightarrow$ list]\index[func]{pyp_utils!sort_dict_by_values()}
sort\_dict\_by\_values() returns a dictionary where the keys in the dictionary are sorted ascending value, first on value and then on key within value. The implementation was taken from John Hunter's contribution to a newsgroup thread: \url{http://groups-beta.google.com/group/comp.lang.python/browse}\_thread/thread/bbc259f8454e4d3f/cc686f4cd795feb4?q=python+\%22sorted+dictionary\%22=1=en\#cc686f4cd795feb4
\begin{description}
\item[\emph{mydict}] A non-empty Python dictionary.
\item[Returns:] A list of tuples sorted in ascending order.
\end{description}

\item[\textbf{string\_to\_table\_name(instring)} $\Rightarrow$ string]\index[func]{pyp_utils!string_to_table_name()}
string\_to\_table\_name() takes an arbitrary string and returns a string that is safe to use as an SQLite table name.
\begin{description}
\item[\emph{instring}] A string that will be converted to an SQLite-safe table name.
\item[Returns:] A string that is safe to use as an SQLite table name.
\end{description}

\item[\textbf{trim\_pedigree\_to\_year(pedobj, year)} $\Rightarrow$ list]\index[func]{pyp_utils!trim_pedigree_to_year()}
trim\_pedigree\_to\_year() takes pedigrees and removes all individuals who were not born in birthyear 'year'.
\begin{description}
\item[\emph{myped}] A PyPedal pedigree object.
\item[\emph{year}] A birthyear.
\item[Returns:] A pedigree containing only animals born in the given birthyear or an ampty list (on failure).
\end{description}

\end{description}
%\chapter{Tutorial}
\label{cha:tutorial}

\begin{quote} 
   This chapter provides a tutorial for \PYPEDAL{}.  The sample pedigree files may be found in the directory \texttt{} in the distribution. \footnote{Please let me know of any additions to this tutorial that you feel would be helpful.}
\end{quote}

We are going to start the actual tutorial in this chapter.  First, however, we will describe some key concepts that will help you work successfully with \PYPEDAL{}.  You can find a more detailed explanation of \PYPEDAL{} components in chapter \ref{chapter:api}.

\section{A Few Important Concepts}
To make the most of \PYPEDAL{} you, the user, need to have a solid understanding of your dataset as well as of the \PYPEDAL{} API.  While Python is an object-oriented programming language, \PYPEDAL{} is at heart a procedural tool.  One of the exceptions to this rule is what \PYPEDAL{} terms a pedigree, which is a Python list containing Animal() objects.  The first step in most \PYPEDAL{} analyses is to read your pedigree into \PYPEDAL{} from a textfile.  After that, you will spend most of your time passing your pedigree from one procedure to another.  But always remember that the elements in the pedigree are objects!


\section{A Gentle Introduction to PyPedal}
For this tutorial we are going to use a sample pedigree from Hartl and Clark'a (1989) "Principles of Population Genetics (Second Edition)" (Figure 5, p. 242).  The pedigree is provided as \textbf{hartl.ped} in the distribution in the \texttt{tutorial} subdirectory,  There is also an accompanying Python program, \textbf{hartl.py}.

\subsection{The Anatomy of a Pedigree File}
Obviously you need a pedigree file in order to work with \PYPEDAL{}.  There are a couple of things that you need to know about pedigree files and at least one thing that is helpful to know.  Pedigree files must contain a format code, and the format code \textbf{must} precede the first animal record.  A complete list of pedigree codes appears in section \ref{sec:pedigree-format-codes}.  Each animal record must appear on a separate line in the pedigree file.  An animal record consists of at least an animal ID, a sire ID, and a dam ID; the IDs are separated by a delimiter, usually a comma or a space.  More information may be required on a linde depending on the pedigree format used.  Missing parents should be coded as `0'.  Parents do not need to have their own entry in the pedigree if THEIR parents are unknown; the \texttt{preprocess()} procedure is clever enough to add the needed records automatically.  Comment lines, which begin with `\#', may appear anywhere in the file; they are ignored by the preprocessor.
\begin{verbatim}
# Great tit pedigree from Hartl and Clark (1989), figure 5, p. 242.
# Used in PyPedal tutorial.
% asd
1 0 0
2 0 0
3 0 0
4 1 2
5 1 2
6 3 4
7 3 4
8 5 0
9 0 6
10 7 0
11 8 0
12 9 11
13 12 7
14 10 11
15 13 14
\end{verbatim}
This pedigree contains fifteen animals, including three founders (animals with neither parent known), in the familiar 'animal sire dam' format.

\subsection{The Anatomy of a Program}
The \textbf{hartl.ped} program is fairly simple, but it demonstrates some of the things that you can easily do with \PYPEDAL{}.  Please note that while I have placed these comamnds in a file, you can also walk through the steps using the Python command line.  Most of the print statements are there to provide feedback while the program is running.  It is not a big deal with a small pedigree, but it is nice to know that something is happening when you throw a large pedigree at \PYPEDAL().  I have put in line numbers for ease os reference, but if you are working along with the tutorial at the command line you should not type in the line numbers.

\begin{verbatim}
001 print 'Starting pypedal.py at %s' % asctime(localtime(time()))
002 print '\tPreprocessing pedigree at %s' % asctime(localtime(time()))
003 example = preprocess('hartl.ped',sepchar=' ')
004 example = renumber(example,'example',io='yes')
005 print '\tCalling set_ancestor_flag at %s' % asctime(localtime(time()))
006 set_ancestor_flag(example,'example',io='yes')
007 print '\tCollecting pedigree metadata at %s' % asctime(localtime(time()))
008 example_meta = Pedigree(example,'example.ped','example_meta')
009 print '\tCalling a_effective_founders_lacy() at %s' % asctime(localtime(time()))
010 a_effective_founders_lacy(example,filetag='example')
011 print '\tCalling a_effective_founders_boichard() at %s' % asctime(localtime(time()))
012 a_effective_founders_boichard(example,filetag='example')
013 print '\tCalling a_effective_ancestors_definite() at %s' % asctime(localtime(time()))
014 a_effective_ancestors_definite(example,filetag='example')
015 print '\tCalling a_effective_ancestors_indefinite() at %s' % asctime(localtime(time()))
016 a_effective_ancestors_indefinite(example,filetag='example',n=10)
017 print '\tCalling related_animals() at %s' % asctime(localtime(time()))
018 list_a = related_animals(example[14].animalID,example)
019 print list_a
020 print '\tCalling related_animals() at %s' % asctime(localtime(time()))
021 list_b = related_animals(example[9].animalID,example)
022 print list_b
023 print '\tCalling common_ancestors() at %s' % asctime(localtime(time()))
024 list_r = common_ancestors(example[14].animalID,example[9].animalID,example)
025 print list_r
026 print 'Stopping pypedal.py at %s' % asctime(localtime(time()))
\end{verbatim}

\subsection{Reading PyPedal Output}
\begin{verbatim}
Starting pypedal.py at Mon Apr 19 15:28:53 2004
        Preprocessing pedigree at Mon Apr 19 15:28:53 2004
        Calling set_ancestor_flag at Mon Apr 19 15:28:53 2004
        Collecting pedigree metadata at Mon Apr 19 15:28:53 2004
PEDIGREE example_meta (example.ped)
        Records:                15
        Unique Sires:           9
        Unique Dams:            7
        Unique Gens:            1
        Unique Years:           1
        Unique Founders:        3
        Pedigree Code:          asd
        Calling inbreeding() at Mon Apr 19 15:28:53 2004
{1: 0.0, 2: 0.0, 3: 0.0, 4: 0.0, 5: 0.0, 6: 0.0, 7: 0.0, 8: 0.0, 9: 0.0, 10: 0.0, 11: 0.0, 12: 0.015625, 13: 0.078125, 14: 0.015625, 15: 0.14453125}
        Calling a_effective_founders_lacy() at Mon Apr 19 15:28:53 2004
============================================================
animals:        15
founders:       3
descendants:    12
f_e:            7.205
============================================================
        Calling a_effective_founders_boichard() at Mon Apr 19 15:28:53 2004
============================================================
animals:        15
founders:       3
descendants:    12
f_e:            5.856
============================================================
        Calling a_effective_ancestors_definite() at Mon Apr 19 15:28:53 2004
============================================================
animals:        15
founders:       0
descendants:    15
f_a:            0.000
============================================================
        Calling a_effective_ancestors_indefinite() at Mon Apr 19 15:28:53 2004
------------------------------------------------------------
WARNING: (pyp_metrics/a_effective_ancestors_indefinite()): Setting n (10) to be equal to the actual number of founders (0) in the pedigree!
============================================================
animals:        15
founders:       0
descendants:    15
f_l:            0.000
f_u:            1.000
============================================================
        Calling related_animals() at Mon Apr 19 15:28:53 2004
[15, 13, 7, 3, 4, 1, 2, 12, 9, 6, 11, 8, 5, 14, 10]
        Calling related_animals() at Mon Apr 19 15:28:53 2004
[10, 7, 3, 4, 1, 2]
        Calling common_ancestors() at Mon Apr 19 15:28:53 2004
[1, 2, 3, 4, 7, 10]
Stopping pypedal.py at Mon Apr 19 15:28:53 2004
\end{verbatim}

\chapter{Glossary}
\label{cha:glossary}
\begin{quote}
This chapter provides a glossary of terms.\footnote{Please let me know of any additions to this list which
you feel would be helpful.}
\end{quote}
\begin{description}
\item[coefficient of inbreeding] Probability that two alleles selected at random are identical by descent.
\end{description}
\begin{description}
\item[coefficient of relationship] Proportion of genes that two individuals share on average.
\end{description}
\begin{description}
\item[effective ancestor number] The number of equally-contributing ancestors, not necessarily founders, needed to produce a population with the heterozygosity of the studied population \cite{ref352}.
\end{description}
\begin{description}
\item[effective founder number] The number of equally-contributing founders needed to produce a population with the heterozygosity of the studied population \cite{ref640}.
\end{description}
\begin{description}
\item[effective population size] The effective population size is the size of an ideal population that would lose heterozygosity at a rate equal to that of the studied population \cite{ref91}.
\end{description}
\begin{description}
\item[founder] An animal with unknown parents that is assumed to be unrelated to all other founders.
\end{description}
\begin{description}
\item[internal report] A \PyPedal() report that is intended for use by other \PyPedal() procedures, such as plotting
routines, and not for printing.
\end{description}
\begin{description}
\item[numerator relationship matrix] Matrix of additive genetic covariances among the animals in a population.
\end{description}
\begin{description}
\item[pedigree] A \PYPEDAL{} pedigree consists of a Python list containing instances of \PYPEDAL{} NewAnimal{} objects.
\end{description}
\begin{description}
\item[renumbering] Many calculations require that the animals in a pedigree be ordered from oldest to youngest, with sires and dams preceding offspring, and renumbered  starting with 1.  This is a computational necessity, and results in an animal's ID (\texttt{animalID}) being changed to reflect that animal's order in the pedigree.  All animals have their original IDs stored in their \texttt{originalName} attribute.
\end{description}

\bibliographystyle{chicago}
\bibliography{references}

\printindex
\printindex[func]

\end{document}