\chapter{Glossary}
\label{cha:glossary}
\begin{quote}
This chapter provides a glossary of terms.\footnote{Please let me know of any additions to this list which
you feel would be helpful.}
\end{quote}
\begin{description}
\item[coefficient of inbreeding] Probability that two alleles selected at random are identical by descent.
\end{description}
\begin{description}
\item[coefficient of relationship] Proportion of genes that two individuals share on average.
\end{description}
\begin{description}
\item[effective ancestor number] The number of equally-contributing ancestors, not necessarily founders, needed to produce a population with the heterozygosity of the studied population \cite{ref352}.
\end{description}
\begin{description}
\item[effective founder number] The number of equally-contributing founders needed to produce a population with the heterozygosity of the studied population \cite{ref640}.
\end{description}
\begin{description}
\item[effective population size] The effective population size is the size of an ideal population that would lose heterozygosity at a rate equal to that of the studied population \cite{ref91}.
\end{description}
\begin{description}
\item[founder] An animal with unknown parents that is assumed to be unrelated to all other founders.
\end{description}
\begin{description}
\item[internal report] A \PyPedal() report that is intended for use by other \PyPedal() procedures, such as plotting
routines, and not for printing.
\end{description}
\begin{description}
\item[numerator relationship matrix] Matrix of additive genetic covariances among the animals in a population.
\end{description}
\begin{description}
\item[pedigree] A \PYPEDAL{} pedigree consists of a Python list containing instances of \PYPEDAL{} NewAnimal{} objects.
\end{description}
\begin{description}
\item[renumbering] Many calculations require that the animals in a pedigree be ordered from oldest to youngest, with sires and dams preceding offspring, and renumbered  starting with 1.  This is a computational necessity, and results in an animal's ID (\texttt{animalID}) being changed to reflect that animal's order in the pedigree.  All animals have their original IDs stored in their \texttt{originalName} attribute.
\end{description}