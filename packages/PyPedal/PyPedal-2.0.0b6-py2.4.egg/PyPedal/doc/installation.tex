\chapter{Installing PyPedal}
\label{cha:installation}
\index{installation}
\begin{quote}
This chapter explains how to install and test \PyPedal{} under Posix-type operating systems and Microsoft Windows.
\end{quote}
\section{Overview of installation}
\label{sec:installation-overview}
Before we can begin the tutorial, you need install and test Python, Numarray and some other Python extensions, and \PyPedal{} itself. The extensions that you need to install in order to use all of the features of \PyPedal{} are listed in Table \ref{tbl:extensions}.  Note that some extensions need to be installed before others: \module{Numarray} should be installed first, SQLite must be installed before \module{pysqlite}, and \module{pyparsing} and Graphviz must be installed before \module{pydot}.

If you do not install one or more optional modules you will still be able to use \PyPedal{}, although some features may not be available to you.  Details on installing the extensions listed above can be found on their respective websites.  All of these extensions are available for Unix-type operating systems (e.g. Linux, Mac OS X) as well as for Microsoft Windows; most sites also provide binary installers for Windows.  Python extensions can usually be installed by unzipping/untaring the archives, entering the folder, and issuing the command \samp{python setup.py install} as a root/administrative user.
\begin{center}
    \begin{table}
        \caption{Third-party extensions used by PyPedal.}
        \label{tbl:extensions}
        \centerline{
        \begin{tabular}{llp{4in}}
            \hline
            Extension & Function & URL \\
            \hline
            elementtree & Lightweight XML processing & \url{http://effbot.org/zone/element-index.htm}\\
            Graphviz & Draw directed graphs & \url{http://www.research.att.com/sw/tools/graphviz/}\\
            matplotlib & Plotting, matrix visualization & \url{http://matplotlib.sourceforge.net/}\\
            NetworkX & Network analysis & \url{https://networkx.lanl.gov/}\\
            Numarray & Array manipulation & \url{http://www.stsci.edu/resources/software_hardware/numarray}\\
            PIL & Image processing & \url{http://effbot.org/zone/pil-index.htm}\\
            pydot & Interface to Graphviz & \url{http://dkbza.org/pydot.html}\\
            pyparsing & Text parsing & \url{http://pyparsing.sourceforge.net/}\\
            pysqlite & Interface to SQLite & \url{http://initd.org/tracker/pysqlite}\\
            PythonDoc & Generate API documentation & \url{http://effbot.org/zone/pythondoc.htm}\\
            ReportLab & Generate PDF documents & \url{http://www.reportlab.org/}\\
            SQLite & Lightweight SQL database & \url{http://www.sqlite.org/}\\
            testoob & Advanced unit testing & \url{http://testoob.sourceforge.net/}\\
            \hline
        \end{tabular}}
    \end{table}
\end{center}
\section{Testing the Python installation}
The first step is to install Python 2.4 (or later) if you haven't already done so. Python is available at
\url{http://sourceforge.net/projects/python/}.  Click on the link corresponding to your platform, and follow the instructions
presented there. Python can usually be started by typing \samp{python} at the shell (Posix) or double-clicking on the Python interpreter (Windows).  When you start Python you should see a message such as:
\begin{verbatim}
Python 2.4 (#1, Feb 25 2005, 12:30:11)
[GCC 3.3.3] on linux2
Type "help", "copyright", "credits" or "license" for more information.
\end{verbatim}
If you have problems getting Python to work, contact your local support person or e-mail  \ulink{python-help@python.org}{mailto:python-help@python.org} for help.
\section{Installing PyPedal}
\label{sec:installing-pypedal}
In order to get \PyPedal{}, visit the official website at \url{http://pypedal.sourceforge.net/}.  Click on the "Sourceforge Page" link, click on the "Download PyPedal" button, and select the latest file release. Files whose names end in ".tar.gz" are source code releases. The other files are binaries for a given platform (if any are available).

The CVS repository on the Sourceforge site is not in synch with the development tree; to get the latest version you should download the source code release.
\subsection{Installing on Unix, Linux, and Mac OSX}
\label{sec:installing-unix}
\index{installation!installation on Linux}
The source distribution should be uncompressed and unpacked as follows (for example):
\begin{verbatim}
gunzip pypedal-2.0.0a20.tar.gz
tar xf pypedal-2.0.0a20.tar.gz
\end{verbatim}
Follow the instructions in the top-level directory for compilation and installation. Installation is usually as simple as:
\begin{verbatim}
python setup.py install
\end{verbatim}
\paragraph*{Important Tip} \label{sec:tip:from-pypedal-import} Just like all Python modules and packages, the \PyPedal{} module can be invoked using either the \samp{import PyPedal} form, or the \samp{from PyPedal import ...} form.  All of the code samples will assume that they have been preceded by statements such as:
\begin{verbatim}
>>> from PyPedal import <module-name>
\end{verbatim}
A complete list of modules is provided in Chapter \ref{cha:api}.
\subsection{Installing on Windows}
\label{sec:installing-windows}
\index{installation!installation on Windows}
To install \PyPedal{}, you need to be logged into an account with Administrator privileges.  As a general rule, always remove any old version of \PyPedal{} before installing the next version.

Please note that we have lightly tested \PyPedal{} on Windows XP, but cannot guarantee that it runs without problems on Win-32 platforms!  \PyPedal{} should install and run properly on Win-32 as long as the dependencies mentioned above are satisfied.

\index{installation!installation on Windows!environment variables}
In order to get your installation working correctly you will need to set some environment variables.  Under Windows XP you access those variables by right-clicking on the \emph{My Computer} icon on your desktop, selecting \emph{Properties}, selecting the \emph{Advanced} tab, and clicking the \emph{Environment Variables} button.  First, add \texttt{;C:\textbackslash{}Python24} to the \envvar{PATH} by selecting it in the \emph{User Variables} list and clicking \emph{Edit}.  Next, create a \envvar{PYTHONPATH} environment variable by clicking the \texttt{New} button under \texttt{User Variables}, entering the path to the \PyPedal{} directory in the \texttt{Variable value} field.

\index{installation!installation on Windows!SQLite}
The documentation for SQLite for Windows is kind of vague.  I got it to work by downloading the files \file{sqlite-3_2_7.zip} and \file{sqlitedll-3_2_7.zip} and extracting their contents into \texttt{C:\textbackslash{}Windows}.  Your mileage may vary.
\subsubsection{Installation from source}
\label{sec:installation-from-source}
\index{installation!installation from source}
\begin{enumerate}
\item Unpack the distribution: (NOTE: You may have to download an "unzipping" utility)
\begin{verbatim}
C:\> unzip PyPedal.zip
C:\> cd PyPedal
\end{verbatim}
\item Build it using the distutils defaults:
\begin{verbatim}
C:\PyPedal> python setup.py install
\end{verbatim}
This installs \PyPedal{} in \texttt{C:\textbackslash{}python24\textbackslash{}site-packages}.
\end{enumerate}
\subsubsection{Installation from self-installing executable}
\label{sec:installation-self-installing}
\index{installation!self-installing executable}
\begin{enumerate}
\item Click on the executable's icon to run the installer.
\item Click "next" several times.  I have not experimented with customizing the installation directory and don't recommend changing any of the installation defaults.  If you do, and have problems, please let me know.
\item Assuming everything else goes smoothly, click "finish".
\end{enumerate}
\subsubsection{Installation on Cygwin}
\label{sec:installation-cygwin}
\index{installation!installation on Cygwin}
No information on installing \PyPedal{} on Cygwin is available.  If you manage to get it working, please let me know.
\section{Testing the PyPedal Installation}
\label{sec:installation-testing}
\index{installation!testing the installation}
To find out if you have correctly installed \PyPedal{}, type \samp{import PyPedal} at the Python prompt. You'll see one of two
behaviors (throughout this document user input and Python interpreter output will be emphasized
as shown in the block below):
\begin{verbatim}
>>> import PyPedal
Traceback (innermost last):
File "<stdin>", line 1, in ?
ImportError: No module named PyPedal
\end{verbatim}
indicating that you don't have \PyPedal{} installed, or:
\begin{verbatim}
>>> import PyPedal
>>> PyPedal.__version__.version
'2.0.0b4'
\end{verbatim}
indicating that \PyPedal{} is installed.