\documentclass[10pt]{article}
\usepackage{fullpage, graphicx, url}
\setlength{\parskip}{1ex}
\setlength{\parindent}{0ex}
\title{The pyp\_db Module}
\begin{document}
\section*{The pyp\_db Module}


 pyp\_db contains a set of procedures for ...
\subsection*{Module Contents}
\begin{description}
\item[\textbf{createPedigreeDatabase(dbname='pypedal')}
 ⇒ integer [\#]]

 createPedigreeDatabase() creates a new database in SQLite.
\begin{description}
\item[\emph{dbname}
] The name of the database to create.
\item[Returns:] A 1 on successful database creation, a 0 otherwise.

\end{description}
\\ 

\item[\textbf{createPedigreeTable(curs, tablename='example')}
 ⇒ integer [\#]]

 createPedigreeDatabase() creates a new pedigree table in a SQLite database.
\begin{description}
\item[\emph{tablename}
] The name of the table to create.
\item[Returns:] A 1 on successful table creation, a 0 otherwise.

\end{description}
\\ 

\item[\textbf{databaseQuery(sql, curs=0, dbname='pypedal')}
 ⇒ string [\#]]

 databaseQuery() executes an SQLite query. This is a wrapper function used by the reporting functions that need to fetch data from SQLite. I wrote it so that any changes that need to be made in the way PyPedal talks to SQLite will only need to be changed in one place.
\begin{description}
\item[\emph{sql}
] A string containing an SQL query.
\item[\emph{\_curs}
] An [optional] SQLite cursor.
\item[\emph{dbname}
] The database into which the pedigree will be loaded.
\item[Returns:] The results of the query, or 0 if no resultset.

\end{description}
\\ 

\item[\textbf{getCursor(dbname='pypedal')}
 ⇒ cursor [\#]]

 getCursor() creates a database connection and returns a cursor on success or a 0 on failure. It isvery useful for non-trivial queries because it creates SQLite aggrefates before returning the cursor. The reporting routines in pyp\_reports make heavy use of getCursor().
\begin{description}
\item[\emph{dbname}
] The database into which the pedigree will be loaded.
\item[Returns:] An SQLite cursor if the database exists, a 0 otherwise.

\end{description}
\\ 

\item[\textbf{loadPedigreeTable(pedobj)}
 ⇒ integer [\#]]

 loadPedigreeDatabase() takes a PyPedal pedigree object and loads the animal records in that pedigree into an SQLite table.
\begin{description}
\item[\emph{pedobj}
] A PyPedal pedigree object.
\item[\emph{dbname}
] The database into which the pedigree will be loaded.
\item[\emph{tablename}
] The table into which the pedigree will be loaded.
\item[Returns:] A 1 on successful table load, a 0 otherwise.

\end{description}
\\ 

\item[\textbf{PypMean()}
 (class) [\#]]

 PypMean is a user-defined aggregate for SQLite for returning means from queries.


 For more information about this class, see \emph{The PypMean Class}
.

\item[\textbf{PypSSD()}
 (class) [\#]]

 PypSSD is a user-defined aggregate for SQLite for returning sample standard deviations from queries.


 For more information about this class, see \emph{The PypSSD Class}
.

\item[\textbf{PypSum()}
 (class) [\#]]

 PypSum is a user-defined aggregate for SQLite for returning sums from queries.


 For more information about this class, see \emph{The PypSum Class}
.

\item[\textbf{PypSVar()}
 (class) [\#]]

 PypSVar is a user-defined aggregate for SQLite for returning sample variances from queries.


 For more information about this class, see \emph{The PypSVar Class}
.

\item[\textbf{tableCountRows(dbname='pypedal', tablename='example')}
 ⇒ integer [\#]]

 tableCountRows() returns the number of rows in a table.
\begin{description}
\item[\emph{dbname}
] The database into which the pedigree will be loaded.
\item[\emph{tablename}
] The table into which the pedigree will be loaded.
\item[Returns:] The number of rows in the table 1 or 0.

\end{description}
\\ 

\item[\textbf{tableDropRows(dbname='pypedal', tablename='example')}
 ⇒ integer [\#]]

 tableDropRows() drops all of the data from an existing table.
\begin{description}
\item[\emph{dbname}
] The database from which data will be dropped.
\item[\emph{tablename}
] The table from which data will be dropped.
\item[Returns:] A 1 if the data were dropped, a 0 otherwise.

\end{description}
\\ 

\item[\textbf{tableDropTable(dbname='pypedal', tablename='example')}
 ⇒ integer [\#]]

 tableDropTable() drops a table from the database.
\begin{description}
\item[\emph{dbname}
] The database from which the table will be dropped.
\item[\emph{tablename}
] The table which will be dropped.
\item[Returns:] 1

\end{description}
\\ 

\item[\textbf{tableExists(dbname='pypedal', tablename='example')}
 ⇒ integer [\#]]

 tableExists() queries the sqlite\_master view in an SQLite database to determine whether or not a table exists.
\begin{description}
\item[\emph{dbname}
] The database into which the pedigree will be loaded.
\item[\emph{tablename}
] The table into which the pedigree will be loaded.
\item[Returns:] A 1 if the table exists, a 0 otherwise.

\end{description}
\\ 


\end{description}
\subsection*{The PypMean Class}
\begin{description}
\item[\textbf{PypMean()}
 (class) [\#]]

 PypMean is a user-defined aggregate for SQLite for returning means from queries.


\end{description}
\subsection*{The PypSSD Class}
\begin{description}
\item[\textbf{PypSSD()}
 (class) [\#]]

 PypSSD is a user-defined aggregate for SQLite for returning sample standard deviations from queries.


\end{description}
\subsection*{The PypSum Class}
\begin{description}
\item[\textbf{PypSum()}
 (class) [\#]]

 PypSum is a user-defined aggregate for SQLite for returning sums from queries.


\end{description}
\subsection*{The PypSVar Class}
\begin{description}
\item[\textbf{PypSVar()}
 (class) [\#]]

 PypSVar is a user-defined aggregate for SQLite for returning sample variances from queries.


\end{description}

\end{document}
