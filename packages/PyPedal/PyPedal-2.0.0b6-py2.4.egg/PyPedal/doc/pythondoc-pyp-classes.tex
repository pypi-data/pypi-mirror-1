

 pyp\_classes contains two base classes that are used by PyPedal, the Animal() class and the Pedigree() class. What most PyPedal routines recognize as a pedigree is actually just a Python list of Animal() objects. An instance of a Pedigree() object is a collection of METADATA about a list of Animals(). I know that this is confusing, and it is going to change by the time that PyPedal 2.0.0 final is released.
\subsection*{Module Contents}
\begin{description}
\item[\textbf{Animal(animalID, sireID, damID, gen='0', by=1900, sex='u', fa=0., name='u', alleles=['', ''], breed='u', age=-999, alive=-999)}
 (class) [\#]]

 The Animal() class is holds animals records read from a pedigree file.


 For more information about this class, see \emph{The Animal Class}
.

\item[\textbf{Pedigree(myped, inputfile, name, pedcode='asd', reord=0, renum=0, debug=0)}
 (class) [\#]]

 The Pedigree() class stores metadata about pedigrees.


 For more information about this class, see \emph{The Pedigree Class}
.


\end{description}
\subsection*{The Animal Class}
\begin{description}
\item[\textbf{Animal(animalID, sireID, damID, gen='0', by=1900, sex='u', fa=0., name='u', alleles=['', ''], breed='u', age=-999, alive=-999)}
 (class) [\#]]

 The Animal() class is holds animals records read from a pedigree file.

\item[\textbf{\_\_init\_\_(animalID, sireID, damID, gen='0', by=1900, sex='u', fa=0., name='u', alleles=['', ''], breed='u', age=-999, alive=-999)}
 ⇒ object [\#]]

 \_\_init\_\_() initializes an Animal() object.
\begin{description}
\item[\emph{self}
] Reference to the current Animal() object
\item[\emph{animalID}
] Animal ID number
\item[\emph{sireID}
] Sire ID number
\item[\emph{damID}
] Dam ID number
\item[\emph{gen}
] Generation to which the animal belongs
\item[\emph{by}
] Birthyear of the animal
\item[\emph{sex}
] Sex of the animal (m|f|u)
\item[\emph{fa}
] Coefficient of inbreeding of the animal
\item[\emph{name}
] Name of animal
\item[\emph{alleles}
] A two-element array of strings, which represent allelotypes.
\item[\emph{breed}
] Breed of animal
\item[\emph{age}
] Age of animal
\item[\emph{alive}
] Status of animal (alive or dead)
\item[Returns:] An instance of an Animal() object populated with data

\end{description}
\\ 

\item[\textbf{pad\_id()}
 ⇒ integer [\#]]

 pad\_id() takes an Animal ID, pads it to fifteen digits, and prepends the birthyear (or 1950 if the birth year is unknown). The order of elements is: birthyear, animalID, count of zeros, zeros.
\begin{description}
\item[\emph{self}
] Reference to the current Animal() object
\item[Returns:] A padded ID number that is supposed to be unique across animals

\end{description}
\\ 

\item[\textbf{printme()}
 [\#]]

 printme() prints a summary of the data stored in the Animal() object.
\begin{description}
\item[\emph{self}
] Reference to the current Animal() object

\end{description}
\\ 

\item[\textbf{stringme()}
 [\#]]

 stringme() returns a summary of the data stored in the Animal() object as a string.
\begin{description}
\item[\emph{self}
] Reference to the current Animal() object

\end{description}
\\ 

\item[\textbf{trap()}
 [\#]]

 trap() checks for common errors in Animal() objects
\begin{description}
\item[\emph{self}
] Reference to the current Animal() object

\end{description}
\\ 


\end{description}
\subsection*{The Pedigree Class}
\begin{description}
\item[\textbf{Pedigree(myped, inputfile, name, pedcode='asd', reord=0, renum=0, debug=0)}
 (class) [\#]]

 The Pedigree() class stores metadata about pedigrees. Hopefully this will help improve performance in some procedures, as well as provide some useful summary data.

\item[\textbf{\_\_init\_\_(myped, inputfile, name, pedcode='asd', reord=0, renum=0, debug=0)}
 ⇒ object [\#]]

 \_\_init\_\_() initializes a Pedigree metata object.
\begin{description}
\item[\emph{self}
] Reference to the current Pedigree() object
\item[\emph{myped}
] A PyPedal pedigree
\item[\emph{inputfile}
] The name of the file from which the pedigree was loaded
\item[\emph{name}
] The name assigned to the PyPedal pedigree
\item[\emph{pedcode}
] The format code for the PyPedal pedigree
\item[\emph{reord}
] Flag indicating whether or not the pedigree is reordered (0|1)
\item[\emph{renum}
] Flag indicating whether or not the pedigree is renumbered (0|1)
\item[Returns:] An instance of a Pedigree() object populated with data

\end{description}
\\ 

\item[\textbf{fileme()}
 [\#]]

 fileme() writes the metada stored in the Pedigree() object to disc.
\begin{description}
\item[\emph{self}
] Reference to the current Pedigree() object

\end{description}
\\ 

\item[\textbf{nud()}
 ⇒ integer-and-list [\#]]

 nud() returns the number of unique dams in the pedigree along with a list of the dams
\begin{description}
\item[\emph{self}
] Reference to the current Pedigree() object
\item[Returns:] The number of unique dams in the pedigree and a list of those dams

\end{description}
\\ 

\item[\textbf{nuf()}
 ⇒ integer-and-list [\#]]

 nuf() returns the number of unique founders in the pedigree along with a list of the founders
\begin{description}
\item[\emph{self}
] Reference to the current Pedigree() object
\item[Returns:] The number of unique founders in the pedigree and a list of those founders

\end{description}
\\ 

\item[\textbf{nug()}
 ⇒ integer-and-list [\#]]

 nug() returns the number of unique generations in the pedigree along with a list of the generations
\begin{description}
\item[\emph{self}
] Reference to the current Pedigree() object
\item[Returns:] The number of unique generations in the pedigree and a list of those generations

\end{description}
\\ 

\item[\textbf{nus()}
 ⇒ integer-and-list [\#]]

 nus() returns the number of unique sires in the pedigree along with a list of the sires
\begin{description}
\item[\emph{self}
] Reference to the current Pedigree() object
\item[Returns:] The number of unique sires in the pedigree and a list of those sires

\end{description}
\\ 

\item[\textbf{nuy()}
 ⇒ integer-and-list [\#]]

 nuy() returns the number of unique birthyears in the pedigree along with a list of the birthyears
\begin{description}
\item[\emph{self}
] Reference to the current Pedigree() object
\item[Returns:] The number of unique birthyears in the pedigree and a list of those birthyears

\end{description}
\\ 

\item[\textbf{printme()}
 [\#]]

 printme() prints a summary of the metadata stored in the Pedigree() object.
\begin{description}
\item[\emph{self}
] Reference to the current Pedigree() object

\end{description}
\\ 

\item[\textbf{stringme()}
 [\#]]

 stringme() returns a summary of the metadata stored in the pedigree as a string.
\begin{description}
\item[\emph{self}
] Reference to the current Pedigree() object

\end{description}
\\ 


\end{description}

