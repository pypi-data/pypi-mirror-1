\chapter{API}
\label{cha:api}
\begin{quote}
This chapter provides an overview of the \PYPEDAL{} Application Programming Interface (API).  More simply, it is a reference to the various classes, methods, and procedures that make up the \PYPEDAL{} module.
\end{quote}
\section{Some Background}
A complete list of the core \PyPedal() modules is presented in Table \ref{tbl:pypedal-modules}.  Using the \PyPedal{} API is quite simple.  The following discussion assumes that you have imported each of the Python modules using, e.g.,  \samp{from PyPedal import pyp_utils} rather than \samp{from PyPedal.pyp_utils import *}.  The latter is poor style and can result in namespace pollution; this is not known to be a problem with \PyPedal{}, but I offer no guarantees that this will remain the case.  In order to access a function in the \module{pyp_utils} module, such as \function{pyp_nice_time()}, you use a dotted notation with a '.' separating the module name and the function name.  For example:
\begin{verbatim}
[jcole@aipl440 jcole]$ python
Python 2.4 (#1, Feb 25 2005, 12:30:11)
[GCC 3.3.3] on linux2
Type "help", "copyright", "credits" or "license" for more information.
>>> from PyPedal import pyp_utils
>>> pyp_utils.pyp_nice_time()
'Mon Aug 15 16:27:38 2005'
\end{verbatim}
\begin{center}
    \begin{table}
        \caption{PyPedal modules.}
        \label{tbl:pypedal-modules}
        \centerline{
        \begin{tabular}{llp{4in}}
            \hline
            Module Name & Description \\
            \hline
            pyp\_db & \parbox[t]{4in}{Working with SQLite relational databases: create databases, add/drop tables, load PyPedal pedigrees into tables.} \\
            pyp\_demog & \parbox[t]{4in}{Generate demographic reports, age distributions, for the pedigreed population.} \\
            pyp\_graphics & \parbox[t]{4in}{Visualize pedigrees and numerator relationship matrices (NRM).} \\
            pyp\_io & \parbox[t]{4in}{Save and load NRM and inverses of NRM; write pedigrees to formats used by other packages.} \\
            pyp\_metrics & \parbox[t]{4in}{Compute metrics on pedigrees: effective founder and ancestor numbers, effective number of founder genomes, pedigree completeness.  Tools for identifying related animals, calculating coefficients of inbreeding and relationship, and computing expected offspring inbreeding from matings.} \\
            pyp\_network & \parbox[t]{4in}{Convert pedigrees directed graphs.} \\
            pyp\_newclasses & \parbox[t]{4in}{Pedigree, animal, and metadata classes used by PyPedal.} \\
            pyp\_nrm & \parbox[t]{4in}{Creating, decompose, and inverting NRM, and recurse through pedigrees.} \\
            pyp\_reports & \parbox[t]{4in}{Create reports from pedigree database (loaded in pyp_db).} \\
            pyp\_template & \parbox[t]{4in}{Template for developers to use when adding new features to \PyPedal{} (Chapter \ref{cha:newfeatures}).} \\
            pyp\_utils & \parbox[t]{4in}{Load, reorder and renumber pedigrees; set flags in individual animal records; string and date-time tools.} \\
            \hline
        \end{tabular}}
    \end{table}
\end{center}
\section{pyp\_db}
pyp\_db\index[func]{pyp_db} contains a set of procedures for ...

\subsection*{Module Contents}

\begin{description}
\item[\textbf{createPedigreeDatabase(dbname='pypedal')} $\Rightarrow$ integer]\index[func]{pyp_db!createPedigreeDatabase()}
createPedigreeDatabase() creates a new database in SQLite.
\begin{description}
\item[\emph{dbname}] The name of the database to create.
\item[Returns:] A 1 on successful database creation, a 0 otherwise.
\end{description}

\item[\textbf{createPedigreeTable(curs, tablename='example')} $\Rightarrow$ integer]\index[func]{pyp_db!createPedigreeTable()}
createPedigreeDatabase() creates a new pedigree table in a SQLite database.
\begin{description}
\item[\emph{tablename}] The name of the table to create.
\item[Returns:] A 1 on successful table creation, a 0 otherwise.
\end{description}

\item[\textbf{databaseQuery(sql, curs=0, dbname='pypedal')} $\Rightarrow$ string]\index[func]{pyp_db!databaseQuery()}
databaseQuery() executes an SQLite query. This is a wrapper function used by the reporting functions that need to fetch data from SQLite. I wrote it so that any changes that need to be made in the way PyPedal talks to SQLite will only need to be changed in one place.
\begin{description}
\item[\emph{sql}] A string containing an SQL query.
\item[\emph{\_curs}] An [optional] SQLite cursor.
\item[\emph{dbname}] The database into which the pedigree will be loaded.
\item[Returns:] The results of the query, or 0 if no resultset.
\end{description}

\item[\textbf{getCursor(dbname='pypedal')} $\Rightarrow$ cursor]\index[func]{pyp_db!getCursor()}
getCursor() creates a database connection and returns a cursor on success or a 0 on failure. It isvery useful for non-trivial queries because it creates SQLite aggrefates before returning the cursor. The reporting routines in pyp\_reports make heavy use of getCursor().
\begin{description}
\item[\emph{dbname}] The database into which the pedigree will be loaded.
\item[Returns:] An SQLite cursor if the database exists, a 0 otherwise.
\end{description}

\item[\textbf{loadPedigreeTable(pedobj)} $\Rightarrow$ integer]\index[func]{pyp_db!loadPedigreeTable()}
loadPedigreeDatabase() takes a PyPedal pedigree object and loads the animal records in that pedigree into an SQLite table.
\begin{description}
\item[\emph{pedobj}] A PyPedal pedigree object.
\item[\emph{dbname}] The database into which the pedigree will be loaded.
\item[\emph{tablename}] The table into which the pedigree will be loaded.
\item[Returns:] A 1 on successful table load, a 0 otherwise.
\end{description}

\item[\textbf{tableCountRows(dbname='pypedal', tablename='example')} $\Rightarrow$ integer]\index[func]{pyp_db!tableCountRows()}
tableCountRows() returns the number of rows in a table.
\begin{description}
\item[\emph{dbname}] The database into which the pedigree will be loaded.
\item[\emph{tablename}] The table into which the pedigree will be loaded.
\item[Returns:] The number of rows in the table 1 or 0.
\end{description}

\item[\textbf{tableDropRows(dbname='pypedal', tablename='example')} $\Rightarrow$ integer]\index[func]{pyp_db!tableDropRows()}
tableDropRows() drops all of the data from an existing table.
\begin{description}
\item[\emph{dbname}] The database into which the pedigree will be loaded.
\item[\emph{tablename}] The table into which the pedigree will be loaded.
\item[Returns:] A 1 if the data were dropped, a 0 otherwise.
\end{description}

\item[\textbf{tableExists(dbname='pypedal', tablename='example')} $\Rightarrow$ integer]\index[func]{pyp_db!tableExists()}
tableExists() queries the sqlite\_master view in an SQLite database to determine whether or not a table exists.
\begin{description}
\item[\emph{dbname}] The database into which the pedigree will be loaded.
\item[\emph{tablename}] The table into which the pedigree will be loaded.
\item[Returns:] A 1 if the table exists, a 0 otherwise.
\end{description}
\end{description}

\subsection*{The PypMean Class}
\begin{description}
\item[\textbf{PypMean()} (class)]\index[func]{pyp_db!PypMean}
PypMean is a user-defined aggregate for SQLite for returning means from queries.
\end{description}

\subsection*{The PypSSD Class}\index[func]{pyp_db!PypSSD}
\begin{description}
\item[\textbf{PypSSD()} (class)]
PypSSD is a user-defined aggregate for SQLite for returning sample standard deviations from queries.
\end{description}

\subsection*{The PypSum Class}
\begin{description}
\item[\textbf{PypSum()} (class)]\index[func]{pyp_db!PypSum}
PypSum is a user-defined aggregate for SQLite for returning sums from queries.
\end{description}

\subsection*{The PypSVar Class}
\begin{description}
\item[\textbf{PypSVar()} (class)]\index[func]{pyp_db!PypSVar}
PypSVar is a user-defined aggregate for SQLite for returning sample variances from queries.
\end{description}
\section{pyp\_demog}
pyp\_demog\index[func]{pyp_demog} contains a set of procedures for demographic calculations on the population describe in a pedigree.
\subsection*{Module Contents}

\begin{description}
\item[\textbf{age\_distribution(pedobj, sex=1)} $\Rightarrow$ None]\index[func]{pyp_demog!age\_distribution()}
age\_distribution() computes histograms of the age distribution of males and females in the population. You can also stratify by sex to get individual histograms.
\begin{description}
\item[\emph{myped}] An instance of a PyPedal NewPedigree object.
\item[\emph{sex}] A flag which determines whether or not to stratify by sex.
\end{description}

\item[\textbf{founders\_by\_year(pedobj)} $\Rightarrow$ dictionary]\index[func]{pyp_demog!founders\_by\_year()}
founders\_by\_year() returns a dictionary containing the number of founders in each birthyear.
\begin{description}
\item[\emph{pedobj}] A PyPedal pedigree object.
\item[Returns:] dict A dictionary containing entries for each sex/gender code defined in the global SEX\_CODE\_MAP.
\end{description}

\item[\textbf{set\_age\_units(units='year')} $\Rightarrow$ None]\index[func]{pyp_demog!set\_age\_units()}
set\_age\_units() defines a global variable, BASE\_DEMOGRAPHIC\_UNIT.
\begin{description}
\item[\emph{units}] The base unit for age computations ('year'|'month'|'day').
\item[Returns:] None
\end{description}

\item[\textbf{set\_base\_year(year=1950)} $\Rightarrow$ None]\index[func]{pyp_demog!age\_distribution()}
set\_base\_year() defines a global variable, BASE\_DEMOGRAPHIC\_YEAR.
\begin{description}
\item[\emph{year}] The year to be used as a base for computing ages.
\item[Returns:] None
\end{description}

\item[\textbf{sex\_ratio(pedobj)} $\Rightarrow$ dictionary]\index[func]{pyp_demog!sex\_ratio()}
sex\_ratio() returns a dictionary containing the proportion of males and females in the population.
\begin{description}
\item[\emph{myped}] An instance of a PyPedal NewPedigree object.
\item[Returns:] dict A dictionary containing entries for each sex/gender code defined in the global SEX\_CODE\_MAP.
\end{description}

\end{description}
\section{pyp\_graphics}
pyp\_graphics\index[func]{pyp_graphics} contains routines for working with graphics in PyPedal, such as creating directed graphs from pedigrees using PyDot and visualizing relationship matrices using Rick Muller's spy and pcolor routines (\url{http://aspn.activestate.com/ASPN/Cookbook/Python/}). The Python Imaging Library (\url{http://www.pythonware.com/products/pil/}), matplotlib (\url{http://matplotlib.sourceforge.net/}), Graphviz (\url{http://www.graphviz.org/}), and pydot (\url{http://dkbza.org/pydot.html}) are required by one or more routines in this module. They ARE NOT distributed with PyPedal and must be installed by the end-user! Note that the matplotlib functionality in PyPedal requires only the Agg backend, which means that you do not have to install GTK/PyGTK or WxWidgets/PyWxWidgets just to use PyPedal. Please consult the sites above for licensing and installation information.

\subsection*{Module Contents}

\begin{description}
\item[\textbf{draw\_pedigree(pedobj, gfilename='pedigree', gtitle='My\_Pedigree', gformat='jpg', gsize='f', gdot='1')} $\Rightarrow$ integer]\index[func]{pyp_graphics!draw\_pedigree()}
draw\_pedigree() uses the pydot bindings to the graphviz library -- if they are available on your system -- to produce a directed graph of your pedigree with paths of inheritance as edges and animals as nodes. If there is more than one generation in the pedigree as determind by the ``gen'' attributes of the anumals in the pedigree, draw\_pedigree() will use subgraphs to try and group animals in the same generation together in the drawing.
\begin{description}
\item[\emph{pedobj}] A PyPedal pedigree object.
\item[\emph{gfilename}] The name of the file to which the pedigree should be drawn
\item[\emph{gtitle}] The title of the graph.
\item[\emph{gsize}] The size of the graph: 'f': full-size, 'l': letter-sized page.
\item[\emph{gdot}] Whether or not to write the dot code for the pedigree graph to a file (can produce large files).
\item[Returns:] A 1 for success and a 0 for failure.
\end{description}

\item[\textbf{pcolor\_matrix\_pylab(A, fname='pcolor\_matrix\_matplotlib')} $\Rightarrow$ lists]\index[func]{pyp_graphics!pcolor\_matrix\_pylab()}
pcolor\_matrix\_pylab() implements a matlab-like 'pcolor' function to display the large elements of a matrix in pseudocolor using the Python Imaging Library.
\begin{description}
\item[\emph{A}] Input Numpy matrix (such as a numerator relationship matrix).
\item[\emph{fname}] Output filename to which to dump the graphics (default 'tmp.png')
\item[\emph{do\_outline}] Whether or not to print an outline around the block (default 0)
\item[\emph{height}] The height of the image (default 300)
\item[\emph{width}] The width of the image (default 300)
\item[Returns:] A list of Animal() objects; a pedigree metadata object.
\end{description}

\item[\textbf{plot\_founders\_by\_year(pedobj, gfilename='founders\_by\_year', gtitle='Founders by Birthyear')} $\Rightarrow$ integer]\index[func]{pyp_graphics!plot\_founders\_by\_year()}
founders\_by\_year() uses matplotlib -- if available on your system -- to produce a bar graph of the number (count) of founders in each birthyear.
\begin{description}
\item[\emph{pedobj}] A PyPedal pedigree object.
\item[\emph{gfilename}] The name of the file to which the pedigree should be drawn
\item[\emph{gtitle}] The title of the graph.
\item[Returns:] A 1 for success and a 0 for failure.
\end{description}

\item[\textbf{plot\_founders\_pct\_by\_year(pedobj, gfilename='founders\_pct\_by\_year', gtitle='Founders by Birthyear')} $\Rightarrow$ integer]\index[func]{pyp_graphics!plot\_founders\_pct\_by\_year()}
founders\_pct\_by\_year() uses matplotlib -- if available on your system -- to produce a line graph of the frequency (percentage) of founders in each birthyear.
\begin{description}
\item[\emph{pedobj}] A PyPedal pedigree object.
\item[\emph{gfilename}] The name of the file to which the pedigree should be drawn
\item[\emph{gtitle}] The title of the graph.
\item[Returns:] A 1 for success and a 0 for failure.
\end{description}

\item[\textbf{rmuller\_get\_color(a, cmin, cmax)} $\Rightarrow$ integer]\index[func]{pyp_graphics!rmuller\_get\_color()}
rmuller\_get\_color() Converts a float value to one of a continuous range of colors using recipe 9.10 from the Python Cookbook.
\begin{description}
\item[\emph{a}] Float value to convert to a color.
\item[\emph{cmin}] Minimum value in array (?).
\item[\emph{cmax}] Maximum value in array (?).
\item[Returns:] An RGB triplet.
\end{description}

\item[\textbf{rmuller\_pcolor\_matrix\_pil(A, fname='tmp.png', do\_outline=0, height=300, width=300)} $\Rightarrow$ lists]\index[func]{pyp_graphics!rmuller\_pcolor\_matrix\_pil()}
rmuller\_pcolor\_matrix\_pil() implements a matlab-like 'pcolor' function to display the large elements of a matrix in pseudocolor using the Python Imaging Library.
\begin{description}
\item[\emph{A}] Input Numpy matrix (such as a numerator relationship matrix).
\item[\emph{fname}] Output filename to which to dump the graphics (default 'tmp.png')
\item[\emph{do\_outline}] Whether or not to print an outline around the block (default 0)
\item[\emph{height}] The height of the image (default 300)
\item[\emph{width}] The width of the image (default 300)
\item[Returns:] A list of Animal() objects; a pedigree metadata object.
\end{description}

\item[\textbf{rmuller\_spy\_matrix\_pil(A, fname='tmp.png', cutoff=0.1, do\_outline=0, height=300, width=300)} $\Rightarrow$ lists]\index[func]{pyp_graphics!rmuller\_spy\_matrix\_pil()}
rmuller\_spy\_matrix\_pil() implements a matlab-like 'spy' function to display the sparsity of a matrix using the Python Imaging Library.
\begin{description}
\item[\emph{A}] Input Numpy matrix (such as a numerator relationship matrix).
\item[\emph{fname}] Output filename to which to dump the graphics (default 'tmp.png')
\item[\emph{cutoff}] Threshold value for printing an element (default 0.1)
\item[\emph{do\_outline}] Whether or not to print an outline around the block (default 0)
\item[\emph{height}] The height of the image (default 300)
\item[\emph{width}] The width of the image (default 300)
\item[Returns:] A list of Animal() objects; a pedigree metadata object.
\end{description}

\item[\textbf{spy\_matrix\_pylab(A, fname='spy\_matrix\_matplotlib')} $\Rightarrow$ lists]\index[func]{pyp_graphics!spy\_matrix\_pylab()}
spy\_matrix\_pylab() implements a matlab-like 'pcolor' function to display the large elements of a matrix in pseudocolor using the Python Imaging Library.
\begin{description}
\item[\emph{A}] Input Numpy matrix (such as a numerator relationship matrix).
\item[\emph{fname}] Output filename to which to dump the graphics (default 'tmp.png')
\item[\emph{do\_outline}] Whether or not to print an outline around the block (default 0)
\item[\emph{height}] The height of the image (default 300)
\item[\emph{width}] The width of the image (default 300)
\item[Returns:] A list of Animal() objects; a pedigree metadata object.
\end{description}

\end{description}
\section{pyp\_io}
pyp\_io\index[func]{pyp_io} contains several procedures for writing structures to and reading them from disc (e.g. using pickle() to store and retrieve A and A-inverse). It also includes a set of functions used to render strings as HTML or plaintext for use in generating output files.

\subsection*{Module Contents}

\begin{description}
\item[\textbf{a\_inverse\_from\_file(inputfile)} $\Rightarrow$ matrix]\index[func]{pyp_io!a\_inverse\_from\_file()}
a\_inverse\_from\_file() uses the Python pickle system for persistent objects to read the inverse of a relationship matrix from a file.
\begin{description}
\item[\emph{inputfile}] The name of the input file.
\item[Returns:] The inverse of a numerator relationship matrix.
\end{description}

\item[\textbf{a\_inverse\_to\_file(pedobj, ainv='')}]\index[func]{pyp_io!a\_inverse\_to\_file()}
a\_inverse\_to\_file() uses the Python pickle system for persistent objects to write the inverse of a relationship matrix to a file.
\begin{description}
\item[\emph{pedobj}] A PyPedal pedigree object.
\item[\emph{filetag}] A descriptor prepended to output file names.
\end{description}

\item[\textbf{dissertation\_pedigree\_to\_file(pedobj)}]\index[func]{pyp_io!dissertation\_pedigree\_to\_file()}
dissertation\_pedigree\_to\_file() takes a pedigree in 'asdxfg' format and writes is to a file.
\begin{description}
\item[\emph{pedobj}] A PyPedal pedigree object.
\end{description}

\item[\textbf{dissertation\_pedigree\_to\_pedig\_format(pedobj)}]\index[func]{pyp_io!dissertation\_pedigree\_to\_pedig\_format()}
dissertation\_pedigree\_to\_pedig\_format() takes a pedigree in 'asdbxfg' format, formats it into the form used by Didier Boichard's 'pedig' suite of programs, and writes it to a file.
\begin{description}
\item[\emph{pedobj}] A PyPedal pedigree object.
\end{description}

\item[\textbf{dissertation\_pedigree\_to\_pedig\_format\_mask(pedobj)}]\index[func]{pyp_io!dissertation\_pedigree\_to\_pedig\_format\_mask()}
dissertation\_pedigree\_to\_pedig\_format\_mask() Takes a pedigree in 'asdbxfg' format, formats it into the form used by Didier Boichard's 'pedig' suite of programs, and writes it to a file. THIS FUNCTION MASKS THE GENERATION ID WITH A FAKE BIRTH YEAR AND WRITES THE FAKE BIRTH YEAR TO THE FILE INSTEAD OF THE TRUE BIRTH YEAR. THIS IS AN ATTEMPT TO FOOL PEDIG TO GET f\_e, f\_a et al. BY GENERATION.
\begin{description}
\item[\emph{pedobj}] A PyPedal pedigree object.
\end{description}

\item[\textbf{dissertation\_pedigree\_to\_pedig\_interest\_format(pedobj)}]\index[func]{pyp_io!dissertation\_pedigree\_to\_pedig\_interest\_format()}
dissertation\_pedigree\_to\_pedig\_interest\_format() takes a pedigree in 'asdbxfg' format, formats it into the form used by Didier Boichard's parente program for the studied individuals file.
\begin{description}
\item[\emph{pedobj}] A PyPedal pedigree object.
\end{description}

\item[\textbf{pickle\_pedigree(pedobj, filename='')} $\Rightarrow$ integer]\index[func]{pyp_io!pickle\_pedigree()}
pickle\_pedigree() pickles a pedigree.
\begin{description}
\item[\emph{pedobj}] An instance of a PyPedal pedigree object.
\item[\emph{filename}] The name of the file to which the pedigree object should be pickled (optional).
\item[Returns:] A 1 on success, a 0 otherwise.
\end{description}

\item[\textbf{pyp\_file\_footer(ofhandle, caller=''Unknown PyPedal routine'')} $\Rightarrow$ None]\index[func]{pyp_io!pyp\_file\_footer()}
pyp\_file\_footer()
\begin{description}
\item[\emph{ofhandle}] A Python file handle.
\item[\emph{caller}] A string indicating the name of the calling routine.
\item[Returns:] None
\end{description}

\item[\textbf{pyp\_file\_header(ofhandle, caller=''Unknown PyPedal routine'')} $\Rightarrow$ integer]
pyp\_file\_header()\index[func]{pyp_io!pyp\_file\_header()}
\begin{description}
\item[\emph{ofhandle}] A Python file handle.
\item[\emph{caller}] A string indicating the name of the calling routine.
\item[Returns:] None
\end{description}

\item[\textbf{renderTitle(title\_string, title\_level=''1'')} $\Rightarrow$ integer]\index[func]{pyp_io!renderTitle()}
renderTitle() ... Produced HTML output by default.

\item[\textbf{unpickle\_pedigree(filename='')} $\Rightarrow$ object]\index[func]{pyp_io!unpickle\_pedigree()}
unpickle\_pedigree() reads a pickled pedigree in from a file and returns the unpacked pedigree object.
\begin{description}
\item[\emph{filename}] The name of the pickle file.
\item[Returns:] An instance of a NewPedigree object on success, a 0 otherwise.
\end{description}

\end{description}
\section{pyp\_metrics}
pyp\_metrics\index[func]{pyp_metrics} contains a set of procedures for calculating metrics on PyPedal pedigree objects. These metrics include coefficients of inbreeding and relationship as well as effective founder number, effective population size, and effective ancestor number.

\subsection*{Module Contents}
\begin{description}
\item[\textbf{a\_coefficients(pedobj, a='', method='nrm')} $\Rightarrow$ dictionary]\index[func]{pyp_metrics!a_coefficients()}
a\_coefficients() writes population average coefficients of inbreeding and relationship to a file, as well as individual animal IDs and coefficients of inbreeding. Some pedigrees are too large for fast\_a\_matrix() or fast\_a\_matrix\_r() -- an array that large cannot be allocated due to memory restrictions -- and will result in a value of -999.9 for all outputs.
\begin{description}
\item[\emph{pedobj}] A PyPedal pedigree object.
\item[\emph{a}] A numerator relationship matrix (optional).
\item[\emph{method}] If no relationship matrix is passed, determines which procedure should be called to build one (nrm|frm).
\item[Returns:] A dictionary of non-zero individual inbreeding coefficients.
\end{description}

\item[\textbf{a\_effective\_ancestors\_definite(pedobj, a='', gen='')} $\Rightarrow$ float]\index[func]{pyp_metrics!a_effective_ancestors()}
a\_effective\_ancestors\_definite() uses the algorithm in Appendix B of \citeN{ref352} to compute the effective ancestor number for a myped pedigree. NOTE: One problem here is that if you pass a pedigree WITHOUT generations and error is not thrown. You simply end up wth a list of generations that contains the default value for Animal() objects, 0. Boichard's algorithm requires information about the generation of animals. If you do not provide an input pedigree with generations things may not work. By default the most recent generation -- the generation with the largest generation ID -- will be used as the reference population.
\begin{description}
\item[\emph{pedobj}] A PyPedal pedigree object.
\item[\emph{a}] A numerator relationship matrix (optional).
\item[\emph{gen}] Generation of interest.
\item[Returns:] The effective founder number.
\end{description}

\item[\textbf{a\_effective\_ancestors\_indefinite(pedobj, a='', gen='', n=25)} $\Rightarrow$ float]\index[func]{pyp_metrics!a_effective_ancestors_indefinite()}
a\_effective\_ancestors\_indefinite() uses the approach outlined on pages 9 and 10 of Boichard et al. \cite{ref352} to compute approximate upper and lower bounds for f\_a. This is much more tractable for large pedigrees than the exact computation provided in a\_effective\_ancestors\_definite(). NOTE: One problem here is that if you pass a pedigree WITHOUT generations and error is not thrown. You simply end up wth a list of generations that contains the default value for Animal() objects, 0. NOTE: If you pass a value of n that is greater than the actual number of ancestors in the pedigree then strange things happen. As a stop-gap, a\_effective\_ancestors\_indefinite() will detect that case and replace n with the number of founders - 1. Boichard's algorithm requires information about the GENERATION of animals. If you do not provide an input pedigree with generations things may not work. By default the most recent generation -- the generation with the largest generation ID -- will be used as the reference population.
\begin{description}
\item[\emph{pedobj}] A PyPedal pedigree object.
\item[\emph{a}] A numerator relationship matrix (optional).
\item[\emph{gen}] Generation of interest.
\item[Returns:] The effective founder number.
\end{description}

\item[\textbf{a\_effective\_founders\_boichard(pedobj, a='', gen='')} $\Rightarrow$ float]\index[func]{pyp_metrics!a_effective_founders_boichard()}
a\_effective\_founders\_boichard() uses the algorithm in Appendix A of \citeN{ref352} to compute the effective founder number for pedobj. Note that results from this function will not necessarily match those from a\_effective\_founders\_lacy(). Boichard's algorithm requires information about the GENERATION of animals. If you do not provide an input pedigree with generations things may not work. By default the most recent generation -- the generation with the largest generation ID -- will be used as the reference population.
\begin{description}
\item[\emph{pedobj}] A PyPedal pedigree object.
\item[\emph{a}] A numerator relationship matrix (optional).
\item[\emph{gen}] Generation of interest.
\item[Returns:] The effective founder number.
\end{description}

\item[\textbf{a\_effective\_founders\_lacy(pedobj, a='')} $\Rightarrow$ float]\index[func]{pyp_metrics!a_effective_founders_lacy()}
a\_effective\_founders\_lacy() calculates the number of effective founders in a pedigree using the exact method of \citeN{ref640}.
\begin{description}
\item[\emph{pedobj}] A PyPedal pedigree object.
\item[\emph{a}] A numerator relationship matrix (optional).
\item[Returns:] The effective founder number.
\end{description}

\item[\textbf{common\_ancestors(anim\_a, anim\_b, pedobj)} $\Rightarrow$ list]\index[func]{pyp_metrics!common_ancestors()}
common\_ancestors() returns a list of the ancestors that two animals share in common.
\begin{description}
\item[\emph{anim\_a}] The renumbered ID of the first animal, a.
\item[\emph{anim\_b}] The renumbered ID of the second animal, b.
\item[\emph{pedobj}] A PyPedal pedigree object.
\item[Returns:] A list of animals related to anim\_a AND anim\_b
\end{description}

\item[\textbf{descendants(anid, pedobj, \_desc)} $\Rightarrow$ list]\index[func]{pyp_metrics!descendants()}
descendants() uses pedigree metadata to walk a pedigree and return a list of all of the descendants of a given animal.
\begin{description}
\item[\emph{anid}] An animal ID
\item[\emph{pedobj}] A Python list of PyPedal Animal() objects.
\item[\emph{\_desc}] A Python dictionary of descendants of animal anid.
\item[Returns:] A list of descendants of anid.
\end{description}

\item[\textbf{effective\_founder\_genomes(pedobj, rounds=10)} $\Rightarrow$ float]\index[func]{pyp_metrics!effective_founder_genomes()}
effective\_founder\_genomes() simulates the random segregation of founder alleles through a pedigree after the method of \citeN{ref1719}. At present only two alleles are simulated for each founder. Summary statistics are computed on the most recent generation.
\begin{description}
\item[\emph{pedobj}] A PyPedal pedigree object.
\item[\emph{rounds}] The number of times to simulate segregation through the entire pedigree.
\item[Returns:] The effective number of founder genomes over based on 'rounds' gene-drop simulations.
\end{description}

\item[\textbf{effective\_founders\_lacy(pedobj)} $\Rightarrow$ float]\index[func]{pyp_metrics!effective_founders_lacy()}
effective\_founders\_lacy() calculates the number of effective founders in a pedigree using the exact method of  \citeN{ref640}. This version of the routine a\_effective\_founders\_lacy() is designed to work with larger pedigrees as it forms ``familywise'' relationship matrices rather than a ``populationwise'' relationship matrix.
\begin{description}
\item[\emph{pedobj}] A PyPedal pedigree object.
\item[Returns:] The effective founder number.
\end{description}

\item[\textbf{fast\_a\_coefficients(pedobj, a='', method='nrm', debug=0)} $\Rightarrow$ dictionary]\index[func]{pyp_metrics!fast_a_coefficients()}
a\_fast\_coefficients() writes population average coefficients of inbreeding and relationship to a file, as well as individual animal IDs and coefficients of inbreeding. It returns a list of non-zero individual CoI.
\begin{description}
\item[\emph{pedobj}] A PyPedal pedigree object.
\item[\emph{a}] A numerator relationship matrix (optional).
\item[\emph{method}] If no relationship matrix is passed, determines which procedure should be called to build one (nrm|frm).
\item[Returns:] A dictionary of non-zero individual inbreeding coefficients.
\end{description}

\item[\textbf{founder\_descendants(pedobj)} $\Rightarrow$ dictionary [\#]]\index[func]{pyp_metrics!founder_descendants()}
founder\_descendants() returns a dictionary containing a list of descendants of each founder in the pedigree.
\begin{description}
\item[\emph{pedojb}] An instance of a PyPedal NewPedigree object.
\end{description}

\item[\textbf{generation\_lengths(pedobj, units='y')} $\Rightarrow$ dictionary]\index[func]{pyp_metrics!generation_lengths()}
generation\_lengths() computes the average age of parents at the time of birth of their first offspring. This is implies that selection decisions are made at the time of birth of of the first offspring. Average ages are computed for each of four paths: sire-son, sire-daughter, dam-son, and dam-daughter. An overall mean is computed, as well. IT IS IMPORTANT to note that if you DO NOT provide birthyears in your pedigree file that the returned dictionary will contain only zeroes! This is because when no birthyer is provided a default value (1900) is assigned to all animals in the pedigree.
\begin{description}
\item[\emph{pedobj}] A PyPedal pedigree object.
\item[\emph{units}] A character indicating the units in which the generation lengths should be returned.
\item[Returns:] A dictionary containing the five average ages.
\end{description}

\item[\textbf{generation\_lengths\_all(pedobj, units='y')} $\Rightarrow$ dictionary]\index[func]{pyp_metrics!generation_lengths_all()}
generation\_lengths\_all() computes the average age of parents at the time of birth of their offspring. The computation is made using birth years for all known offspring of sires and dams, which implies discrete generations. Average ages are computed for each of four paths: sire-son, sire-daughter, dam-son, and dam-daughter. An overall mean is computed, as well. IT IS IMPORTANT to note that if you DO NOT provide birthyears in your pedigree file that the returned dictionary will contain only zeroes! This is because when no birthyear is provided a default value (1900) is assigned to all animals in the pedigree.
\begin{description}
\item[\emph{pedobj}] A PyPedal pedigree object.
\item[\emph{units}] A character indicating the units in which the generation lengths should be returned.
\item[Returns:] A dictionary containing the five average ages.
\end{description}

\item[\textbf{mating\_coi(anim\_a, anim\_b, pedobj)} $\Rightarrow$ float]\index[func]{pyp_metrics!mating_coi()}
mating\_coi() returns the coefficient of inbreeding of offspring of a mating between two animals, anim\_a and anim\_b.
\begin{description}
\item[\emph{anim\_a}] The renumbered ID of an animal, a.
\item[\emph{anim\_b}] The renumbered ID of an animal, b.
\item[\emph{pedobj}] A PyPedal pedigree object.
\item[Returns:] The coefficient of relationship of anim\_a and anim\_b
\end{description}

\item[\textbf{min\_max\_f(pedobj, a='', n=10)} $\Rightarrow$ list]\index[func]{pyp_metrics!min_max_f()}
min\_max\_f() takes a pedigree and returns a list of the individuals with the n largest and n smallest coefficients of inbreeding. Individuals with CoI of zero are not included.
\begin{description}
\item[\emph{pedobj}] A PyPedal pedigree object.
\item[\emph{a}] A numerator relationship matrix (optional).
\item[\emph{n}] An integer (optional, default is 10).
\item[Returns:] Lists of the individuals with the n largest and the n smallest CoI in the pedigree as (ID, CoI) tuples.
\end{description}

\item[\textbf{num\_equiv\_gens(pedobj)} $\Rightarrow$ dictionary]\index[func]{pyp_metrics!num_equiv_gens()}
num\_equiv\_gens() computes the number of equivalent generations as the sum of (1/2)\^{}n, where n is the number of generations separating an individual and each of its known ancestors.
\begin{description}
\item[\emph{pedobj}] A PyPedal pedigree object.
\item[Returns:] A dictionary containing the five average ages.
\end{description}

\item[\textbf{num\_traced\_gens(pedobj)} $\Rightarrow$ dictionary]\index[func]{pyp_metrics!num_traced_gens()}
num\_traced\_gens() is computed as the number of generations separating offspring from the oldest known ancestor in in each selection path. Ancestors with unknown parents are assigned to generation 0. See Valera at al. \cite{Valera2005a} for details.
\begin{description}
\item[\emph{pedobj}] A PyPedal pedigree object.
\item[Returns:] A dictionary containing the five average ages.
\end{description}

\item[\textbf{partial\_inbreeding(pedobj)} $\Rightarrow$ dictionary]\index[func]{pyp_metrics!partial_inbreeding()}
partial\_inbreeding() computes the number of equivalent generations as the sum of $\frac{1}{2}^{n}$, where n is the number of generations separating an individual and each of its known ancestors.
\begin{description}
\item[\emph{pedobj}] A PyPedal pedigree object.
\item[Returns:] A dictionary containing the five average ages.
\end{description}

\item[\textbf{pedigree\_completeness(pedobj, gens=4)}]\index[func]{pyp_metrics!pedigree_completeness()}
pedigree\_completeness() computes the proportion of known ancestors in the pedigree of each animal in the population for a user-determined number of generations. Also, the mean pedcomps for all animals and for all animals that are not founders are computed as summary statistics.  This is similar to pedigree completeness as computed by \citeN{ref615}, but with some of the modifications of VanRaden (2003) (\url{http://www.aipl.arsusda.gov/reference/changes/eval0311.html}).
\begin{description}
\item[\emph{pedobj}] A PyPedal pedigree object.
\item[\emph{gens}] The number of generations the pedigree should be traced for completeness.
\end{description}

\item[\textbf{related\_animals(anim\_a, pedobj)} $\Rightarrow$ list]\index[func]{pyp_metrics!related_animals()}
related\_animals() returns a list of the ancestors of an animal.
\begin{description}
\item[\emph{anim\_a}] The renumbered ID of an animal, a.
\item[\emph{pedobj}] A PyPedal pedigree object.
\item[Returns:] A list of animals related to anim\_a
\end{description}

\item[\textbf{relationship(anim\_a, anim\_b, pedobj)} $\Rightarrow$ float]\index[func]{pyp_metrics!relationship()}
relationship() returns the coefficient of relationship for two animals, anim\_a and anim\_b.
\begin{description}
\item[\emph{anim\_a}] The renumbered ID of an animal, a.
\item[\emph{anim\_b}] The renumbered ID of an animal, b.
\item[\emph{pedobj}] A PyPedal pedigree object.
\item[Returns:] The coefficient of relationship of anim\_a and anim\_b
\end{description}

\item[\textbf{theoretical\_ne\_from\_metadata(pedobj)} $\Rightarrow$ None]\index[func]{pyp_metrics!theoretical_ne_from_metadata()}
theoretical\_ne\_from\_metadata() computes the theoretical effective population size based on the number of sires and dams contained in a pedigree metadata object. Writes results to an output file.
\begin{description}
\item[\emph{pedobj}] A PyPedal pedigree object.
\end{description}
\end{description}
\section{pyp\_newclasses}
pyp\_newclasses\index[func]{pyp_newclasses} contains the new class structure that will be a part of PyPedal 2.0.0Final. It includes a master class to which most of the computational routines will be bound as methods, a NewAnimal() class, and a PedigreeMetadata() class.

\subsection*{Module Contents}

\begin{description}
\item[\textbf{NewAMatrix(kw)} (class)]\index[func]{pyp_newclasses!NewAMatrix}
NewAMatrix provides an instance of a numerator relationship matrix as a Numarray array of floats with some convenience methods.
For more information about this class, see \emph{The NewAMatrix Class}

\item[\textbf{NewAnimal(locations, data, mykw)} (class)]\index[func]{pyp_newclasses!NewAnimal}
The NewAnimal() class is holds animals records read from a pedigree file.
For more information about this class, see \emph{The NewAnimal Class}

\item[\textbf{NewPedigree(kw)} (class)]\index[func]{pyp_newclasses!NewPedigree}
The NewPedigree class is the main data structure for PyP 2.0.0Final.
For more information about this class, see \emph{The NewPedigree Class}

\item[\textbf{PedigreeMetadata(myped, kw)} (class)]\index[func]{pyp_newclasses!PedigreeMetadata}
The PedigreeMetadata() class stores metadata about pedigrees.
For more information about this class, see \emph{The PedigreeMetadata Class}
\end{description}

\subsection*{The NewAMatrix Class}
\begin{description}
\item[\textbf{NewAMatrix(kw)} (class)]
NewAMatrix provides an instance of a numerator relationship matrix as a Numarray array of floats with some convenience methods. The idea here is to provide a wrapper around a NRM so that it is easier to work with. For large pedigrees it can take a long time to compute the elements of A, so there is real value in providing an easy way to save and retrieve a NRM once it has been formed.

\item[\textbf{form\_a\_matrix(pedigree)} $\Rightarrow$ integer]\index[func]{pyp_newclasses!NewAMatrix!form_a_matrix()}
form\_a\_matrix() calls pyp\_nrm/fast\_a\_matrix() or pyp\_nrm/fast\_a\_matrix\_r() to form a NRM from a pedigree.
\begin{description}
\item[\emph{pedigree}] The pedigree used to form the NRM.
\item[Returns:] A NRM on success, 0 on failure.
\end{description}

\item[\textbf{info()} $\Rightarrow$ None]\index[func]{pyp_newclasses!NewAMatrix!info()}
info() uses the info() method of Numarray arrays to dump some information about the NRM. This is of use predominantly for debugging.
\begin{description}
\item[\emph{None}]
\item[Returns:] None
\end{description}

\item[\textbf{load(nrm\_filename)} $\Rightarrow$ integer]\index[func]{pyp_newclasses!NewAMatrix!load()}
load() uses the Numarray Array Function ``fromfile()'' to load an array from a binary file. If the load is successful, self.nrm contains the matrix.
\begin{description}
\item[\emph{nrm\_filename}] The file from which the matrix should be read.
\item[Returns:] A load status indicator (0: failed, 1: success).
\end{description}

\item[\textbf{save(nrm\_filename)} $\Rightarrow$ integer]\index[func]{pyp_newclasses!NewAMatrix!save()}
save() uses the Numarray method ``tofile()'' to save an array to a binary file.
\begin{description}
\item[\emph{nrm\_filename}] The file to which the matrix should be written.
\item[Returns:] A save status indicator (0: failed, 1: success).
\end{description}

\end{description}

\subsection*{The NewAnimal Class}
\begin{description}
\item[\textbf{NewAnimal(locations, data, mykw)} (class)]
The NewAnimal() class is holds animals records read from a pedigree file.

\item[\textbf{\_\_init\_\_(locations, data, mykw)} $\Rightarrow$ object]\index[func]{pyp_newclasses!NewAnimal!\_\_init\_\_()}
\_\_init\_\_() initializes a NewAnimal() object.
\begin{description}
\item[\emph{locations}] A dictionary containing the locations of variables in the input line.
\item[\emph{data}] The line of input read from the pedigree file.
\item[Returns:] An instance of a NewAnimal() object populated with data
\end{description}

\item[\textbf{pad\_id()} $\Rightarrow$ integer]\index[func]{pyp_newclasses!NewAnimal!pad_id()}
pad\_id() takes an Animal ID, pads it to fifteen digits, and prepends the birthyear (or 1950 if the birth year is unknown). The order of elements is: birthyear, animalID, count of zeros, zeros.
\begin{description}
\item[\emph{self}] Reference to the current Animal() object
\item[Returns:] A padded ID number that is supposed to be unique across animals
\end{description}

\item[\textbf{printme()} $\Rightarrow$ None]\index[func]{pyp_newclasses!NewAnimal!printme()}
printme() prints a summary of the data stored in the Animal() object.
\begin{description}
\item[\emph{self}] Reference to the current Animal() object
\end{description}


\item[\textbf{string\_to\_int(idstring)} $\Rightarrow$ None]\index[func]{pyp_newclasses!NewAnimal!string_to_int()}
string\_to\_int() takes an Animal/Sire/Dam ID as a string and returns a string that can be represented as an integer by replacing each character in the string with its corresponding ASCII table value.

\item[\textbf{stringme()} $\Rightarrow$ None]\index[func]{pyp_newclasses!NewAnimal!stringme()}
stringme() returns a summary of the data stored in the Animal() object as a string.
\begin{description}
\item[\emph{self}] Reference to the current Animal() object
\end{description}


\item[\textbf{trap()} $\Rightarrow$ None]\index[func]{pyp_newclasses!NewAnimal!trap()}
trap() checks for common errors in Animal() objects
\begin{description}
\item[\emph{self}] Reference to the current Animal() object
\end{description}

\end{description}

\subsection*{The NewPedigree Class}
\begin{description}
\item[\textbf{NewPedigree(kw)} (class)]
The NewPedigree class is the main data structure for PyP 2.0.0Final.

\item[\textbf{load(pedsource='file')} $\Rightarrow$ None]\index[func]{pyp_newclasses!NewPedigree!load()}
load() wraps several processes useful for loading and preparing a pedigree for use in an analysis, including reading the animals into a list of animal objects, forming lists of sires and dams, checking for common errors, setting ancestor flags, and renumbering the pedigree.
\begin{description}
\item[\emph{renum}] Flag to indicate whether or not the pedigree is to be renumbered.
\item[\emph{alleles}] Flag to indicate whether or not pyp\_metrics/effective\_founder\_genomes() should be called for a single round to assign alleles.
\item[Returns:] None
\end{description}

\item[\textbf{preprocess()} $\Rightarrow$ None]\index[func]{pyp_newclasses!NewPedigree!preprocess()}
preprocess() processes a pedigree file, which includes reading the animals into a list of animal objects, forming lists of sires and dams, and checking for common errors.
\begin{description}
\item[\emph{None}]
\item[Returns:] None
\end{description}

\item[\textbf{renumber()} $\Rightarrow$ None]\index[func]{pyp_newclasses!NewPedigree!renumber()}
renumber() updates the ID map after a pedigree has been renumbered so that all references are to renumbered rather than original IDs.
\begin{description}
\item[\emph{None}]
\item[Returns:] None
\end{description}

\item[\textbf{save(filename='', outformat='o', idformat='o')} $\Rightarrow$ integer]\index[func]{pyp_newclasses!NewPedigree!save()}
save() writes a PyPedal pedigree to a user-specified file. The saved pedigree includes all fields recognized by PyPedal, not just the original fields read from the input pedigree file.
\begin{description}
\item[\emph{filename}] The file to which the pedigree should be written.
\item[\emph{outformat}] The format in which the pedigree should be written: 'o' for original (as read) and 'l' for long version (all available variables).
\item[\emph{idformat}] Write 'o' (original) or 'r' (renumbered) animal, sire, and dam IDs.
\item[Returns:] A save status indicator (0: failed, 1: success)
\end{description}

\item[\textbf{updateidmap()} $\Rightarrow$ None]\index[func]{pyp_newclasses!NewPedigree!updateidmap()}
updateidmap() updates the ID map after a pedigree has been renumbered so that all references are to renumbered rather than original IDs.
\begin{description}
\item[\emph{None}]
\item[Returns:] None
\end{description}

\end{description}

\subsection*{The PedigreeMetadata Class}
\begin{description}
\item[\textbf{PedigreeMetadata(myped, kw)} (class)]
The PedigreeMetadata() class stores metadata about pedigrees. Hopefully this will help improve performance in some procedures, as well as provide some useful summary data.

\item[\textbf{\_\_init\_\_(myped, kw)} $\Rightarrow$ object]\index[func]{pyp_newclasses!PedigreeMetadata!\_\_init\_\_()}
\_\_init\_\_() initializes a PedigreeMetadata object.
\begin{description}
\item[\emph{self}] Reference to the current Pedigree() object
\item[\emph{myped}] A PyPedal pedigree.
\item[\emph{kw}] A dictionary of options.
\item[Returns:] An instance of a Pedigree() object populated with data
\end{description}

\item[\textbf{fileme()} $\Rightarrow$ None]\index[func]{pyp_newclasses!PedigreeMetadata!fileme()}
fileme() writes the metada stored in the Pedigree() object to disc.
\begin{description}
\item[\emph{self}] Reference to the current Pedigree() object
\end{description}

\item[\textbf{nud()} $\Rightarrow$ integer-and-list]\index[func]{pyp_newclasses!PedigreeMetadata!nud()}
nud() returns the number of unique dams in the pedigree along with a list of the dams
\begin{description}
\item[\emph{self}] Reference to the current Pedigree() object
\item[Returns:] The number of unique dams in the pedigree and a list of those dams
\end{description}

\item[\textbf{nuf()} $\Rightarrow$ integer-and-list]\index[func]{pyp_newclasses!PedigreeMetadata!nuf()}
nuf() returns the number of unique founders in the pedigree along with a list of the founders
\begin{description}
\item[\emph{self}] Reference to the current Pedigree() object
\item[Returns:] The number of unique founders in the pedigree and a list of those founders
\end{description}

\item[\textbf{nug()} $\Rightarrow$ integer-and-list]\index[func]{pyp_newclasses!PedigreeMetadata!nug()}
nug() returns the number of unique generations in the pedigree along with a list of the generations
\begin{description}
\item[\emph{self}] Reference to the current Pedigree() object
\item[Returns:] The number of unique generations in the pedigree and a list of those generations
\end{description}

\item[\textbf{nus()} $\Rightarrow$ integer-and-list]\index[func]{pyp_newclasses!PedigreeMetadata!nus()}
nus() returns the number of unique sires in the pedigree along with a list of the sires
\begin{description}
\item[\emph{self}] Reference to the current Pedigree() object
\item[Returns:] The number of unique sires in the pedigree and a list of those sires
\end{description}

\item[\textbf{nuy()} $\Rightarrow$ integer-and-list]\index[func]{pyp_newclasses!PedigreeMetadata!nuy()}
nuy() returns the number of unique birthyears in the pedigree along with a list of the birthyears
\begin{description}
\item[\emph{self}] Reference to the current Pedigree() object
\item[Returns:] The number of unique birthyears in the pedigree and a list of those birthyears
\end{description}

\item[\textbf{printme()} $\Rightarrow$ None]\index[func]{pyp_newclasses!PedigreeMetadata!printme()}
printme() prints a summary of the metadata stored in the Pedigree() object.
\begin{description}
\item[\emph{self}] Reference to the current Pedigree() object
\end{description}

\item[\textbf{stringme()} $\Rightarrow$ None]\index[func]{pyp_newclasses!PedigreeMetadata!stringme()}
stringme() returns a summary of the metadata stored in the pedigree as a string.
\begin{description}
\item[\emph{self}] Reference to the current Pedigree() object
\end{description}

\end{description}
\section{pyp\_nrm}
pyp\_nrm\index[func]{pyp_nrm} contains several procedures for computing numerator relationship matrices and for performing operations on those matrices. It also contains routines for computing CoI on large pedigrees using the recursive method of VanRaden \cite{VanRaden1992a}.

\subsection*{Module Contents}

\begin{description}
\item[\textbf{a\_decompose(pedobj)} $\Rightarrow$ matrices] \index[func]{pyp_nrm!a\_decompose()}
Form the decomposed form of A, TDT', directly from a pedigree (after Henderson \cite{ref143}, Mrode \cite{ref224}). Return D, a diagonal matrix, and T, a lower triagular matrix such that A = TDT'.
\begin{description}
\item[\emph{pedobj}] A PyPedal pedigree object.
\item[Returns:] A diagonal matrix, D, and a lower triangular matrix, T.
\end{description}

\item[\textbf{a\_inverse\_df(pedobj)} $\Rightarrow$ matrix] \index[func]{pyp_nrm!a\_inverse\_df()}
Directly form the inverse of A from the pedigree file - accounts for inbreeding - using the method of Quaas \cite{ref235}.
\begin{description}
\item[\emph{pedobj}] A PyPedal pedigree object.
\item[Returns:] The inverse of the NRM, A, accounting for inbreeding.
\end{description}

\item[\textbf{a\_inverse\_dnf(pedobj, filetag='\_a\_inverse\_dnf\_')} $\Rightarrow$ matrix] \index[func]{pyp_nrm!a\_inverse\_dnf()}
Form the inverse of A directly using the method of Henderson \cite{ref143} which does not account for inbreeding.
\begin{description}
\item[\emph{pedobj}] A PyPedal pedigree object.
\item[Returns:] The inverse of the NRM, A, not accounting for inbreeding.
\end{description}

\item[\textbf{a\_matrix(pedobj, save=0)} $\Rightarrow$ array] \index[func]{pyp_nrm!a\_matrix()}
a\_matrix() is used to form a numerator relationship matrix from a pedigree. DEPRECATED. use fast\_a\_matrix() instead.
\begin{description}
\item[\emph{pedobj}] A PyPedal pedigree object.
\item[\emph{save}] Flag to indicate whether or not the relationship matrix is written to a file.
\item[Returns:] The NRM as a numarray matrix.
\end{description}

\item[\textbf{fast\_a\_matrix(pedigree, pedopts, save=0)} $\Rightarrow$ matrix] \index[func]{pyp_nrm!fast\_a\_matrix()}
Form a numerator relationship matrix from a pedigree. fast\_a\_matrix() is a hacked version of a\_matrix() modified to try and improve performance. Lists of animal, sire, and dam IDs are formed and accessed rather than myped as it is much faster to access a member of a simple list rather than an attribute of an object in a list. Further note that only the diagonal and upper off-diagonal of A are populated. This is done to save n(n+1) / 2 matrix writes. For a 1000-element array, this saves 500,500 writes.
\begin{description}
\item[\emph{pedigree}] A PyPedal pedigree.
\item[\emph{pedopts}] PyPedal options.
\item[\emph{save}] Flag to indicate whether or not the relationship matrix is written to a file.
\item[Returns:] The NRM as Numarray matrix.
\end{description}

\item[\textbf{fast\_a\_matrix\_r(pedigree, pedopts, save=0)} $\Rightarrow$ matrix] \index[func]{pyp_nrm!fast\_a\_matrix\_r()}
Form a relationship matrix from a pedigree. fast\_a\_matrix\_r() differs from fast\_a\_matrix() in that the coefficients of relationship are corrected for the inbreeding of the parents.
\begin{description}
\item[\emph{pedobj}] A PyPedal pedigree object.
\item[\emph{save}] Flag to indicate whether or not the relationship matrix is written to a file.
\item[Returns:] A relationship as Numarray matrix.
\end{description}

\item[\textbf{form\_d\_nof(pedobj)} $\Rightarrow$ matrix] \index[func]{pyp_nrm!form\_d\_nof()}
Form the diagonal matrix, D, used in decomposing A and forming the direct inverse of A. This function does not write output to a file - if you need D in a file, use the a\_decompose() function. form\_d() is a convenience function used by other functions. Note that inbreeding is not considered in the formation of D.
\begin{description}
\item[\emph{pedobj}] A PyPedal pedigree object.
\item[Returns:] A diagonal matrix, D.
\end{description}

\item[\textbf{inbreeding(pedobj, method='tabular')} $\Rightarrow$ dictionary] \index[func]{pyp_nrm!inbreeding()}
inbreeding() is a proxy function used to dispatch pedigrees to the appropriate function for computing CoI. By default, small pedigrees $<$ 10,000 animals) are processed with the tabular method directly. For larger pedigrees, or if requested, the recursive method of VanRaden \cite{VanRaden1992a} is used.
\begin{description}
\item[\emph{pedobj}] A PyPedal pedigree object.
\item[\emph{method}] Keyword indicating which method of computing CoI should be used (tabular|vanraden).
\item[Returns:] A dictionary of CoI keyed to renumbered animal IDs.
\end{description}

\item[\textbf{inbreeding\_tabular(pedobj)} $\Rightarrow$ dictionary] \index[func]{pyp_nrm!inbreeding\_tabular()}
inbreeding\_tabular() computes CoI using the tabular method by calling fast\_a\_matrix() to form the NRM directly. In order for this routine to return successfully requires that you are able to allocate a matrix of floats of dimension len(myped)**2.
\begin{description}
\item[\emph{pedobj}] A PyPedal pedigree object.
\item[Returns:] A dictionary of CoI keyed to renumbered animal IDs
\end{description}

\item[\textbf{inbreeding\_vanraden(pedobj, cleanmaps=1)} $\Rightarrow$ dictionary] \index[func]{pyp_nrm!inbreeding\_vanraden()}
inbreeding\_vanraden() uses VanRaden's \cite{VanRaden1992a} method for computing coefficients of inbreeding in a large pedigree. The method works as follows: 1. Take a large pedigree and order it from youngest animal to oldest (n, n-1, ..., 1); 2. Recurse through the pedigree to find all of the ancestors of that animal n; 3. Reorder and renumber that ``subpedigree''; 4. Compute coefficients of inbreeding for that ``subpedigree'' using the tabular method (Emik and Terrill \cite{Emik1949a}); 5. Put the coefficients of inbreeding in a dictionary; 6. Repeat 2 - 5 for animals n-1 through 1; the process is slowest for the early pedigrees and fastest for the later pedigrees.
\begin{description}
\item[\emph{pedobj}] A PyPedal pedigree object.
\item[\emph{cleanmaps}] Flag to denote whether or not subpedigree ID maps should be delete after they are used (0|1)
\item[Returns:] A dictionary of CoI keyed to renumbered animal IDs
\end{description}

\item[\textbf{recurse\_pedigree(pedobj, anid, \_ped)} $\Rightarrow$ list] \index[func]{pyp_nrm!recurse\_pedigree()}
recurse\_pedigree() performs the recursion needed to build the subpedigrees used by inbreeding\_vanraden(). For the animal with animalID anid recurse\_pedigree() will recurse through the pedigree myped and add references to the relatives of anid to the temporary pedigree, \_ped.
\begin{description}
\item[\emph{pedobj}] A PyPedal pedigree.
\item[\emph{anid}] The ID of the animal whose relatives are being located.
\item[\emph{\_ped}] A temporary PyPedal pedigree that stores references to relatives of anid.
\item[Returns:] A list of references to the relatives of anid contained in myped.
\end{description}

\item[\textbf{recurse\_pedigree\_idonly(pedobj, anid, \_ped)} $\Rightarrow$ list] \index[func]{pyp_nrm!recurse\_pedigree\_idonly()}
recurse\_pedigree\_idonly() performs the recursion needed to build subpedigrees.
\begin{description}
\item[\emph{pedobj}] A PyPedal pedigree.
\item[\emph{anid}] The ID of the animal whose relatives are being located.
\item[\emph{\_ped}] A PyPedal list that stores the animalIDs of relatives of anid.
\item[Returns:] A list of animalIDs of the relatives of anid contained in myped.
\end{description}

\item[\textbf{recurse\_pedigree\_n(pedobj, anid, \_ped, depth=3)} $\Rightarrow$ list] \index[func]{pyp_nrm!recurse\_pedigree\_n()}
recurse\_pedigree\_n() recurses to build a pedigree of depth n. A depth less than 1 returns the animal whose relatives were to be identified.
\begin{description}
\item[\emph{pedobj}] A PyPedal pedigree.
\item[\emph{anid}] The ID of the animal whose relatives are being located.
\item[\emph{\_ped}] A temporary PyPedal pedigree that stores references to relatives of anid.
\item[\emph{depth}] The depth of the pedigree to return.
\item[Returns:] A list of references to the relatives of anid contained in myped.
\end{description}

\item[\textbf{recurse\_pedigree\_onesided(pedobj, anid, \_ped, side)} $\Rightarrow$ list] \index[func]{pyp_nrm!recurse\_pedigree\_onesided()}
recurse\_pedigree\_onsided() recurses to build a subpedigree from either the sire or dam side of a pedigree.
\begin{description}
\item[\emph{pedobj}] A PyPedal pedigree.
\item[\emph{side}] The side to build: 's' for sire and 'd' for dam.
\item[\emph{anid}] The ID of the animal whose relatives are being located.
\item[\emph{\_ped}] A temporary PyPedal pedigree that stores references to relatives of anid.
\item[Returns:] A list of references to the relatives of anid contained in myped.
\end{description}

\end{description}
\section{pyp\_reports}
pyp\_reports contains a set of procedures for generating reports
\index[func]{pyp_reports}
\label{sec:pyp-reports}
\subsection*{Module Contents}
\begin{description}
\item[\textbf{\_pdfCreateTitlePage(canv, \_pdfSettings, reporttitle='', reportauthor='')} $\Rightarrow$ None]
\index[func]{pyp_reports!_pdfCreateTitlePage()}
\label{sec:pyp-reports-pdf-create-title-page}
\_pdfCreateTitlePage() adds a title page to a ReportLab canvas object.
\begin{description}
\item[\emph{canv}] An instance of a ReportLab Canvas object.
\item[\emph{\_pdfSettings}] An options dictionary created by \_pdfInitialize().
\index[func]{pyp_reports!_pdfSettings()}
\label{sec:pyp-reports-pdf-settings}
\item[Returns:] None
\end{description}
\item[\textbf{\_pdfDrawPageFrame(canv, \_pdfSettings)} $\Rightarrow$ None]
\_pdfDrawPageFrame() nicely frames page contents and includes the document title in a header and the page number in a footer.
\index[func]{pyp_reports!_pdfDrawPageFrame()}
\label{sec:pyp-reports-pdf-draw-page-frame}
\begin{description}
\item[\emph{canv}] An instance of a ReportLab Canvas object.
\item[\emph{\_pdfSettings}] An options dictionary created by \_pdfInitialize().
\item[Returns:] None
\end{description}
\item[\textbf{\_pdfInitialize(pedobj)} $\Rightarrow$ dictionary]
\_pdfInitialize() returns a dictionary of metadata that is used for report generation.
\index[func]{pyp_reports!_pdfInitialize()}
\label{sec:pyp-reports-pdf-initialize}
\begin{description}
\item[\emph{pedobj}] A PyPedal pedigree object.
\item[Returns:] A dictionary of metadata that is used for report generation.
\end{description}
\item[\textbf{meanMetricBy(pedobj, metric='fa', byvar='by')} $\Rightarrow$ dictionary]
meanMetricBy() returns a dictionary of means keyed by levels of the 'byvar' that can be used to draw graphs or prepare reports of summary statistics.
\index[func]{pyp_reports!meanMetricBy()}
\label{sec:pyp-reports-mean-metric-by}
\begin{description}
\item[\emph{pedobj}] A PyPedal pedigree object.
\item[\emph{metric}] The variable to summarize on a BY variable.
\item[\emph{byvar}] The variable on which to group the metric.
\item[Returns:] A dictionary containing means for the metric variable keyed to levels of the byvar.
\end{description}
\item[\textbf{pdfPedigreeMetadata(pedobj, titlepage=0, reporttitle='', reportauthor='', reportfile='')} $\Rightarrow$ integer ]
pdfPedigreeMetadata() produces a report, in PDF format, of the metadata from the input pedigree. It is intended for use as a template for custom printed reports.
\index[func]{pyp_reports!pdfPedigreeMetadata()}
\label{sec:pyp-reports-pdf-pedigree-metadata}
\begin{description}
\item[\emph{pedobj}] A PyPedal pedigree object.
\item[\emph{titlepage}] Show (1) or hide (0) the title page.
\item[\emph{reporttitle}] Title of report; if '', \_pdfTitle is used.
\item[\emph{reportauthor}] Author/preparer of report.
\item[\emph{reportfile}] Optional name of file to which the report should be written.
\item[Returns:] A 1 on success, 0 otherwise.
\end{description}
\end{description}
\section{pyp\_utils}
pyp\_utils\index[func]{pyp_utils} contains a set of procedures for creating and operating on PyPedal pedigrees. This includes routines for reordering and renumbering pedigrees, as well as for modifying pedigrees.

\subsection*{Module Contents}

\begin{description}
\item[\textbf{assign\_offspring(pedobj)} $\Rightarrow$ integer]\index[func]{pyp_utils!assign_offspring()}
assign\_offspring() assigns offspring to their parent(s)'s unknown sex offspring list (well, dictionary).
\begin{description}
\item[\emph{myped}] An instance of a NewPedigree object.
\item[Returns:] 0 for failure and 1 for success.
\end{description}

\item[\textbf{assign\_sexes(pedobj)} $\Rightarrow$ integer]\index[func]{pyp_utils!assign_sexes()}
assign\_sexes() assigns a sex to every animal in the pedigree using sire and daughter lists for improved accuracy.
\begin{description}
\item[\emph{pedobj}] A renumbered and reordered PyPedal pedigree object.
\item[Returns:] 0 for failure and 1 for success.
\end{description}

\item[\textbf{delete\_id\_map(filetag='\_renumbered\_')} $\Rightarrow$ integer]\index[func]{pyp_utils!delete_id_map()}
delete\_id\_map() checks to see if an ID map for the given filetag exists. If the file exists, it is deleted.
\begin{description}
\item[\emph{filetag}] A descriptor prepended to output file names that is used to determine name of the file to delete.
\item[Returns:] A flag indicating whether or not the file was successfully deleted (0|1)
\end{description}

\item[\textbf{fast\_reorder(myped, filetag='\_new\_reordered\_', io='no', debug=0)} $\Rightarrow$ list]\index[func]{pyp_utils!fast_reorder()}
fast\_reorder() renumbers a pedigree such that parents precede their offspring in the pedigree. In order to minimize overhead as much as is reasonably possible, a list of animal IDs that have already been seen is kept. Whenever a parent that is not in the seen list is encountered, the offspring of that parent is moved to the end of the pedigree. This should ensure that the pedigree is properly sorted such that all parents precede their offspring. myped is reordered in place. fast\_reorder() uses dictionaries to renumber the pedigree based on paddedIDs.
\begin{description}
\item[\emph{myped}] A PyPedal pedigree object.
\item[\emph{filetag}] A descriptor prepended to output file names.
\item[\emph{io}] Indicates whether or not to write the reordered pedigree to a file (yes|no).
\item[\emph{debug}] Flag to indicate whether or not debugging messages are written to STDOUT.
\item[Returns:] A reordered PyPedal pedigree.
\end{description}

\item[\textbf{load\_id\_map(filetag='\_renumbered\_')} $\Rightarrow$ dictionary]\index[func]{pyp_utils!load_id_map()}
load\_id\_map() reads an ID map from the file generated by pyp\_utils/renumber() into a dictionary. There is a VERY similar function, pyp\_io/id\_map\_from\_file(), that is deprecated because it is much more fragile that this procedure.
\begin{description}
\item[\emph{filetag}] A descriptor prepended to output file names that is used to determine the input file name.
\item[Returns:] A dictionary whose keys are renumbered IDs and whose values are original IDs or an empty dictionary (on failure).
\end{description}

\item[\textbf{pedigree\_range(pedobj, n)} $\Rightarrow$ list]\index[func]{pyp_utils!pedigree_range()}
pedigree\_range() takes a renumbered pedigree and removes all individuals with a renumbered ID $>$ n. The reduced pedigree is returned. Assumes that the input pedigree is sorted on animal key in ascending order.
\begin{description}
\item[\emph{myped}] A PyPedal pedigree object.
\item[\emph{n}] A renumbered animalID.
\item[Returns:] A pedigree containing only animals born in the given birthyear or an empty list (on failure).
\end{description}

\item[\textbf{pyp\_nice\_time()} $\Rightarrow$ string]\index[func]{pyp_utils!pyp_nice_time()}
pyp\_nice\_time() returns the current date and time formatted as, e.g., Wed Mar 30 10:26:31 2005.
\begin{description}
\item[\emph{None}]
\item[Returns:] A string containing the formatted date and time.
\end{description}

\item[\textbf{renumber(myped, filetag='\_renumbered\_', io='no', outformat='0', debug=0)} $\Rightarrow$ list]\index[func]{pyp_utils!renumber()}
renumber() takes a pedigree as input and renumbers it such that the oldest animal in the pedigree has an ID of '1' and the n-th animal has an ID of 'n'. If the pedigree is not ordered from oldest to youngest such that all offspring precede their offspring, the pedigree will be reordered. The renumbered pedigree is written to disc in 'asd' format and a map file that associates sequential IDs with original IDs is also written.
\begin{description}
\item[\emph{myped}] A PyPedal pedigree object.
\item[\emph{filetag}] A descriptor prepended to output file names.
\item[\emph{io}] Indicates whether or not to write the renumbered pedigree to a file (yes|no).
\item[\emph{outformat}] Flag to indicate whether or not to write an asd pedigree (0) or a full pedigree (1).
\item[\emph{debug}] Flag to indicate whether or not progress messages are written to stdout.
\item[Returns:] A reordered PyPedal pedigree.
\end{description}

\item[\textbf{reorder(myped, filetag='\_reordered\_', io='no')} $\Rightarrow$ list]\index[func]{pyp_utils!reorder()}
reorder() renumbers a pedigree such that parents precede their offspring in the pedigree. In order to minimize overhead as much as is reasonably possible, a list of animal IDs that have already been seen is kept. Whenever a parent that is not in the seen list is encountered, the offspring of that parent is moved to the end of the pedigree. This should ensure that the pedigree is properly sorted such that all parents precede their offspring. myped is reordered in place. reorder() is VERY slow, but I am pretty sure that it works correctly.
\begin{description}
\item[\emph{myped}] A PyPedal pedigree object.
\item[\emph{filetag}] A descriptor prepended to output file names.
\item[\emph{io}] Indicates whether or not to write the reordered pedigree to a file (yes|no).
\item[Returns:] A reordered PyPedal pedigree.
\end{description}

\item[\textbf{reverse\_string(mystring)} $\Rightarrow$ string]\index[func]{pyp_utils!reverse_string()}
reverse\_string() reverses the input string and returns the reversed version.
\begin{description}
\item[\emph{mystring}] A non-empty Python string.
\item[Returns:] The input string with the order of its characters reversed.
\end{description}

\item[\textbf{set\_age(pedobj)} $\Rightarrow$ integer]\index[func]{pyp_utils!set_age()}
set\_age() Computes ages for all animals in a pedigree based on the global BASE\_DEMOGRAPHIC\_YEAR defined in pyp\_demog.py. If the by is unknown, the inferred generation is used. If the inferred generation is unknown, the age is set to -999.
\begin{description}
\item[\emph{pedobj}] A PyPedal pedigree object.
\item[Returns:] 0 for failure and 1 for success.
\end{description}

\item[\textbf{set\_ancestor\_flag(pedobj)} $\Rightarrow$ integer]\index[func]{pyp_utils!set_ancestor_flag()}
set\_ancestor\_flag() loops through a pedigree to build a dictionary of all of the parents in the pedigree. It then sets the ancestor flags for the parents. set\_ancestor\_flag() expects a reordered and renumbered pedigree as input!
\begin{description}
\item[\emph{pedobj}] A PyPedal NewPedigree object.
\item[Returns:] 0 for failure and 1 for success.
\end{description}

\item[\textbf{set\_generation(pedobj)} $\Rightarrow$ integer]\index[func]{pyp_utils!set_generation()}
set\_generation() Works through a pedigree to infer the generation to which an animal belongs based on founders belonging to generation 1. The igen assigned to an animal as the larger of sire.igen+1 and dam.igen+1. This routine assumes that myped is reordered and renumbered.
\begin{description}
\item[\emph{pedobj}] A PyPedal NewPedigree object.
\item[Returns:] 0 for failure and 1 for success.
\end{description}

\item[\textbf{set\_species(pedobj, species='u')} $\Rightarrow$ integer]\index[func]{pyp_utils!set_species()}
set\_species() assigns a specie to every animal in the pedigree.
\begin{description}
\item[\emph{pedobj}] A PyPedal pedigree object.
\item[\emph{species}] A PyPedal string.
\item[Returns:] 0 for failure and 1 for success.
\end{description}

\item[\textbf{simple\_histogram\_dictionary(mydict, histchar='*', histstep=5)} $\Rightarrow$ dictionary]\index[func]{pyp_utils!simple_histogram_dictionary()}
simple\_histogram\_dictionary() returns a dictionary containing a simple, text histogram. The input dictionary is assumed to contain keys which are distinct levels and values that are counts.
\begin{description}
\item[\emph{mydict}] A non-empty Python dictionary.
\item[\emph{histchar}] The character used to draw the histogram (default is '*').
\item[\emph{histstep}] Used to determine the number of bins (stars) in the diagram.
\item[Returns:] A dictionary containing the histogram by level or an empty dictionary (on failure).
\end{description}

\item[\textbf{sort\_dict\_by\_keys(mydict)} $\Rightarrow$ dictionary]\index[func]{pyp_utils!sort_dict_by_keys()}
sort\_dict\_by\_keys() returns a dictionary where the values in the dictionary in the order obtained by sorting the keys. Taken from the routine sortedDictValues3 in the ``Python Cookbook'', p. 39.
\begin{description}
\item[\emph{mydict}] A non-empty Python dictionary.
\item[Returns:] The input dictionary with keys sorted in ascending order or an empty dictionary (on failure).
\end{description}

\item[\textbf{sort\_dict\_by\_values(first, second)} $\Rightarrow$ list]\index[func]{pyp_utils!sort_dict_by_values()}
sort\_dict\_by\_values() returns a dictionary where the keys in the dictionary are sorted ascending value, first on value and then on key within value. The implementation was taken from John Hunter's contribution to a newsgroup thread: \url{http://groups-beta.google.com/group/comp.lang.python/browse}\_thread/thread/bbc259f8454e4d3f/cc686f4cd795feb4?q=python+\%22sorted+dictionary\%22=1=en\#cc686f4cd795feb4
\begin{description}
\item[\emph{mydict}] A non-empty Python dictionary.
\item[Returns:] A list of tuples sorted in ascending order.
\end{description}

\item[\textbf{string\_to\_table\_name(instring)} $\Rightarrow$ string]\index[func]{pyp_utils!string_to_table_name()}
string\_to\_table\_name() takes an arbitrary string and returns a string that is safe to use as an SQLite table name.
\begin{description}
\item[\emph{instring}] A string that will be converted to an SQLite-safe table name.
\item[Returns:] A string that is safe to use as an SQLite table name.
\end{description}

\item[\textbf{trim\_pedigree\_to\_year(pedobj, year)} $\Rightarrow$ list]\index[func]{pyp_utils!trim_pedigree_to_year()}
trim\_pedigree\_to\_year() takes pedigrees and removes all individuals who were not born in birthyear 'year'.
\begin{description}
\item[\emph{myped}] A PyPedal pedigree object.
\item[\emph{year}] A birthyear.
\item[Returns:] A pedigree containing only animals born in the given birthyear or an ampty list (on failure).
\end{description}

\end{description}