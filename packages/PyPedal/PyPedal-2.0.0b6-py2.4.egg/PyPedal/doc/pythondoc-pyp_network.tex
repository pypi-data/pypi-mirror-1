\documentclass[10pt]{article}
\usepackage{fullpage, graphicx, url}
\setlength{\parskip}{1ex}
\setlength{\parindent}{0ex}
\title{The pyp\_network Module}
\begin{document}
\section*{The pyp\_network Module}


 pyp\_network contains a set of procedures for working with pedigrees as directed graphs.
\subsection*{Module Contents}
\begin{description}
\item[\textbf{count\_offspring(pedgraph, anid)}
 ⇒ integer [\#]]

 immediate\_family() returns the number of offspring of an animal.
\begin{description}
\item[\emph{pedgraph}
] An instance of a networkx DiGraph.
\item[\emph{anid}
] The animal for whom offspring are to be counted.
\item[Returns:] Count of offspring.

\end{description}
\\ 

\item[\textbf{find\_ancestors(pedgraph, anid, \_ancestors=[])}
 ⇒ list [\#]]

 find\_ancestors() identifies the ancestors of an animal and returns them in a list.
\begin{description}
\item[\emph{pedgraph}
] An instance of a networkx DiGraph.
\item[\emph{anid}
] The animal for whom ancestors are to be found.
\item[\emph{\_ancestors}
] The list of ancestors already found.
\item[Returns:] List of ancestors of anid.

\end{description}
\\ 

\item[\textbf{find\_descendants(pedgraph, anid, \_descendants=[])}
 ⇒ list [\#]]

 find\_descendants() identifies the descendants of an animal and returns them in a list.
\begin{description}
\item[\emph{pedgraph}
] An instance of a networkx DiGraph.
\item[\emph{anid}
] The animal for whom descendants are to be found.
\item[\emph{\_descendants}
] The list of descendants already found.
\item[Returns:] List of descendants of anid.

\end{description}
\\ 

\item[\textbf{get\_founder\_descendants(pedgraph)}
 ⇒ dictionary [\#]]

 get\_founder\_descendants() returns a dictionary containing a list of descendants of each founder in the pedigree.
\begin{description}
\item[\emph{pedgraph}
] An instance of a NetworkX DiGraph.
\item[Returns:] A dictionary containing a list of descendants for each founder in the graph.

\end{description}
\\ 

\item[\textbf{immediate\_family(pedgraph, anid)}
 ⇒ list [\#]]

 immediate\_family() returns parents and offspring of an animal.
\begin{description}
\item[\emph{pedgraph}
] An instance of a networkx DiGraph.
\item[\emph{anid}
] The animal for whom immediate family are to be found.
\item[Returns:] List of immediate family members of anid.

\end{description}
\\ 

\item[\textbf{most\_influential\_offspring(pedgraph, anid, resolve='all')}
 ⇒ dictionary [\#]]

 most\_influential\_offspring() returns the most influential offspring of an animal as measured by their number of offspring.
\begin{description}
\item[\emph{pedgraph}
] An instance of a networkx DiGraph.
\item[\emph{anid}
] The animal for whom the most influential offspring is to be found.
\item[\emph{resolve}
] Indicates how ties should be handled ('first'|'last'|'all').
\item[Returns:] The most influential offspring of anid.

\end{description}
\\ 

\item[\textbf{offspring\_influence(pedgraph, anid)}
 ⇒ dictionary [\#]]

 offspring\_influence() returns the number of grand-children by each child of a given animal.
\begin{description}
\item[\emph{pedgraph}
] An instance of a networkx DiGraph.
\item[\emph{anid}
] The animal for whom grand-progeny are to be counted.
\item[Returns:] A dictionary of counts of progeny per child.

\end{description}
\\ 

\item[\textbf{ped\_to\_graph(pedobj, oid=0)}
 ⇒ graph [\#]]

 ped\_to\_graph() Takes a PyPedal pedigree object and returns a networkx XDiGraph object.
\begin{description}
\item[\emph{pedobj}
] A PyPedal pedigree object.
\item[\emph{oid}
] Flag indicating if original (1) or renumbered (0) IDs should be used.
\item[Returns:] DiGraph object

\end{description}
\\ 


\end{description}

\end{document}
