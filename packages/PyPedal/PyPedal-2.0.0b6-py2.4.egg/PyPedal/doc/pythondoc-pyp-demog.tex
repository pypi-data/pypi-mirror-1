

 pyp\_demog contains a set of procedures for demographic calculations on the population describe in a pedigree.
\subsection*{Module Contents}
\begin{description}
\item[\textbf{age\_distribution(pedobj, sex=1)}
 [\#]]

 age\_distribution() computes histograms of the age distribution of males and females in the population. You can also stratify by sex to get individual histograms.
\begin{description}
\item[\emph{myped}
] An instance of a PyPedal NewPedigree object.
\item[\emph{sex}
] A flag which determines whether or not to stratify by sex.

\end{description}
\\ 

\item[\textbf{founders\_by\_year(pedobj)}
 ⇒ dictionary [\#]]

 founders\_by\_year() returns a dictionary containing the number of founders in each birthyear.
\begin{description}
\item[\emph{pedobj}
] A PyPedal pedigree object.
\item[Returns:] dict A dictionary containing entries for each sex/gender code defined in the global SEX\_CODE\_MAP.

\end{description}
\\ 

\item[\textbf{set\_age\_units(units='year')}
 ⇒ None [\#]]

 set\_age\_units() defines a global variable, BASE\_DEMOGRAPHIC\_UNIT.
\begin{description}
\item[\emph{units}
] The base unit for age computations ('year'|'month'|'day').
\item[Returns:] None

\end{description}
\\ 

\item[\textbf{set\_base\_year(year=1900)}
 ⇒ None [\#]]

 set\_base\_year() defines a global variable, BASE\_DEMOGRAPHIC\_YEAR.
\begin{description}
\item[\emph{year}
] The year to be used as a base for computing ages.
\item[Returns:] None

\end{description}
\\ 

\item[\textbf{sex\_ratio(pedobj)}
 ⇒ dictionary [\#]]

 sex\_ratio() returns a dictionary containing the proportion of males and females in the population.
\begin{description}
\item[\emph{myped}
] An instance of a PyPedal NewPedigree object.
\item[Returns:] dict A dictionary containing entries for each sex/gender code defined in the global SEX\_CODE\_MAP.

\end{description}
\\ 


\end{description}

