\chapter{HOWTOs}
\label{cha:howtos}
\begin{quote}
In this chapter, examples of common operations are presented.
\end{quote}
\section{Basic Tasks}
\label{sec:howto-basic-operations}
\index{how do I!basic tasks}
\subsection{How do I load a pedigree from a file?}
\label{sec:howto-load-pedigree}
\index{how do I!basic tasks!load a pedigree}
Each pedigree that you read must be passed its own dictionary of options that must have at least a pedigree file name (\var{pedfile}) and a pedigree format string (\var{pedformat}).  You then call \method{pyp_newclasses.NewPedigree()} and pass the options dictionary as an argument.  The following code fragment demonstrates how to read a pedigree file:
\begin{verbatim}
options = {}
options['pedfile'] = 'new_lacy.ped'
options['pedformat'] = 'asd'

example1 = pyp_newclasses.NewPedigree(options)
example1.load()
\end{verbatim}
The options dictionary may be named anything you like.  In this manual, and in the example programs distributed with \PyPedal{}, \var{options} is the name used.
\subsection{How do I load multiple pedigrees in one program?}
\label{sec:howto-load-multiple-pedigrees}
\index{how do I!basic tasks!load multiple pedigrees}
A \PyPedal{} program can load more than one pedigree at a time.  Each pedigree must be passed its own options dictionary, and the pedigrees must have different names.  This is easily done by creating a dictionary with global options and customizing it for each pedigree.  Once you have created a pedigree by calling \method{pyp\_newclasses.NewPedigree('options')} you can change the options dictionary without affecting that pedigree (a pedigree stores a copy of the options dictionary in its \member{kw} attribute).  The following code fragment demonstrates how to read two pedigree files in a single program:
\begin{verbatim}
#   Create the empty options dictionary
options = {}

#   Read the first pedigree
options['pedfile'] = 'new_lacy.ped'
options['pedformat'] = 'asd'
options['pedname'] = 'Lacy Pedigree'
example1 = pyp_newclasses.NewPedigree(options)
example1.load()

#   Read the second pedigree
options['pedfile'] = 'new_boichard.ped'
options['pedformat'] = 'asdg'
options['pedname'] = 'Boichard Pedigree'
example2 = pyp_newclasses.NewPedigree(options)
example2.load()
\end{verbatim}
Note that \var{pedformat} only needs to be changed if the two pedigrees have different formats.  You do not even have to change \var{pedfile}.
\subsection{How do I renumber a pedigree?}
\label{sec:howto-renumber-pedigree}
\index{how do I!basic tasks!renumber a pedigree}
Set the \member{renumber} option to \samp{1} before you load the pedigree.
\begin{verbatim}
options = {}
options['renumber'] = 1
options['pedfile'] = 'new_lacy.ped'
options['pedformat'] = 'asd'
if __name__ == '__main__':
    example1 = pyp_newclasses.NewPedigree(options)
    example1.load()
\end{verbatim}
If you do not renumber a pedigree at load time and choose to renumber it later you must set the \member{renumber} option and call the pedigree's \method{renumber()} method:
\begin{verbatim}
example.kw['renumber'] = 1
example.renumber()
\end{verbatim}
For more details on pedigree renumbering see Section \ref{sec:renumbering}.
\subsection{How do I turn off output messages?}
\label{sec:howto-turn-off-messages}
\index{how do I!basic tasks!turn off output}
You may want to suppress the output that is normally written to STDOUT by scripts.  You do this by setting the \member{messages} option:
\begin{verbatim}
options['messages'] = 'quiet'
\end{verbatim}
The default setting for \member{messages} is \samp{verbose}, which produces lots of messages.
\section{Calculating Measures of Genetic Variation}
\label{sec:howto-genetic-variation}
\index{how do I!calculate genetic variation}
\subsection{How do I calculate coefficients of inbreeding?}
\label{sec:howto-calculate-inbreeding}
\index{how do I!calculate genetic variation!coefficients of inbreeding}
This requires that you have a renumbered pedigree (HOWTO \ref{sec:howto-renumber-pedigree}).
\begin{verbatim}
options = {}
options['renumber'] = 1
options['pedfile'] = 'new_lacy.ped'
options['pedformat'] = 'asd'
example1 = pyp_newclasses.NewPedigree(options)
example1.load()
example_inbreeding = pyp_nrm.inbreeding(example)
print example_inbreeding
\end{verbatim}
The dictionary returned by \function{pyp_nrm.inbreeding(example)}, \var{example_inbreeding}, contains two dictionaries: \var{fx} contains coefficients of inbreeding (COI) keyed to renumbered animal IDs and \var{metadata} contains summary statistics.  \var{metadata} also contains two dictionaries: \var{all} contains summary statistics for all animals, while \var{nonzero} contains summary statistics for only animals with non-zero coefficients of inbreeding.  If you print \var{example_inbreeding} you'll get the following:
\begin{verbatim}
{'fx': {1: 0.0, 2: 0.0, 3: 0.0, 4: 0.0, 5: 0.0, 6: 0.0, 7: 0.0, 8: 0.0, 9: 0.0,
10: 0.0, 11: 0.0, 12: 0.0, 13: 0.0, 14: 0.0, 15: 0.0, 16: 0.0, 17: 0.0, 18: 0.0,
19: 0.0, 20: 0.0, 21: 0.0, 22: 0.0, 23: 0.0, 24: 0.0, 25: 0.0, 26: 0.0, 27: 0.0,
28: 0.25, 29: 0.0, 30: 0.0, 31: 0.25, 32: 0.0, 33: 0.0, 34: 0.0, 35: 0.0, 36: 0.0,
37: 0.0, 38: 0.21875, 39: 0.0, 40: 0.0625, 41: 0.0, 42: 0.0, 43: 0.03125, 44: 0.0,
45: 0.0, 46: 0.0, 47: 0.0},
'metadata': {'nonzero': {'f_max': 0.25, 'f_avg': 0.16250000000000001,
'f_rng': 0.21875, 'f_sum': 0.8125, 'f_min': 0.03125, 'f_count': 5},
'all': {'f_max': 0.25, 'f_avg': 0.017287234042553192, 'f_rng': 0.25,
'f_sum': 0.8125, 'f_min': 0.0, 'f_count': 47}}}
\end{verbatim}
Obtaining the COI for a given animal, say 28, is simple:
\begin{verbatim}
>>> print example_inbreeding['fx'][28]
'0.25'
\end{verbatim}
To print the mean COI for the pedigree:
\begin{verbatim}
>>> print example_inbreeding['metadata']['all']['f_avg']
'0.017287234042553192'
\end{verbatim}
\section{Databases and Report Generation}
\label{sec:howto-databases-and-reports}
\index{how do I!databases and reports}
\subsection{How do I load a pedigree into a database?}
\label{sec:howto-load-pedigree-db}
\index{how do I!databases and reports!load a pedigree}
The \module{pyp\_reports} module (\ref{sec:pyp-reports}) uses the \module{pyp\_db} module (Section \ref{sec:pyp-db})
to store and manipulate a pedigree in an SQLite database.  In order to use these tools you must first load your pedigree into
the database.  This is done with a call to \function{pyp\_db.loadPedigreeTable()}:
\begin{verbatim}
options = {}
options['pedfile'] = 'hartlandclark.ped'
options['pedname'] = 'Pedigree from van Noordwijck and Scharloo (1981)'
options['pedformat'] = 'asdb'

example = pyp_newclasses.NewPedigree(options)
example.load()

pyp_nrm.inbreeding(example)
pyp_db.loadPedigreeTable(example)
\end{verbatim}
The routines in \module{pyp\_reports} will check to see if your pedigree has already been loaded; if it
has not, a table will be created and populated for you.
\subsection{How do I update a pedigree in the database?}
\label{sec:howto-pedigree-db-update-table}
\index{how do I!databases and reports!update pedigree table}
Changes to a \PyPedal{} pedigree object are not automatically saved to the database.  If you have changed
your pedigree, such as by calculating coefficients of inbreeding, and you want those changes visible to the
database you have to call \function{pyp\_db.loadPedigreeTable()} again.  \textbf{IMPORTANT NOTE:} If you call
\function{pyp\_db.loadPedigreeTable()} after you have already loaded your pedigree into the database it will
drop the existing table and reload it; all data in the existing table will be lost!  In the following
example, the pedigree is written to table \textbf{hartlandclark} in the database \textbf{pypedal}:
\begin{verbatim}
options = {}
options['pedfile'] = 'hartlandclark.ped'
options['pedname'] = 'Pedigree from van Noordwijck and Scharloo (1981)'
options['pedformat'] = 'asdb'

example = pyp_newclasses.NewPedigree(options)
example.load()

pyp_db.loadPedigreeTable(example)
\end{verbatim}
\member{pypedal} is the default database name used by \PyPedal{}, and can be changed using a pedigree's \member{database_name} option.  By default, table names are formed from the pedigree file name.  A table name can be specified using a pedigree's \member{dbtable_name} option.  Continuing the above example, suppose that I calculated coefficients of inbreeding on my pedigree and want to store the resulting pedigree in a new table named \var{noordwijck_and_scharloo_inbreeding}:
\begin{verbatim}
options['dbtable_name'] = 'noordwijck_and_scharloo_inbreeding'
pyp_nrm.inbreeding(example)
pyp_db.loadPedigreeTable(example)
\end{verbatim}
You should see messages in the log telling you that the table has been created and populated:
\begin{verbatim}
Tue, 29 Nov 2005 11:24:22 WARNING  Table noordwijck_and_scharloo_inbreeding does
                                   not exist in database pypedal!
Tue, 29 Nov 2005 11:24:22 INFO     Table noordwijck_and_scharloo_inbreeding
                                   created in database pypedal!
\end{verbatim}
\section{Contribute a HOWTO}
\label{sec:howto-contribute}
\index{how do I!contribute a HOWTO}
Users are invited to contribute HOWTOs demonstrating how to solve problems they've found interesting.  In order for such HOWTOs to be considered for inclusion in this manual they must be licensed under the GNU Free Documentation License version 1.2 or later (\url{http://www.gnu.org/copyleft/fdl.html}).  Authorship will be acknowledged, and copyright will remain with the author of the HOWTO.