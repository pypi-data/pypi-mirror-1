\documentclass[10pt]{article}
\usepackage{fullpage, graphicx, url}
\setlength{\parskip}{1ex}
\setlength{\parindent}{0ex}
\title{The pyp\_io Module}
\begin{document}
\section*{The pyp\_io Module}


 pyp\_io contains several procedures for writing structures to and reading them from disc (e.g. using pickle() to store and retrieve A and A-inverse). It also includes a set of functions used to render strings as HTML or plaintext for use in generating output files.
\subsection*{Module Contents}
\begin{description}
\item[\textbf{a\_inverse\_from\_file(inputfile)}
 ⇒ matrix [\#]]

 a\_inverse\_from\_file() uses the Python pickle system for persistent objects to read the inverse of a relationship matrix from a file.
\begin{description}
\item[\emph{inputfile}
] The name of the input file.
\item[Returns:] The inverse of a numerator relationship matrix.

\end{description}
\\ 

\item[\textbf{a\_inverse\_to\_file(pedobj, ainv='')}
 ⇒ integer [\#]]

 a\_inverse\_to\_file() uses the Python pickle system for persistent objects to write the inverse of a relationship matrix to a file.
\begin{description}
\item[\emph{pedobj}
] A PyPedal pedigree object.
\item[\emph{filetag}
] A descriptor prepended to output file names.
\item[Returns:] True (1) on success, false (0) on failure

\end{description}
\\ 

\item[\textbf{dissertation\_pedigree\_to\_file(pedobj)}
 ⇒ integer [\#]]

 dissertation\_pedigree\_to\_file() takes a pedigree in 'asdxfg' format and writes is to a file.
\begin{description}
\item[\emph{pedobj}
] A PyPedal pedigree object.
\item[Returns:] True (1) on success, false (0) on failure

\end{description}
\\ 

\item[\textbf{dissertation\_pedigree\_to\_pedig\_format(pedobj)}
 ⇒ integer [\#]]

 dissertation\_pedigree\_to\_pedig\_format() takes a pedigree in 'asdbxfg' format, formats it into the form used by Didier Boichard's 'pedig' suite of programs, and writes it to a file.
\begin{description}
\item[\emph{pedobj}
] A PyPedal pedigree object.
\item[Returns:] True (1) on success, false (0) on failure

\end{description}
\\ 

\item[\textbf{dissertation\_pedigree\_to\_pedig\_format\_mask(pedobj)}
 ⇒ integer [\#]]

 dissertation\_pedigree\_to\_pedig\_format\_mask() Takes a pedigree in 'asdbxfg' format, formats it into the form used by Didier Boichard's 'pedig' suite of programs, and writes it to a file. THIS FUNCTION MASKS THE GENERATION ID WITH A FAKE BIRTH YEAR AND WRITES THE FAKE BIRTH YEAR TO THE FILE INSTEAD OF THE TRUE BIRTH YEAR. THIS IS AN ATTEMPT TO FOOL PEDIG TO GET f\_e, f\_a et al. BY GENERATION.
\begin{description}
\item[\emph{pedobj}
] A PyPedal pedigree object.
\item[Returns:] True (1) on success, false (0) on failure

\end{description}
\\ 

\item[\textbf{dissertation\_pedigree\_to\_pedig\_interest\_format(pedobj)}
 ⇒ integer [\#]]

 dissertation\_pedigree\_to\_pedig\_interest\_format() takes a pedigree in 'asdbxfg' format, formats it into the form used by Didier Boichard's parente program for the studied individuals file.
\begin{description}
\item[\emph{pedobj}
] A PyPedal pedigree object.
\item[Returns:] True (1) on success, false (0) on failure

\end{description}
\\ 

\item[\textbf{pickle\_pedigree(pedobj, filename='')}
 ⇒ integer [\#]]

 pickle\_pedigree() pickles a pedigree.
\begin{description}
\item[\emph{pedobj}
] An instance of a PyPedal pedigree object.
\item[\emph{filename}
] The name of the file to which the pedigree object should be pickled (optional).
\item[Returns:] A 1 on success, a 0 otherwise.

\end{description}
\\ 

\item[\textbf{pyp\_file\_footer(ofhandle, caller=''Unknown PyPedal routine'')}
 [\#]]

 pyp\_file\_footer()
\begin{description}
\item[\emph{ofhandle}
] A Python file handle.
\item[\emph{caller}
] A string indicating the name of the calling routine.
\item[Returns:] None

\end{description}
\\ 

\item[\textbf{pyp\_file\_header(ofhandle, caller=''Unknown PyPedal routine'')}
 [\#]]

 pyp\_file\_header()
\begin{description}
\item[\emph{ofhandle}
] A Python file handle.
\item[\emph{caller}
] A string indicating the name of the calling routine.
\item[Returns:] None

\end{description}
\\ 

\item[\textbf{renderTitle(title\_string, title\_level=''1'')}
 [\#]]

 renderTitle() ... Produced HTML output by default.

\item[\textbf{summary\_inbreeding(f\_metadata)}
 ⇒ string [\#]]

 summary\_inbreeding() returns a string representation of the data contained in the 'metadata' dictionary contained in the output dictionary returned by pyp\_nrm/pyp\_inbreeding().
\begin{description}
\item[\emph{f\_metadata}
] Dictionary of inbreeding metadata.
\item[Returns:] A string on success, a 0 otherwise.

\end{description}
\\ 

\item[\textbf{unpickle\_pedigree(filename='')}
 ⇒ object [\#]]

 unpickle\_pedigree() reads a pickled pedigree in from a file and returns the unpacked pedigree object.
\begin{description}
\item[\emph{filename}
] The name of the pickle file.
\item[Returns:] An instance of a NewPedigree object on success, a 0 otherwise.

\end{description}
\\ 


\end{description}

\end{document}
