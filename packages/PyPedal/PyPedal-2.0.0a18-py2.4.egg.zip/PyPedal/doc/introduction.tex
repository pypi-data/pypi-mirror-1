\chapter{Introduction}
\label{cha:introduction}

\begin{quote}
   This chapter introduces the \PyPedal{} module for Python 2.4 and outlines the rest
   of the document.
\end{quote}

\PYPEDAL{} (\textbf{P}ython \textbf{Ped}igree An\textbf{al}ysis) is a tool for analyzing pedigree files.  It calculates several quantitative measures of allelic and genoytpic diversity from pedigrees, including average coefficients of inbreeding and relationship, effective number of founders, and effective number of ancestors.  Some qualitative checks are performed in order to catch some common mistakes, such as parents with more recent birthdates or ID numbers than their offspring.  Currently, \PYPEDAL{} only makes use of information on pedigree structure.  Allelotypes can be assigned to founders (or read from the pedigree file) for use in gene-dropping simulations to compute effective number of  founder genomes, but no other measures of alleic diversity are currently supported.  Some tools for non-interactive pedigree visualization are also provided.

Routines are also provided for the decomposition of $A$ and the direct formation of $A^{1}$ with and without accounting for inbreeding.  These are of academic rather than practical interest, but if a simple script is needed for the inversion of a reasonably-sized pedigree \PYPEDAL{} is up to the task.

\PYPEDAL{} is a Python (\url(http://www.python.org/))language module that may be called by other Python programs or used interactively from the Python interpreter.  You must have Python 2.4 installed in order to use \PYPEDAL() as \PYPEDAL() makes use of some version-specific features found only in 2.4.  The Numarray module must be installed in order for you to use \PYPEDAL(), and may be found at \url{http://www.stsci.edu/resources/software_hardware/numarray}.  In addition, \PYPEDAL() will make use of the following modules if they are installed:
\begin{itemize}
\item Graphviz (\url(http://www.research.att.com/sw/tools/graphviz/)) using the PyDot (\url(http://dkbza.org/pydot.html)) module is needed by the draw_pedigree() routine.
\item matplotlib (\url(http://matplotlib.sourceforge.net/)) is used to draw histograms and line graphs.
\item The Python Imaging Library (\url(http://www.pythonware.com/products/pil/)) is used to visualize numerator relationship matrices.
\end{itemize}

This document is the ``official'' documentation for \PyPedal{}. It includes a tutorial and is the most authoritative source of information about \PyPedal{} with the exception of the source code. The tutorial material will walk you through a set of manipulations of a simple pedigree.  All users of \PYPEDAL{} are encouraged to follow the tutorial with a working \PYPEDAL{} installation. The best way to learn is by doing --- the aim of this tutorial is to guide you along this "doing."

This manual contains:
\begin{description}
\item[Installing PyPedal] Chapter \ref{cha:installation} provides information
   on testing Python and installing PyPedal.
\item[PyPedal Tutorial] Chapter \ref{cha:tutorial} provides information
   on testing Python and installing PyPedal.
\item[High-Level Overview] Chapter \ref{cha:overview} gives a
   high-level overview of the components of the \PyPedal{} system as a whole.
\item[Glossary] Appendix \ref{cha:glossary} gives a glossary of terms.
\end{description}

\section{Implemented Features}
PyPedal is currently capable of doing the following things:
\begin{itemize}
\item Reading pedigree files in several formats;
\item Checking pedigree integrity (duplicate IDs, parents younger than offspring, etc.);
\item Generating summary information such as frequency of appearance in the pedigree file;
\item Computation of the numerator relationship matrix ($A$) from a pedigree file using the tabular method;
\item Inbreeding calculations for large pedigrees using Van{R}aden's (1992) recursive algorithm;
\item Computation of average total and average individual coefficients of inbreeding and relationship;
\item Decomposition of $A$ into $T$ and $D$ such that $A=TDT'$;
\item Computation of the direct inverse of $A$ (not accounting for inbreeding) using the method of Henderson (1976);
\item Computation of the direct inverse of $A$ (accounting for inbreeding) using the method of Quaas (1976);
\item Storage of $A$ and its inverse between user sessions as persistent Python objects using the pickle module to avoid unnecessary calculations;
\item Calculation of theoretical effective population size;
\item Calculation of actual effective population size based on the change in population average inbreeding;
\item Computation of effective founder number using the exact algorithm of Lacy (1989);
\item Computation of effective founder number using the approximate algorithm of Boichard et al. (1996);
\item Computation of effective ancestor number using the algorithm of Boichard et al. (1996);
\item Selection of subpedigrees containing all ancestors of an animal;
\item Identification of the common relatives of two animals;
\item Output to ASCII text files, including matrices, coefficients of inbreeding and relationship, and summary information;
\item Reordering and renumbering of pedigree files.
\end{itemize}

\section{Planned Features}
The following features are not yet implemented in PyPedal, but will probably be added in a future release:
\begin{itemize}
\item Direct calculation of the inverse of $A$ accounting for inbreeding using the method of Luo and Meuwissen;
\item Calculation of some measure of effective family number (inspired by a post of D. Gianola's to the Animal Geneticists Discussion Group email list on 30 January 2001);
\item Representation of pedigrees as an algebraic structure (i.e. graphs);
\item Identification of disconnected subgroups (if any) in a pedigree;
\item Fast operations on graphs;
\end{itemize}

\section{Where to get information and code}
\PYPEDAL{} and its documentation are available at (\ulink{sourceforge.net}{http://pypedal.sourceforge.net/}). The Numarray web site is: \url{http://numpy.sourceforge.net/}. The Python web site is \url{http://www.python.org/}.

\section{Acknowledgments}
\PYPEDAL{} was initially written to support the author's dissertation research while at Louisiana State University, Baton Rouge (\url{http://www.lsu.edu/}).  It lay fallow for some time but has recently come under active development again.  This is due in part to a request from colleagues at the University of Minnesota that led to the inclusion of new functionality in \PYPEDAL{}.  The author wishes to thank Dr. Paul Van{R}aden for very helpful suggestions for improving the ability of \PYPEDAL{} to handle certain computations in very large pedigrees.