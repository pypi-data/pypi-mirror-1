\chapter{Installing PyPedal}
\label{cha:installation}

\begin{quote}
   This chapter explains how to install and test \PYPEDAL{} from either the source distribution or from the binary distribution.
\end{quote}

Before we can begin the tutorial, we need to make sure that you can install and test Python, the Numeric or Numarray extension, and the \PYPEDAL{} extension.

\section{Testing the Python installation}

The first step is to install Python if you haven't already. Python is available from the Python project page at \url{http://sourceforge.net/projects/python/}.  Click on the link corresponding to your platform, and follow the instructions
described there. \PYPEDAL{} requires version 2.3 as a minimum.  When installed, starting Python by typing python at the shell or double-clicking on the Python interpreter should give a prompt such as:
\begin{verbatim}
Python 2.3.3 (#2, Feb 17 2004, 11:45:40)
[GCC 3.3.2 (Mandrake Linux 10.0 3.3.2-6mdk)] on linux2
Type "help", "copyright", "credits" or "license" for more information.
\end{verbatim}
If you have problems getting Python to work, consider contacting your local support person or e-mailing \ulink{python-help@python.org}{mailto:python-help@python.org} for help. If neither solution works, consider posting on the
\ulink{comp.lang.python}{news:comp.lang.python} newsgroup (details on the newsgroup/mailing list are available at
\url{http://www.python.org/psa/MailingLists.html\#clp}).


\section{Testing the Numarray Python Extension Installation}

The standard Python distribution does not come, as of this writing, with the
numarray Python extensions installed, but your system administrator may have
installed them already. To find out if your Python interpreter has numarray
installed, type \samp{import numarray} at the Python prompt. You'll see one of
two behaviors (throughout this document user input and python interpreter
output will be emphasized as shown in the block below):
\begin{verbatim}
>>> import numarray
Traceback (innermost last):
File "<stdin>", line 1, in ?
ImportError: No module named numarray
\end{verbatim}
indicating that you don't have numarray installed, or:
\begin{verbatim}
>>> import numarray
>>> numarray.__version__
'0.9'
\end{verbatim}
indicating that numarray is installed. If it is installed, you can skip the next section and go ahead to section \ref{sec:installing-pypedal}.  If you don't, you have to get and install the numarray extensions as described on the Numarray website at \url{http://www.stsci.edu/resources/software_hardware/numarray}.

\section{Installing PyPedal}
\label{sec:installing-pypedal}

In order to get \PYPEDAL{}, visit the official website at \url{http://sourceforge.net/projects/pypedal}.  Click on the "PyPedal" release and you will be presented with a list of the available files. The files whose names end in ".tar.gz" are source code releases. The other files are binaries for a given platform (if any are available).

It is possible to get the latest sources directly from our CVS repository using the facilities described at SourceForge. Note that while every effort is made to ensure that the repository is always ``good'', direct use of the repository is subject to more errors than using a standard release.

\subsection{Installing on Unix, Linux, and Mac OSX}
\label{sec:installing-unix}

The source distribution should be uncompressed and unpacked as follows (for
example):
\begin{verbatim}
gunzip pypedal-2.0.0a12.tar.gz
tar xf pypedal-2.0.0a12.tar.gz
\end{verbatim}
Follow the instructions in the top-level directory for compilation and installation. Note that there are options you must consider before beginning.  Installation is usually as simple as:
\begin{verbatim}
python setup.py install
\end{verbatim}
or:
\begin{verbatim}
python setupall.py install
\end{verbatim}
There are currently no extra packages for \PYPEDAL{}.

\paragraph*{Important Tip} \label{sec:tip:from-pypedal-import} Just like all Python modules and packages, the \PYPEDAL{} module can be invoked using either the \samp{import PyPedal} form, or the \samp{from PyPedal import ...} form.  Because most of the functions we'll talk about are in the numarray module, in this document, all of the code samples will assume that they have been preceded
by a statement:
\begin{verbatim}
>>> from numarray PyPedal *
\end{verbatim}


\subsection{Installing on Windows}
\label{sec:installing-windows}

To install numarray, you need to be in an account with Administrator privileges.  As a general rule, always remove (or hide) any old version of \PYPEDAL{} before installing the next version.

Please note that we have \textbf{NOT} tested \PYPEDAL{} on any Win-32 platforms!  However, \PYPEDAL{} should install and run properly on Win-32 as long as the dependencies mentioned above are satisfied.


\subsubsection{Installation from source}

\begin{enumerate}
\item Unpack the distribution: (NOTE: You may have to download an "unzipping" utility)
\begin{verbatim}
C:\> unzip PyPedal.zip 
C:\> cd PyPedal
\end{verbatim}
\item Build it using the distutils defaults:
\begin{verbatim}
C:\pyPedal> python setup.py install
\end{verbatim}
This installs \PYPEDAL{} in \texttt{C:\textbackslash{}pythonXX} where XX is the version number of your python installation, e.g. 20, 21, etc.
\end{enumerate}

\subsubsection{Installation from self-installing executable}

\begin{enumerate}
\item Click on the executable's icon to run the installer.
\item Click "next" several times.  I have not experimented with customizing the installation directory and don't recommend changing any of the installation defaults.  If you do and have problems, let us know.
\item Assuming everything else goes smoothly, click "finish".
\end{enumerate}


\subsubsection{Installation on Cygwin}

No information on installing \PYPEDAL{} on Cygwin is available.  If you manage to get it working, let us know.


\section{Testing the PyPedal Python Extension Installation}

To find out if you have correctly installed \PYPEDAL{}, type \samp{import PyPedal} at the Python prompt. You'll see one of
two behaviors (throughout this document user input and Python interpreter output will be emphasized as shown in the block below):
\begin{verbatim}
>>> import PyPedal
Traceback (innermost last):
File "<stdin>", line 1, in ?
ImportError: No module named PyPedal
\end{verbatim}
indicating that you don't have \PYPEDAL{} installed, or:
\begin{verbatim}
>>> import PyPedal
>>> PyPedal.__version__
'2.0.0a1'
\end{verbatim}
indicating that \PYPEDAL{} is installed.


\section{At the SourceForge...}
\label{sec:at-sourceforge}

The SourceForge project page for numarray is at
\url{http://sourceforge.net/projects/pyedal}. On this project page you will find
links to:
\begin{description}
\item[The PyPedal Discussion List] You can subscribe to a discussion list about \PYPEDAL{} using the project page at SourceForge. The list is a good place to ask questions and get help. Send mail to pyedal-discussion@lists.sourceforge.net.  There is also a pypedal-discussion group that you may join.
\item[The Web Site] Click on "home page" to get to the \PYPEDAL{} Home Page, which has links to documentation and other resources.
\item[Bugs and Patches] Bug tracking and patch-management facilities is provided on the SourceForge project page.
\item[FTP Site] The FTP Site contains this documentation in several formats, plus maybe some other goodies we have lying around.
\end{description}