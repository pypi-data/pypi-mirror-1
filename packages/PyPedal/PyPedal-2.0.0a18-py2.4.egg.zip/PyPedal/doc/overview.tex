\chapter{High-Level Overview}
\label{cha:high-level-overview}

\begin{quote} 
   In this chapter, a high-level overview of \PYPEDAL{} is provided, giving
   the reader the definitions of the key components of the system. This section
   defines the concepts used by the remaining sections.
\end{quote}

\section{The PyPedal Object Model}
\label{sec:pypedal-objects}

At the heart of \PyPedal{} are four different types of objects.  These objects
combine data and the code that operate on those data into one convenient package.
Although most \PyPedal{} users will only work directly with one or two of these
objects it is worthwhile to know what they are.  An instance of the
\textbf{NewPedigree} class stores a pedigree read from an input file as well as
metadata about that pedigree.  The pedigree is a Python list of \textbf{NewAnimal}
objects.  Information about the pedigree, such as the number and identity of founders,
is contained in an instance of the \textbf{PedigreeMetadata} class.

The fourth \PyPedal{} class, \textbf{New{AM}atrix}, is used to manipulate numerator
relationship matrices (NRM).  When working with large pedigrees it can take a long
time to compute the elements of a NRM, and having an easy way to save and restore
them is quite convenient.

`\constant{u1}'

Here is an example of Python code using the NewPedigree object (\texttt{examples/new_lacy.py}):
\begin{verbatim}
import pyp_newclasses, pyp_nrm. pyp_metrics
from pyp_utils import pyp_nice_time

options = {}
options['messages'] = 'verbose'
options['renumber'] = 0
options['counter'] = 5

if __name__ == '__main__':
    print 'Starting pypedal.py at %s' % (pyp_nice_time())
    
		# Example taken from Lacy (1989), Appendix A. 
    options['pedfile'] = 'new_lacy.ped'
    options['pedformat'] = 'asd'
    options['pedname'] = 'Lacy Pedigree'
    example = pyp_newclasses.NewPedigree(options)
    example.load()
    if example.kw['messages'] == 'verbose':
        print '[INFO]: Calling pyp_metrics.effective_founders_lacy at %s' % (pyp_nice_time())
    pyp_metrics.effective_founders_lacy(example)
\end{verbatim}
See section \ref{sec:tip:from-pypedal-import}.

\section{Pedigree Files}
\label{sec:pedigree-files}
Pedigree files consist of plain-text files (also know as ASCII or flatfiles) whose rows contain
records on individual animals and whose columns contain different variables.  The columns are
delimited (separated from one another) by some character such as a space or a tab (\t).  Pedigree
files may also contain comments (notes) about the pedigree that are ignored by \PyPedal{}; comments
always begin with an octothorpe (\#).  For example, the following pedigree contains records for 13
animals, and each record contains three variables (animal ID, sire ID, and dam ID):
\begin{verbatim}
# This pedigree is taken from Boichard et al. (1997).
# Each records contains an animal ID, a sire ID, and
# a dam ID.
1 0 0
2 0 0
3 0 0
4 0 0
5 2 3
6 0 0
7 5 6
8 0 0
9 1 2
10 4 5
11 7 8
12 7 8
13 7 8
\end{verbatim}
If you need to change the default column separator, which is a space (' '), set the
\texttt{sepchar} option to the desired value.  For example, if your columns are
tab-delimited you would set the option as:
\begin{verbatim}
options['sepchar'] = '\t'
\end{verbatim}
\subsection{Pedigree Format Codes}
\label{sec:pedigree-format-codes}
Pedigree format codes consisting of a string of characters are used to describe
the contents of a pedigree file.  The simplest pedigree file that can be read by \PyPedal{}
is shown above; the pedigree format for this file is \texttt{asd}.  A pedigree format is required
for reading a pedigree; there is no default code used, and \PyPedal{} wil halt with an error if you
do not specify one.  You specify the format using an option statement at the start of your program:
\begin{verbatim}
options['pedformat'] = 'asd'
\end{verbatim}
Please note that the format codes are case-sensitive, which means that 'a'
is considered to be a different character than 'A'.  The codes currently
recognized by \PyPedal{} are:
\begin{itemize}
\item a = animal (REQUIRED)
\item s = sire (REQUIRED)
\item d = dam (REQUIRED)
\item g = generation
\item x = sex
\item b = birthyear (YYYY)
\item f = inbreeding
\item r = breed
\item n = name
\item y = birthdate in "MMDDYYYY" format
\item l = alive (1) or dead (0)
\item e = age
\item A = animal ID as a string (cannot contain sepchar)
\item S = sire ID as a string (cannot contain sepchar)
\item D = dam ID as a string (cannot contain sepchar)
\item L = alleles (two alleles separated by a non-null character)
\end{itemize}
As noted, all pedigrees must contain columns corresponding to animals, sires, and dams.

Ufuncs are covered in detail in "Ufuncs" on page~\pageref{cha:ufuncs}.