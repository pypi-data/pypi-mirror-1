% Complete documentation on the extended LaTeX markup used for Python
% documentation is available in ``Documenting Python'', which is part
% of the standard documentation for Python.  It may be found online
% at:
%
%     http://www.python.org/doc/current/doc/doc.html

\documentclass[hyperref]{manual}

% latex2html doesn't know [T1]{fontenc}, so we cannot use that:(

\usepackage{amsmath}
\usepackage[latin1]{inputenc}
\usepackage{textcomp}
\usepackage{fullpage}
\usepackage{graphicx}
\usepackage{url}
\usepackage{index}
\usepackage{chicago}

% The commands of this document do not reset module names at section level
% (nor at chapter level).
% --> You have to do that manually when a new module starts!
%     (use \py@reset)
%begin{latexonly}
\makeatletter
\renewcommand{\section}{\@startsection{section}{1}{\z@}%
   {-3.5ex \@plus -1ex \@minus -.2ex}%
   {2.3ex \@plus.2ex}%
   {\reset@font\Large\py@HeaderFamily}}
\makeatother
%end{latexonly}

% additional mathematical functions
\DeclareMathOperator{\abs}{abs}

% provide a cross-linking command for the index
%begin{latexonly}
%\newcommand*\see[2]{\protect\seename #1}
%\newcommand*{\seename}{$\to$}
%end{latexonly}

% some convenience declarations
\newcommand{\pypedal}{PyPedal}
\newcommand{\PyPedal}{PyPedal}  % Only beginning of sentence, otherwise use \pypedal{}
\newcommand{\PYPEDAL}{PyPedal}
\newcommand{\python}{Python}

% mark internal comments
% for any published version switch to the second (empty) definition of the macro!
% \newcommand{\remark}[1]{(\textbf{Note to authors: #1})}
\newcommand{\remark}[1]{}

\title{PyPedal:\\Software for pedigree analysis}

\author{John B. Cole, PhD}

\authoraddress{Animal Improvement Programs Laboratory, Agricultural Research Service, USDA, Room 306 Bldg 005 BARC-West, 10300 Baltimore Avenue, Beltsville, MD 20705-2350}

% I use date to indicate the manual-updates,
% release below gives the matching software version.
\date{August 11, 2005}          % update before release!
                                % Use an explicit date so that reformatting
                                % doesn't cause a new date to be used.  Setting
                                % the date to \today can be used during draft
                                % stages to make it easier to handle versions.

\release{2.0.0a19}                 % (software) release version;
\setshortversion{2.0}         % this is used to define the \version macro

\makeindex                      % tell \index to actually write the .idx file
\newindex{func}{fdx}{fnd}{Function Index}

\begin{document}

\maketitle

% This makes the contents more accessible from the front page of the HTML.
\ifhtml
\part*{General}
\chapter*{Front Matter}
\label{front}
\fi

\section*{Legal Notice}
\label{sec:legal-notice}

Copyright (c) 2002, 2003, 2004, 2005.  John B. Cole.  All rights reserved.

Permission to use, copy, modify, and distribute this software for any purpose
without fee is hereby granted, provided that this entire notice is included in
all copies of any software which is or includes a copy or modification of this
software and in all copies of the supporting documentation for such software.

\subsection*{Disclaimer}

The author of this software does not make any warranty, express or implied, or
assume any liability or responsibility for the accuracy, completeness, or
usefulness of any information, apparatus, product, or process disclosed, or
represent that its use would not infringe privately-owned rights. Reference
herein to any specific commercial products, process, or service by trade name,
trademark, manufacturer, or otherwise, does not necessarily constitute or imply
its endorsement, recommendation, or favoring by the United States Government or
the author. The views and opinions of authors expressed herein do not necessarily
state or reflect those of the United States Government and shall not be used for
advertising or product endorsement purposes.

%% Local Variables:
%% mode: LaTeX
%% mode: auto-fill
%% fill-column: 79
%% indent-tabs-mode: nil
%% ispell-dictionary: "american"
%% reftex-fref-is-default: nil
%% TeX-auto-save: t
%% TeX-command-default: "pdfeLaTeX"
%% TeX-master: "numarray"
%% TeX-parse-self: t
%% End:

\tableofcontents
\listoftables

\declaremodule{extension}{pypedal}
\moduleauthor{John B. Cole}{jcole@aipl.arsusda.gov}
\modulesynopsis{Py{P}edal}
\input{lesser}
\chapter{Introduction}
\label{cha:introduction}

\begin{quote}
This chapter introduces the \PyPedal{} module for Python 2.4 and outlines the rest of the document.
\end{quote}

\PYPEDAL{} (\textbf{P}ython \textbf{Ped}igree An\textbf{al}ysis) is a tool for analyzing pedigree files.  It calculates several quantitative measures of allelic and genoytpic diversity from pedigrees, including average coefficients of inbreeding and relationship, effective number of founders, and effective number of ancestors.  Some qualitative checks are performed in order to catch some common mistakes, such as parents with more recent birthdates or ID numbers than their offspring.  Tools for pedigree visualization and report generation are provided.  Currently, \PYPEDAL{} only makes use of information on pedigree structure.  Allelotypes can be assigned to founders (or read from the pedigree file) for use in gene-dropping simulations to compute effective number of  founder genomes, but no other measures of alleic diversity are currently supported.

\PYPEDAL{} is a Python (\url(http://www.python.org/))language module that may be called by other Python programs or used interactively from the Python interpreter.  You must have Python 2.4 installed in order to use \PYPEDAL() as \PYPEDAL() makes use of some version-specific features found only in 2.4.  The Numarray module must be installed in order for you to use \PYPEDAL(), and may be found at \url{http://www.stsci.edu/resources/software_hardware/numarray}.

This document is the ``official'' documentation for \PyPedal{}. It includes a tutorial and is the most authoritative source of information about \PyPedal{} with the exception of the source code. The tutorial material will walk you through a set of manipulations of a simple pedigree.  All users of \PYPEDAL{} are encouraged to follow the tutorial with a working \PYPEDAL{} installation. The best way to learn is by doing --- the aim of this tutorial is to guide you along this "doing."

This content of this manual is broken down as follows:
\begin{description}
\item[License] Chapter \ref{cha:license} describes the license under which \PyPedal{} is distributed.  It is important that you review the license before using the program.
\item[Installing PyPedal] Chapter \ref{cha:installation} provides information
   on testing Python and installing PyPedal.
\item[High-Level Overview] Chapter \ref{cha:high-level-overview} gives a
   high-level overview of the components of the \PyPedal{} system as a whole.
\item[Applications Programming Interface] Chapter \ref{cha:api} includes a complete reference, including useage notes, for all functions in all \PyPedal{}. modules.
\item[PyPedal Tutorial] Chapter \ref{cha:tutorial} provides a gentle introduction to \PyPedal{}.
\item[Glossary] Chapter \ref{cha:glossary} provides a glossary of terms.
\item[References and Indices] are provided at the end of the manual.
\end{description}

\section{Implemented Features}
PyPedal is currently capable of doing the following things:
\begin{itemize}
\item Reading pedigree files in user-defined formats;
\item Checking pedigree integrity (duplicate IDs, parents younger than offspring, etc.);
\item Generating summary information such as frequency of appearance in the pedigree file;
\item Computation of the numerator relationship matrix ($A$) from a pedigree file using the tabular method;
\item Inbreeding calculations for large pedigrees;
\item Computation of average total and average individual coefficients of inbreeding and relationship;
\item Decomposition of $A$ into $T$ and $D$ such that $A=TDT'$;
\item Computation of the direct inverse of $A$ (not accounting for inbreeding) using the method of Henderson \cite{ref143};
\item Computation of the direct inverse of $A$ (accounting for inbreeding) using the method of \citeN{ref235};
\item Storage of $A$ and its inverse between user sessions as persistent Python objects using the pickle module to avoid unnecessary calculations;
\item Calculation of theoretical effective population size;
\item Calculation of actual effective population size based on the change in population average inbreeding;
\item Computation of effective founder number using the exact algorithm of \citeN{ref640};
\item Computation of effective founder number using the approximate algorithm of \citeN{ref352};
\item Computation of effective ancestor number using the algorithm of \citeN{ref352};
\item Selection of subpedigrees containing all ancestors of an animal;
\item Identification of the common relatives of two animals;
\item Output to ASCII text files, including matrices, coefficients of inbreeding and relationship, and summary information;
\item Reordering and renumbering of pedigree files.
\end{itemize}
A full list of features, including notes on useage and computational details, is provided in Chapter \ref{cha:api}.  \PYPEDAL{} has been used to perform calculations on pedigrees as large as 100,000 animals and has used in scientific research \cite{Cole2004a}.

\section{Where to get information and code}
\PYPEDAL{} and its documentation are available at: \url{http://pypedal.sourceforge.net/}. The Numarray web site is: \url{http://numpy.sourceforge.net/}. The Python web site is \url{http://www.python.org/}.

\section{Acknowledgments}
\PYPEDAL{} was initially written to support the author's dissertation research while at Louisiana State University, Baton Rouge (\url{http://www.lsu.edu/}).  It lay fallow for some time but has recently come under active development again.  This is due in part to a request from colleagues at the University of Minnesota that led to the inclusion of new functionality in \PYPEDAL{}.  The author wishes to thank Dr. Paul Van{R}aden for very helpful suggestions for improving the ability of \PYPEDAL{} to handle certain computations in very large pedigrees.  Additional feedback in the form of bug reports, feature requests, and discussion of computing strategies was provided by Edward H. Hagen (Institute for Theoretical Biology, Humboldt-Universit�zu Berlin), Kathy Hanford (University of Nebraska, Lincoln), Thomas von Hassell, and Gianluca Saba.
\chapter{Installing PyPedal}
\label{cha:installation}

\begin{quote}
   This chapter explains how to install and test \PYPEDAL{} from either the source distribution or from the binary distribution.
\end{quote}

Before we can begin the tutorial, we need to make sure that you can install and test Python, the Numeric or Numarray extension, and the \PYPEDAL{} extension.

\section{Testing the Python installation}

The first step is to install Python if you haven't already. Python is available from the Python project page at \url{http://sourceforge.net/projects/python/}.  Click on the link corresponding to your platform, and follow the instructions
described there. \PYPEDAL{} requires version 2.3 as a minimum.  When installed, starting Python by typing python at the shell or double-clicking on the Python interpreter should give a prompt such as:
\begin{verbatim}
Python 2.3.3 (#2, Feb 17 2004, 11:45:40)
[GCC 3.3.2 (Mandrake Linux 10.0 3.3.2-6mdk)] on linux2
Type "help", "copyright", "credits" or "license" for more information.
\end{verbatim}
If you have problems getting Python to work, consider contacting your local support person or e-mailing \ulink{python-help@python.org}{mailto:python-help@python.org} for help. If neither solution works, consider posting on the
\ulink{comp.lang.python}{news:comp.lang.python} newsgroup (details on the newsgroup/mailing list are available at
\url{http://www.python.org/psa/MailingLists.html\#clp}).


\section{Testing the Numarray Python Extension Installation}

The standard Python distribution does not come, as of this writing, with the
numarray Python extensions installed, but your system administrator may have
installed them already. To find out if your Python interpreter has numarray
installed, type \samp{import numarray} at the Python prompt. You'll see one of
two behaviors (throughout this document user input and python interpreter
output will be emphasized as shown in the block below):
\begin{verbatim}
>>> import numarray
Traceback (innermost last):
File "<stdin>", line 1, in ?
ImportError: No module named numarray
\end{verbatim}
indicating that you don't have numarray installed, or:
\begin{verbatim}
>>> import numarray
>>> numarray.__version__
'0.9'
\end{verbatim}
indicating that numarray is installed. If it is installed, you can skip the next section and go ahead to section \ref{sec:installing-pypedal}.  If you don't, you have to get and install the numarray extensions as described on the Numarray website at \url{http://www.stsci.edu/resources/software_hardware/numarray}.

\section{Installing PyPedal}
\label{sec:installing-pypedal}

In order to get \PYPEDAL{}, visit the official website at \url{http://sourceforge.net/projects/pypedal}.  Click on the "PyPedal" release and you will be presented with a list of the available files. The files whose names end in ".tar.gz" are source code releases. The other files are binaries for a given platform (if any are available).

It is possible to get the latest sources directly from our CVS repository using the facilities described at SourceForge. Note that while every effort is made to ensure that the repository is always ``good'', direct use of the repository is subject to more errors than using a standard release.

\subsection{Installing on Unix, Linux, and Mac OSX}
\label{sec:installing-unix}

The source distribution should be uncompressed and unpacked as follows (for
example):
\begin{verbatim}
gunzip pypedal-2.0.0a12.tar.gz
tar xf pypedal-2.0.0a12.tar.gz
\end{verbatim}
Follow the instructions in the top-level directory for compilation and installation. Note that there are options you must consider before beginning.  Installation is usually as simple as:
\begin{verbatim}
python setup.py install
\end{verbatim}
or:
\begin{verbatim}
python setupall.py install
\end{verbatim}
There are currently no extra packages for \PYPEDAL{}.

\paragraph*{Important Tip} \label{sec:tip:from-pypedal-import} Just like all Python modules and packages, the \PYPEDAL{} module can be invoked using either the \samp{import PyPedal} form, or the \samp{from PyPedal import ...} form.  Because most of the functions we'll talk about are in the numarray module, in this document, all of the code samples will assume that they have been preceded
by a statement:
\begin{verbatim}
>>> from numarray PyPedal *
\end{verbatim}


\subsection{Installing on Windows}
\label{sec:installing-windows}

To install numarray, you need to be in an account with Administrator privileges.  As a general rule, always remove (or hide) any old version of \PYPEDAL{} before installing the next version.

Please note that we have \textbf{NOT} tested \PYPEDAL{} on any Win-32 platforms!  However, \PYPEDAL{} should install and run properly on Win-32 as long as the dependencies mentioned above are satisfied.


\subsubsection{Installation from source}

\begin{enumerate}
\item Unpack the distribution: (NOTE: You may have to download an "unzipping" utility)
\begin{verbatim}
C:\> unzip PyPedal.zip 
C:\> cd PyPedal
\end{verbatim}
\item Build it using the distutils defaults:
\begin{verbatim}
C:\pyPedal> python setup.py install
\end{verbatim}
This installs \PYPEDAL{} in \texttt{C:\textbackslash{}pythonXX} where XX is the version number of your python installation, e.g. 20, 21, etc.
\end{enumerate}

\subsubsection{Installation from self-installing executable}

\begin{enumerate}
\item Click on the executable's icon to run the installer.
\item Click "next" several times.  I have not experimented with customizing the installation directory and don't recommend changing any of the installation defaults.  If you do and have problems, let us know.
\item Assuming everything else goes smoothly, click "finish".
\end{enumerate}


\subsubsection{Installation on Cygwin}

No information on installing \PYPEDAL{} on Cygwin is available.  If you manage to get it working, let us know.


\section{Testing the PyPedal Python Extension Installation}

To find out if you have correctly installed \PYPEDAL{}, type \samp{import PyPedal} at the Python prompt. You'll see one of
two behaviors (throughout this document user input and Python interpreter output will be emphasized as shown in the block below):
\begin{verbatim}
>>> import PyPedal
Traceback (innermost last):
File "<stdin>", line 1, in ?
ImportError: No module named PyPedal
\end{verbatim}
indicating that you don't have \PYPEDAL{} installed, or:
\begin{verbatim}
>>> import PyPedal
>>> PyPedal.__version__
'2.0.0a1'
\end{verbatim}
indicating that \PYPEDAL{} is installed.


\section{At the SourceForge...}
\label{sec:at-sourceforge}

The SourceForge project page for numarray is at
\url{http://sourceforge.net/projects/pyedal}. On this project page you will find
links to:
\begin{description}
\item[The PyPedal Discussion List] You can subscribe to a discussion list about \PYPEDAL{} using the project page at SourceForge. The list is a good place to ask questions and get help. Send mail to pyedal-discussion@lists.sourceforge.net.  There is also a pypedal-discussion group that you may join.
\item[The Web Site] Click on "home page" to get to the \PYPEDAL{} Home Page, which has links to documentation and other resources.
\item[Bugs and Patches] Bug tracking and patch-management facilities is provided on the SourceForge project page.
\item[FTP Site] The FTP Site contains this documentation in several formats, plus maybe some other goodies we have lying around.
\end{description}
\chapter{High-Level Overview}
\label{cha:high-level-overview}

\begin{quote} 
   In this chapter, a high-level overview of \PYPEDAL{} is provided, giving
   the reader the definitions of the key components of the system. This section
   defines the concepts used by the remaining sections.
\end{quote}

\section{The PyPedal Object Model}
\label{sec:pypedal-objects}

At the heart of \PyPedal{} are four different types of objects.  These objects
combine data and the code that operate on those data into one convenient package.
Although most \PyPedal{} users will only work directly with one or two of these
objects it is worthwhile to know what they are.  An instance of the
\textbf{NewPedigree} class stores a pedigree read from an input file as well as
metadata about that pedigree.  The pedigree is a Python list of \textbf{NewAnimal}
objects.  Information about the pedigree, such as the number and identity of founders,
is contained in an instance of the \textbf{PedigreeMetadata} class.

The fourth \PyPedal{} class, \textbf{New{AM}atrix}, is used to manipulate numerator
relationship matrices (NRM).  When working with large pedigrees it can take a long
time to compute the elements of a NRM, and having an easy way to save and restore
them is quite convenient.

`\constant{u1}'

Here is an example of Python code using the NewPedigree object (\texttt{examples/new_lacy.py}):
\begin{verbatim}
import pyp_newclasses, pyp_nrm. pyp_metrics
from pyp_utils import pyp_nice_time

options = {}
options['messages'] = 'verbose'
options['renumber'] = 0
options['counter'] = 5

if __name__ == '__main__':
    print 'Starting pypedal.py at %s' % (pyp_nice_time())
    
		# Example taken from Lacy (1989), Appendix A. 
    options['pedfile'] = 'new_lacy.ped'
    options['pedformat'] = 'asd'
    options['pedname'] = 'Lacy Pedigree'
    example = pyp_newclasses.NewPedigree(options)
    example.load()
    if example.kw['messages'] == 'verbose':
        print '[INFO]: Calling pyp_metrics.effective_founders_lacy at %s' % (pyp_nice_time())
    pyp_metrics.effective_founders_lacy(example)
\end{verbatim}
See section \ref{sec:tip:from-pypedal-import}.

\section{Pedigree Files}
\label{sec:pedigree-files}
Pedigree files consist of plain-text files (also know as ASCII or flatfiles) whose rows contain
records on individual animals and whose columns contain different variables.  The columns are
delimited (separated from one another) by some character such as a space or a tab (\t).  Pedigree
files may also contain comments (notes) about the pedigree that are ignored by \PyPedal{}; comments
always begin with an octothorpe (\#).  For example, the following pedigree contains records for 13
animals, and each record contains three variables (animal ID, sire ID, and dam ID):
\begin{verbatim}
# This pedigree is taken from Boichard et al. (1997).
# Each records contains an animal ID, a sire ID, and
# a dam ID.
1 0 0
2 0 0
3 0 0
4 0 0
5 2 3
6 0 0
7 5 6
8 0 0
9 1 2
10 4 5
11 7 8
12 7 8
13 7 8
\end{verbatim}
If you need to change the default column separator, which is a space (' '), set the
\texttt{sepchar} option to the desired value.  For example, if your columns are
tab-delimited you would set the option as:
\begin{verbatim}
options['sepchar'] = '\t'
\end{verbatim}
\subsection{Pedigree Format Codes}
\label{sec:pedigree-format-codes}
Pedigree format codes consisting of a string of characters are used to describe
the contents of a pedigree file.  The simplest pedigree file that can be read by \PyPedal{}
is shown above; the pedigree format for this file is \texttt{asd}.  A pedigree format is required
for reading a pedigree; there is no default code used, and \PyPedal{} wil halt with an error if you
do not specify one.  You specify the format using an option statement at the start of your program:
\begin{verbatim}
options['pedformat'] = 'asd'
\end{verbatim}
Please note that the format codes are case-sensitive, which means that 'a'
is considered to be a different character than 'A'.  The codes currently
recognized by \PyPedal{} are:
\begin{itemize}
\item a = animal (REQUIRED)
\item s = sire (REQUIRED)
\item d = dam (REQUIRED)
\item g = generation
\item x = sex
\item b = birthyear (YYYY)
\item f = inbreeding
\item r = breed
\item n = name
\item y = birthdate in "MMDDYYYY" format
\item l = alive (1) or dead (0)
\item e = age
\item A = animal ID as a string (cannot contain sepchar)
\item S = sire ID as a string (cannot contain sepchar)
\item D = dam ID as a string (cannot contain sepchar)
\item L = alleles (two alleles separated by a non-null character)
\end{itemize}
As noted, all pedigrees must contain columns corresponding to animals, sires, and dams.

Ufuncs are covered in detail in "Ufuncs" on page~\pageref{cha:ufuncs}.
\chapter{API}
\label{cha:api}

\begin{quote}
This chapter provides an overview of the \PYPEDAL{} Application Programming Interface (API).  More simply, it is a reference to the various classes, methods, and procedures that make up the \PYPEDAL{} module.
\end{quote}

\section{Some Background}
Using the \PyPedal{} API is quite simple.  The following discussion assumes that you have imported each of the Python modules using, e.g., \texttt{import pyp_utils} rather than \texttt{from pyp_utils import *}.  The latter is poor style and can result in namespace pollution; this is not known to be a problem with \PyPedal{}, but I offer no guarantees that this will remain so.  In order to access a function in the \texttt{pyp_utils} module, such as \texttt{pyp_nice_time()}, you use a dotted notation with a '.' separating the module name and the function name.  (I can't help but feel that I am being somewhat sloppy in conflating modules and libraries, but there you are.)  Anyway, on to an example:
\begin{verbatim}
[jcole@aipl440 jcole]$ python
Python 2.4 (#1, Feb 25 2005, 12:30:11)
[GCC 3.3.3] on linux2
Type "help", "copyright", "credits" or "license" for more information.
>>> import pyp_utils
>>> pyp_utils.pyp_nice_time()
'Mon Aug 15 16:27:38 2005'
\end{verbatim}

\section{pyp\_db}
\input{pyp-db}

\section{pyp\_demog}
\input{pyp-demog}

\section{pyp\_graphics}
\input{pyp-graphics}

\section{pyp\_io}
\input{pyp-io}

\section{pyp\_metrics}
\input{pyp-metrics}

\section{pyp\_newclasses}
\input{pyp-newclasses}

\section{pyp\_nrm}
\input{pyp-nrm}

\section{pyp\_reports}
\input{pyp-reports}

\section{pyp\_utils}
\input{pyp-utils}
\input{tutorial}
\chapter{Glossary}
\label{cha:glossary}
\begin{quote}
This chapter provides a glossary of terms.\footnote{Please let me know of any additions to this list which
you feel would be helpful.}
\end{quote}
\begin{description}
\item[coefficient of inbreeding] Probability that two alleles selected at random are identical by descent.
\end{description}
\begin{description}
\item[coefficient of relationship] Proportion of genes that two individuals share on average.
\end{description}
\begin{description}
\item[effective ancestor number] The number of equally-contributing ancestors, not necessarily founders, needed to produce a population with the heterozygosity of the studied population \cite{ref352}.
\end{description}
\begin{description}
\item[effective founder number] The number of equally-contributing founders needed to produce a population with the heterozygosity of the studied population \cite{ref640}.
\end{description}
\begin{description}
\item[effective population size] The effective population size is the size of an ideal population that would lose heterozygosity at a rate equal to that of the studied population \cite{ref91}.
\end{description}
\begin{description}
\item[founder] An animal with unknown parents that is assumed to be unrelated to all other founders.
\end{description}
\begin{description}
\item[internal report] A \PyPedal() report that is intended for use by other \PyPedal() procedures, such as plotting
routines, and not for printing.
\end{description}
\begin{description}
\item[numerator relationship matrix] Matrix of additive genetic covariances among the animals in a population.
\end{description}
\begin{description}
\item[pedigree] A \PYPEDAL{} pedigree consists of a Python list containing instances of \PYPEDAL{} NewAnimal{} objects.
\end{description}
\begin{description}
\item[renumbering] Many calculations require that the animals in a pedigree be ordered from oldest to youngest, with sires and dams preceding offspring, and renumbered  starting with 1.  This is a computational necessity, and results in an animal's ID (\texttt{animalID}) being changed to reflect that animal's order in the pedigree.  All animals have their original IDs stored in their \texttt{originalName} attribute.
\end{description}

\bibliographystyle{chicago}
\bibliography{references}

\printindex[func]
\printindex

\end{document}