\documentclass[10pt]{article}
\usepackage{fullpage, graphicx, url}
\setlength{\parskip}{1ex}
\setlength{\parindent}{0ex}
\title{The pyp\_newclasses Module}
\begin{document}
\section*{The pyp\_newclasses Module}


 pyp\_newclasses contains the new class structure that will be a part of PyPedal 2.0.0Final. It includes a master class to which most of the computational routines will be bound as methods, a NewAnimal() class, and a PedigreeMetadata() class.
\subsection*{Module Contents}
\begin{description}
\item[\textbf{NewAMatrix(kw)}
 (class) [\#]]

 NewAMatrix provides an instance of a numerator relationship matrix as a Numarray array of floats with some convenience methods.


 For more information about this class, see \emph{The NewAMatrix Class}
.

\item[\textbf{NewAnimal(locations, data, mykw)}
 (class) [\#]]

 The NewAnimal() class is holds animals records read from a pedigree file.


 For more information about this class, see \emph{The NewAnimal Class}
.

\item[\textbf{NewPedigree(kw)}
 (class) [\#]]

 The NewPedigree class is the main data structure for PyP 2.0.0Final.


 For more information about this class, see \emph{The NewPedigree Class}
.

\item[\textbf{PedigreeMetadata(myped, kw)}
 (class) [\#]]

 The PedigreeMetadata() class stores metadata about pedigrees.


 For more information about this class, see \emph{The PedigreeMetadata Class}
.


\end{description}
\subsection*{The NewAMatrix Class}
\begin{description}
\item[\textbf{NewAMatrix(kw)}
 (class) [\#]]

 NewAMatrix provides an instance of a numerator relationship matrix as a Numarray array of floats with some convenience methods. The idea here is to provide a wrapper around a NRM so that it is easier to work with. For large pedigrees it can take a long time to compute the elements of A, so there is real value in providing an easy way to save and retrieve a NRM once it has been formed.

\item[\textbf{form\_a\_matrix(pedigree)}
 ⇒ integer [\#]]

 form\_a\_matrix() calls pyp\_nrm/fast\_a\_matrix() or pyp\_nrm/fast\_a\_matrix\_r() to form a NRM from a pedigree.
\begin{description}
\item[\emph{pedigree}
] The pedigree used to form the NRM.
\item[Returns:] A NRM on success, 0 on failure.

\end{description}
\\ 

\item[\textbf{info()}
 ⇒ None [\#]]

 info() uses the info() method of Numarray arrays to dump some information about the NRM. This is of use predominantly for debugging.
\begin{description}
\item[\emph{None}
]
\item[Returns:] None

\end{description}
\\ 

\item[\textbf{load(nrm\_filename)}
 ⇒ integer [\#]]

 load() uses the Numarray Array Function ``fromfile()'' to load an array from a binary file. If the load is successful, self.nrm contains the matrix.
\begin{description}
\item[\emph{nrm\_filename}
] The file from which the matrix should be read.
\item[Returns:] A load status indicator (0: failed, 1: success).

\end{description}
\\ 

\item[\textbf{save(nrm\_filename)}
 ⇒ integer [\#]]

 save() uses the Numarray method ``tofile()'' to save an array to a binary file.
\begin{description}
\item[\emph{nrm\_filename}
] The file to which the matrix should be written.
\item[Returns:] A save status indicator (0: failed, 1: success).

\end{description}
\\ 


\end{description}
\subsection*{The NewAnimal Class}
\begin{description}
\item[\textbf{NewAnimal(locations, data, mykw)}
 (class) [\#]]

 The NewAnimal() class is holds animals records read from a pedigree file.

\item[\textbf{\_\_init\_\_(locations, data, mykw)}
 ⇒ object [\#]]

 \_\_init\_\_() initializes a NewAnimal() object.
\begin{description}
\item[\emph{locations}
] A dictionary containing the locations of variables in the input line.
\item[\emph{data}
] The line of input read from the pedigree file.
\item[Returns:] An instance of a NewAnimal() object populated with data

\end{description}
\\ 

\item[\textbf{pad\_id()}
 ⇒ integer [\#]]

 pad\_id() takes an Animal ID, pads it to fifteen digits, and prepends the birthyear (or 1950 if the birth year is unknown). The order of elements is: birthyear, animalID, count of zeros, zeros.
\begin{description}
\item[\emph{self}
] Reference to the current Animal() object
\item[Returns:] A padded ID number that is supposed to be unique across animals

\end{description}
\\ 

\item[\textbf{printme()}
 [\#]]

 printme() prints a summary of the data stored in the Animal() object.
\begin{description}
\item[\emph{self}
] Reference to the current Animal() object

\end{description}
\\ 

\item[\textbf{string\_to\_int(idstring)}
 [\#]]

 string\_to\_int() takes an Animal/Sire/Dam ID as a string and returns a string that can be represented as an integer by replacing each character in the string with its corresponding ASCII table value.

\item[\textbf{stringme()}
 [\#]]

 stringme() returns a summary of the data stored in the Animal() object as a string.
\begin{description}
\item[\emph{self}
] Reference to the current Animal() object

\end{description}
\\ 

\item[\textbf{trap()}
 [\#]]

 trap() checks for common errors in Animal() objects
\begin{description}
\item[\emph{self}
] Reference to the current Animal() object

\end{description}
\\ 


\end{description}
\subsection*{The NewPedigree Class}
\begin{description}
\item[\textbf{NewPedigree(kw)}
 (class) [\#]]

 The NewPedigree class is the main data structure for PyP 2.0.0Final.

\item[\textbf{load(pedsource='file')}
 ⇒ None [\#]]

 load() wraps several processes useful for loading and preparing a pedigree for use in an analysis, including reading the animals into a list of animal objects, forming lists of sires and dams, checking for common errors, setting ancestor flags, and renumbering the pedigree.
\begin{description}
\item[\emph{renum}
] Flag to indicate whether or not the pedigree is to be renumbered.
\item[\emph{alleles}
] Flag to indicate whether or not pyp\_metrics/effective\_founder\_genomes() should be called for a single round to assign alleles.
\item[Returns:] None

\end{description}
\\ 

\item[\textbf{preprocess()}
 ⇒ None [\#]]

 preprocess() processes a pedigree file, which includes reading the animals into a list of animal objects, forming lists of sires and dams, and checking for common errors.
\begin{description}
\item[\emph{None}
]
\item[Returns:] None

\end{description}
\\ 

\item[\textbf{renumber()}
 ⇒ None [\#]]

 renumber() updates the ID map after a pedigree has been renumbered so that all references are to renumbered rather than original IDs.
\begin{description}
\item[\emph{None}
]
\item[Returns:] None

\end{description}
\\ 

\item[\textbf{save(filename='', outformat='o', idformat='o')}
 ⇒ integer [\#]]

 save() writes a PyPedal pedigree to a user-specified file. The saved pedigree includes all fields recognized by PyPedal, not just the original fields read from the input pedigree file.
\begin{description}
\item[\emph{filename}
] The file to which the pedigree should be written.
\item[\emph{outformat}
] The format in which the pedigree should be written: 'o' for original (as read) and 'l' for long version (all available variables).
\item[\emph{idformat}
] Write 'o' (original) or 'r' (renumbered) animal, sire, and dam IDs.
\item[Returns:] A save status indicator (0: failed, 1: success)

\end{description}
\\ 

\item[\textbf{updateidmap()}
 ⇒ None [\#]]

 updateidmap() updates the ID map after a pedigree has been renumbered so that all references are to renumbered rather than original IDs.
\begin{description}
\item[\emph{None}
]
\item[Returns:] None

\end{description}
\\ 


\end{description}
\subsection*{The PedigreeMetadata Class}
\begin{description}
\item[\textbf{PedigreeMetadata(myped, kw)}
 (class) [\#]]

 The PedigreeMetadata() class stores metadata about pedigrees. Hopefully this will help improve performance in some procedures, as well as provide some useful summary data.

\item[\textbf{\_\_init\_\_(myped, kw)}
 ⇒ object [\#]]

 \_\_init\_\_() initializes a PedigreeMetadata object.
\begin{description}
\item[\emph{self}
] Reference to the current Pedigree() object
\item[\emph{myped}
] A PyPedal pedigree.
\item[\emph{kw}
] A dictionary of options.
\item[Returns:] An instance of a Pedigree() object populated with data

\end{description}
\\ 

\item[\textbf{fileme()}
 [\#]]

 fileme() writes the metada stored in the Pedigree() object to disc.
\begin{description}
\item[\emph{self}
] Reference to the current Pedigree() object

\end{description}
\\ 

\item[\textbf{nud()}
 ⇒ integer-and-list [\#]]

 nud() returns the number of unique dams in the pedigree along with a list of the dams
\begin{description}
\item[\emph{self}
] Reference to the current Pedigree() object
\item[Returns:] The number of unique dams in the pedigree and a list of those dams

\end{description}
\\ 

\item[\textbf{nuf()}
 ⇒ integer-and-list [\#]]

 nuf() returns the number of unique founders in the pedigree along with a list of the founders
\begin{description}
\item[\emph{self}
] Reference to the current Pedigree() object
\item[Returns:] The number of unique founders in the pedigree and a list of those founders

\end{description}
\\ 

\item[\textbf{nug()}
 ⇒ integer-and-list [\#]]

 nug() returns the number of unique generations in the pedigree along with a list of the generations
\begin{description}
\item[\emph{self}
] Reference to the current Pedigree() object
\item[Returns:] The number of unique generations in the pedigree and a list of those generations

\end{description}
\\ 

\item[\textbf{nus()}
 ⇒ integer-and-list [\#]]

 nus() returns the number of unique sires in the pedigree along with a list of the sires
\begin{description}
\item[\emph{self}
] Reference to the current Pedigree() object
\item[Returns:] The number of unique sires in the pedigree and a list of those sires

\end{description}
\\ 

\item[\textbf{nuy()}
 ⇒ integer-and-list [\#]]

 nuy() returns the number of unique birthyears in the pedigree along with a list of the birthyears
\begin{description}
\item[\emph{self}
] Reference to the current Pedigree() object
\item[Returns:] The number of unique birthyears in the pedigree and a list of those birthyears

\end{description}
\\ 

\item[\textbf{printme()}
 [\#]]

 printme() prints a summary of the metadata stored in the Pedigree() object.
\begin{description}
\item[\emph{self}
] Reference to the current Pedigree() object

\end{description}
\\ 

\item[\textbf{stringme()}
 [\#]]

 stringme() returns a summary of the metadata stored in the pedigree as a string.
\begin{description}
\item[\emph{self}
] Reference to the current Pedigree() object

\end{description}
\\ 


\end{description}

\end{document}
