\documentclass[10pt]{article}
\usepackage{fullpage, graphicx, url}
\setlength{\parskip}{1ex}
\setlength{\parindent}{0ex}
\title{The pyp\_nrm Module}
\begin{document}
\section*{The pyp\_nrm Module}


 pyp\_nrm contains several procedures for computing numerator relationship matrices and for performing operations on those matrices. It also contains routines for computing CoI on large pedigrees using the recursive method of VanRaden (1992).
\subsection*{Module Contents}
\begin{description}
\item[\textbf{a\_decompose(pedobj)}
 ⇒ matrices [\#]]

 Form the decomposed form of A, TDT', directly from a pedigree (after Henderson, 1976; Thompson, 1977; Mrode, 1996). Return D, a diagonal matrix, and T, a lower triagular matrix such that A = TDT'.
\begin{description}
\item[\emph{pedobj}
] A PyPedal pedigree object.
\item[Returns:] A diagonal matrix, D, and a lower triangular matrix, T.

\end{description}
\\ 

\item[\textbf{a\_inverse\_df(pedobj)}
 ⇒ matrix [\#]]

 Directly form the inverse of A from the pedigree file - accounts for inbreeding - using the method of Quaas (1976).
\begin{description}
\item[\emph{pedobj}
] A PyPedal pedigree object.
\item[Returns:] The inverse of the NRM, A, accounting for inbreeding.

\end{description}
\\ 

\item[\textbf{a\_inverse\_dnf(pedobj, filetag='\_a\_inverse\_dnf\_')}
 ⇒ matrix [\#]]

 Form the inverse of A directly using the method of Henderson (1976) which does not account for inbreeding.
\begin{description}
\item[\emph{pedobj}
] A PyPedal pedigree object.
\item[Returns:] The inverse of the NRM, A, not accounting for inbreeding.

\end{description}
\\ 

\item[\textbf{a\_matrix(pedobj, save=0)}
 ⇒ array [\#]]

 a\_matrix() is used to form a numerator relationship matrix from a pedigree. DEPRECATED. use fast\_a\_matrix() instead.
\begin{description}
\item[\emph{pedobj}
] A PyPedal pedigree object.
\item[\emph{save}
] Flag to indicate whether or not the relationship matrix is written to a file.
\item[Returns:] The NRM as a numarray matrix.

\end{description}
\\ 

\item[\textbf{fast\_a\_matrix(pedigree, pedopts, save=0, method='dense')}
 ⇒ matrix [\#]]

 Form a numerator relationship matrix from a pedigree. fast\_a\_matrix() is a hacked version of a\_matrix() modified to try and improve performance. Lists of animal, sire, and dam IDs are formed and accessed rather than myped as it is much faster to access a member of a simple list rather than an attribute of an object in a list. Further note that only the diagonal and upper off-diagonal of A are populated. This is done to save n(n+1) / 2 matrix writes. For a 1000-element array, this saves 500,500 writes.
\begin{description}
\item[\emph{pedigree}
] A PyPedal pedigree.
\item[\emph{pedopts}
] PyPedal options.
\item[\emph{save}
] Flag to indicate whether or not the relationship matrix is written to a file.
\item[\emph{method}
] Use dense or sparse matrix storage.
\item[Returns:] The NRM as Numarray matrix.

\end{description}
\\ 

\item[\textbf{fast\_a\_matrix\_r(pedigree, pedopts, save=0, method='dense')}
 ⇒ matrix [\#]]

 Form a relationship matrix from a pedigree. fast\_a\_matrix\_r() differs from fast\_a\_matrix() in that the coefficients of relationship are corrected for the inbreeding of the parents.
\begin{description}
\item[\emph{pedobj}
] A PyPedal pedigree object.
\item[\emph{save}
] Flag to indicate whether or not the relationship matrix is written to a file.
\item[Returns:] A relationship as Numarray matrix.

\end{description}
\\ 

\item[\textbf{form\_d\_nof(pedobj)}
 ⇒ matrix [\#]]

 Form the diagonal matrix, D, used in decomposing A and forming the direct inverse of A. This function does not write output to a file - if you need D in a file, use the a\_decompose() function. form\_d() is a convenience function used by other functions. Note that inbreeding is not considered in the formation of D.
\begin{description}
\item[\emph{pedobj}
] A PyPedal pedigree object.
\item[Returns:] A diagonal matrix, D.

\end{description}
\\ 

\item[\textbf{inbreeding(pedobj, method='tabular')}
 ⇒ dictionary [\#]]

 inbreeding() is a proxy function used to dispatch pedigrees to the appropriate function for computing CoI. By default, small pedigrees $<$ 10,000 animals) are processed with the tabular method directly. For larger pedigrees, or if requested, the recursive method of VanRaden (1992) is used. 
\begin{description}
\item[\emph{pedobj}
] A PyPedal pedigree object.
\item[\emph{method}
] Keyword indicating which method of computing CoI should be used (tabular|vanraden).
\item[Returns:] A dictionary of CoI keyed to renumbered animal IDs.

\end{description}
\\ 

\item[\textbf{inbreeding\_tabular(pedobj)}
 ⇒ dictionary [\#]]

 inbreeding\_tabular() computes CoI using the tabular method by calling fast\_a\_matrix() to form the NRM directly. In order for this routine to return successfully requires that you are able to allocate a matrix of floats of dimension len(myped)**2.
\begin{description}
\item[\emph{pedobj}
] A PyPedal pedigree object.
\item[Returns:] A dictionary of CoI keyed to renumbered animal IDs

\end{description}
\\ 

\item[\textbf{inbreeding\_vanraden(pedobj, cleanmaps=1)}
 ⇒ dictionary [\#]]

 inbreeding\_vanraden() uses VanRaden's (1992) method for computing coefficients of inbreeding in a large pedigree. The method works as follows: 1. Take a large pedigree and order it from youngest animal to oldest (n, n-1, ..., 1); 2. Recurse through the pedigree to find all of the ancestors of that animal n; 3. Reorder and renumber that ``subpedigree''; 4. Compute coefficients of inbreeding for that ``subpedigree'' using the tabular method (Emik and Terrill, 1949); 5. Put the coefficients of inbreeding in a dictionary; 6. Repeat 2 - 5 for animals n-1 through 1; the process is slowest for the early pedigrees and fastest for the later pedigrees.
\begin{description}
\item[\emph{pedobj}
] A PyPedal pedigree object.
\item[\emph{cleanmaps}
] Flag to denote whether or not subpedigree ID maps should be deleted after they are used (0|1)
\item[Returns:] A dictionary of CoI keyed to renumbered animal IDs

\end{description}
\\ 

\item[\textbf{recurse\_pedigree(pedobj, anid, \_ped)}
 ⇒ list [\#]]

 recurse\_pedigree() performs the recursion needed to build the subpedigrees used by inbreeding\_vanraden(). For the animal with animalID anid recurse\_pedigree() will recurse through the pedigree myped and add references to the relatives of anid to the temporary pedigree, \_ped.
\begin{description}
\item[\emph{pedobj}
] A PyPedal pedigree.
\item[\emph{anid}
] The ID of the animal whose relatives are being located.
\item[\emph{\_ped}
] A temporary PyPedal pedigree that stores references to relatives of anid.
\item[Returns:] A list of references to the relatives of anid contained in myped.

\end{description}
\\ 

\item[\textbf{recurse\_pedigree\_idonly(pedobj, anid, \_ped)}
 ⇒ list [\#]]

 recurse\_pedigree\_idonly() performs the recursion needed to build subpedigrees.
\begin{description}
\item[\emph{pedobj}
] A PyPedal pedigree.
\item[\emph{anid}
] The ID of the animal whose relatives are being located.
\item[\emph{\_ped}
] A PyPedal list that stores the animalIDs of relatives of anid.
\item[Returns:] A list of animalIDs of the relatives of anid contained in myped.

\end{description}
\\ 

\item[\textbf{recurse\_pedigree\_idonly\_side(pedobj, anid, \_ped, side='s')}
 ⇒ list [\#]]

 recurse\_pedigree\_idonly\_side() performs the recursion needed to build a subpedigree containing only animal IDs for either all sires or all dams. That is, a pedigree would go sire-paternal grandsire-paternal great-grandsire, etc.
\begin{description}
\item[\emph{pedobj}
] A PyPedal pedigree.
\item[\emph{anid}
] The ID of the animal whose relatives are being located.
\item[\emph{\_ped}
] A PyPedal list that stores the animalIDs of relatives of anid.
\item[\emph{side}
] The side of the pedigree to follow ('s'|'d').
\item[Returns:] A list of animalIDs of the relatives of anid contained in myped.

\end{description}
\\ 

\item[\textbf{recurse\_pedigree\_n(pedobj, anid, \_ped, depth=3)}
 ⇒ list [\#]]

 recurse\_pedigree\_n() recurses to build a pedigree of depth n. A depth less than 1 returns the animal whose relatives were to be identified.
\begin{description}
\item[\emph{pedobj}
] A PyPedal pedigree.
\item[\emph{anid}
] The ID of the animal whose relatives are being located.
\item[\emph{\_ped}
] A temporary PyPedal pedigree that stores references to relatives of anid.
\item[\emph{depth}
] The depth of the pedigree to return.
\item[Returns:] A list of references to the relatives of anid contained in myped.

\end{description}
\\ 

\item[\textbf{recurse\_pedigree\_onesided(pedobj, anid, \_ped, side)}
 ⇒ list [\#]]

 recurse\_pedigree\_onsided() recurses to build a subpedigree from either the sire or dam side of a pedigree.
\begin{description}
\item[\emph{pedobj}
] A PyPedal pedigree.
\item[\emph{side}
] The side to build: 's' for sire and 'd' for dam.
\item[\emph{anid}
] The ID of the animal whose relatives are being located.
\item[\emph{\_ped}
] A temporary PyPedal pedigree that stores references to relatives of anid.
\item[Returns:] A list of references to the relatives of anid contained in myped.

\end{description}
\\ 


\end{description}

\end{document}
