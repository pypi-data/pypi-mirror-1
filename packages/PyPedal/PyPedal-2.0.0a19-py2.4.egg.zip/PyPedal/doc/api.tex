\chapter{API}
\label{cha:api}

\begin{quote}
This chapter provides an overview of the \PYPEDAL{} Application Programming Interface (API).  More simply, it is a reference to the various classes, methods, and procedures that make up the \PYPEDAL{} module.
\end{quote}

\section{Some Background}
Using the \PyPedal{} API is quite simple.  The following discussion assumes that you have imported each of the Python modules using, e.g., \texttt{import pyp_utils} rather than \texttt{from pyp_utils import *}.  The latter is poor style and can result in namespace pollution; this is not known to be a problem with \PyPedal{}, but I offer no guarantees that this will remain so.  In order to access a function in the \texttt{pyp_utils} module, such as \texttt{pyp_nice_time()}, you use a dotted notation with a '.' separating the module name and the function name.  (I can't help but feel that I am being somewhat sloppy in conflating modules and libraries, but there you are.)  Anyway, on to an example:
\begin{verbatim}
[jcole@aipl440 jcole]$ python
Python 2.4 (#1, Feb 25 2005, 12:30:11)
[GCC 3.3.3] on linux2
Type "help", "copyright", "credits" or "license" for more information.
>>> import pyp_utils
>>> pyp_utils.pyp_nice_time()
'Mon Aug 15 16:27:38 2005'
\end{verbatim}

\section{pyp\_db}
\input{pyp-db}

\section{pyp\_demog}
\input{pyp-demog}

\section{pyp\_graphics}
\input{pyp-graphics}

\section{pyp\_io}
\input{pyp-io}

\section{pyp\_metrics}
\input{pyp-metrics}

\section{pyp\_newclasses}
\input{pyp-newclasses}

\section{pyp\_nrm}
\input{pyp-nrm}

\section{pyp\_reports}
\input{pyp-reports}

\section{pyp\_utils}
\input{pyp-utils}