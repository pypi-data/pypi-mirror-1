\documentclass{manual}

\title{The Vision Egg Programmer's Manual}

\author{Andrew Straw}

\date{\today}			% update before release!
\release{1.1.2}			% software release, not documentation
\setreleaseinfo{}		% empty for final release
\setshortversion{1.1.2}		% major.minor only for software

\makeindex                      % tell \index to actually write the
                                % .idx file
\makemodindex                   % ... and the module index as well.

\begin{document}

\maketitle

\ifhtml
\chapter*{Front Matter\label{front}}
\fi

Copyright \copyright{} 2001-2008 Andrew Straw
All rights reserved.

\begin{abstract}

\noindent
The Vision Egg was designed to perform two primary tasks.  The first
task is the drawing of computer graphics using OpenGL.  The optional
second task is to handle the flow control of your program to
coordinate events on your computer in a precisely timed way.

These are challenging tasks, and the Vision Egg does much of the work
for you. However, to make full use of the Vision Egg, you should
understand the basics.  This is an overview of the main components of
the VisionEgg itself.

Note before starting: The Vision Egg is fundamentally object oriented
in nature, and this document assumes you are familiar with terms such
as ''class'' and ''instance''.  If you are not, please find some
information on the topic of object oriented programming.  As you write
scipts, you will also need to consult the Python, Numeric, pygame, or
other documentation.
\end{abstract}

\tableofcontents

\chapter{Coordinating events \label{coordinating events}}

There are several ways to organize the sequence of your experiments
using the Vision Egg.  You can write your own custom flow control and
event handling, using the Vision Egg solely for drawing graphics.
This is often useful in psychophysics experiments where interaction
with a subject is interleaved with presentation of stimuli.

Alternatively, you can make use of the classes in
\module{VisionEgg.FlowControl}.  For example, \class{Presentation} is a class
that maintains an association between the parameters of stimuli and
their control functions, calls these functions, and initiates drawing
of the stimuli.  There are several ways to use the
\class{Presentation} class described below.  This mode of operation is
useful for electrophysiology experiments.

\section{Custom flow control and event handling}

By writing your own custom flow control code, you have much more
flexibility in designing experiments, but also less of the work
involved has been done for you.  Perhaps the best place to start is
simply to look at some examples.  See the demonstration scripts
dots_simple_loop.py, mouseTarget_user_loop.py, and multi_stim.py.
Each of these programs has its own main loop which performs the same
role as the \class{Presentation} class's \method{go} and
\method{run_forever} methods, which are described further in this chapter.

\section{Using the Presentation class: Running a single trial}

Most of the Vision Egg demonstration scripts run a single trial and
then quit. From a programming perspective, this is the easiest way to
get started. The timing and control of events within a single trial is
performed by a ``go loop'', which is entered by calling the
\method{go()} method of the \class{Presentation} class.

A cycle of the go loop consists of updating any relevant stimulus
parameters, clearing the framebuffer, and calling the stimuli to draw
themselves.  The buffers are swapped and the cycle begins again,
usually after waiting for the vertical blanking interval (see the
section in this manual on double buffering).  When waiting for the
vertical blanking interval (``sync swap buffers'') is enabled, cycles
through the ``go loop'' never occur faster than the frame rate.  If
the go loop is burdened with lots of calculations or if the operating
system takes the CPU away from the Vision Egg, the cycle through the
go loop is not completed before the video card begins drawing the next
frame and therefore a frame is skipped.

A go loop can run indefinitly or have its duration limited to a
duration measured in seconds or in number of frames drawn.  (Measuring
duration based on frames drawn is only meaningful when buffer swapping
is synchronized with the vertical blanking interval and frame skipping
would be particularly undesirable in this case.)

\section{Using the Presentation class: Continuous operation}

Often, the visual stimulus needs to continue running between trials.
At a minimum this could be used to keep the display constant and to
prevent the Vision Egg from quitting, but could also be used to
maintain a moving pattern on the display between trials.  In addition,
it may be necessary to trigger a go loop with a minimum of latency
after the receipt of some signal, such as a digital input on the
parallel port.

To use the Vision Egg in this manner, the \method{run_forever()}
method of \class{Presentation} is called, which begins a loop that
performs the same tasks as a go loop with the exception that functions
controlling stimulus parameters are informed that it is a ``between go
loops'' state.  At any point this \method{run_forever} loop can create
a go loop, which returns control back to the \method{run_forever} loop
when done.  Alternatively, if the controlling functions for stimulus
parameters operate between go loops, the entire experiment could be
run without entering a go loop.  (This could also be acheived by
starting a go loop with a duration parameter set to ``forever''.)

\chapter{Hierarchy of graphical objects \label{hierarchy}}

Currently, the Vision Egg supports only a single screen (window).
However, it is designed to run simultaneously in multiple screens, so
once this capability is available (perhaps in pyglet), the following
priciples will continue to apply.

Each screen contains a list of least one ``viewport''. A viewport is
defined to occupy a rectangular region of the screen and define how
and where objects are drawn. The default viewport created with each
screen fills the entire screen. In the Vision Egg \class{Viewport}
class, the screen position and size are specified in addition to the
projection.  The projection, specified by the \class{Projection}
class, transforms 3D ``eye coordinates'' into ``clip coordinates''
according to, for example, an orthographic or perspective projection.
(Eye coordinates are the 3D coordinates of objects referenced from the
observers eye in arbitrary units.  Clip coordinates are used to
compute the final position of the 3D object on the 2D screen.)  The
default \class{Projection} created with a \class{Viewport} is an
orthographic projection that maps eye coordinates in a one to one
manner to pixel coordinates, allowing specification of object position
in absolute pixels. For more information, consult section 2.11,
``Coordinate Transformations'' of the OpenGL Specification.

Multiple instances of the \class{Viewport} class may occupy the same
region of the screen.  This could be used, for example, to overlay
objects with different projections such as in the targetBackground
demo.  The order of the list of viewports is important, with the first
in the list being drawn first and later viewports are drawn on top of
earlier viewports.

An instance of the \class{Viewport} class keeps an ordered list of the
objects it draws.  Objects to be drawn on top of other objects should
be drawn last and therefore placed last in the list.

The objects a viewport drawns are all instances of the
\class{Stimulus} class. The name ``Stimulus'' is perhaps a slightly
inaccurate because instances of this class only define how to draw a
set of graphics primitives. So for example, there are
\class{SinGrating2D} and \class{TextureStimulus} subclasses of the
\class{Stimulus} class.

The Vision Egg draws objects in a hierarchical manner.  First, the
screen(s) calls each of its viewports in turn.  Each viewport calls
each of its stimuli in turn.  In this way, the occlusion of objects
can be controlled by drawing order without employing more advanced
concepts such as depth testing (which is also possible).

\chapter{Controlling stimulus parameters in realtime \label{controllers}}

When using the \class{Presentation} class, you have a powerful method
of updating parameters in realtime available to you.  ``Controllers''
are instances of the class \class{Controller}.  A controller is called
at pre-defined intervals and updates the value of some stimulus
parameter.  For example, in the ``target'' demo script, the ``center''
parameter of a \class{Target2D} stimulus is updated on every frame by
a function which computes position based upon the current time.  You
can also control parameters without using controllers by simply
changing the values as your program executes.

Instances of \class{Controller} are called by instances of the
\class{Presentation} class.  After creating an instance of
\class{Controller}, it must be ``registered'' by calling the
\method{add_controller} method of \class{Presentation}, during which
the stimulus parameter under control is specified.  The
\class{Presentation} takes care of calling the controller from this
point. Specifically, the \method{during_go_eval()} is called during a
\method{go()} loop, and \method{between_go_eval()} is called by
\method{between_presentations()} (during \method{run_forever()}, for
example.)  These ``eval'' methods return a value which becomes the
new value of the parameter being controller.

The frequency with which \method{during_go_eval()} and
\method{between_go_eval()} are evaluated is determined by the
\var{eval_frequency} attribute of the controller.  The default
\var{eval_frequency} is every frame.

The \var{temporal_variables} attribute of the controller specifies
what temporal variables the ``eval'' methods have available to base
calculations on.  The default value is TIME_SEC_SINCE_GO, so when
\method{during_go_eval()} is called, the instance will have an
attribute \var{time_sec_since_go} set to the time since the onset of
the \method{go()} loop.

For more information, see the documentation for the \class{Controller}
and \class{Presentation} classes in the \module{VisionEgg.Core}
module.

\chapter{Other general information \label{other info}}

\section{Double buffering}

The Vision Egg operates in double buffered mode.  This means that the
contents of the ``front'' framebuffer are read by the video card to
produce the on-screen display.  Meanwhile, clearing and drawing
operations always occur on the back framebuffer, which becomes the
front buffer on a buffer swap (also called flip).  This way, an
incomplete frame is never drawn to the screen.  However, if the
buffers are swapped while the display is only part-way through the
front buffer, the top and bottom parts of the display will contain
frames drawn at different times and thus lead to a tearing artifact.
For this reason the default behavior of the Vision Egg is to
synchronize buffer swapping with the vertical blanking period,
ensuring that tearing does not happen.  (Synchronizing buffer swapping
is not available for some video drivers on the linux platform, and may
frequently be overriden in the Displays control panel in Windows.)

\section{File layout}

Several directories are created in a Vision Egg installation.  The
files used when a Python script imports any Vision Egg module are in
the standard Python ``site-packages'' directory.  Next, the Vision Egg
system directory contains data files such as sample images used by the
demo scripts and a site-wide configuration file.  A user-specific
configuration file is looked for is the VisionEgg home directory.  On
your system, you can determine the exact location of these directories
by running the ``check-config.py'' script, which is distributed with
the Vision Egg.  If you have installed the Vision Egg from source, you
will also have the source directory.

\section{Priority control}

The Vision Egg can take advantage of operating system dependent
methods of setting the priority of an application.  The performance
and features vary from platform to platform.  The options available
from OS specific system calls are available in the Vision Egg.  Before
trying something new, do not attempt to increase your script's
priority, because this may result in locking up the computer.

\section{The log file: Warnings and errors}

The Vision Egg uses the standard Python logging package to log
information including warnings and errors to two locations by default:
the standard error stream (as standard for Python scripts) and to a
file called ``VisionEgg.log'' in the current directory.  The standard
error stream is typically printed on the console, which may only be
visible when running your script from the command line.  If your
script (or modules it imports) raise a
\exception{SyntaxError}, the Vision Egg will be unable to start and
therefore unable to copy the exception traceback to the log file, and
viewing the standard error is the only way to see what went wrong.
Therefore, if your log file does not display an error but your script
is not executing, run it from the command line.  Also, on Mac OS X,
the standard error output is only visible through the Console.app
available in ``/Applications/Utilities''.

You can increase the verbosity of the output by doing something like
``VisionEgg.logger.setLevel( VisionEgg.logging.DEBUG )'' in your
script.

\section{Configuration options}

A number of options that control behavior of the Vision Egg are
available.  These options are first determined when the
\module{VisionEgg.Configuration} module is loaded (by the base module
\module{VisionEgg}, for example).  This module first loads variables
from the ``VisionEgg.cfg'' file and then checks for environment
variables that override these values.

The values set by \module{VisionEgg.Configuration} may be overriden at
any time by re-assigning the appropriate variable.  For example:

\begin{verbatim}
import VisionEgg

# Turn off GUI window when calling create_default_screen()
VisionEgg.config.VISIONEGG_GUI_INIT = 0
\end{verbatim}

\section{For more information}

A significant amount of documentation is contained within the source
code as ''docstrings'' --- special comments within the code.  These
docstrings are often the best source of information for a particular
class or function. In particular, the fundamentally important classes
in the \module{VisionEgg.Core} module are well documented.  You can
either browse the source code itself, look at the library reference
compiled from the source, or use a utility such as PyDoc to compile
your own reference from the source.

The Vision Egg mailing list is another source of valuble information.
Sign up and browse the archives through the Vision Egg website.

For installation instructions, the Vision Egg website provides the
most up-to-date, platform-specific information.

To create your own stimuli you need to know OpenGL.  To learn more
about OpenGL, you may want to begin with ``The Red Book'' (The OpenGL
Programming Guide, The Official Guide to Learning).  The OpenGL
specification is also useful (available online).

% %begin{latexonly}
% \renewcommand{\indexname}{Module Index}
% %end{latexonly}
% \input{modvisionegg.ind}              % Module Index

% %begin{latexonly}
% \renewcommand{\indexname}{Index}
% %end{latexonly}
% \documentclass{manual}

\title{The Vision Egg Programmer's Manual}

\author{Andrew Straw}

\date{\today}			% update before release!
\release{1.1.1}			% software release, not documentation
\setreleaseinfo{}		% empty for final release
\setshortversion{1.1.1}		% major.minor only for software

\makeindex                      % tell \index to actually write the
                                % .idx file
\makemodindex                   % ... and the module index as well.

\begin{document}

\maketitle

\ifhtml
\chapter*{Front Matter\label{front}}
\fi

Copyright \copyright{} 2001-2008 Andrew Straw
All rights reserved.

\begin{abstract}

\noindent
The Vision Egg was designed to perform two primary tasks.  The first
task is the drawing of computer graphics using OpenGL.  The optional
second task is to handle the flow control of your program to
coordinate events on your computer in a precisely timed way.

These are challenging tasks, and the Vision Egg does much of the work
for you. However, to make full use of the Vision Egg, you should
understand the basics.  This is an overview of the main components of
the VisionEgg itself.

Note before starting: The Vision Egg is fundamentally object oriented
in nature, and this document assumes you are familiar with terms such
as ''class'' and ''instance''.  If you are not, please find some
information on the topic of object oriented programming.  As you write
scipts, you will also need to consult the Python, Numeric, pygame, or
other documentation.
\end{abstract}

\tableofcontents

\chapter{Coordinating events \label{coordinating events}}

There are several ways to organize the sequence of your experiments
using the Vision Egg.  You can write your own custom flow control and
event handling, using the Vision Egg solely for drawing graphics.
This is often useful in psychophysics experiments where interaction
with a subject is interleaved with presentation of stimuli.

Alternatively, you can make use of the classes in
\module{VisionEgg.FlowControl}.  For example, \class{Presentation} is a class
that maintains an association between the parameters of stimuli and
their control functions, calls these functions, and initiates drawing
of the stimuli.  There are several ways to use the
\class{Presentation} class described below.  This mode of operation is
useful for electrophysiology experiments.

\section{Custom flow control and event handling}

By writing your own custom flow control code, you have much more
flexibility in designing experiments, but also less of the work
involved has been done for you.  Perhaps the best place to start is
simply to look at some examples.  See the demonstration scripts
dots_simple_loop.py, mouseTarget_user_loop.py, and multi_stim.py.
Each of these programs has its own main loop which performs the same
role as the \class{Presentation} class's \method{go} and
\method{run_forever} methods, which are described further in this chapter.

\section{Using the Presentation class: Running a single trial}

Most of the Vision Egg demonstration scripts run a single trial and
then quit. From a programming perspective, this is the easiest way to
get started. The timing and control of events within a single trial is
performed by a ``go loop'', which is entered by calling the
\method{go()} method of the \class{Presentation} class.

A cycle of the go loop consists of updating any relevant stimulus
parameters, clearing the framebuffer, and calling the stimuli to draw
themselves.  The buffers are swapped and the cycle begins again,
usually after waiting for the vertical blanking interval (see the
section in this manual on double buffering).  When waiting for the
vertical blanking interval (``sync swap buffers'') is enabled, cycles
through the ``go loop'' never occur faster than the frame rate.  If
the go loop is burdened with lots of calculations or if the operating
system takes the CPU away from the Vision Egg, the cycle through the
go loop is not completed before the video card begins drawing the next
frame and therefore a frame is skipped.

A go loop can run indefinitly or have its duration limited to a
duration measured in seconds or in number of frames drawn.  (Measuring
duration based on frames drawn is only meaningful when buffer swapping
is synchronized with the vertical blanking interval and frame skipping
would be particularly undesirable in this case.)

\section{Using the Presentation class: Continuous operation}

Often, the visual stimulus needs to continue running between trials.
At a minimum this could be used to keep the display constant and to
prevent the Vision Egg from quitting, but could also be used to
maintain a moving pattern on the display between trials.  In addition,
it may be necessary to trigger a go loop with a minimum of latency
after the receipt of some signal, such as a digital input on the
parallel port.

To use the Vision Egg in this manner, the \method{run_forever()}
method of \class{Presentation} is called, which begins a loop that
performs the same tasks as a go loop with the exception that functions
controlling stimulus parameters are informed that it is a ``between go
loops'' state.  At any point this \method{run_forever} loop can create
a go loop, which returns control back to the \method{run_forever} loop
when done.  Alternatively, if the controlling functions for stimulus
parameters operate between go loops, the entire experiment could be
run without entering a go loop.  (This could also be acheived by
starting a go loop with a duration parameter set to ``forever''.)

\chapter{Hierarchy of graphical objects \label{hierarchy}}

Currently, the Vision Egg supports only a single screen (window).
However, it is designed to run simultaneously in multiple screens, so
once this capability is available (perhaps in pyglet), the following
priciples will continue to apply.

Each screen contains a list of least one ``viewport''. A viewport is
defined to occupy a rectangular region of the screen and define how
and where objects are drawn. The default viewport created with each
screen fills the entire screen. In the Vision Egg \class{Viewport}
class, the screen position and size are specified in addition to the
projection.  The projection, specified by the \class{Projection}
class, transforms 3D ``eye coordinates'' into ``clip coordinates''
according to, for example, an orthographic or perspective projection.
(Eye coordinates are the 3D coordinates of objects referenced from the
observers eye in arbitrary units.  Clip coordinates are used to
compute the final position of the 3D object on the 2D screen.)  The
default \class{Projection} created with a \class{Viewport} is an
orthographic projection that maps eye coordinates in a one to one
manner to pixel coordinates, allowing specification of object position
in absolute pixels. For more information, consult section 2.11,
``Coordinate Transformations'' of the OpenGL Specification.

Multiple instances of the \class{Viewport} class may occupy the same
region of the screen.  This could be used, for example, to overlay
objects with different projections such as in the targetBackground
demo.  The order of the list of viewports is important, with the first
in the list being drawn first and later viewports are drawn on top of
earlier viewports.

An instance of the \class{Viewport} class keeps an ordered list of the
objects it draws.  Objects to be drawn on top of other objects should
be drawn last and therefore placed last in the list.

The objects a viewport drawns are all instances of the
\class{Stimulus} class. The name ``Stimulus'' is perhaps a slightly
inaccurate because instances of this class only define how to draw a
set of graphics primitives. So for example, there are
\class{SinGrating2D} and \class{TextureStimulus} subclasses of the
\class{Stimulus} class.

The Vision Egg draws objects in a hierarchical manner.  First, the
screen(s) calls each of its viewports in turn.  Each viewport calls
each of its stimuli in turn.  In this way, the occlusion of objects
can be controlled by drawing order without employing more advanced
concepts such as depth testing (which is also possible).

\chapter{Controlling stimulus parameters in realtime \label{controllers}}

When using the \class{Presentation} class, you have a powerful method
of updating parameters in realtime available to you.  ``Controllers''
are instances of the class \class{Controller}.  A controller is called
at pre-defined intervals and updates the value of some stimulus
parameter.  For example, in the ``target'' demo script, the ``center''
parameter of a \class{Target2D} stimulus is updated on every frame by
a function which computes position based upon the current time.  You
can also control parameters without using controllers by simply
changing the values as your program executes.

Instances of \class{Controller} are called by instances of the
\class{Presentation} class.  After creating an instance of
\class{Controller}, it must be ``registered'' by calling the
\method{add_controller} method of \class{Presentation}, during which
the stimulus parameter under control is specified.  The
\class{Presentation} takes care of calling the controller from this
point. Specifically, the \method{during_go_eval()} is called during a
\method{go()} loop, and \method{between_go_eval()} is called by
\method{between_presentations()} (during \method{run_forever()}, for
example.)  These ``eval'' methods return a value which becomes the
new value of the parameter being controller.

The frequency with which \method{during_go_eval()} and
\method{between_go_eval()} are evaluated is determined by the
\var{eval_frequency} attribute of the controller.  The default
\var{eval_frequency} is every frame.

The \var{temporal_variables} attribute of the controller specifies
what temporal variables the ``eval'' methods have available to base
calculations on.  The default value is TIME_SEC_SINCE_GO, so when
\method{during_go_eval()} is called, the instance will have an
attribute \var{time_sec_since_go} set to the time since the onset of
the \method{go()} loop.

For more information, see the documentation for the \class{Controller}
and \class{Presentation} classes in the \module{VisionEgg.Core}
module.

\chapter{Other general information \label{other info}}

\section{Double buffering}

The Vision Egg operates in double buffered mode.  This means that the
contents of the ``front'' framebuffer are read by the video card to
produce the on-screen display.  Meanwhile, clearing and drawing
operations always occur on the back framebuffer, which becomes the
front buffer on a buffer swap (also called flip).  This way, an
incomplete frame is never drawn to the screen.  However, if the
buffers are swapped while the display is only part-way through the
front buffer, the top and bottom parts of the display will contain
frames drawn at different times and thus lead to a tearing artifact.
For this reason the default behavior of the Vision Egg is to
synchronize buffer swapping with the vertical blanking period,
ensuring that tearing does not happen.  (Synchronizing buffer swapping
is not available for some video drivers on the linux platform, and may
frequently be overriden in the Displays control panel in Windows.)

\section{File layout}

Several directories are created in a Vision Egg installation.  The
files used when a Python script imports any Vision Egg module are in
the standard Python ``site-packages'' directory.  Next, the Vision Egg
system directory contains data files such as sample images used by the
demo scripts and a site-wide configuration file.  A user-specific
configuration file is looked for is the VisionEgg home directory.  On
your system, you can determine the exact location of these directories
by running the ``check-config.py'' script, which is distributed with
the Vision Egg.  If you have installed the Vision Egg from source, you
will also have the source directory.

\section{Priority control}

The Vision Egg can take advantage of operating system dependent
methods of setting the priority of an application.  The performance
and features vary from platform to platform.  The options available
from OS specific system calls are available in the Vision Egg.  Before
trying something new, do not attempt to increase your script's
priority, because this may result in locking up the computer.

\section{The log file: Warnings and errors}

The Vision Egg uses the standard Python logging package to log
information including warnings and errors to two locations by default:
the standard error stream (as standard for Python scripts) and to a
file called ``VisionEgg.log'' in the current directory.  The standard
error stream is typically printed on the console, which may only be
visible when running your script from the command line.  If your
script (or modules it imports) raise a
\exception{SyntaxError}, the Vision Egg will be unable to start and
therefore unable to copy the exception traceback to the log file, and
viewing the standard error is the only way to see what went wrong.
Therefore, if your log file does not display an error but your script
is not executing, run it from the command line.  Also, on Mac OS X,
the standard error output is only visible through the Console.app
available in ``/Applications/Utilities''.

You can increase the verbosity of the output by doing something like
``VisionEgg.logger.setLevel( VisionEgg.logging.DEBUG )'' in your
script.

\section{Configuration options}

A number of options that control behavior of the Vision Egg are
available.  These options are first determined when the
\module{VisionEgg.Configuration} module is loaded (by the base module
\module{VisionEgg}, for example).  This module first loads variables
from the ``VisionEgg.cfg'' file and then checks for environment
variables that override these values.

The values set by \module{VisionEgg.Configuration} may be overriden at
any time by re-assigning the appropriate variable.  For example:

\begin{verbatim}
import VisionEgg

# Turn off GUI window when calling create_default_screen()
VisionEgg.config.VISIONEGG_GUI_INIT = 0
\end{verbatim}

\section{For more information}

A significant amount of documentation is contained within the source
code as ''docstrings'' --- special comments within the code.  These
docstrings are often the best source of information for a particular
class or function. In particular, the fundamentally important classes
in the \module{VisionEgg.Core} module are well documented.  You can
either browse the source code itself, look at the library reference
compiled from the source, or use a utility such as PyDoc to compile
your own reference from the source.

The Vision Egg mailing list is another source of valuble information.
Sign up and browse the archives through the Vision Egg website.

For installation instructions, the Vision Egg website provides the
most up-to-date, platform-specific information.

To create your own stimuli you need to know OpenGL.  To learn more
about OpenGL, you may want to begin with ``The Red Book'' (The OpenGL
Programming Guide, The Official Guide to Learning).  The OpenGL
specification is also useful (available online).

% %begin{latexonly}
% \renewcommand{\indexname}{Module Index}
% %end{latexonly}
% \input{modvisionegg.ind}              % Module Index

% %begin{latexonly}
% \renewcommand{\indexname}{Index}
% %end{latexonly}
% \documentclass{manual}

\title{The Vision Egg Programmer's Manual}

\author{Andrew Straw}

\date{\today}			% update before release!
\release{1.1.1}			% software release, not documentation
\setreleaseinfo{}		% empty for final release
\setshortversion{1.1.1}		% major.minor only for software

\makeindex                      % tell \index to actually write the
                                % .idx file
\makemodindex                   % ... and the module index as well.

\begin{document}

\maketitle

\ifhtml
\chapter*{Front Matter\label{front}}
\fi

Copyright \copyright{} 2001-2008 Andrew Straw
All rights reserved.

\begin{abstract}

\noindent
The Vision Egg was designed to perform two primary tasks.  The first
task is the drawing of computer graphics using OpenGL.  The optional
second task is to handle the flow control of your program to
coordinate events on your computer in a precisely timed way.

These are challenging tasks, and the Vision Egg does much of the work
for you. However, to make full use of the Vision Egg, you should
understand the basics.  This is an overview of the main components of
the VisionEgg itself.

Note before starting: The Vision Egg is fundamentally object oriented
in nature, and this document assumes you are familiar with terms such
as ''class'' and ''instance''.  If you are not, please find some
information on the topic of object oriented programming.  As you write
scipts, you will also need to consult the Python, Numeric, pygame, or
other documentation.
\end{abstract}

\tableofcontents

\chapter{Coordinating events \label{coordinating events}}

There are several ways to organize the sequence of your experiments
using the Vision Egg.  You can write your own custom flow control and
event handling, using the Vision Egg solely for drawing graphics.
This is often useful in psychophysics experiments where interaction
with a subject is interleaved with presentation of stimuli.

Alternatively, you can make use of the classes in
\module{VisionEgg.FlowControl}.  For example, \class{Presentation} is a class
that maintains an association between the parameters of stimuli and
their control functions, calls these functions, and initiates drawing
of the stimuli.  There are several ways to use the
\class{Presentation} class described below.  This mode of operation is
useful for electrophysiology experiments.

\section{Custom flow control and event handling}

By writing your own custom flow control code, you have much more
flexibility in designing experiments, but also less of the work
involved has been done for you.  Perhaps the best place to start is
simply to look at some examples.  See the demonstration scripts
dots_simple_loop.py, mouseTarget_user_loop.py, and multi_stim.py.
Each of these programs has its own main loop which performs the same
role as the \class{Presentation} class's \method{go} and
\method{run_forever} methods, which are described further in this chapter.

\section{Using the Presentation class: Running a single trial}

Most of the Vision Egg demonstration scripts run a single trial and
then quit. From a programming perspective, this is the easiest way to
get started. The timing and control of events within a single trial is
performed by a ``go loop'', which is entered by calling the
\method{go()} method of the \class{Presentation} class.

A cycle of the go loop consists of updating any relevant stimulus
parameters, clearing the framebuffer, and calling the stimuli to draw
themselves.  The buffers are swapped and the cycle begins again,
usually after waiting for the vertical blanking interval (see the
section in this manual on double buffering).  When waiting for the
vertical blanking interval (``sync swap buffers'') is enabled, cycles
through the ``go loop'' never occur faster than the frame rate.  If
the go loop is burdened with lots of calculations or if the operating
system takes the CPU away from the Vision Egg, the cycle through the
go loop is not completed before the video card begins drawing the next
frame and therefore a frame is skipped.

A go loop can run indefinitly or have its duration limited to a
duration measured in seconds or in number of frames drawn.  (Measuring
duration based on frames drawn is only meaningful when buffer swapping
is synchronized with the vertical blanking interval and frame skipping
would be particularly undesirable in this case.)

\section{Using the Presentation class: Continuous operation}

Often, the visual stimulus needs to continue running between trials.
At a minimum this could be used to keep the display constant and to
prevent the Vision Egg from quitting, but could also be used to
maintain a moving pattern on the display between trials.  In addition,
it may be necessary to trigger a go loop with a minimum of latency
after the receipt of some signal, such as a digital input on the
parallel port.

To use the Vision Egg in this manner, the \method{run_forever()}
method of \class{Presentation} is called, which begins a loop that
performs the same tasks as a go loop with the exception that functions
controlling stimulus parameters are informed that it is a ``between go
loops'' state.  At any point this \method{run_forever} loop can create
a go loop, which returns control back to the \method{run_forever} loop
when done.  Alternatively, if the controlling functions for stimulus
parameters operate between go loops, the entire experiment could be
run without entering a go loop.  (This could also be acheived by
starting a go loop with a duration parameter set to ``forever''.)

\chapter{Hierarchy of graphical objects \label{hierarchy}}

Currently, the Vision Egg supports only a single screen (window).
However, it is designed to run simultaneously in multiple screens, so
once this capability is available (perhaps in pyglet), the following
priciples will continue to apply.

Each screen contains a list of least one ``viewport''. A viewport is
defined to occupy a rectangular region of the screen and define how
and where objects are drawn. The default viewport created with each
screen fills the entire screen. In the Vision Egg \class{Viewport}
class, the screen position and size are specified in addition to the
projection.  The projection, specified by the \class{Projection}
class, transforms 3D ``eye coordinates'' into ``clip coordinates''
according to, for example, an orthographic or perspective projection.
(Eye coordinates are the 3D coordinates of objects referenced from the
observers eye in arbitrary units.  Clip coordinates are used to
compute the final position of the 3D object on the 2D screen.)  The
default \class{Projection} created with a \class{Viewport} is an
orthographic projection that maps eye coordinates in a one to one
manner to pixel coordinates, allowing specification of object position
in absolute pixels. For more information, consult section 2.11,
``Coordinate Transformations'' of the OpenGL Specification.

Multiple instances of the \class{Viewport} class may occupy the same
region of the screen.  This could be used, for example, to overlay
objects with different projections such as in the targetBackground
demo.  The order of the list of viewports is important, with the first
in the list being drawn first and later viewports are drawn on top of
earlier viewports.

An instance of the \class{Viewport} class keeps an ordered list of the
objects it draws.  Objects to be drawn on top of other objects should
be drawn last and therefore placed last in the list.

The objects a viewport drawns are all instances of the
\class{Stimulus} class. The name ``Stimulus'' is perhaps a slightly
inaccurate because instances of this class only define how to draw a
set of graphics primitives. So for example, there are
\class{SinGrating2D} and \class{TextureStimulus} subclasses of the
\class{Stimulus} class.

The Vision Egg draws objects in a hierarchical manner.  First, the
screen(s) calls each of its viewports in turn.  Each viewport calls
each of its stimuli in turn.  In this way, the occlusion of objects
can be controlled by drawing order without employing more advanced
concepts such as depth testing (which is also possible).

\chapter{Controlling stimulus parameters in realtime \label{controllers}}

When using the \class{Presentation} class, you have a powerful method
of updating parameters in realtime available to you.  ``Controllers''
are instances of the class \class{Controller}.  A controller is called
at pre-defined intervals and updates the value of some stimulus
parameter.  For example, in the ``target'' demo script, the ``center''
parameter of a \class{Target2D} stimulus is updated on every frame by
a function which computes position based upon the current time.  You
can also control parameters without using controllers by simply
changing the values as your program executes.

Instances of \class{Controller} are called by instances of the
\class{Presentation} class.  After creating an instance of
\class{Controller}, it must be ``registered'' by calling the
\method{add_controller} method of \class{Presentation}, during which
the stimulus parameter under control is specified.  The
\class{Presentation} takes care of calling the controller from this
point. Specifically, the \method{during_go_eval()} is called during a
\method{go()} loop, and \method{between_go_eval()} is called by
\method{between_presentations()} (during \method{run_forever()}, for
example.)  These ``eval'' methods return a value which becomes the
new value of the parameter being controller.

The frequency with which \method{during_go_eval()} and
\method{between_go_eval()} are evaluated is determined by the
\var{eval_frequency} attribute of the controller.  The default
\var{eval_frequency} is every frame.

The \var{temporal_variables} attribute of the controller specifies
what temporal variables the ``eval'' methods have available to base
calculations on.  The default value is TIME_SEC_SINCE_GO, so when
\method{during_go_eval()} is called, the instance will have an
attribute \var{time_sec_since_go} set to the time since the onset of
the \method{go()} loop.

For more information, see the documentation for the \class{Controller}
and \class{Presentation} classes in the \module{VisionEgg.Core}
module.

\chapter{Other general information \label{other info}}

\section{Double buffering}

The Vision Egg operates in double buffered mode.  This means that the
contents of the ``front'' framebuffer are read by the video card to
produce the on-screen display.  Meanwhile, clearing and drawing
operations always occur on the back framebuffer, which becomes the
front buffer on a buffer swap (also called flip).  This way, an
incomplete frame is never drawn to the screen.  However, if the
buffers are swapped while the display is only part-way through the
front buffer, the top and bottom parts of the display will contain
frames drawn at different times and thus lead to a tearing artifact.
For this reason the default behavior of the Vision Egg is to
synchronize buffer swapping with the vertical blanking period,
ensuring that tearing does not happen.  (Synchronizing buffer swapping
is not available for some video drivers on the linux platform, and may
frequently be overriden in the Displays control panel in Windows.)

\section{File layout}

Several directories are created in a Vision Egg installation.  The
files used when a Python script imports any Vision Egg module are in
the standard Python ``site-packages'' directory.  Next, the Vision Egg
system directory contains data files such as sample images used by the
demo scripts and a site-wide configuration file.  A user-specific
configuration file is looked for is the VisionEgg home directory.  On
your system, you can determine the exact location of these directories
by running the ``check-config.py'' script, which is distributed with
the Vision Egg.  If you have installed the Vision Egg from source, you
will also have the source directory.

\section{Priority control}

The Vision Egg can take advantage of operating system dependent
methods of setting the priority of an application.  The performance
and features vary from platform to platform.  The options available
from OS specific system calls are available in the Vision Egg.  Before
trying something new, do not attempt to increase your script's
priority, because this may result in locking up the computer.

\section{The log file: Warnings and errors}

The Vision Egg uses the standard Python logging package to log
information including warnings and errors to two locations by default:
the standard error stream (as standard for Python scripts) and to a
file called ``VisionEgg.log'' in the current directory.  The standard
error stream is typically printed on the console, which may only be
visible when running your script from the command line.  If your
script (or modules it imports) raise a
\exception{SyntaxError}, the Vision Egg will be unable to start and
therefore unable to copy the exception traceback to the log file, and
viewing the standard error is the only way to see what went wrong.
Therefore, if your log file does not display an error but your script
is not executing, run it from the command line.  Also, on Mac OS X,
the standard error output is only visible through the Console.app
available in ``/Applications/Utilities''.

You can increase the verbosity of the output by doing something like
``VisionEgg.logger.setLevel( VisionEgg.logging.DEBUG )'' in your
script.

\section{Configuration options}

A number of options that control behavior of the Vision Egg are
available.  These options are first determined when the
\module{VisionEgg.Configuration} module is loaded (by the base module
\module{VisionEgg}, for example).  This module first loads variables
from the ``VisionEgg.cfg'' file and then checks for environment
variables that override these values.

The values set by \module{VisionEgg.Configuration} may be overriden at
any time by re-assigning the appropriate variable.  For example:

\begin{verbatim}
import VisionEgg

# Turn off GUI window when calling create_default_screen()
VisionEgg.config.VISIONEGG_GUI_INIT = 0
\end{verbatim}

\section{For more information}

A significant amount of documentation is contained within the source
code as ''docstrings'' --- special comments within the code.  These
docstrings are often the best source of information for a particular
class or function. In particular, the fundamentally important classes
in the \module{VisionEgg.Core} module are well documented.  You can
either browse the source code itself, look at the library reference
compiled from the source, or use a utility such as PyDoc to compile
your own reference from the source.

The Vision Egg mailing list is another source of valuble information.
Sign up and browse the archives through the Vision Egg website.

For installation instructions, the Vision Egg website provides the
most up-to-date, platform-specific information.

To create your own stimuli you need to know OpenGL.  To learn more
about OpenGL, you may want to begin with ``The Red Book'' (The OpenGL
Programming Guide, The Official Guide to Learning).  The OpenGL
specification is also useful (available online).

% %begin{latexonly}
% \renewcommand{\indexname}{Module Index}
% %end{latexonly}
% \input{modvisionegg.ind}              % Module Index

% %begin{latexonly}
% \renewcommand{\indexname}{Index}
% %end{latexonly}
% \documentclass{manual}

\title{The Vision Egg Programmer's Manual}

\author{Andrew Straw}

\date{\today}			% update before release!
\release{1.1.1}			% software release, not documentation
\setreleaseinfo{}		% empty for final release
\setshortversion{1.1.1}		% major.minor only for software

\makeindex                      % tell \index to actually write the
                                % .idx file
\makemodindex                   % ... and the module index as well.

\begin{document}

\maketitle

\ifhtml
\chapter*{Front Matter\label{front}}
\fi

Copyright \copyright{} 2001-2008 Andrew Straw
All rights reserved.

\begin{abstract}

\noindent
The Vision Egg was designed to perform two primary tasks.  The first
task is the drawing of computer graphics using OpenGL.  The optional
second task is to handle the flow control of your program to
coordinate events on your computer in a precisely timed way.

These are challenging tasks, and the Vision Egg does much of the work
for you. However, to make full use of the Vision Egg, you should
understand the basics.  This is an overview of the main components of
the VisionEgg itself.

Note before starting: The Vision Egg is fundamentally object oriented
in nature, and this document assumes you are familiar with terms such
as ''class'' and ''instance''.  If you are not, please find some
information on the topic of object oriented programming.  As you write
scipts, you will also need to consult the Python, Numeric, pygame, or
other documentation.
\end{abstract}

\tableofcontents

\chapter{Coordinating events \label{coordinating events}}

There are several ways to organize the sequence of your experiments
using the Vision Egg.  You can write your own custom flow control and
event handling, using the Vision Egg solely for drawing graphics.
This is often useful in psychophysics experiments where interaction
with a subject is interleaved with presentation of stimuli.

Alternatively, you can make use of the classes in
\module{VisionEgg.FlowControl}.  For example, \class{Presentation} is a class
that maintains an association between the parameters of stimuli and
their control functions, calls these functions, and initiates drawing
of the stimuli.  There are several ways to use the
\class{Presentation} class described below.  This mode of operation is
useful for electrophysiology experiments.

\section{Custom flow control and event handling}

By writing your own custom flow control code, you have much more
flexibility in designing experiments, but also less of the work
involved has been done for you.  Perhaps the best place to start is
simply to look at some examples.  See the demonstration scripts
dots_simple_loop.py, mouseTarget_user_loop.py, and multi_stim.py.
Each of these programs has its own main loop which performs the same
role as the \class{Presentation} class's \method{go} and
\method{run_forever} methods, which are described further in this chapter.

\section{Using the Presentation class: Running a single trial}

Most of the Vision Egg demonstration scripts run a single trial and
then quit. From a programming perspective, this is the easiest way to
get started. The timing and control of events within a single trial is
performed by a ``go loop'', which is entered by calling the
\method{go()} method of the \class{Presentation} class.

A cycle of the go loop consists of updating any relevant stimulus
parameters, clearing the framebuffer, and calling the stimuli to draw
themselves.  The buffers are swapped and the cycle begins again,
usually after waiting for the vertical blanking interval (see the
section in this manual on double buffering).  When waiting for the
vertical blanking interval (``sync swap buffers'') is enabled, cycles
through the ``go loop'' never occur faster than the frame rate.  If
the go loop is burdened with lots of calculations or if the operating
system takes the CPU away from the Vision Egg, the cycle through the
go loop is not completed before the video card begins drawing the next
frame and therefore a frame is skipped.

A go loop can run indefinitly or have its duration limited to a
duration measured in seconds or in number of frames drawn.  (Measuring
duration based on frames drawn is only meaningful when buffer swapping
is synchronized with the vertical blanking interval and frame skipping
would be particularly undesirable in this case.)

\section{Using the Presentation class: Continuous operation}

Often, the visual stimulus needs to continue running between trials.
At a minimum this could be used to keep the display constant and to
prevent the Vision Egg from quitting, but could also be used to
maintain a moving pattern on the display between trials.  In addition,
it may be necessary to trigger a go loop with a minimum of latency
after the receipt of some signal, such as a digital input on the
parallel port.

To use the Vision Egg in this manner, the \method{run_forever()}
method of \class{Presentation} is called, which begins a loop that
performs the same tasks as a go loop with the exception that functions
controlling stimulus parameters are informed that it is a ``between go
loops'' state.  At any point this \method{run_forever} loop can create
a go loop, which returns control back to the \method{run_forever} loop
when done.  Alternatively, if the controlling functions for stimulus
parameters operate between go loops, the entire experiment could be
run without entering a go loop.  (This could also be acheived by
starting a go loop with a duration parameter set to ``forever''.)

\chapter{Hierarchy of graphical objects \label{hierarchy}}

Currently, the Vision Egg supports only a single screen (window).
However, it is designed to run simultaneously in multiple screens, so
once this capability is available (perhaps in pyglet), the following
priciples will continue to apply.

Each screen contains a list of least one ``viewport''. A viewport is
defined to occupy a rectangular region of the screen and define how
and where objects are drawn. The default viewport created with each
screen fills the entire screen. In the Vision Egg \class{Viewport}
class, the screen position and size are specified in addition to the
projection.  The projection, specified by the \class{Projection}
class, transforms 3D ``eye coordinates'' into ``clip coordinates''
according to, for example, an orthographic or perspective projection.
(Eye coordinates are the 3D coordinates of objects referenced from the
observers eye in arbitrary units.  Clip coordinates are used to
compute the final position of the 3D object on the 2D screen.)  The
default \class{Projection} created with a \class{Viewport} is an
orthographic projection that maps eye coordinates in a one to one
manner to pixel coordinates, allowing specification of object position
in absolute pixels. For more information, consult section 2.11,
``Coordinate Transformations'' of the OpenGL Specification.

Multiple instances of the \class{Viewport} class may occupy the same
region of the screen.  This could be used, for example, to overlay
objects with different projections such as in the targetBackground
demo.  The order of the list of viewports is important, with the first
in the list being drawn first and later viewports are drawn on top of
earlier viewports.

An instance of the \class{Viewport} class keeps an ordered list of the
objects it draws.  Objects to be drawn on top of other objects should
be drawn last and therefore placed last in the list.

The objects a viewport drawns are all instances of the
\class{Stimulus} class. The name ``Stimulus'' is perhaps a slightly
inaccurate because instances of this class only define how to draw a
set of graphics primitives. So for example, there are
\class{SinGrating2D} and \class{TextureStimulus} subclasses of the
\class{Stimulus} class.

The Vision Egg draws objects in a hierarchical manner.  First, the
screen(s) calls each of its viewports in turn.  Each viewport calls
each of its stimuli in turn.  In this way, the occlusion of objects
can be controlled by drawing order without employing more advanced
concepts such as depth testing (which is also possible).

\chapter{Controlling stimulus parameters in realtime \label{controllers}}

When using the \class{Presentation} class, you have a powerful method
of updating parameters in realtime available to you.  ``Controllers''
are instances of the class \class{Controller}.  A controller is called
at pre-defined intervals and updates the value of some stimulus
parameter.  For example, in the ``target'' demo script, the ``center''
parameter of a \class{Target2D} stimulus is updated on every frame by
a function which computes position based upon the current time.  You
can also control parameters without using controllers by simply
changing the values as your program executes.

Instances of \class{Controller} are called by instances of the
\class{Presentation} class.  After creating an instance of
\class{Controller}, it must be ``registered'' by calling the
\method{add_controller} method of \class{Presentation}, during which
the stimulus parameter under control is specified.  The
\class{Presentation} takes care of calling the controller from this
point. Specifically, the \method{during_go_eval()} is called during a
\method{go()} loop, and \method{between_go_eval()} is called by
\method{between_presentations()} (during \method{run_forever()}, for
example.)  These ``eval'' methods return a value which becomes the
new value of the parameter being controller.

The frequency with which \method{during_go_eval()} and
\method{between_go_eval()} are evaluated is determined by the
\var{eval_frequency} attribute of the controller.  The default
\var{eval_frequency} is every frame.

The \var{temporal_variables} attribute of the controller specifies
what temporal variables the ``eval'' methods have available to base
calculations on.  The default value is TIME_SEC_SINCE_GO, so when
\method{during_go_eval()} is called, the instance will have an
attribute \var{time_sec_since_go} set to the time since the onset of
the \method{go()} loop.

For more information, see the documentation for the \class{Controller}
and \class{Presentation} classes in the \module{VisionEgg.Core}
module.

\chapter{Other general information \label{other info}}

\section{Double buffering}

The Vision Egg operates in double buffered mode.  This means that the
contents of the ``front'' framebuffer are read by the video card to
produce the on-screen display.  Meanwhile, clearing and drawing
operations always occur on the back framebuffer, which becomes the
front buffer on a buffer swap (also called flip).  This way, an
incomplete frame is never drawn to the screen.  However, if the
buffers are swapped while the display is only part-way through the
front buffer, the top and bottom parts of the display will contain
frames drawn at different times and thus lead to a tearing artifact.
For this reason the default behavior of the Vision Egg is to
synchronize buffer swapping with the vertical blanking period,
ensuring that tearing does not happen.  (Synchronizing buffer swapping
is not available for some video drivers on the linux platform, and may
frequently be overriden in the Displays control panel in Windows.)

\section{File layout}

Several directories are created in a Vision Egg installation.  The
files used when a Python script imports any Vision Egg module are in
the standard Python ``site-packages'' directory.  Next, the Vision Egg
system directory contains data files such as sample images used by the
demo scripts and a site-wide configuration file.  A user-specific
configuration file is looked for is the VisionEgg home directory.  On
your system, you can determine the exact location of these directories
by running the ``check-config.py'' script, which is distributed with
the Vision Egg.  If you have installed the Vision Egg from source, you
will also have the source directory.

\section{Priority control}

The Vision Egg can take advantage of operating system dependent
methods of setting the priority of an application.  The performance
and features vary from platform to platform.  The options available
from OS specific system calls are available in the Vision Egg.  Before
trying something new, do not attempt to increase your script's
priority, because this may result in locking up the computer.

\section{The log file: Warnings and errors}

The Vision Egg uses the standard Python logging package to log
information including warnings and errors to two locations by default:
the standard error stream (as standard for Python scripts) and to a
file called ``VisionEgg.log'' in the current directory.  The standard
error stream is typically printed on the console, which may only be
visible when running your script from the command line.  If your
script (or modules it imports) raise a
\exception{SyntaxError}, the Vision Egg will be unable to start and
therefore unable to copy the exception traceback to the log file, and
viewing the standard error is the only way to see what went wrong.
Therefore, if your log file does not display an error but your script
is not executing, run it from the command line.  Also, on Mac OS X,
the standard error output is only visible through the Console.app
available in ``/Applications/Utilities''.

You can increase the verbosity of the output by doing something like
``VisionEgg.logger.setLevel( VisionEgg.logging.DEBUG )'' in your
script.

\section{Configuration options}

A number of options that control behavior of the Vision Egg are
available.  These options are first determined when the
\module{VisionEgg.Configuration} module is loaded (by the base module
\module{VisionEgg}, for example).  This module first loads variables
from the ``VisionEgg.cfg'' file and then checks for environment
variables that override these values.

The values set by \module{VisionEgg.Configuration} may be overriden at
any time by re-assigning the appropriate variable.  For example:

\begin{verbatim}
import VisionEgg

# Turn off GUI window when calling create_default_screen()
VisionEgg.config.VISIONEGG_GUI_INIT = 0
\end{verbatim}

\section{For more information}

A significant amount of documentation is contained within the source
code as ''docstrings'' --- special comments within the code.  These
docstrings are often the best source of information for a particular
class or function. In particular, the fundamentally important classes
in the \module{VisionEgg.Core} module are well documented.  You can
either browse the source code itself, look at the library reference
compiled from the source, or use a utility such as PyDoc to compile
your own reference from the source.

The Vision Egg mailing list is another source of valuble information.
Sign up and browse the archives through the Vision Egg website.

For installation instructions, the Vision Egg website provides the
most up-to-date, platform-specific information.

To create your own stimuli you need to know OpenGL.  To learn more
about OpenGL, you may want to begin with ``The Red Book'' (The OpenGL
Programming Guide, The Official Guide to Learning).  The OpenGL
specification is also useful (available online).

% %begin{latexonly}
% \renewcommand{\indexname}{Module Index}
% %end{latexonly}
% \input{modvisionegg.ind}              % Module Index

% %begin{latexonly}
% \renewcommand{\indexname}{Index}
% %end{latexonly}
% \input{visionegg.ind}                 % Index

\end{document}
                 % Index

\end{document}
                 % Index

\end{document}
                 % Index

\end{document}
