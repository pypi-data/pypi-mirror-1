\documentclass[10pt]{article}
\usepackage{fullpage}
\usepackage{amsmath}
\usepackage{graphicx}
\usepackage{py2tex}

\let\jnlstyle=\rm
\def\refjnl#1{{\jnlstyle#1}}
\def\aj{\refjnl{AJ}}
\def\araa{\refjnl{ARA\&A}}
\def\apj{\refjnl{ApJ}}
\def\apjl{\refjnl{ApJ}}
\def\apjs{\refjnl{ApJS}}
\def\ao{\refjnl{Appl.~Opt.}}
\def\apss{\refjnl{Ap\&SS}}
\def\aap{\refjnl{A\&A}}
\def\aapr{\refjnl{A\&A~Rev.}}
\def\aaps{\refjnl{A\&AS}}
\def\azh{\refjnl{AZh}}
\def\baas{\refjnl{BAAS}}
\def\jrasc{\refjnl{JRASC}}
\def\memras{\refjnl{MmRAS}}
\def\mnras{\refjnl{MNRAS}}
\def\pra{\refjnl{Phys.~Rev.~A}}
\def\prb{\refjnl{Phys.~Rev.~B}}
\def\prc{\refjnl{Phys.~Rev.~C}}
\def\prd{\refjnl{Phys.~Rev.~D}}
\def\pre{\refjnl{Phys.~Rev.~E}}
\def\prl{\refjnl{Phys.~Rev.~Lett.}}
\def\pasp{\refjnl{PASP}}
\def\pasj{\refjnl{PASJ}}
\def\qjras{\refjnl{QJRAS}}
\def\skytel{\refjnl{S\&T}}
\def\solphys{\refjnl{Sol.~Phys.}}
\def\sovast{\refjnl{Soviet~Ast.}}
\def\ssr{\refjnl{Space~Sci.~Rev.}}
\def\zap{\refjnl{ZAp}}
\def\nat{\refjnl{Nature}}
\def\iaucirc{\refjnl{IAU~Circ.}}
\def\aplett{\refjnl{Astrophys.~Lett.}}
\def\apspr{\refjnl{Astrophys.~Space~Phys.~Res.}}
\def\bain{\refjnl{Bull.~Astron.~Inst.~Netherlands}}
\def\fcp{\refjnl{Fund.~Cosmic~Phys.}}
\def\gca{\refjnl{Geochim.~Cosmochim.~Acta}}
\def\grl{\refjnl{Geophys.~Res.~Lett.}}
\def\jcp{\refjnl{J.~Chem.~Phys.}}

\title{Astronomical Interferometry in PYthon (AIPY)\\
Version 0.6.2}

\author{Aaron Parsons}
\date{05 September 2008}

\begin{document}
\maketitle
\tableofcontents 

\section{Introduction}

This package collects together tools for radio astronomical interferometry.  In
addition to pure-python phasing, calibration, imaging, and
deconvolution code, this package includes interfaces to MIRIAD (a Fortran
interferometry package), HEALPix (a package for representing spherical data
sets), routines from SciPy for fitting, and the PyFITS and PyEphem packages
verbatim.  It can be found on the web at
{\it http://setiathome.berkeley.edu/\~{}aparsons/aipy}

AIPY is under active development as a part of the NSF-funded project
PAPER: the Precision Array for Probing the Epoch of Reionization--a 
low-frequency interferometry experiment for detecting ionization resulting
from the formation of the first starts and galaxies.

AIPY is free software; you can redistribute it and/or modify it under
the terms of the GNU General Public License as published by the Free Software
Foundation; either version 2 of the License, or (at your option) any later
version.

\subsection{Acknowledgement}

The subpackage "optimize" is released under GPL by SCIPY developers, and
has been modified to include only pure-python implementations.

The C source code for "cephes" (in cephes/cephes) is part of the CEPHES Math
Library and is released under GPL.  The python wrapper code is released under
GPL by SCIPY developers, and has been modified to remove Fortran dependencies.

The C source code for "healpix" (in healpix/cxx) is released under GPL
by the HEALPix collaboration, and is included in AIPY verbatim.

The C/FORTRAN source code for "miriad" (in miriad/mirsrc) is released under
GPL. It is the work of many authors, and is included in AIPY verbatim.

\section{Installation}

\subsection{Requirements}

\begin{itemize}
\item[] *nix or MacOs.  I don't think AIPY installs on Windows.
\item[] Python 2.4 or better.  {\it http://www.python.org}
\item[] Numpy 1.0.4 or better. {\it http://numpy.scipy.org}
\item[] PyEphem 3.7.2.3 or better. {\it http://rhodesmill.org/pyephem}
\item[] PyFITS 1.1 or better. {\it http://www.stsci.edu/resources/software\_hardware/pyfits}
\item[] Matplotlib 0.91 or better. {\it http:://matplotlib.sourceforge.net}
\item[] Matplotlib-Basemap 0.9 or better {\it http:://matplotlib.sourceforge.net}
\end{itemize}

\subsection{Install As Root}

\begin{verbatim}
$ sudo python setup.py install
\end{verbatim}

\subsection{Install As User}

\begin{verbatim}
$ python setup.py install --install-lib <module_dir> --install-scripts <scripts_dir>
\end{verbatim}

This puts the python module in <module\_dir> and the command-line scripts
in <scripts\_dir>.  The next thing is to tell python where to look
for the AIPY module.  This is done by setting the PYTHONPATH shell variable
to point to <module\_dir>.  If you are using bash, add the following line
to you .bashrc file:
\begin{verbatim}
export PYTHONPATH=PYTHONPATH:<module_dir>
\end{verbatim}

\clearpage
\input miriad.tex
\clearpage
\input ant.tex
\clearpage
\input sim.tex
\clearpage
\input fit.tex
\clearpage
\input loc.tex
\clearpage
\input src.tex
\clearpage
\input img.tex
\clearpage
\input map.tex
\clearpage
\input coord.tex
\clearpage
\input deconv.tex
\clearpage
\input healpix.tex
\clearpage
\input scripting.tex
\clearpage
\input rfi.tex
\clearpage

\bibliographystyle{plain}   % or "unsrt", "alpha", "abbrv", etc.
\bibliography{biblio}       % use data in file "biblio.bib"

\end{document}


