\documentclass{manual}

\title{Big Python Manual}

\author{Your Name Here}

% Please at least include a long-lived email address;
% the rest is at your discretion.
\authoraddress{
	Organization name, if applicable \\
	Street address, if you want to use it \\
	E-mail: \email{your-email@your.domain}
}

\date{April 30, 1999}		% update before release!
				% Use an explicit date so that reformatting
				% doesn't cause a new date to be used.  Setting
				% the date to \today can be used during draft
				% stages to make it easier to handle versions.

\release{x.y}			% release version; this is used to define the
				% \version macro

\makeindex			% tell \index to actually write the .idx file
\makemodindex			% If this contains a lot of module sections.


\begin{document}

\maketitle

% This makes the contents more accessible from the front page of the HTML.
\ifhtml
\chapter*{Front Matter\label{front}}
\fi

%\section*{Legal Notice}
\label{sec:legal-notice}

Copyright (c) 2002, 2003, 2004, 2005.  John B. Cole.  All rights reserved.

Permission to use, copy, modify, and distribute this software for any purpose
without fee is hereby granted, provided that this entire notice is included in
all copies of any software which is or includes a copy or modification of this
software and in all copies of the supporting documentation for such software.

\subsection*{Disclaimer}

The author of this software does not make any warranty, express or implied, or
assume any liability or responsibility for the accuracy, completeness, or
usefulness of any information, apparatus, product, or process disclosed, or
represent that its use would not infringe privately-owned rights. Reference
herein to any specific commercial products, process, or service by trade name,
trademark, manufacturer, or otherwise, does not necessarily constitute or imply
its endorsement, recommendation, or favoring by the United States Government or
the author. The views and opinions of authors expressed herein do not necessarily
state or reflect those of the United States Government and shall not be used for
advertising or product endorsement purposes.

%% Local Variables:
%% mode: LaTeX
%% mode: auto-fill
%% fill-column: 79
%% indent-tabs-mode: nil
%% ispell-dictionary: "american"
%% reftex-fref-is-default: nil
%% TeX-auto-save: t
%% TeX-command-default: "pdfeLaTeX"
%% TeX-master: "numarray"
%% TeX-parse-self: t
%% End:

\begin{abstract}

\noindent
Big Python is a special version of Python for users who require larger 
keys on their keyboards.  It accomodates their special needs by ...

\end{abstract}

\tableofcontents


\chapter{...}

My chapter.


\appendix
\chapter{...}

My appendix.

The \code{\e appendix} markup need not be repeated for additional
appendices.


%
%  The ugly "%begin{latexonly}" pseudo-environments are really just to
%  keep LaTeX2HTML quiet during the \renewcommand{} macros; they're
%  not really valuable.
%
%  If you don't want the Module Index, you can remove all of this up
%  until the second \input line.
%
%begin{latexonly}
\renewcommand{\indexname}{Module Index}
%end{latexonly}
\input{mod\jobname.ind}		% Module Index

%begin{latexonly}
\renewcommand{\indexname}{Index}
%end{latexonly}
\input{\jobname.ind}			% Index

\end{document}
