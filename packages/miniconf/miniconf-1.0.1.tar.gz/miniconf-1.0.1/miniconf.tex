\documentclass{howto}

% Crude reformatting of the original module documentation as a standalone howto.
%
\title{\module{miniconf} --- Plain text Python built-in objects persistence}

\release{1.0.1}

\author{Sylvain Fourmanoit}
\authoraddress{\email{syfou@users.sourceforge.net}}

\begin{document}
\maketitle

\begin{abstract}
\noindent
\module{miniconf} is a pure-Python module providing data persistence of a subset
of built-in objects as plain text. It has been initially submitted as a patch
against Python 2.5 tree
(\url{http://sourceforge.net/tracker/?func=detail\&atid=355470\&aid=1527597\&group_id=5470}).
It is distributed here as a standalone module, in the hope it will be useful to
the community, as well as to collect feedback. It should work on both Python 2.4
as well as on Python 2.5.
\end{abstract}

\tableofcontents

\section{Usage}
\module{miniconf} provides data persistence of a subset of Python built-in
objects as plain text. Objects are loaded from, and dumped to valid Python
source composed of assignment statements of the form
\code{identifier = value}. Supported objects are:

\begin{itemize}
\item dictionaries (\code{dict})
\item lists and tuples (\code{list} and \code{tuple})
\item integers and floats (\code{int} and \code{float})
\item plain and unicode strings (\code{str} and \code{unicode})
\item booleans (\code{bool})
\item \code{None}
\end{itemize}

Arbitrarily complex objects composed of these types can be
handled. \module{miniconf} is restricted to these types because they can be
easily reconstructed from literal representation in Python, entirely known at
byte-compilation, without the need to execute the code.

\module{miniconf} aims at providing an easy, elegant way to store (dump) and
retrieve (load) configurations\footnote{Hence the name, \module{miniconf}. See
\refmodule{ConfigParser} for a different but more thorough, higher-level
implementation of configuration file handlers.} and other simple datasets in a
human-readable, familiar pythonic notation, while preventing unwanted injection
of external code.

Basically, \module{miniconf} exposes two important functions, \function{load}
and \function{dump}, as well as one helper function, \function{format_header}:

\begin{funcdesc}{load}{buf\optional{, pedantic=False}}
Load configuration from string \var{buf}. If \var{pedantic} is \code{True},
un-loadable elements will raise \exception{TypeError} exceptions instead of
being silently dropped. On success, return a dictionary containing the parsed
built-in objects, indexed by assignment names. On error, raise a
\exception{SyntaxError}, \exception{TypeError} or \exception{ValueError}
exception.
\end{funcdesc}

\begin{funcdesc}{dump}{data\optional{, comments=\{\}, pedantic=True, ...}}
Dump configuration from dictionary \var{data}, prepending string comments from
corresponding values (i.e. associated to the matching keys in \var{data}) in
\var{comments}. If \var{pedantic} is \code{True}, a \exception{TypeError}
exception will be raised if some value in data is a derivate class of an
otherwise supported type. Return the formatted string on success, or raise a
\exception{TypeError} or \exception{ValueError} on error. \var{data} dictionary
is dumped in identifiers' alphabetical order. Valid keywords are optional
arguments to the low-level \class{PrettyPrinter} object (see the
\function{pformat} function from the \module{pprint} module).

If every lines in a comment string are already starting with a pound sign thus
making the string an already valid Python comment, such string is preserved
untouched in the output. If not, the comment string will be formatted using
\function{format_header}, using the same width used by the
\class{PrettyPrinter}. Basically, this means you are free to either have
comments automatically formatted and wrapped as a single paragraph, or use your
own layout if you want, as long as the whole string keeps being a valid Python
comment.

Values associated with special \code{'--top--'} and \code{'--bottom--'} keys, if
they exist in comments, will be respectively included at the beginning and end
of the return string; same formatting rules apply to them.
\end{funcdesc}

\begin{funcdesc}{format_header}{header\optional{, width=80}}
Turn \var{header} into a valid Python comment by prepending a pound sing
followed by a space to each line, and wrap it to the given width. Return the
result as a single, potentially multi-lines string.
\end{funcdesc}

\section{Example}

Load objects from a textual representation, stored in \var{snippet}:

\begin{verbatim}
>>> import miniconf
>>> snippet = 'spam = [ 1, True, ("test", None) ]; egg = -2'
>>> config = miniconf.load(snippet)
>>> print config
{'egg': -2, 'spam': [1, True, ('test', None)]}
\end{verbatim}

Of course, \var{config} could have been constructed from \var{snippet} as well
by doing:

\begin{verbatim}
>>> config = {}
>>> exec snippet in config
>>> del config['__builtins__']
\end{verbatim}

The whole point of using \module{miniconf} instead of the exec statement is that
it is safer, since no arbitrary code execution ever occurs: the code is only
parsed, not executed, and the objects are reconstructed from the snippet
abstract syntax tree. In practice, it makes user access to simple pythonic data
structure possible without having to fear injection of unwanted third-party
code.

Finally, let's modify \var{config} a little and dump it back:

\begin{verbatim}
>>> config['egg'] = u'new_value'
>>> config['new'] = range(10)
>>> print miniconf.dump(config)
egg = u'new_value'

new = [0, 1, 2, 3, 4, 5, 6, 7, 8, 9]

spam = [1, True, ('test', None)]
\end{verbatim}

\section{Explanation on pedantry}

\function{load} and \function{dump} have two modes of operation: pedantic and
non-pedantic.

\begin{itemize}
\item
On \function{load}, the pedantic argument indicates whether the function should
bail out on un-loadable constructs instead or just ignore them. Default is not
to be pedantic on load.

\item
On \function{dump}, the pedantic argument indicates if the function should bail
out when dumping objects that could only be partially restored later on (because
they belong to a derivate class of a supported type). A successfully dump will
always be re-loadable anyway regardless of the pedantry. Default is to be
pedantic on dump.
\end{itemize}

A non-pedantic load or dump does not ensure that the operation will not raise an
exception (such as a \exception{SyntaxError} because of an unparsable buffer
during a load), only that the level of tolerance to the minor problems just
described will be greater.

\section{Limitations}

\module{miniconf} has a few limitations:

\begin{itemize}
\item
It only supports a subset of built-ins types (see above).

\item
It is not preemptiple: concurrent calls to \function{dump} or \function{load}
will have unpredictable results and must be avoided.

\item
It is strictly string-based: it loads from them, and dumps to them. Potential
race conditions with the underlying sources and destinations such as real files
has to be dealt with externally (handling locks, etc.).

\item
\function{dump} and \function{load} are not inverse functions, because
\function{load} is not injective; any special content or formatting in the
source code (comments, un-loadable objects, non-assignments statements, lexical
format) will be discarded at load time. See for instance:

\begin{verbatim}
>>> from miniconf import load, dump
>>> print dump(load('spam = "egg" # this comment will be lost'))
spam = 'egg'
\end{verbatim}

Basically, this means that an external agent (user or application)
cannot add an element such as a comment to a snippet, and have it
preserved next time the program will have loaded, then dumped it
back. Of course, one can very well choose not to systematically dump
the data over the source of the next load, which alleviates this
limitation.
\end{itemize}

\end{document}
