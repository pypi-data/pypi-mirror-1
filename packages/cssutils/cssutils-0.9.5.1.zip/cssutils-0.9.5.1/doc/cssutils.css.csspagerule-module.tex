%
% API Documentation for cssutils
% Module cssutils.css.csspagerule
%
% Generated by epydoc 3.0.1
% [Fri Feb 01 19:05:21 2008]
%

%%%%%%%%%%%%%%%%%%%%%%%%%%%%%%%%%%%%%%%%%%%%%%%%%%%%%%%%%%%%%%%%%%%%%%%%%%%
%%                          Module Description                           %%
%%%%%%%%%%%%%%%%%%%%%%%%%%%%%%%%%%%%%%%%%%%%%%%%%%%%%%%%%%%%%%%%%%%%%%%%%%%

    \index{cssutils \textit{(package)}!cssutils.css \textit{(package)}!cssutils.css.csspagerule \textit{(module)}|(}
\section{Module cssutils.css.csspagerule}

    \label{cssutils:css:csspagerule}

CSSPageRule implements DOM Level 2 CSS CSSPageRule.
\textbf{Version:} \$LastChangedRevision: 953 \$



\textbf{Date:} \$LastChangedDate: 2008-01-27 17:44:48 +0100 (So, 27 Jan 2008) \$



\textbf{Author:} \$LastChangedBy: cthedot \$




%%%%%%%%%%%%%%%%%%%%%%%%%%%%%%%%%%%%%%%%%%%%%%%%%%%%%%%%%%%%%%%%%%%%%%%%%%%
%%                           Class Description                           %%
%%%%%%%%%%%%%%%%%%%%%%%%%%%%%%%%%%%%%%%%%%%%%%%%%%%%%%%%%%%%%%%%%%%%%%%%%%%

    \index{cssutils \textit{(package)}!cssutils.css \textit{(package)}!cssutils.css.csspagerule \textit{(module)}!cssutils.css.csspagerule.CSSPageRule \textit{(class)}|(}
\subsection{Class CSSPageRule}

    \label{cssutils:css:csspagerule:CSSPageRule}
\begin{tabular}{cccccccccc}
% Line for object, linespec=[False, False, False]
\multicolumn{2}{r}{\settowidth{\BCL}{object}\multirow{2}{\BCL}{object}}
&&
&&
&&
  \\\cline{3-3}
  &&\multicolumn{1}{c|}{}
&&
&&
&&
  \\
% Line for cssutils.util.Base, linespec=[False, False]
\multicolumn{4}{r}{\settowidth{\BCL}{cssutils.util.Base}\multirow{2}{\BCL}{cssutils.util.Base}}
&&
&&
  \\\cline{5-5}
  &&&&\multicolumn{1}{c|}{}
&&
&&
  \\
% Line for cssutils.css.cssrule.CSSRule, linespec=[False]
\multicolumn{6}{r}{\settowidth{\BCL}{cssutils.css.cssrule.CSSRule}\multirow{2}{\BCL}{cssutils.css.cssrule.CSSRule}}
&&
  \\\cline{7-7}
  &&&&&&\multicolumn{1}{c|}{}
&&
  \\
&&&&&&\multicolumn{2}{l}{\textbf{cssutils.css.csspagerule.CSSPageRule}}
\end{tabular}


The CSSPageRule interface represents a @page rule within a CSS style
sheet. The @page rule is used to specify the dimensions, orientation,
margins, etc. of a page box for paged media.


%___________________________________________________________________________

\hypertarget{properties}{}
\pdfbookmark[3]{Properties}{properties}
\paragraph*{Properties}
\label{properties}
\begin{description}
\item[{cssText: of type DOMString}] \leavevmode 
The parsable textual representation of this rule

\item[{selectorText: of type DOMString}] \leavevmode 
The parsable textual representation of the page selector for the rule.

\item[{style: of type CSSStyleDeclaration}] \leavevmode 
The declaration-block of this rule.

\end{description}


%___________________________________________________________________________

\hypertarget{cssutils-only}{}
\pdfbookmark[4]{cssutils only}{cssutils-only}
\subparagraph*{cssutils only}
\label{cssutils-only}
\begin{description}
\item[{atkeyword:}] \leavevmode 
the literal keyword used

\end{description}

Inherits properties from CSSRule


%___________________________________________________________________________

\hypertarget{format}{}
\pdfbookmark[3]{Format}{format}
\paragraph*{Format}
\label{format}
\begin{quote}{\ttfamily \raggedright \noindent
page~\\
~~:~PAGE{\_}SYM~S*~pseudo{\_}page?~S*~\\
~~~~LBRACE~S*~declaration~{[}~';'~S*~declaration~{]}*~'{\}}'~S*~\\
~~;~\\
pseudo{\_}page~\\
~~:~':'~IDENT~{\#}~:first,~:left,~:right~in~CSS~2.1~\\
~~;
}\end{quote}

%%%%%%%%%%%%%%%%%%%%%%%%%%%%%%%%%%%%%%%%%%%%%%%%%%%%%%%%%%%%%%%%%%%%%%%%%%%
%%                                Methods                                %%
%%%%%%%%%%%%%%%%%%%%%%%%%%%%%%%%%%%%%%%%%%%%%%%%%%%%%%%%%%%%%%%%%%%%%%%%%%%

  \subsubsection{Methods}

    \vspace{0.5ex}

\hspace{.8\funcindent}\begin{boxedminipage}{\funcwidth}

    \raggedright \textbf{\_\_init\_\_}(\textit{self}, \textit{selectorText}={\tt None}, \textit{style}={\tt None}, \textit{parentRule}={\tt None}, \textit{parentStyleSheet}={\tt None}, \textit{readonly}={\tt False})

    \vspace{-1.5ex}

    \rule{\textwidth}{0.5\fboxrule}
\setlength{\parskip}{2ex}

if readonly allows setting of properties in constructor only
\begin{description}
\item[{selectorText}] \leavevmode 
type string

\item[{style}] \leavevmode 
CSSStyleDeclaration for this CSSStyleRule

\end{description}
\setlength{\parskip}{1ex}
      Overrides: object.\_\_init\_\_

    \end{boxedminipage}

    \vspace{0.5ex}

\hspace{.8\funcindent}\begin{boxedminipage}{\funcwidth}

    \raggedright \textbf{\_\_repr\_\_}(\textit{self})

\setlength{\parskip}{2ex}
    repr(x)

\setlength{\parskip}{1ex}
      Overrides: object.\_\_repr\_\_ 	extit{(inherited documentation)}

    \end{boxedminipage}

    \vspace{0.5ex}

\hspace{.8\funcindent}\begin{boxedminipage}{\funcwidth}

    \raggedright \textbf{\_\_str\_\_}(\textit{self})

\setlength{\parskip}{2ex}
    str(x)

\setlength{\parskip}{1ex}
      Overrides: object.\_\_str\_\_ 	extit{(inherited documentation)}

    \end{boxedminipage}


\large{\textbf{\textit{Inherited from object}}}

\begin{quote}
\_\_delattr\_\_(), \_\_getattribute\_\_(), \_\_hash\_\_(), \_\_new\_\_(), \_\_reduce\_\_(), \_\_reduce\_ex\_\_(), \_\_setattr\_\_()
\end{quote}

%%%%%%%%%%%%%%%%%%%%%%%%%%%%%%%%%%%%%%%%%%%%%%%%%%%%%%%%%%%%%%%%%%%%%%%%%%%
%%                              Properties                               %%
%%%%%%%%%%%%%%%%%%%%%%%%%%%%%%%%%%%%%%%%%%%%%%%%%%%%%%%%%%%%%%%%%%%%%%%%%%%

  \subsubsection{Properties}

    \vspace{-1cm}
\hspace{\varindent}\begin{longtable}{|p{\varnamewidth}|p{\vardescrwidth}|l}
\cline{1-2}
\cline{1-2} \centering \textbf{Name} & \centering \textbf{Description}& \\
\cline{1-2}
\endhead\cline{1-2}\multicolumn{3}{r}{\small\textit{continued on next page}}\\\endfoot\cline{1-2}
\endlastfoot\raggedright c\-s\-s\-T\-e\-x\-t\- & \raggedright (DOM) The parsable textual representation of the rule.&\\
\cline{1-2}
\raggedright s\-e\-l\-e\-c\-t\-o\-r\-T\-e\-x\-t\- & \raggedright (DOM) The parsable textual representation of the page selector for the rule.&\\
\cline{1-2}
\raggedright s\-t\-y\-l\-e\- & \raggedright (DOM) The declaration-block of this rule set.&\\
\cline{1-2}
\multicolumn{2}{|l|}{\textit{Inherited from cssutils.css.cssrule.CSSRule \textit{(Section \ref{cssutils:css:cssrule:CSSRule})}}}\\
\multicolumn{2}{|p{\varwidth}|}{\raggedright parentRule, parentStyleSheet, typeString}\\
\cline{1-2}
\multicolumn{2}{|l|}{\textit{Inherited from object}}\\
\multicolumn{2}{|p{\varwidth}|}{\raggedright \_\_class\_\_}\\
\cline{1-2}
\end{longtable}


%%%%%%%%%%%%%%%%%%%%%%%%%%%%%%%%%%%%%%%%%%%%%%%%%%%%%%%%%%%%%%%%%%%%%%%%%%%
%%                            Class Variables                            %%
%%%%%%%%%%%%%%%%%%%%%%%%%%%%%%%%%%%%%%%%%%%%%%%%%%%%%%%%%%%%%%%%%%%%%%%%%%%

  \subsubsection{Class Variables}

    \vspace{-1cm}
\hspace{\varindent}\begin{longtable}{|p{\varnamewidth}|p{\vardescrwidth}|l}
\cline{1-2}
\cline{1-2} \centering \textbf{Name} & \centering \textbf{Description}& \\
\cline{1-2}
\endhead\cline{1-2}\multicolumn{3}{r}{\small\textit{continued on next page}}\\\endfoot\cline{1-2}
\endlastfoot\raggedright t\-y\-p\-e\- & \raggedright The type of this rule, as defined by a CSSRule type constant.
Overwritten in derived classes.

The expectation is that binding-specific casting methods can be used to
cast down from an instance of the CSSRule interface to the specific
derived interface implied by the type.
(Casting not for this Python implementation I guess...)

\textbf{Value:} 
{\tt 6}&\\
\cline{1-2}
\multicolumn{2}{|l|}{\textit{Inherited from cssutils.css.cssrule.CSSRule \textit{(Section \ref{cssutils:css:cssrule:CSSRule})}}}\\
\multicolumn{2}{|p{\varwidth}|}{\raggedright CHARSET\_RULE, COMMENT, FONT\_FACE\_RULE, IMPORT\_RULE, MEDIA\_RULE, NAMESPACE\_RULE, PAGE\_RULE, STYLE\_RULE, UNKNOWN\_RULE}\\
\cline{1-2}
\end{longtable}

    \index{cssutils \textit{(package)}!cssutils.css \textit{(package)}!cssutils.css.csspagerule \textit{(module)}!cssutils.css.csspagerule.CSSPageRule \textit{(class)}|)}
    \index{cssutils \textit{(package)}!cssutils.css \textit{(package)}!cssutils.css.csspagerule \textit{(module)}|)}
