%
% API Documentation for cssutils
% Module cssutils.stylesheets.stylesheet
%
% Generated by epydoc 3.0.1
% [Fri Feb 01 19:05:22 2008]
%

%%%%%%%%%%%%%%%%%%%%%%%%%%%%%%%%%%%%%%%%%%%%%%%%%%%%%%%%%%%%%%%%%%%%%%%%%%%
%%                          Module Description                           %%
%%%%%%%%%%%%%%%%%%%%%%%%%%%%%%%%%%%%%%%%%%%%%%%%%%%%%%%%%%%%%%%%%%%%%%%%%%%

    \index{cssutils \textit{(package)}!cssutils.stylesheets \textit{(package)}!cssutils.stylesheets.stylesheet \textit{(module)}|(}
\section{Module cssutils.stylesheets.stylesheet}

    \label{cssutils:stylesheets:stylesheet}

StyleSheet implements DOM Level 2 Style Sheets StyleSheet.
\textbf{Version:} \$LastChangedRevision: 950 \$



\textbf{Date:} \$LastChangedDate: 2008-01-27 17:28:41 +0100 (So, 27 Jan 2008) \$



\textbf{Author:} \$LastChangedBy: cthedot \$




%%%%%%%%%%%%%%%%%%%%%%%%%%%%%%%%%%%%%%%%%%%%%%%%%%%%%%%%%%%%%%%%%%%%%%%%%%%
%%                           Class Description                           %%
%%%%%%%%%%%%%%%%%%%%%%%%%%%%%%%%%%%%%%%%%%%%%%%%%%%%%%%%%%%%%%%%%%%%%%%%%%%

    \index{cssutils \textit{(package)}!cssutils.stylesheets \textit{(package)}!cssutils.stylesheets.stylesheet \textit{(module)}!cssutils.stylesheets.stylesheet.StyleSheet \textit{(class)}|(}
\subsection{Class StyleSheet}

    \label{cssutils:stylesheets:stylesheet:StyleSheet}
\begin{tabular}{cccccccc}
% Line for object, linespec=[False, False]
\multicolumn{2}{r}{\settowidth{\BCL}{object}\multirow{2}{\BCL}{object}}
&&
&&
  \\\cline{3-3}
  &&\multicolumn{1}{c|}{}
&&
&&
  \\
% Line for cssutils.util.Base, linespec=[False]
\multicolumn{4}{r}{\settowidth{\BCL}{cssutils.util.Base}\multirow{2}{\BCL}{cssutils.util.Base}}
&&
  \\\cline{5-5}
  &&&&\multicolumn{1}{c|}{}
&&
  \\
&&&&\multicolumn{2}{l}{\textbf{cssutils.stylesheets.stylesheet.StyleSheet}}
\end{tabular}

\textbf{Known Subclasses:} cssutils.css.cssstylesheet.CSSStyleSheet


The StyleSheet interface is the abstract base interface
for any type of style sheet. It represents a single style
sheet associated with a structured document.

In HTML, the StyleSheet interface represents either an
external style sheet, included via the HTML LINK element,
or an inline STYLE element (-ch: also an @import stylesheet?).

In XML, this interface represents
an external style sheet, included via a style sheet
processing instruction.

%%%%%%%%%%%%%%%%%%%%%%%%%%%%%%%%%%%%%%%%%%%%%%%%%%%%%%%%%%%%%%%%%%%%%%%%%%%
%%                                Methods                                %%
%%%%%%%%%%%%%%%%%%%%%%%%%%%%%%%%%%%%%%%%%%%%%%%%%%%%%%%%%%%%%%%%%%%%%%%%%%%

  \subsubsection{Methods}

    \vspace{0.5ex}

\hspace{.8\funcindent}\begin{boxedminipage}{\funcwidth}

    \raggedright \textbf{\_\_init\_\_}(\textit{self}, \textit{type}={\tt \texttt{'}\texttt{text/css}\texttt{'}}, \textit{href}={\tt None}, \textit{media}={\tt None}, \textit{title}={\tt \texttt{u'}\texttt{}\texttt{'}}, \textit{disabled}={\tt None}, \textit{ownerNode}={\tt None}, \textit{parentStyleSheet}={\tt None})

    \vspace{-1.5ex}

    \rule{\textwidth}{0.5\fboxrule}
\setlength{\parskip}{2ex}
\begin{description}
\item[{type: readonly}] \leavevmode 
This specifies the style sheet language for this
style sheet. The style sheet language is specified
as a content type (e.g. ``text/css''). The content
type is often specified in the ownerNode. Also see
the type attribute definition for the LINK element
in HTML 4.0, and the type pseudo-attribute for the
XML style sheet processing instruction.

\item[{href: readonly}] \leavevmode 
If the style sheet is a linked style sheet, the value
of this attribute is its location. For inline style
sheets, the value of this attribute is None. See the
href attribute definition for the LINK element in HTML
4.0, and the href pseudo-attribute for the XML style
sheet processing instruction.

\item[{media: of type MediaList, readonly}] \leavevmode 
The intended destination media for style information.
The media is often specified in the ownerNode. If no
media has been specified, the MediaList will be empty.
See the media attribute definition for the LINK element
in HTML 4.0, and the media pseudo-attribute for the XML
style sheet processing instruction. Modifying the media
list may cause a change to the attribute disabled.

\item[{title: readonly}] \leavevmode 
The advisory title. The title is often specified in
the ownerNode. See the title attribute definition for
the LINK element in HTML 4.0, and the title
pseudo-attribute for the XML style sheet processing
instruction.

\item[{disabled: False if the style sheet is applied to the}] \leavevmode 
document. True if it is not. Modifying this attribute
may cause a new resolution of style for the document.
A stylesheet only applies if both an appropriate medium
definition is present and the disabled attribute is False.
So, if the media doesn't apply to the current user agent,
the disabled attribute is ignored.

\item[{ownerNode: of type Node, readonly}] \leavevmode 
The node that associates this style sheet with the
document. For HTML, this may be the corresponding LINK
or STYLE element. For XML, it may be the linking
processing instruction. For style sheets that are
included by other style sheets, the value of this
attribute is None.

\item[{parentStyleSheet: of type StyleSheet, readonly}] \leavevmode 
For style sheet languages that support the concept
of style sheet inclusion, this attribute represents
the including style sheet, if one exists. If the style
sheet is a top-level style sheet, or the style sheet
language does not support inclusion, the value of this
attribute is None.

\end{description}
\setlength{\parskip}{1ex}
      Overrides: object.\_\_init\_\_

    \end{boxedminipage}


\large{\textbf{\textit{Inherited from object}}}

\begin{quote}
\_\_delattr\_\_(), \_\_getattribute\_\_(), \_\_hash\_\_(), \_\_new\_\_(), \_\_reduce\_\_(), \_\_reduce\_ex\_\_(), \_\_repr\_\_(), \_\_setattr\_\_(), \_\_str\_\_()
\end{quote}

%%%%%%%%%%%%%%%%%%%%%%%%%%%%%%%%%%%%%%%%%%%%%%%%%%%%%%%%%%%%%%%%%%%%%%%%%%%
%%                              Properties                               %%
%%%%%%%%%%%%%%%%%%%%%%%%%%%%%%%%%%%%%%%%%%%%%%%%%%%%%%%%%%%%%%%%%%%%%%%%%%%

  \subsubsection{Properties}

    \vspace{-1cm}
\hspace{\varindent}\begin{longtable}{|p{\varnamewidth}|p{\vardescrwidth}|l}
\cline{1-2}
\cline{1-2} \centering \textbf{Name} & \centering \textbf{Description}& \\
\cline{1-2}
\endhead\cline{1-2}\multicolumn{3}{r}{\small\textit{continued on next page}}\\\endfoot\cline{1-2}
\endlastfoot\raggedright p\-a\-r\-e\-n\-t\-S\-t\-y\-l\-e\-S\-h\-e\-e\-t\- & &\\
\cline{1-2}
\multicolumn{2}{|l|}{\textit{Inherited from object}}\\
\multicolumn{2}{|p{\varwidth}|}{\raggedright \_\_class\_\_}\\
\cline{1-2}
\end{longtable}

    \index{cssutils \textit{(package)}!cssutils.stylesheets \textit{(package)}!cssutils.stylesheets.stylesheet \textit{(module)}!cssutils.stylesheets.stylesheet.StyleSheet \textit{(class)}|)}
    \index{cssutils \textit{(package)}!cssutils.stylesheets \textit{(package)}!cssutils.stylesheets.stylesheet \textit{(module)}|)}
