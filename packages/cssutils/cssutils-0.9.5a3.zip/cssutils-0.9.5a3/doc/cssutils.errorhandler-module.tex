%
% API Documentation for cssutils
% Module cssutils.errorhandler
%
% Generated by epydoc 3.0.1
% [Fri Feb 01 19:05:21 2008]
%

%%%%%%%%%%%%%%%%%%%%%%%%%%%%%%%%%%%%%%%%%%%%%%%%%%%%%%%%%%%%%%%%%%%%%%%%%%%
%%                          Module Description                           %%
%%%%%%%%%%%%%%%%%%%%%%%%%%%%%%%%%%%%%%%%%%%%%%%%%%%%%%%%%%%%%%%%%%%%%%%%%%%

    \index{cssutils \textit{(package)}!cssutils.errorhandler \textit{(module)}|(}
\section{Module cssutils.errorhandler}

    \label{cssutils:errorhandler}

cssutils ErrorHandler
\begin{description}
\item[{ErrorHandler}] \leavevmode 
used as log with usual levels (debug, info, warn, error)

if instanciated with \texttt{raiseExceptions=True} raises exeptions instead
of logging

\item[{log}] \leavevmode 
defaults to instance of ErrorHandler for any kind of log message from
lexerm, parser etc.
\begin{itemize}
\item {} 
raiseExceptions = {[}False, True{]}

\item {} 
setloglevel(loglevel)

\end{itemize}

\end{description}
\textbf{Version:} \$LastChangedRevision: 878 \$



\textbf{Date:} \$LastChangedDate: 2008-01-19 22:57:14 +0100 (Sa, 19 Jan 2008) \$



\textbf{Author:} \$LastChangedBy: cthedot \$




%%%%%%%%%%%%%%%%%%%%%%%%%%%%%%%%%%%%%%%%%%%%%%%%%%%%%%%%%%%%%%%%%%%%%%%%%%%
%%                           Class Description                           %%
%%%%%%%%%%%%%%%%%%%%%%%%%%%%%%%%%%%%%%%%%%%%%%%%%%%%%%%%%%%%%%%%%%%%%%%%%%%

    \index{cssutils \textit{(package)}!cssutils.errorhandler \textit{(module)}!cssutils.errorhandler.ErrorHandler \textit{(class)}|(}
\subsection{Class ErrorHandler}

    \label{cssutils:errorhandler:ErrorHandler}
\begin{tabular}{cccccccc}
% Line for object, linespec=[False, False]
\multicolumn{2}{r}{\settowidth{\BCL}{object}\multirow{2}{\BCL}{object}}
&&
&&
  \\\cline{3-3}
  &&\multicolumn{1}{c|}{}
&&
&&
  \\
% Line for cssutils.errorhandler.\_ErrorHandler, linespec=[False]
\multicolumn{4}{r}{\settowidth{\BCL}{cssutils.errorhandler.\_ErrorHandler}\multirow{2}{\BCL}{cssutils.errorhandler.\_ErrorHandler}}
&&
  \\\cline{5-5}
  &&&&\multicolumn{1}{c|}{}
&&
  \\
&&&&\multicolumn{2}{l}{\textbf{cssutils.errorhandler.ErrorHandler}}
\end{tabular}


Singleton, see {\_}ErrorHandler

%%%%%%%%%%%%%%%%%%%%%%%%%%%%%%%%%%%%%%%%%%%%%%%%%%%%%%%%%%%%%%%%%%%%%%%%%%%
%%                                Methods                                %%
%%%%%%%%%%%%%%%%%%%%%%%%%%%%%%%%%%%%%%%%%%%%%%%%%%%%%%%%%%%%%%%%%%%%%%%%%%%

  \subsubsection{Methods}

    \vspace{0.5ex}

\hspace{.8\funcindent}\begin{boxedminipage}{\funcwidth}

    \raggedright \textbf{\_\_init\_\_}(\textit{self}, \textit{log}={\tt None}, \textit{defaultloglevel}={\tt 20}, \textit{raiseExceptions}={\tt True})

\setlength{\parskip}{2ex}

inits log if none given
\begin{description}
\item[{log}] \leavevmode 
for parse messages, default logs to sys.stderr

\item[{defaultloglevel}] \leavevmode 
if none give this is logging.DEBUG

\item[{raiseExceptions}] \leavevmode \begin{itemize}
\item {} 
True: Errors will be reported to the calling app,
e.g. during building

\item {} 
False: Errors will be written to the log, this is the
default behaviour when parsing

\end{itemize}

\end{description}
\setlength{\parskip}{1ex}
      Overrides: object.\_\_init\_\_ 	extit{(inherited documentation)}

    \end{boxedminipage}


\large{\textbf{\textit{Inherited from cssutils.errorhandler.\_ErrorHandler}}}

\begin{quote}
\_\_getattr\_\_(), setlog(), setloglevel()
\end{quote}

\large{\textbf{\textit{Inherited from object}}}

\begin{quote}
\_\_delattr\_\_(), \_\_getattribute\_\_(), \_\_hash\_\_(), \_\_new\_\_(), \_\_reduce\_\_(), \_\_reduce\_ex\_\_(), \_\_repr\_\_(), \_\_setattr\_\_(), \_\_str\_\_()
\end{quote}

%%%%%%%%%%%%%%%%%%%%%%%%%%%%%%%%%%%%%%%%%%%%%%%%%%%%%%%%%%%%%%%%%%%%%%%%%%%
%%                              Properties                               %%
%%%%%%%%%%%%%%%%%%%%%%%%%%%%%%%%%%%%%%%%%%%%%%%%%%%%%%%%%%%%%%%%%%%%%%%%%%%

  \subsubsection{Properties}

    \vspace{-1cm}
\hspace{\varindent}\begin{longtable}{|p{\varnamewidth}|p{\vardescrwidth}|l}
\cline{1-2}
\cline{1-2} \centering \textbf{Name} & \centering \textbf{Description}& \\
\cline{1-2}
\endhead\cline{1-2}\multicolumn{3}{r}{\small\textit{continued on next page}}\\\endfoot\cline{1-2}
\endlastfoot\multicolumn{2}{|l|}{\textit{Inherited from object}}\\
\multicolumn{2}{|p{\varwidth}|}{\raggedright \_\_class\_\_}\\
\cline{1-2}
\end{longtable}


%%%%%%%%%%%%%%%%%%%%%%%%%%%%%%%%%%%%%%%%%%%%%%%%%%%%%%%%%%%%%%%%%%%%%%%%%%%
%%                            Class Variables                            %%
%%%%%%%%%%%%%%%%%%%%%%%%%%%%%%%%%%%%%%%%%%%%%%%%%%%%%%%%%%%%%%%%%%%%%%%%%%%

  \subsubsection{Class Variables}

    \vspace{-1cm}
\hspace{\varindent}\begin{longtable}{|p{\varnamewidth}|p{\vardescrwidth}|l}
\cline{1-2}
\cline{1-2} \centering \textbf{Name} & \centering \textbf{Description}& \\
\cline{1-2}
\endhead\cline{1-2}\multicolumn{3}{r}{\small\textit{continued on next page}}\\\endfoot\cline{1-2}
\endlastfoot\raggedright i\-n\-s\-t\-a\-n\-c\-e\- & \raggedright \textbf{Value:} 
{\tt None}&\\
\cline{1-2}
\end{longtable}

    \index{cssutils \textit{(package)}!cssutils.errorhandler \textit{(module)}!cssutils.errorhandler.ErrorHandler \textit{(class)}|)}
    \index{cssutils \textit{(package)}!cssutils.errorhandler \textit{(module)}|)}
