%
% API Documentation for cssutils
% Module cssutils.css.cssnamespacerule
%
% Generated by epydoc 3.0.1
% [Fri Feb 01 19:05:21 2008]
%

%%%%%%%%%%%%%%%%%%%%%%%%%%%%%%%%%%%%%%%%%%%%%%%%%%%%%%%%%%%%%%%%%%%%%%%%%%%
%%                          Module Description                           %%
%%%%%%%%%%%%%%%%%%%%%%%%%%%%%%%%%%%%%%%%%%%%%%%%%%%%%%%%%%%%%%%%%%%%%%%%%%%

    \index{cssutils \textit{(package)}!cssutils.css \textit{(package)}!cssutils.css.cssnamespacerule \textit{(module)}|(}
\section{Module cssutils.css.cssnamespacerule}

    \label{cssutils:css:cssnamespacerule}

CSSNamespaceRule currently implements
\href{http://www.w3.org/TR/2006/WD-css3-namespace-20060828/}{http://www.w3.org/TR/2006/WD-css3-namespace-20060828/}
\begin{description}
\item[{The following changes have been done:}] \leavevmode 
1. the url() syntax is not implemented as it may (?) be deprecated
anyway

\end{description}
\textbf{Version:} \$LastChangedRevision: 950 \$



\textbf{Date:} \$LastChangedDate: 2008-01-27 17:28:41 +0100 (So, 27 Jan 2008) \$



\textbf{Author:} \$LastChangedBy: cthedot \$




%%%%%%%%%%%%%%%%%%%%%%%%%%%%%%%%%%%%%%%%%%%%%%%%%%%%%%%%%%%%%%%%%%%%%%%%%%%
%%                           Class Description                           %%
%%%%%%%%%%%%%%%%%%%%%%%%%%%%%%%%%%%%%%%%%%%%%%%%%%%%%%%%%%%%%%%%%%%%%%%%%%%

    \index{cssutils \textit{(package)}!cssutils.css \textit{(package)}!cssutils.css.cssnamespacerule \textit{(module)}!cssutils.css.cssnamespacerule.CSSNamespaceRule \textit{(class)}|(}
\subsection{Class CSSNamespaceRule}

    \label{cssutils:css:cssnamespacerule:CSSNamespaceRule}
\begin{tabular}{cccccccccc}
% Line for object, linespec=[False, False, False]
\multicolumn{2}{r}{\settowidth{\BCL}{object}\multirow{2}{\BCL}{object}}
&&
&&
&&
  \\\cline{3-3}
  &&\multicolumn{1}{c|}{}
&&
&&
&&
  \\
% Line for cssutils.util.Base, linespec=[False, False]
\multicolumn{4}{r}{\settowidth{\BCL}{cssutils.util.Base}\multirow{2}{\BCL}{cssutils.util.Base}}
&&
&&
  \\\cline{5-5}
  &&&&\multicolumn{1}{c|}{}
&&
&&
  \\
% Line for cssutils.css.cssrule.CSSRule, linespec=[False]
\multicolumn{6}{r}{\settowidth{\BCL}{cssutils.css.cssrule.CSSRule}\multirow{2}{\BCL}{cssutils.css.cssrule.CSSRule}}
&&
  \\\cline{7-7}
  &&&&&&\multicolumn{1}{c|}{}
&&
  \\
&&&&&&\multicolumn{2}{l}{\textbf{cssutils.css.cssnamespacerule.CSSNamespaceRule}}
\end{tabular}


Represents an @namespace rule within a CSS style sheet.

The @namespace at-rule declares a namespace prefix and associates
it with a given namespace (a string). This namespace prefix can then be
used in namespace-qualified names such as those described in the
Selectors Module {[}SELECT{]} or the Values and Units module {[}CSS3VAL{]}.


%___________________________________________________________________________

\hypertarget{properties}{}
\pdfbookmark[3]{Properties}{properties}
\paragraph*{Properties}
\label{properties}
\begin{description}
\item[{cssText: of type DOMString}] \leavevmode 
The parsable textual representation of this rule

\item[{namespaceURI: of type DOMString}] \leavevmode 
The namespace URI (a simple string!) which is bound to the given
prefix. If no prefix is set (\texttt{CSSNamespaceRule.prefix=='{}'})
the namespace defined by \texttt{namespaceURI} is set as the default
namespace.

\item[{prefix: of type DOMString}] \leavevmode 
The prefix used in the stylesheet for the given
\texttt{CSSNamespaceRule.nsuri}. If prefix is empty namespaceURI sets a
default namespace for the stylesheet.

\end{description}


%___________________________________________________________________________

\hypertarget{cssutils-only}{}
\pdfbookmark[4]{cssutils only}{cssutils-only}
\subparagraph*{cssutils only}
\label{cssutils-only}
\begin{description}
\item[{atkeyword:}] \leavevmode 
the literal keyword used

\end{description}

Inherits properties from CSSRule


%___________________________________________________________________________

\hypertarget{format}{}
\pdfbookmark[3]{Format}{format}
\paragraph*{Format}
\label{format}
\begin{description}
\item[{namespace}] \leavevmode 
: NAMESPACE{\_}SYM S* {[}namespace{\_}prefix S*{]}? {[}STRING{\textbar}URI{]} S* ';' S*
;

\item[{namespace{\_}prefix}] \leavevmode 
: IDENT
;

\end{description}

%%%%%%%%%%%%%%%%%%%%%%%%%%%%%%%%%%%%%%%%%%%%%%%%%%%%%%%%%%%%%%%%%%%%%%%%%%%
%%                                Methods                                %%
%%%%%%%%%%%%%%%%%%%%%%%%%%%%%%%%%%%%%%%%%%%%%%%%%%%%%%%%%%%%%%%%%%%%%%%%%%%

  \subsubsection{Methods}

    \vspace{0.5ex}

\hspace{.8\funcindent}\begin{boxedminipage}{\funcwidth}

    \raggedright \textbf{\_\_init\_\_}(\textit{self}, \textit{namespaceURI}={\tt None}, \textit{prefix}={\tt None}, \textit{cssText}={\tt None}, \textit{parentRule}={\tt None}, \textit{parentStyleSheet}={\tt None}, \textit{readonly}={\tt False})

    \vspace{-1.5ex}

    \rule{\textwidth}{0.5\fboxrule}
\setlength{\parskip}{2ex}

Do not use as positional but as keyword parameters only!

If readonly allows setting of properties in constructor only

format namespace:
\begin{quote}{\ttfamily \raggedright \noindent
:~NAMESPACE{\_}SYM~S*~{[}namespace{\_}prefix~S*{]}?~{[}STRING|URI{]}~S*~';'~S*~\\
;
}\end{quote}
\setlength{\parskip}{1ex}
      \textbf{Parameters}
      \vspace{-1ex}

      \begin{quote}
        \begin{Ventry}{xxxxxxxxxxxxxxxx}

          \item[namespaceURI]


The namespace URI (a simple string!) which is bound to the
given prefix. If no prefix is set
(\texttt{CSSNamespaceRule.prefix=='{}'}) the namespace defined by
namespaceURI is set as the default namespace
          \item[prefix]


The prefix used in the stylesheet for the given
\texttt{CSSNamespaceRule.uri}.
          \item[cssText]


if no namespaceURI is given cssText must be given to set
a namespaceURI as this is readonly later on
          \item[parentStyleSheet]


sheet where this rule belongs to
        \end{Ventry}

      \end{quote}

      Overrides: object.\_\_init\_\_

    \end{boxedminipage}

    \vspace{0.5ex}

\hspace{.8\funcindent}\begin{boxedminipage}{\funcwidth}

    \raggedright \textbf{\_\_repr\_\_}(\textit{self})

\setlength{\parskip}{2ex}
    repr(x)

\setlength{\parskip}{1ex}
      Overrides: object.\_\_repr\_\_ 	extit{(inherited documentation)}

    \end{boxedminipage}

    \vspace{0.5ex}

\hspace{.8\funcindent}\begin{boxedminipage}{\funcwidth}

    \raggedright \textbf{\_\_str\_\_}(\textit{self})

\setlength{\parskip}{2ex}
    str(x)

\setlength{\parskip}{1ex}
      Overrides: object.\_\_str\_\_ 	extit{(inherited documentation)}

    \end{boxedminipage}


\large{\textbf{\textit{Inherited from object}}}

\begin{quote}
\_\_delattr\_\_(), \_\_getattribute\_\_(), \_\_hash\_\_(), \_\_new\_\_(), \_\_reduce\_\_(), \_\_reduce\_ex\_\_(), \_\_setattr\_\_()
\end{quote}

%%%%%%%%%%%%%%%%%%%%%%%%%%%%%%%%%%%%%%%%%%%%%%%%%%%%%%%%%%%%%%%%%%%%%%%%%%%
%%                              Properties                               %%
%%%%%%%%%%%%%%%%%%%%%%%%%%%%%%%%%%%%%%%%%%%%%%%%%%%%%%%%%%%%%%%%%%%%%%%%%%%

  \subsubsection{Properties}

    \vspace{-1cm}
\hspace{\varindent}\begin{longtable}{|p{\varnamewidth}|p{\vardescrwidth}|l}
\cline{1-2}
\cline{1-2} \centering \textbf{Name} & \centering \textbf{Description}& \\
\cline{1-2}
\endhead\cline{1-2}\multicolumn{3}{r}{\small\textit{continued on next page}}\\\endfoot\cline{1-2}
\endlastfoot\raggedright n\-a\-m\-e\-s\-p\-a\-c\-e\-U\-R\-I\- & \raggedright URI (string!) of the defined namespace.&\\
\cline{1-2}
\raggedright p\-r\-e\-f\-i\-x\- & \raggedright Prefix used for the defined namespace.&\\
\cline{1-2}
\raggedright p\-a\-r\-e\-n\-t\-S\-t\-y\-l\-e\-S\-h\-e\-e\-t\- & \raggedright Containing CSSStyleSheet.&\\
\cline{1-2}
\raggedright c\-s\-s\-T\-e\-x\-t\- & \raggedright (DOM attribute) The parsable textual representation.&\\
\cline{1-2}
\multicolumn{2}{|l|}{\textit{Inherited from cssutils.css.cssrule.CSSRule \textit{(Section \ref{cssutils:css:cssrule:CSSRule})}}}\\
\multicolumn{2}{|p{\varwidth}|}{\raggedright parentRule, typeString}\\
\cline{1-2}
\multicolumn{2}{|l|}{\textit{Inherited from object}}\\
\multicolumn{2}{|p{\varwidth}|}{\raggedright \_\_class\_\_}\\
\cline{1-2}
\end{longtable}


%%%%%%%%%%%%%%%%%%%%%%%%%%%%%%%%%%%%%%%%%%%%%%%%%%%%%%%%%%%%%%%%%%%%%%%%%%%
%%                            Class Variables                            %%
%%%%%%%%%%%%%%%%%%%%%%%%%%%%%%%%%%%%%%%%%%%%%%%%%%%%%%%%%%%%%%%%%%%%%%%%%%%

  \subsubsection{Class Variables}

    \vspace{-1cm}
\hspace{\varindent}\begin{longtable}{|p{\varnamewidth}|p{\vardescrwidth}|l}
\cline{1-2}
\cline{1-2} \centering \textbf{Name} & \centering \textbf{Description}& \\
\cline{1-2}
\endhead\cline{1-2}\multicolumn{3}{r}{\small\textit{continued on next page}}\\\endfoot\cline{1-2}
\endlastfoot\raggedright t\-y\-p\-e\- & \raggedright The type of this rule, as defined by a CSSRule type constant.
Overwritten in derived classes.

The expectation is that binding-specific casting methods can be used to
cast down from an instance of the CSSRule interface to the specific
derived interface implied by the type.
(Casting not for this Python implementation I guess...)

\textbf{Value:} 
{\tt 7}&\\
\cline{1-2}
\multicolumn{2}{|l|}{\textit{Inherited from cssutils.css.cssrule.CSSRule \textit{(Section \ref{cssutils:css:cssrule:CSSRule})}}}\\
\multicolumn{2}{|p{\varwidth}|}{\raggedright CHARSET\_RULE, COMMENT, FONT\_FACE\_RULE, IMPORT\_RULE, MEDIA\_RULE, NAMESPACE\_RULE, PAGE\_RULE, STYLE\_RULE, UNKNOWN\_RULE}\\
\cline{1-2}
\end{longtable}

    \index{cssutils \textit{(package)}!cssutils.css \textit{(package)}!cssutils.css.cssnamespacerule \textit{(module)}!cssutils.css.cssnamespacerule.CSSNamespaceRule \textit{(class)}|)}
    \index{cssutils \textit{(package)}!cssutils.css \textit{(package)}!cssutils.css.cssnamespacerule \textit{(module)}|)}
