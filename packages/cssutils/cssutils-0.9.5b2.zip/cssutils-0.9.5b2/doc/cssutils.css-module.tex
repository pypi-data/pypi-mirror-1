%
% API Documentation for cssutils
% Package cssutils.css
%
% Generated by epydoc 3.0.1
% [Fri Feb 01 19:05:21 2008]
%

%%%%%%%%%%%%%%%%%%%%%%%%%%%%%%%%%%%%%%%%%%%%%%%%%%%%%%%%%%%%%%%%%%%%%%%%%%%
%%                          Module Description                           %%
%%%%%%%%%%%%%%%%%%%%%%%%%%%%%%%%%%%%%%%%%%%%%%%%%%%%%%%%%%%%%%%%%%%%%%%%%%%

    \index{cssutils \textit{(package)}!cssutils.css \textit{(package)}|(}
\section{Package cssutils.css}

    \label{cssutils:css}

Document Object Model Level 2 CSS
\href{http://www.w3.org/TR/2000/PR-DOM-Level-2-Style-20000927/css.html}{http://www.w3.org/TR/2000/PR-DOM-Level-2-Style-20000927/css.html}
\begin{description}
\item[{currently implemented}] \leavevmode \begin{itemize}
\item {} 
CSSStyleSheet

\item {} 
CSSRuleList

\item {} 
CSSRule

\item {} 
CSSComment (cssutils addon)

\item {} 
CSSCharsetRule

\item {} 
CSSFontFaceRule

\item {} 
CSSImportRule

\item {} 
CSSMediaRule

\item {} 
CSSNamespaceRule (WD)

\item {} 
CSSPageRule

\item {} 
CSSStyleRule

\item {} 
CSSUnkownRule

\item {} 
Selector and SelectorList

\item {} 
CSSStyleDeclaration

\item {} 
CSS2Properties

\item {} 
CSSValue

\item {} 
CSSPrimitiveValue

\item {} 
CSSValueList

\end{itemize}

\item[{todo}] \leavevmode \begin{itemize}
\item {} 
RGBColor, Rect, Counter

\end{itemize}

\end{description}
\textbf{Version:} \$LastChangedRevision: 879 \$



\textbf{Date:} \$LastChangedDate: 2008-01-19 22:57:42 +0100 (Sa, 19 Jan 2008) \$



\textbf{Author:} \$LastChangedBy: cthedot \$




%%%%%%%%%%%%%%%%%%%%%%%%%%%%%%%%%%%%%%%%%%%%%%%%%%%%%%%%%%%%%%%%%%%%%%%%%%%
%%                                Modules                                %%
%%%%%%%%%%%%%%%%%%%%%%%%%%%%%%%%%%%%%%%%%%%%%%%%%%%%%%%%%%%%%%%%%%%%%%%%%%%

\subsection{Modules}

\begin{itemize}
\setlength{\parskip}{0ex}
\item \textbf{csscharsetrule}: 
CSSCharsetRule implements DOM Level 2 CSS CSSCharsetRule.


  \textit{(Section \ref{cssutils:css:csscharsetrule}, p.~\pageref{cssutils:css:csscharsetrule})}

\item \textbf{csscomment}: 
CSSComment is not defined in DOM Level 2 at all but a cssutils defined
class only.


  \textit{(Section \ref{cssutils:css:csscomment}, p.~\pageref{cssutils:css:csscomment})}

\item \textbf{cssfontfacerule}: 
CSSFontFaceRule implements DOM Level 2 CSS CSSFontFaceRule.


  \textit{(Section \ref{cssutils:css:cssfontfacerule}, p.~\pageref{cssutils:css:cssfontfacerule})}

\item \textbf{cssimportrule}: 
CSSImportRule implements DOM Level 2 CSS CSSImportRule.


  \textit{(Section \ref{cssutils:css:cssimportrule}, p.~\pageref{cssutils:css:cssimportrule})}

\item \textbf{cssmediarule}: 
CSSMediaRule implements DOM Level 2 CSS CSSMediaRule.


  \textit{(Section \ref{cssutils:css:cssmediarule}, p.~\pageref{cssutils:css:cssmediarule})}

\item \textbf{cssnamespacerule}: 
CSSNamespaceRule currently implements
\href{http://www.w3.org/TR/2006/WD-css3-namespace-20060828/}{http://www.w3.org/TR/2006/WD-css3-namespace-20060828/}


  \textit{(Section \ref{cssutils:css:cssnamespacerule}, p.~\pageref{cssutils:css:cssnamespacerule})}

\item \textbf{csspagerule}: 
CSSPageRule implements DOM Level 2 CSS CSSPageRule.


  \textit{(Section \ref{cssutils:css:csspagerule}, p.~\pageref{cssutils:css:csspagerule})}

\item \textbf{cssproperties}: 
CSS2Properties (partly!) implements DOM Level 2 CSS CSS2Properties used
by CSSStyleDeclaration


  \textit{(Section \ref{cssutils:css:cssproperties}, p.~\pageref{cssutils:css:cssproperties})}

\item \textbf{cssrule}: 
CSSRule implements DOM Level 2 CSS CSSRule.


  \textit{(Section \ref{cssutils:css:cssrule}, p.~\pageref{cssutils:css:cssrule})}

\item \textbf{cssrulelist}: 
CSSRuleList implements DOM Level 2 CSS CSSRuleList.


  \textit{(Section \ref{cssutils:css:cssrulelist}, p.~\pageref{cssutils:css:cssrulelist})}

\item \textbf{cssstyledeclaration}: 
CSSStyleDeclaration implements DOM Level 2 CSS CSSStyleDeclaration and
extends CSS2Properties


  \textit{(Section \ref{cssutils:css:cssstyledeclaration}, p.~\pageref{cssutils:css:cssstyledeclaration})}

\item \textbf{cssstylerule}: 
CSSStyleRule implements DOM Level 2 CSS CSSStyleRule.


  \textit{(Section \ref{cssutils:css:cssstylerule}, p.~\pageref{cssutils:css:cssstylerule})}

\item \textbf{cssstylesheet}: 
CSSStyleSheet implements DOM Level 2 CSS CSSStyleSheet.


  \textit{(Section \ref{cssutils:css:cssstylesheet}, p.~\pageref{cssutils:css:cssstylesheet})}

\item \textbf{cssunknownrule}: 
CSSUnknownRule implements DOM Level 2 CSS CSSUnknownRule.


  \textit{(Section \ref{cssutils:css:cssunknownrule}, p.~\pageref{cssutils:css:cssunknownrule})}

\item \textbf{cssvalue}: 
CSSValue related classes


  \textit{(Section \ref{cssutils:css:cssvalue}, p.~\pageref{cssutils:css:cssvalue})}

\item \textbf{property}: 
Property is a single CSS property in a CSSStyleDeclaration


  \textit{(Section \ref{cssutils:css:property}, p.~\pageref{cssutils:css:property})}

\item \textbf{selector}: 
Selector is a single Selector of a CSSStyleRule SelectorList.


  \textit{(Section \ref{cssutils:css:selector}, p.~\pageref{cssutils:css:selector})}

\item \textbf{selectorlist}: 
SelectorList is a list of CSS Selector objects.


  \textit{(Section \ref{cssutils:css:selectorlist}, p.~\pageref{cssutils:css:selectorlist})}

\end{itemize}


%%%%%%%%%%%%%%%%%%%%%%%%%%%%%%%%%%%%%%%%%%%%%%%%%%%%%%%%%%%%%%%%%%%%%%%%%%%
%%                           Class Description                           %%
%%%%%%%%%%%%%%%%%%%%%%%%%%%%%%%%%%%%%%%%%%%%%%%%%%%%%%%%%%%%%%%%%%%%%%%%%%%

    \index{cssutils \textit{(package)}!cssutils.css \textit{(package)}!cssutils.css.cssstylesheet \textit{(module)}!cssutils.css.cssstylesheet.CSSStyleSheet \textit{(class)}|(}
\subsection{Class CSSStyleSheet}

    \label{cssutils:css:cssstylesheet:CSSStyleSheet}
\begin{tabular}{cccccccccc}
% Line for object, linespec=[False, False, False]
\multicolumn{2}{r}{\settowidth{\BCL}{object}\multirow{2}{\BCL}{object}}
&&
&&
&&
  \\\cline{3-3}
  &&\multicolumn{1}{c|}{}
&&
&&
&&
  \\
% Line for cssutils.util.Base, linespec=[False, False]
\multicolumn{4}{r}{\settowidth{\BCL}{cssutils.util.Base}\multirow{2}{\BCL}{cssutils.util.Base}}
&&
&&
  \\\cline{5-5}
  &&&&\multicolumn{1}{c|}{}
&&
&&
  \\
% Line for cssutils.stylesheets.stylesheet.StyleSheet, linespec=[False]
\multicolumn{6}{r}{\settowidth{\BCL}{cssutils.stylesheets.stylesheet.StyleSheet}\multirow{2}{\BCL}{cssutils.stylesheets.stylesheet.StyleSheet}}
&&
  \\\cline{7-7}
  &&&&&&\multicolumn{1}{c|}{}
&&
  \\
&&&&&&\multicolumn{2}{l}{\textbf{cssutils.css.cssstylesheet.CSSStyleSheet}}
\end{tabular}


The CSSStyleSheet interface represents a CSS style sheet.


%___________________________________________________________________________

\hypertarget{properties}{}
\pdfbookmark[3]{Properties}{properties}
\paragraph*{Properties}
\label{properties}


%___________________________________________________________________________

\hypertarget{cssom}{}
\pdfbookmark[4]{CSSOM}{cssom}
\subparagraph*{CSSOM}
\label{cssom}
\begin{description}
\item[{cssRules}] \leavevmode 
of type CSSRuleList, (DOM readonly)

\item[{encoding}] \leavevmode 
reflects the encoding of an @charset rule or 'utf-8' (default)
if set to \texttt{None}

\item[{ownerRule}] \leavevmode 
of type CSSRule, readonly (NOT IMPLEMENTED YET)

\end{description}

Inherits properties from stylesheet.StyleSheet


%___________________________________________________________________________

\hypertarget{cssutils}{}
\pdfbookmark[4]{cssutils}{cssutils}
\subparagraph*{cssutils}
\label{cssutils}
\begin{description}
\item[{cssText: string}] \leavevmode 
a textual representation of the stylesheet

\item[{namespaces}] \leavevmode 
reflects set @namespace rules of this rule.
A dict of {\{}prefix: namespaceURI{\}} mapping.

\end{description}


%___________________________________________________________________________

\hypertarget{format}{}
\pdfbookmark[3]{Format}{format}
\paragraph*{Format}
\label{format}
\begin{description}
\item[{stylesheet}] \leavevmode \begin{description}
\item[{: {[} CHARSET{\_}SYM S* STRING S* ';' {]}?}] \leavevmode 
{[}S{\textbar}CDO{\textbar}CDC{]}* {[} import {[}S{\textbar}CDO{\textbar}CDC{]}* {]}*
{[} namespace {[}S{\textbar}CDO{\textbar}CDC{]}* {]}* {\#} according to @namespace WD
{[} {[} ruleset {\textbar} media {\textbar} page {]} {[}S{\textbar}CDO{\textbar}CDC{]}* {]}*

\end{description}

\end{description}

%%%%%%%%%%%%%%%%%%%%%%%%%%%%%%%%%%%%%%%%%%%%%%%%%%%%%%%%%%%%%%%%%%%%%%%%%%%
%%                                Methods                                %%
%%%%%%%%%%%%%%%%%%%%%%%%%%%%%%%%%%%%%%%%%%%%%%%%%%%%%%%%%%%%%%%%%%%%%%%%%%%

  \subsubsection{Methods}

    \vspace{0.5ex}

\hspace{.8\funcindent}\begin{boxedminipage}{\funcwidth}

    \raggedright \textbf{\_\_init\_\_}(\textit{self}, \textit{href}={\tt None}, \textit{media}={\tt None}, \textit{title}={\tt \texttt{u'}\texttt{}\texttt{'}}, \textit{disabled}={\tt None}, \textit{ownerNode}={\tt None}, \textit{parentStyleSheet}={\tt None}, \textit{readonly}={\tt False})

    \vspace{-1.5ex}

    \rule{\textwidth}{0.5\fboxrule}
\setlength{\parskip}{2ex}

init parameters are the same as for stylesheets.StyleSheet
\setlength{\parskip}{1ex}
      Overrides: object.\_\_init\_\_

    \end{boxedminipage}

    \label{cssutils:css:cssstylesheet:CSSStyleSheet:__iter__}
    \index{cssutils \textit{(package)}!cssutils.css \textit{(package)}!cssutils.css.cssstylesheet \textit{(module)}!cssutils.css.cssstylesheet.CSSStyleSheet \textit{(class)}!cssutils.css.cssstylesheet.CSSStyleSheet.\_\_iter\_\_ \textit{(method)}}

    \vspace{0.5ex}

\hspace{.8\funcindent}\begin{boxedminipage}{\funcwidth}

    \raggedright \textbf{\_\_iter\_\_}(\textit{self})

    \vspace{-1.5ex}

    \rule{\textwidth}{0.5\fboxrule}
\setlength{\parskip}{2ex}

generator which iterates over cssRules.
\setlength{\parskip}{1ex}
    \end{boxedminipage}

    \label{cssutils:css:cssstylesheet:CSSStyleSheet:add}
    \index{cssutils \textit{(package)}!cssutils.css \textit{(package)}!cssutils.css.cssstylesheet \textit{(module)}!cssutils.css.cssstylesheet.CSSStyleSheet \textit{(class)}!cssutils.css.cssstylesheet.CSSStyleSheet.add \textit{(method)}}

    \vspace{0.5ex}

\hspace{.8\funcindent}\begin{boxedminipage}{\funcwidth}

    \raggedright \textbf{add}(\textit{self}, \textit{rule})

    \vspace{-1.5ex}

    \rule{\textwidth}{0.5\fboxrule}
\setlength{\parskip}{2ex}

Adds rule to stylesheet at appropriate position.
Same as \texttt{sheet.insertRule(rule, inOrder=True)}.
\setlength{\parskip}{1ex}
    \end{boxedminipage}

    \label{cssutils:css:cssstylesheet:CSSStyleSheet:deleteRule}
    \index{cssutils \textit{(package)}!cssutils.css \textit{(package)}!cssutils.css.cssstylesheet \textit{(module)}!cssutils.css.cssstylesheet.CSSStyleSheet \textit{(class)}!cssutils.css.cssstylesheet.CSSStyleSheet.deleteRule \textit{(method)}}

    \vspace{0.5ex}

\hspace{.8\funcindent}\begin{boxedminipage}{\funcwidth}

    \raggedright \textbf{deleteRule}(\textit{self}, \textit{index})

    \vspace{-1.5ex}

    \rule{\textwidth}{0.5\fboxrule}
\setlength{\parskip}{2ex}

Used to delete a rule from the style sheet.
\setlength{\parskip}{1ex}
      \textbf{Parameters}
      \vspace{-1ex}

      \begin{quote}
        \begin{Ventry}{xxxxx}

          \item[index]


of the rule to remove in the StyleSheet's rule list. For an
index {\textless} 0 \textbf{no} INDEX{\_}SIZE{\_}ERR is raised but rules for
normal Python lists are used. E.g. \texttt{deleteRule(-1)} removes
the last rule in cssRules.
        \end{Ventry}

      \end{quote}

      \textbf{Raises}
    \vspace{-1ex}

      \begin{quote}
        \begin{description}

          \item[\texttt{INDEX\_SIZE\_ERR}]


(self)
Raised if the specified index does not correspond to a rule in
the style sheet's rule list.
          \item[\texttt{NAMESPACE\_ERR}]


(self)
Raised if removing this rule would result in an invalid StyleSheet
          \item[\texttt{NO\_MODIFICATION\_ALLOWED\_ERR}]


(self)
Raised if this style sheet is readonly.
        \end{description}

      \end{quote}

    \end{boxedminipage}

    \label{cssutils:css:cssstylesheet:CSSStyleSheet:insertRule}
    \index{cssutils \textit{(package)}!cssutils.css \textit{(package)}!cssutils.css.cssstylesheet \textit{(module)}!cssutils.css.cssstylesheet.CSSStyleSheet \textit{(class)}!cssutils.css.cssstylesheet.CSSStyleSheet.insertRule \textit{(method)}}

    \vspace{0.5ex}

\hspace{.8\funcindent}\begin{boxedminipage}{\funcwidth}

    \raggedright \textbf{insertRule}(\textit{self}, \textit{rule}, \textit{index}={\tt None}, \textit{inOrder}={\tt False})

    \vspace{-1.5ex}

    \rule{\textwidth}{0.5\fboxrule}
\setlength{\parskip}{2ex}

Used to insert a new rule into the style sheet. The new rule now
becomes part of the cascade.
\setlength{\parskip}{1ex}
      \textbf{Parameters}
      \vspace{-1ex}

      \begin{quote}
        \begin{Ventry}{xxxxxxx}

          \item[rule]


a parsable DOMString, in cssutils also a CSSRule or a
CSSRuleList
          \item[index]


of the rule before the new rule will be inserted.
If the specified index is equal to the length of the
StyleSheet's rule collection, the rule will be added to the end
of the style sheet.
If index is not given or None rule will be appended to rule
list.
          \item[inOrder]


if True the rule will be put to a proper location while
ignoring index but without raising HIERARCHY{\_}REQUEST{\_}ERR.
The resulting index is returned nevertheless
        \end{Ventry}

      \end{quote}

      \textbf{Return Value}
    \vspace{-1ex}

      \begin{quote}

the index within the stylesheet's rule collection
      \end{quote}

      \textbf{Raises}
    \vspace{-1ex}

      \begin{quote}
        \begin{description}

          \item[\texttt{HIERARCHY\_REQUEST\_ERR}]


(self)
Raised if the rule cannot be inserted at the specified index
e.g. if an @import rule is inserted after a standard rule set
or other at-rule.
          \item[\texttt{INDEX\_SIZE\_ERR}]


(self)
Raised if the specified index is not a valid insertion point.
          \item[\texttt{NO\_MODIFICATION\_ALLOWED\_ERR}]


(self)
Raised if this style sheet is readonly.
          \item[\texttt{SYNTAX\_ERR}]


(rule)
Raised if the specified rule has a syntax error and is
unparsable.
        \end{description}

      \end{quote}

    \end{boxedminipage}

    \label{cssutils:css:cssstylesheet:CSSStyleSheet:replaceUrls}
    \index{cssutils \textit{(package)}!cssutils.css \textit{(package)}!cssutils.css.cssstylesheet \textit{(module)}!cssutils.css.cssstylesheet.CSSStyleSheet \textit{(class)}!cssutils.css.cssstylesheet.CSSStyleSheet.replaceUrls \textit{(method)}}

    \vspace{0.5ex}

\hspace{.8\funcindent}\begin{boxedminipage}{\funcwidth}

    \raggedright \textbf{replaceUrls}(\textit{self}, \textit{replacer})

    \vspace{-1.5ex}

    \rule{\textwidth}{0.5\fboxrule}
\setlength{\parskip}{2ex}

\textbf{EXPERIMENTAL}

Utility method to replace all \texttt{url(urlstring)} values in
\texttt{CSSImportRules} and \texttt{CSSStyleDeclaration} objects (properties).

\texttt{replacer} must be a function which is called with a single
argument \texttt{urlstring} which is the current value of url()
excluding \texttt{url(} and \texttt{)}. It still may have surrounding
single or double quotes though.
\setlength{\parskip}{1ex}
    \end{boxedminipage}

    \label{cssutils:css:cssstylesheet:CSSStyleSheet:setSerializer}
    \index{cssutils \textit{(package)}!cssutils.css \textit{(package)}!cssutils.css.cssstylesheet \textit{(module)}!cssutils.css.cssstylesheet.CSSStyleSheet \textit{(class)}!cssutils.css.cssstylesheet.CSSStyleSheet.setSerializer \textit{(method)}}

    \vspace{0.5ex}

\hspace{.8\funcindent}\begin{boxedminipage}{\funcwidth}

    \raggedright \textbf{setSerializer}(\textit{self}, \textit{cssserializer})

    \vspace{-1.5ex}

    \rule{\textwidth}{0.5\fboxrule}
\setlength{\parskip}{2ex}

Sets the global Serializer used for output of all stylesheet
output.
\setlength{\parskip}{1ex}
    \end{boxedminipage}

    \label{cssutils:css:cssstylesheet:CSSStyleSheet:setSerializerPref}
    \index{cssutils \textit{(package)}!cssutils.css \textit{(package)}!cssutils.css.cssstylesheet \textit{(module)}!cssutils.css.cssstylesheet.CSSStyleSheet \textit{(class)}!cssutils.css.cssstylesheet.CSSStyleSheet.setSerializerPref \textit{(method)}}

    \vspace{0.5ex}

\hspace{.8\funcindent}\begin{boxedminipage}{\funcwidth}

    \raggedright \textbf{setSerializerPref}(\textit{self}, \textit{pref}, \textit{value})

    \vspace{-1.5ex}

    \rule{\textwidth}{0.5\fboxrule}
\setlength{\parskip}{2ex}

Sets Preference of CSSSerializer used for output of this
stylesheet. See cssutils.serialize.Preferences for possible
preferences to be set.
\setlength{\parskip}{1ex}
    \end{boxedminipage}

    \vspace{0.5ex}

\hspace{.8\funcindent}\begin{boxedminipage}{\funcwidth}

    \raggedright \textbf{\_\_repr\_\_}(\textit{self})

\setlength{\parskip}{2ex}
    repr(x)

\setlength{\parskip}{1ex}
      Overrides: object.\_\_repr\_\_ 	extit{(inherited documentation)}

    \end{boxedminipage}

    \vspace{0.5ex}

\hspace{.8\funcindent}\begin{boxedminipage}{\funcwidth}

    \raggedright \textbf{\_\_str\_\_}(\textit{self})

\setlength{\parskip}{2ex}
    str(x)

\setlength{\parskip}{1ex}
      Overrides: object.\_\_str\_\_ 	extit{(inherited documentation)}

    \end{boxedminipage}


\large{\textbf{\textit{Inherited from object}}}

\begin{quote}
\_\_delattr\_\_(), \_\_getattribute\_\_(), \_\_hash\_\_(), \_\_new\_\_(), \_\_reduce\_\_(), \_\_reduce\_ex\_\_(), \_\_setattr\_\_()
\end{quote}

%%%%%%%%%%%%%%%%%%%%%%%%%%%%%%%%%%%%%%%%%%%%%%%%%%%%%%%%%%%%%%%%%%%%%%%%%%%
%%                              Properties                               %%
%%%%%%%%%%%%%%%%%%%%%%%%%%%%%%%%%%%%%%%%%%%%%%%%%%%%%%%%%%%%%%%%%%%%%%%%%%%

  \subsubsection{Properties}

    \vspace{-1cm}
\hspace{\varindent}\begin{longtable}{|p{\varnamewidth}|p{\vardescrwidth}|l}
\cline{1-2}
\cline{1-2} \centering \textbf{Name} & \centering \textbf{Description}& \\
\cline{1-2}
\endhead\cline{1-2}\multicolumn{3}{r}{\small\textit{continued on next page}}\\\endfoot\cline{1-2}
\endlastfoot\raggedright c\-s\-s\-T\-e\-x\-t\- & &\\
\cline{1-2}
\raggedright e\-n\-c\-o\-d\-i\-n\-g\- & \raggedright return encoding if @charset rule if given or default of 'utf-8'&\\
\cline{1-2}
\raggedright n\-a\-m\-e\-s\-p\-a\-c\-e\-s\- & \raggedright Namespaces used in this CSSStyleSheet.&\\
\cline{1-2}
\raggedright o\-w\-n\-e\-r\-R\-u\-l\-e\- & \raggedright (DOM attribute) NOT IMPLEMENTED YET&\\
\cline{1-2}
\multicolumn{2}{|l|}{\textit{Inherited from cssutils.stylesheets.stylesheet.StyleSheet \textit{(Section \ref{cssutils:stylesheets:stylesheet:StyleSheet})}}}\\
\multicolumn{2}{|p{\varwidth}|}{\raggedright parentStyleSheet}\\
\cline{1-2}
\multicolumn{2}{|l|}{\textit{Inherited from object}}\\
\multicolumn{2}{|p{\varwidth}|}{\raggedright \_\_class\_\_}\\
\cline{1-2}
\end{longtable}


%%%%%%%%%%%%%%%%%%%%%%%%%%%%%%%%%%%%%%%%%%%%%%%%%%%%%%%%%%%%%%%%%%%%%%%%%%%
%%                            Class Variables                            %%
%%%%%%%%%%%%%%%%%%%%%%%%%%%%%%%%%%%%%%%%%%%%%%%%%%%%%%%%%%%%%%%%%%%%%%%%%%%

  \subsubsection{Class Variables}

    \vspace{-1cm}
\hspace{\varindent}\begin{longtable}{|p{\varnamewidth}|p{\vardescrwidth}|l}
\cline{1-2}
\cline{1-2} \centering \textbf{Name} & \centering \textbf{Description}& \\
\cline{1-2}
\endhead\cline{1-2}\multicolumn{3}{r}{\small\textit{continued on next page}}\\\endfoot\cline{1-2}
\endlastfoot\raggedright t\-y\-p\-e\- & \raggedright \textbf{Value:} 
{\tt \texttt{'}\texttt{text/css}\texttt{'}}&\\
\cline{1-2}
\end{longtable}

    \index{cssutils \textit{(package)}!cssutils.css \textit{(package)}!cssutils.css.cssstylesheet \textit{(module)}!cssutils.css.cssstylesheet.CSSStyleSheet \textit{(class)}|)}

%%%%%%%%%%%%%%%%%%%%%%%%%%%%%%%%%%%%%%%%%%%%%%%%%%%%%%%%%%%%%%%%%%%%%%%%%%%
%%                           Class Description                           %%
%%%%%%%%%%%%%%%%%%%%%%%%%%%%%%%%%%%%%%%%%%%%%%%%%%%%%%%%%%%%%%%%%%%%%%%%%%%

    \index{cssutils \textit{(package)}!cssutils.css \textit{(package)}!cssutils.css.cssrulelist \textit{(module)}!cssutils.css.cssrulelist.CSSRuleList \textit{(class)}|(}
\subsection{Class CSSRuleList}

    \label{cssutils:css:cssrulelist:CSSRuleList}
\begin{tabular}{cccccccc}
% Line for object, linespec=[False, False]
\multicolumn{2}{r}{\settowidth{\BCL}{object}\multirow{2}{\BCL}{object}}
&&
&&
  \\\cline{3-3}
  &&\multicolumn{1}{c|}{}
&&
&&
  \\
% Line for list, linespec=[False]
\multicolumn{4}{r}{\settowidth{\BCL}{list}\multirow{2}{\BCL}{list}}
&&
  \\\cline{5-5}
  &&&&\multicolumn{1}{c|}{}
&&
  \\
&&&&\multicolumn{2}{l}{\textbf{cssutils.css.cssrulelist.CSSRuleList}}
\end{tabular}


The CSSRuleList object represents an (ordered) list of statements.

The items in the CSSRuleList are accessible via an integral index,
starting from 0.

Subclasses a standard Python list so theoretically all standard list
methods are available. Setting methods like \texttt{{\_}{\_}init{\_}{\_}}, \texttt{append},
\texttt{extend} or \texttt{{\_}{\_}setslice{\_}{\_}} are added later on instances of this
class if so desired.
E.g. CSSStyleSheet adds \texttt{append} which is not available in a simple
instance of this class!


%___________________________________________________________________________

\hypertarget{properties}{}
\pdfbookmark[3]{Properties}{properties}
\paragraph*{Properties}
\label{properties}
\begin{description}
\item[{length: of type unsigned long, readonly}] \leavevmode 
The number of CSSRules in the list. The range of valid child rule
indices is 0 to length-1 inclusive.

\end{description}

%%%%%%%%%%%%%%%%%%%%%%%%%%%%%%%%%%%%%%%%%%%%%%%%%%%%%%%%%%%%%%%%%%%%%%%%%%%
%%                                Methods                                %%
%%%%%%%%%%%%%%%%%%%%%%%%%%%%%%%%%%%%%%%%%%%%%%%%%%%%%%%%%%%%%%%%%%%%%%%%%%%

  \subsubsection{Methods}

    \vspace{0.5ex}

\hspace{.8\funcindent}\begin{boxedminipage}{\funcwidth}

    \raggedright \textbf{\_\_init\_\_}(\textit{self}, *\textit{ignored})

    \vspace{-1.5ex}

    \rule{\textwidth}{0.5\fboxrule}
\setlength{\parskip}{2ex}

nothing is set as this must also be defined later
\setlength{\parskip}{1ex}
      \textbf{Return Value}
    \vspace{-1ex}

      \begin{quote}
      new list

      \end{quote}

      Overrides: object.\_\_init\_\_

    \end{boxedminipage}

    \vspace{0.5ex}

\hspace{.8\funcindent}\begin{boxedminipage}{\funcwidth}

    \raggedright \textbf{\_\_setslice\_\_}(\textit{self}, *\textit{ignored})

    \vspace{-1.5ex}

    \rule{\textwidth}{0.5\fboxrule}
\setlength{\parskip}{2ex}

no direct setting possible
\setlength{\parskip}{1ex}
      Overrides: list.\_\_setslice\_\_

    \end{boxedminipage}

    \vspace{0.5ex}

\hspace{.8\funcindent}\begin{boxedminipage}{\funcwidth}

    \raggedright \textbf{\_\_setitem\_\_}(\textit{self}, *\textit{ignored})

    \vspace{-1.5ex}

    \rule{\textwidth}{0.5\fboxrule}
\setlength{\parskip}{2ex}

no direct setting possible
\setlength{\parskip}{1ex}
      Overrides: list.\_\_setitem\_\_

    \end{boxedminipage}

    \vspace{0.5ex}

\hspace{.8\funcindent}\begin{boxedminipage}{\funcwidth}

    \raggedright \textbf{extend}(\textit{self}, *\textit{ignored})

    \vspace{-1.5ex}

    \rule{\textwidth}{0.5\fboxrule}
\setlength{\parskip}{2ex}

no direct setting possible
\setlength{\parskip}{1ex}
      Overrides: list.extend

    \end{boxedminipage}

    \vspace{0.5ex}

\hspace{.8\funcindent}\begin{boxedminipage}{\funcwidth}

    \raggedright \textbf{append}(\textit{self}, *\textit{ignored})

    \vspace{-1.5ex}

    \rule{\textwidth}{0.5\fboxrule}
\setlength{\parskip}{2ex}

no direct setting possible
\setlength{\parskip}{1ex}
      Overrides: list.append

    \end{boxedminipage}

    \label{cssutils:css:cssrulelist:CSSRuleList:item}
    \index{cssutils \textit{(package)}!cssutils.css \textit{(package)}!cssutils.css.cssrulelist \textit{(module)}!cssutils.css.cssrulelist.CSSRuleList \textit{(class)}!cssutils.css.cssrulelist.CSSRuleList.item \textit{(method)}}

    \vspace{0.5ex}

\hspace{.8\funcindent}\begin{boxedminipage}{\funcwidth}

    \raggedright \textbf{item}(\textit{self}, \textit{index})

    \vspace{-1.5ex}

    \rule{\textwidth}{0.5\fboxrule}
\setlength{\parskip}{2ex}

(DOM)
Used to retrieve a CSS rule by ordinal index. The order in this
collection represents the order of the rules in the CSS style
sheet. If index is greater than or equal to the number of rules in
the list, this returns None.

Returns CSSRule, the style rule at the index position in the
CSSRuleList, or None if that is not a valid index.
\setlength{\parskip}{1ex}
    \end{boxedminipage}


\large{\textbf{\textit{Inherited from list}}}

\begin{quote}
\_\_add\_\_(), \_\_contains\_\_(), \_\_delitem\_\_(), \_\_delslice\_\_(), \_\_eq\_\_(), \_\_ge\_\_(), \_\_getattribute\_\_(), \_\_getitem\_\_(), \_\_getslice\_\_(), \_\_gt\_\_(), \_\_hash\_\_(), \_\_iadd\_\_(), \_\_imul\_\_(), \_\_iter\_\_(), \_\_le\_\_(), \_\_len\_\_(), \_\_lt\_\_(), \_\_mul\_\_(), \_\_ne\_\_(), \_\_new\_\_(), \_\_repr\_\_(), \_\_reversed\_\_(), \_\_rmul\_\_(), count(), index(), insert(), pop(), remove(), reverse(), sort()
\end{quote}

\large{\textbf{\textit{Inherited from object}}}

\begin{quote}
\_\_delattr\_\_(), \_\_reduce\_\_(), \_\_reduce\_ex\_\_(), \_\_setattr\_\_(), \_\_str\_\_()
\end{quote}

%%%%%%%%%%%%%%%%%%%%%%%%%%%%%%%%%%%%%%%%%%%%%%%%%%%%%%%%%%%%%%%%%%%%%%%%%%%
%%                              Properties                               %%
%%%%%%%%%%%%%%%%%%%%%%%%%%%%%%%%%%%%%%%%%%%%%%%%%%%%%%%%%%%%%%%%%%%%%%%%%%%

  \subsubsection{Properties}

    \vspace{-1cm}
\hspace{\varindent}\begin{longtable}{|p{\varnamewidth}|p{\vardescrwidth}|l}
\cline{1-2}
\cline{1-2} \centering \textbf{Name} & \centering \textbf{Description}& \\
\cline{1-2}
\endhead\cline{1-2}\multicolumn{3}{r}{\small\textit{continued on next page}}\\\endfoot\cline{1-2}
\endlastfoot\raggedright l\-e\-n\-g\-t\-h\- & \raggedright (DOM) The number of CSSRules in the list.&\\
\cline{1-2}
\multicolumn{2}{|l|}{\textit{Inherited from object}}\\
\multicolumn{2}{|p{\varwidth}|}{\raggedright \_\_class\_\_}\\
\cline{1-2}
\end{longtable}

    \index{cssutils \textit{(package)}!cssutils.css \textit{(package)}!cssutils.css.cssrulelist \textit{(module)}!cssutils.css.cssrulelist.CSSRuleList \textit{(class)}|)}

%%%%%%%%%%%%%%%%%%%%%%%%%%%%%%%%%%%%%%%%%%%%%%%%%%%%%%%%%%%%%%%%%%%%%%%%%%%
%%                           Class Description                           %%
%%%%%%%%%%%%%%%%%%%%%%%%%%%%%%%%%%%%%%%%%%%%%%%%%%%%%%%%%%%%%%%%%%%%%%%%%%%

    \index{cssutils \textit{(package)}!cssutils.css \textit{(package)}!cssutils.css.cssrule \textit{(module)}!cssutils.css.cssrule.CSSRule \textit{(class)}|(}
\subsection{Class CSSRule}

    \label{cssutils:css:cssrule:CSSRule}
\begin{tabular}{cccccccc}
% Line for object, linespec=[False, False]
\multicolumn{2}{r}{\settowidth{\BCL}{object}\multirow{2}{\BCL}{object}}
&&
&&
  \\\cline{3-3}
  &&\multicolumn{1}{c|}{}
&&
&&
  \\
% Line for cssutils.util.Base, linespec=[False]
\multicolumn{4}{r}{\settowidth{\BCL}{cssutils.util.Base}\multirow{2}{\BCL}{cssutils.util.Base}}
&&
  \\\cline{5-5}
  &&&&\multicolumn{1}{c|}{}
&&
  \\
&&&&\multicolumn{2}{l}{\textbf{cssutils.css.cssrule.CSSRule}}
\end{tabular}

\textbf{Known Subclasses:}
cssutils.css.csscharsetrule.CSSCharsetRule,
    cssutils.css.csscomment.CSSComment,
    cssutils.css.cssmediarule.CSSMediaRule,
    cssutils.css.cssnamespacerule.CSSNamespaceRule,
    cssutils.css.csspagerule.CSSPageRule,
    cssutils.css.cssstylerule.CSSStyleRule,
    cssutils.css.cssunknownrule.CSSUnknownRule,
    cssutils.css.cssimportrule.CSSImportRule,
    cssutils.css.cssfontfacerule.CSSFontFaceRule


Abstract base interface for any type of CSS statement. This includes
both rule sets and at-rules. An implementation is expected to preserve
all rules specified in a CSS style sheet, even if the rule is not
recognized by the parser. Unrecognized rules are represented using the
CSSUnknownRule interface.


%___________________________________________________________________________

\hypertarget{properties}{}
\pdfbookmark[3]{Properties}{properties}
\paragraph*{Properties}
\label{properties}
\begin{description}
\item[{cssText: of type DOMString}] \leavevmode 
The parsable textual representation of the rule. This reflects the
current state of the rule and not its initial value.

\item[{parentRule: of type CSSRule, readonly}] \leavevmode 
If this rule is contained inside another rule (e.g. a style rule
inside an @media block), this is the containing rule. If this rule
is not nested inside any other rules, this returns None.

\item[{parentStyleSheet: of type CSSStyleSheet, readonly}] \leavevmode 
The style sheet that contains this rule.

\item[{type: of type unsigned short, readonly}] \leavevmode 
The type of the rule, as defined above. The expectation is that
binding-specific casting methods can be used to cast down from an
instance of the CSSRule interface to the specific derived interface
implied by the type.

\end{description}


%___________________________________________________________________________

\hypertarget{cssutils-only}{}
\pdfbookmark[4]{cssutils only}{cssutils-only}
\subparagraph*{cssutils only}
\label{cssutils-only}
\begin{description}
\item[{seq:}] \leavevmode 
contains sequence of parts of the rule including comments but
excluding @KEYWORD and braces

\item[{typeString: string}] \leavevmode 
A string name of the type of this rule, e.g. 'STYLE{\_}RULE'. Mainly
useful for debugging

\item[{valid:}] \leavevmode 
if this rule is valid

\end{description}

%%%%%%%%%%%%%%%%%%%%%%%%%%%%%%%%%%%%%%%%%%%%%%%%%%%%%%%%%%%%%%%%%%%%%%%%%%%
%%                                Methods                                %%
%%%%%%%%%%%%%%%%%%%%%%%%%%%%%%%%%%%%%%%%%%%%%%%%%%%%%%%%%%%%%%%%%%%%%%%%%%%

  \subsubsection{Methods}

    \vspace{0.5ex}

\hspace{.8\funcindent}\begin{boxedminipage}{\funcwidth}

    \raggedright \textbf{\_\_init\_\_}(\textit{self}, \textit{parentRule}={\tt None}, \textit{parentStyleSheet}={\tt None}, \textit{readonly}={\tt False})

\setlength{\parskip}{2ex}
    x.\_\_init\_\_(...) initializes x; see x.\_\_class\_\_.\_\_doc\_\_ for 
    signature

\setlength{\parskip}{1ex}
      Overrides: object.\_\_init\_\_ 	extit{(inherited documentation)}

    \end{boxedminipage}


\large{\textbf{\textit{Inherited from object}}}

\begin{quote}
\_\_delattr\_\_(), \_\_getattribute\_\_(), \_\_hash\_\_(), \_\_new\_\_(), \_\_reduce\_\_(), \_\_reduce\_ex\_\_(), \_\_repr\_\_(), \_\_setattr\_\_(), \_\_str\_\_()
\end{quote}

%%%%%%%%%%%%%%%%%%%%%%%%%%%%%%%%%%%%%%%%%%%%%%%%%%%%%%%%%%%%%%%%%%%%%%%%%%%
%%                              Properties                               %%
%%%%%%%%%%%%%%%%%%%%%%%%%%%%%%%%%%%%%%%%%%%%%%%%%%%%%%%%%%%%%%%%%%%%%%%%%%%

  \subsubsection{Properties}

    \vspace{-1cm}
\hspace{\varindent}\begin{longtable}{|p{\varnamewidth}|p{\vardescrwidth}|l}
\cline{1-2}
\cline{1-2} \centering \textbf{Name} & \centering \textbf{Description}& \\
\cline{1-2}
\endhead\cline{1-2}\multicolumn{3}{r}{\small\textit{continued on next page}}\\\endfoot\cline{1-2}
\endlastfoot\raggedright c\-s\-s\-T\-e\-x\-t\- & \raggedright (DOM) The parsable textual representation of the rule. This
reflects the current state of the rule and not its initial value.
The initial value is saved, but this may be removed in a future
version!
MUST BE OVERWRITTEN IN SUBCLASS TO WORK!&\\
\cline{1-2}
\raggedright p\-a\-r\-e\-n\-t\-R\-u\-l\-e\- & \raggedright READONLY&\\
\cline{1-2}
\raggedright p\-a\-r\-e\-n\-t\-S\-t\-y\-l\-e\-S\-h\-e\-e\-t\- & \raggedright READONLY&\\
\cline{1-2}
\raggedright t\-y\-p\-e\-S\-t\-r\-i\-n\-g\- & \raggedright Name of this rules type.&\\
\cline{1-2}
\multicolumn{2}{|l|}{\textit{Inherited from object}}\\
\multicolumn{2}{|p{\varwidth}|}{\raggedright \_\_class\_\_}\\
\cline{1-2}
\end{longtable}


%%%%%%%%%%%%%%%%%%%%%%%%%%%%%%%%%%%%%%%%%%%%%%%%%%%%%%%%%%%%%%%%%%%%%%%%%%%
%%                            Class Variables                            %%
%%%%%%%%%%%%%%%%%%%%%%%%%%%%%%%%%%%%%%%%%%%%%%%%%%%%%%%%%%%%%%%%%%%%%%%%%%%

  \subsubsection{Class Variables}

    \vspace{-1cm}
\hspace{\varindent}\begin{longtable}{|p{\varnamewidth}|p{\vardescrwidth}|l}
\cline{1-2}
\cline{1-2} \centering \textbf{Name} & \centering \textbf{Description}& \\
\cline{1-2}
\endhead\cline{1-2}\multicolumn{3}{r}{\small\textit{continued on next page}}\\\endfoot\cline{1-2}
\endlastfoot\raggedright C\-O\-M\-M\-E\-N\-T\- & \raggedright \textbf{Value:} 
{\tt -1}&\\
\cline{1-2}
\raggedright U\-N\-K\-N\-O\-W\-N\-\_\-R\-U\-L\-E\- & \raggedright \textbf{Value:} 
{\tt 0}&\\
\cline{1-2}
\raggedright S\-T\-Y\-L\-E\-\_\-R\-U\-L\-E\- & \raggedright \textbf{Value:} 
{\tt 1}&\\
\cline{1-2}
\raggedright C\-H\-A\-R\-S\-E\-T\-\_\-R\-U\-L\-E\- & \raggedright \textbf{Value:} 
{\tt 2}&\\
\cline{1-2}
\raggedright I\-M\-P\-O\-R\-T\-\_\-R\-U\-L\-E\- & \raggedright \textbf{Value:} 
{\tt 3}&\\
\cline{1-2}
\raggedright M\-E\-D\-I\-A\-\_\-R\-U\-L\-E\- & \raggedright \textbf{Value:} 
{\tt 4}&\\
\cline{1-2}
\raggedright F\-O\-N\-T\-\_\-F\-A\-C\-E\-\_\-R\-U\-L\-E\- & \raggedright \textbf{Value:} 
{\tt 5}&\\
\cline{1-2}
\raggedright P\-A\-G\-E\-\_\-R\-U\-L\-E\- & \raggedright \textbf{Value:} 
{\tt 6}&\\
\cline{1-2}
\raggedright N\-A\-M\-E\-S\-P\-A\-C\-E\-\_\-R\-U\-L\-E\- & \raggedright \textbf{Value:} 
{\tt 7}&\\
\cline{1-2}
\raggedright t\-y\-p\-e\- & \raggedright The type of this rule, as defined by a CSSRule type constant.
Overwritten in derived classes.

The expectation is that binding-specific casting methods can be used to
cast down from an instance of the CSSRule interface to the specific
derived interface implied by the type.
(Casting not for this Python implementation I guess...)

\textbf{Value:} 
{\tt 0}&\\
\cline{1-2}
\end{longtable}

    \index{cssutils \textit{(package)}!cssutils.css \textit{(package)}!cssutils.css.cssrule \textit{(module)}!cssutils.css.cssrule.CSSRule \textit{(class)}|)}

%%%%%%%%%%%%%%%%%%%%%%%%%%%%%%%%%%%%%%%%%%%%%%%%%%%%%%%%%%%%%%%%%%%%%%%%%%%
%%                           Class Description                           %%
%%%%%%%%%%%%%%%%%%%%%%%%%%%%%%%%%%%%%%%%%%%%%%%%%%%%%%%%%%%%%%%%%%%%%%%%%%%

    \index{cssutils \textit{(package)}!cssutils.css \textit{(package)}!cssutils.css.csscomment \textit{(module)}!cssutils.css.csscomment.CSSComment \textit{(class)}|(}
\subsection{Class CSSComment}

    \label{cssutils:css:csscomment:CSSComment}
\begin{tabular}{cccccccccc}
% Line for object, linespec=[False, False, False]
\multicolumn{2}{r}{\settowidth{\BCL}{object}\multirow{2}{\BCL}{object}}
&&
&&
&&
  \\\cline{3-3}
  &&\multicolumn{1}{c|}{}
&&
&&
&&
  \\
% Line for cssutils.util.Base, linespec=[False, False]
\multicolumn{4}{r}{\settowidth{\BCL}{cssutils.util.Base}\multirow{2}{\BCL}{cssutils.util.Base}}
&&
&&
  \\\cline{5-5}
  &&&&\multicolumn{1}{c|}{}
&&
&&
  \\
% Line for cssutils.css.cssrule.CSSRule, linespec=[False]
\multicolumn{6}{r}{\settowidth{\BCL}{cssutils.css.cssrule.CSSRule}\multirow{2}{\BCL}{cssutils.css.cssrule.CSSRule}}
&&
  \\\cline{7-7}
  &&&&&&\multicolumn{1}{c|}{}
&&
  \\
&&&&&&\multicolumn{2}{l}{\textbf{cssutils.css.csscomment.CSSComment}}
\end{tabular}


(cssutils) a CSS comment


%___________________________________________________________________________

\hypertarget{properties}{}
\pdfbookmark[3]{Properties}{properties}
\paragraph*{Properties}
\label{properties}
\begin{description}
\item[{cssText: of type DOMString}] \leavevmode 
The comment text including comment delimiters

\end{description}

Inherits properties from CSSRule


%___________________________________________________________________________

\hypertarget{format}{}
\pdfbookmark[3]{Format}{format}
\paragraph*{Format}
\label{format}
\begin{quote}{\ttfamily \raggedright \noindent
/*...*/
}\end{quote}

%%%%%%%%%%%%%%%%%%%%%%%%%%%%%%%%%%%%%%%%%%%%%%%%%%%%%%%%%%%%%%%%%%%%%%%%%%%
%%                                Methods                                %%
%%%%%%%%%%%%%%%%%%%%%%%%%%%%%%%%%%%%%%%%%%%%%%%%%%%%%%%%%%%%%%%%%%%%%%%%%%%

  \subsubsection{Methods}

    \vspace{0.5ex}

\hspace{.8\funcindent}\begin{boxedminipage}{\funcwidth}

    \raggedright \textbf{\_\_init\_\_}(\textit{self}, \textit{cssText}={\tt None}, \textit{parentRule}={\tt None}, \textit{parentStyleSheet}={\tt None}, \textit{readonly}={\tt False})

\setlength{\parskip}{2ex}
    x.\_\_init\_\_(...) initializes x; see x.\_\_class\_\_.\_\_doc\_\_ for 
    signature

\setlength{\parskip}{1ex}
      Overrides: object.\_\_init\_\_ 	extit{(inherited documentation)}

    \end{boxedminipage}

    \vspace{0.5ex}

\hspace{.8\funcindent}\begin{boxedminipage}{\funcwidth}

    \raggedright \textbf{\_\_repr\_\_}(\textit{self})

\setlength{\parskip}{2ex}
    repr(x)

\setlength{\parskip}{1ex}
      Overrides: object.\_\_repr\_\_ 	extit{(inherited documentation)}

    \end{boxedminipage}

    \vspace{0.5ex}

\hspace{.8\funcindent}\begin{boxedminipage}{\funcwidth}

    \raggedright \textbf{\_\_str\_\_}(\textit{self})

\setlength{\parskip}{2ex}
    str(x)

\setlength{\parskip}{1ex}
      Overrides: object.\_\_str\_\_ 	extit{(inherited documentation)}

    \end{boxedminipage}


\large{\textbf{\textit{Inherited from object}}}

\begin{quote}
\_\_delattr\_\_(), \_\_getattribute\_\_(), \_\_hash\_\_(), \_\_new\_\_(), \_\_reduce\_\_(), \_\_reduce\_ex\_\_(), \_\_setattr\_\_()
\end{quote}

%%%%%%%%%%%%%%%%%%%%%%%%%%%%%%%%%%%%%%%%%%%%%%%%%%%%%%%%%%%%%%%%%%%%%%%%%%%
%%                              Properties                               %%
%%%%%%%%%%%%%%%%%%%%%%%%%%%%%%%%%%%%%%%%%%%%%%%%%%%%%%%%%%%%%%%%%%%%%%%%%%%

  \subsubsection{Properties}

    \vspace{-1cm}
\hspace{\varindent}\begin{longtable}{|p{\varnamewidth}|p{\vardescrwidth}|l}
\cline{1-2}
\cline{1-2} \centering \textbf{Name} & \centering \textbf{Description}& \\
\cline{1-2}
\endhead\cline{1-2}\multicolumn{3}{r}{\small\textit{continued on next page}}\\\endfoot\cline{1-2}
\endlastfoot\raggedright c\-s\-s\-T\-e\-x\-t\- & \raggedright (cssutils) Textual representation of this comment&\\
\cline{1-2}
\multicolumn{2}{|l|}{\textit{Inherited from cssutils.css.cssrule.CSSRule \textit{(Section \ref{cssutils:css:cssrule:CSSRule})}}}\\
\multicolumn{2}{|p{\varwidth}|}{\raggedright parentRule, parentStyleSheet, typeString}\\
\cline{1-2}
\multicolumn{2}{|l|}{\textit{Inherited from object}}\\
\multicolumn{2}{|p{\varwidth}|}{\raggedright \_\_class\_\_}\\
\cline{1-2}
\end{longtable}


%%%%%%%%%%%%%%%%%%%%%%%%%%%%%%%%%%%%%%%%%%%%%%%%%%%%%%%%%%%%%%%%%%%%%%%%%%%
%%                            Class Variables                            %%
%%%%%%%%%%%%%%%%%%%%%%%%%%%%%%%%%%%%%%%%%%%%%%%%%%%%%%%%%%%%%%%%%%%%%%%%%%%

  \subsubsection{Class Variables}

    \vspace{-1cm}
\hspace{\varindent}\begin{longtable}{|p{\varnamewidth}|p{\vardescrwidth}|l}
\cline{1-2}
\cline{1-2} \centering \textbf{Name} & \centering \textbf{Description}& \\
\cline{1-2}
\endhead\cline{1-2}\multicolumn{3}{r}{\small\textit{continued on next page}}\\\endfoot\cline{1-2}
\endlastfoot\raggedright t\-y\-p\-e\- & \raggedright The type of this rule, as defined by a CSSRule type constant.
Overwritten in derived classes.

The expectation is that binding-specific casting methods can be used to
cast down from an instance of the CSSRule interface to the specific
derived interface implied by the type.
(Casting not for this Python implementation I guess...)

\textbf{Value:} 
{\tt -1}&\\
\cline{1-2}
\multicolumn{2}{|l|}{\textit{Inherited from cssutils.css.cssrule.CSSRule \textit{(Section \ref{cssutils:css:cssrule:CSSRule})}}}\\
\multicolumn{2}{|p{\varwidth}|}{\raggedright CHARSET\_RULE, COMMENT, FONT\_FACE\_RULE, IMPORT\_RULE, MEDIA\_RULE, NAMESPACE\_RULE, PAGE\_RULE, STYLE\_RULE, UNKNOWN\_RULE}\\
\cline{1-2}
\end{longtable}

    \index{cssutils \textit{(package)}!cssutils.css \textit{(package)}!cssutils.css.csscomment \textit{(module)}!cssutils.css.csscomment.CSSComment \textit{(class)}|)}

%%%%%%%%%%%%%%%%%%%%%%%%%%%%%%%%%%%%%%%%%%%%%%%%%%%%%%%%%%%%%%%%%%%%%%%%%%%
%%                           Class Description                           %%
%%%%%%%%%%%%%%%%%%%%%%%%%%%%%%%%%%%%%%%%%%%%%%%%%%%%%%%%%%%%%%%%%%%%%%%%%%%

    \index{cssutils \textit{(package)}!cssutils.css \textit{(package)}!cssutils.css.csscharsetrule \textit{(module)}!cssutils.css.csscharsetrule.CSSCharsetRule \textit{(class)}|(}
\subsection{Class CSSCharsetRule}

    \label{cssutils:css:csscharsetrule:CSSCharsetRule}
\begin{tabular}{cccccccccc}
% Line for object, linespec=[False, False, False]
\multicolumn{2}{r}{\settowidth{\BCL}{object}\multirow{2}{\BCL}{object}}
&&
&&
&&
  \\\cline{3-3}
  &&\multicolumn{1}{c|}{}
&&
&&
&&
  \\
% Line for cssutils.util.Base, linespec=[False, False]
\multicolumn{4}{r}{\settowidth{\BCL}{cssutils.util.Base}\multirow{2}{\BCL}{cssutils.util.Base}}
&&
&&
  \\\cline{5-5}
  &&&&\multicolumn{1}{c|}{}
&&
&&
  \\
% Line for cssutils.css.cssrule.CSSRule, linespec=[False]
\multicolumn{6}{r}{\settowidth{\BCL}{cssutils.css.cssrule.CSSRule}\multirow{2}{\BCL}{cssutils.css.cssrule.CSSRule}}
&&
  \\\cline{7-7}
  &&&&&&\multicolumn{1}{c|}{}
&&
  \\
&&&&&&\multicolumn{2}{l}{\textbf{cssutils.css.csscharsetrule.CSSCharsetRule}}
\end{tabular}


The CSSCharsetRule interface represents an @charset rule in a CSS style
sheet. The value of the encoding attribute does not affect the encoding
of text data in the DOM objects; this encoding is always UTF-16
(also in Python?). After a stylesheet is loaded, the value of the
encoding attribute is the value found in the @charset rule. If there
was no @charset in the original document, then no CSSCharsetRule is
created. The value of the encoding attribute may also be used as a hint
for the encoding used on serialization of the style sheet.

The value of the @charset rule (and therefore of the CSSCharsetRule)
may not correspond to the encoding the document actually came in;
character encoding information e.g. in an HTTP header, has priority
(see CSS document representation) but this is not reflected in the
CSSCharsetRule.


%___________________________________________________________________________

\hypertarget{properties}{}
\pdfbookmark[3]{Properties}{properties}
\paragraph*{Properties}
\label{properties}
\begin{description}
\item[{cssText: of type DOMString}] \leavevmode 
The parsable textual representation of this rule

\item[{encoding: of type DOMString}] \leavevmode 
The encoding information used in this @charset rule.

\end{description}

Inherits properties from CSSRule


%___________________________________________________________________________

\hypertarget{format}{}
\pdfbookmark[3]{Format}{format}
\paragraph*{Format}
\label{format}
\begin{description}
\item[{charsetrule:}] \leavevmode 
CHARSET{\_}SYM S* STRING S* ';'

\item[{BUT: Only valid format is:}] \leavevmode 
@charset ``ENCODING'';

\end{description}

%%%%%%%%%%%%%%%%%%%%%%%%%%%%%%%%%%%%%%%%%%%%%%%%%%%%%%%%%%%%%%%%%%%%%%%%%%%
%%                                Methods                                %%
%%%%%%%%%%%%%%%%%%%%%%%%%%%%%%%%%%%%%%%%%%%%%%%%%%%%%%%%%%%%%%%%%%%%%%%%%%%

  \subsubsection{Methods}

    \vspace{0.5ex}

\hspace{.8\funcindent}\begin{boxedminipage}{\funcwidth}

    \raggedright \textbf{\_\_init\_\_}(\textit{self}, \textit{encoding}={\tt None}, \textit{parentRule}={\tt None}, \textit{parentStyleSheet}={\tt None}, \textit{readonly}={\tt False})

    \vspace{-1.5ex}

    \rule{\textwidth}{0.5\fboxrule}
\setlength{\parskip}{2ex}
\begin{description}
\item[{encoding:}] \leavevmode 
a valid character encoding

\item[{readonly:}] \leavevmode 
defaults to False, not used yet

\end{description}

if readonly allows setting of properties in constructor only
\setlength{\parskip}{1ex}
      Overrides: object.\_\_init\_\_

    \end{boxedminipage}

    \vspace{0.5ex}

\hspace{.8\funcindent}\begin{boxedminipage}{\funcwidth}

    \raggedright \textbf{\_\_repr\_\_}(\textit{self})

\setlength{\parskip}{2ex}
    repr(x)

\setlength{\parskip}{1ex}
      Overrides: object.\_\_repr\_\_ 	extit{(inherited documentation)}

    \end{boxedminipage}

    \vspace{0.5ex}

\hspace{.8\funcindent}\begin{boxedminipage}{\funcwidth}

    \raggedright \textbf{\_\_str\_\_}(\textit{self})

\setlength{\parskip}{2ex}
    str(x)

\setlength{\parskip}{1ex}
      Overrides: object.\_\_str\_\_ 	extit{(inherited documentation)}

    \end{boxedminipage}


\large{\textbf{\textit{Inherited from object}}}

\begin{quote}
\_\_delattr\_\_(), \_\_getattribute\_\_(), \_\_hash\_\_(), \_\_new\_\_(), \_\_reduce\_\_(), \_\_reduce\_ex\_\_(), \_\_setattr\_\_()
\end{quote}

%%%%%%%%%%%%%%%%%%%%%%%%%%%%%%%%%%%%%%%%%%%%%%%%%%%%%%%%%%%%%%%%%%%%%%%%%%%
%%                              Properties                               %%
%%%%%%%%%%%%%%%%%%%%%%%%%%%%%%%%%%%%%%%%%%%%%%%%%%%%%%%%%%%%%%%%%%%%%%%%%%%

  \subsubsection{Properties}

    \vspace{-1cm}
\hspace{\varindent}\begin{longtable}{|p{\varnamewidth}|p{\vardescrwidth}|l}
\cline{1-2}
\cline{1-2} \centering \textbf{Name} & \centering \textbf{Description}& \\
\cline{1-2}
\endhead\cline{1-2}\multicolumn{3}{r}{\small\textit{continued on next page}}\\\endfoot\cline{1-2}
\endlastfoot\raggedright e\-n\-c\-o\-d\-i\-n\-g\- & \raggedright (DOM)The encoding information used in this @charset rule.&\\
\cline{1-2}
\raggedright c\-s\-s\-T\-e\-x\-t\- & \raggedright (DOM) The parsable textual representation.&\\
\cline{1-2}
\multicolumn{2}{|l|}{\textit{Inherited from cssutils.css.cssrule.CSSRule \textit{(Section \ref{cssutils:css:cssrule:CSSRule})}}}\\
\multicolumn{2}{|p{\varwidth}|}{\raggedright parentRule, parentStyleSheet, typeString}\\
\cline{1-2}
\multicolumn{2}{|l|}{\textit{Inherited from object}}\\
\multicolumn{2}{|p{\varwidth}|}{\raggedright \_\_class\_\_}\\
\cline{1-2}
\end{longtable}


%%%%%%%%%%%%%%%%%%%%%%%%%%%%%%%%%%%%%%%%%%%%%%%%%%%%%%%%%%%%%%%%%%%%%%%%%%%
%%                            Class Variables                            %%
%%%%%%%%%%%%%%%%%%%%%%%%%%%%%%%%%%%%%%%%%%%%%%%%%%%%%%%%%%%%%%%%%%%%%%%%%%%

  \subsubsection{Class Variables}

    \vspace{-1cm}
\hspace{\varindent}\begin{longtable}{|p{\varnamewidth}|p{\vardescrwidth}|l}
\cline{1-2}
\cline{1-2} \centering \textbf{Name} & \centering \textbf{Description}& \\
\cline{1-2}
\endhead\cline{1-2}\multicolumn{3}{r}{\small\textit{continued on next page}}\\\endfoot\cline{1-2}
\endlastfoot\raggedright t\-y\-p\-e\- & \raggedright The type of this rule, as defined by a CSSRule type constant.
Overwritten in derived classes.

The expectation is that binding-specific casting methods can be used to
cast down from an instance of the CSSRule interface to the specific
derived interface implied by the type.
(Casting not for this Python implementation I guess...)

\textbf{Value:} 
{\tt 2}&\\
\cline{1-2}
\multicolumn{2}{|l|}{\textit{Inherited from cssutils.css.cssrule.CSSRule \textit{(Section \ref{cssutils:css:cssrule:CSSRule})}}}\\
\multicolumn{2}{|p{\varwidth}|}{\raggedright CHARSET\_RULE, COMMENT, FONT\_FACE\_RULE, IMPORT\_RULE, MEDIA\_RULE, NAMESPACE\_RULE, PAGE\_RULE, STYLE\_RULE, UNKNOWN\_RULE}\\
\cline{1-2}
\end{longtable}

    \index{cssutils \textit{(package)}!cssutils.css \textit{(package)}!cssutils.css.csscharsetrule \textit{(module)}!cssutils.css.csscharsetrule.CSSCharsetRule \textit{(class)}|)}

%%%%%%%%%%%%%%%%%%%%%%%%%%%%%%%%%%%%%%%%%%%%%%%%%%%%%%%%%%%%%%%%%%%%%%%%%%%
%%                           Class Description                           %%
%%%%%%%%%%%%%%%%%%%%%%%%%%%%%%%%%%%%%%%%%%%%%%%%%%%%%%%%%%%%%%%%%%%%%%%%%%%

    \index{cssutils \textit{(package)}!cssutils.css \textit{(package)}!cssutils.css.cssmediarule \textit{(module)}!cssutils.css.cssmediarule.CSSMediaRule \textit{(class)}|(}
\subsection{Class CSSMediaRule}

    \label{cssutils:css:cssmediarule:CSSMediaRule}
\begin{tabular}{cccccccccc}
% Line for object, linespec=[False, False, False]
\multicolumn{2}{r}{\settowidth{\BCL}{object}\multirow{2}{\BCL}{object}}
&&
&&
&&
  \\\cline{3-3}
  &&\multicolumn{1}{c|}{}
&&
&&
&&
  \\
% Line for cssutils.util.Base, linespec=[False, False]
\multicolumn{4}{r}{\settowidth{\BCL}{cssutils.util.Base}\multirow{2}{\BCL}{cssutils.util.Base}}
&&
&&
  \\\cline{5-5}
  &&&&\multicolumn{1}{c|}{}
&&
&&
  \\
% Line for cssutils.css.cssrule.CSSRule, linespec=[False]
\multicolumn{6}{r}{\settowidth{\BCL}{cssutils.css.cssrule.CSSRule}\multirow{2}{\BCL}{cssutils.css.cssrule.CSSRule}}
&&
  \\\cline{7-7}
  &&&&&&\multicolumn{1}{c|}{}
&&
  \\
&&&&&&\multicolumn{2}{l}{\textbf{cssutils.css.cssmediarule.CSSMediaRule}}
\end{tabular}


Objects implementing the CSSMediaRule interface can be identified by the
MEDIA{\_}RULE constant. On these objects the type attribute must return the
value of that constant.


%___________________________________________________________________________

\hypertarget{properties}{}
\pdfbookmark[3]{Properties}{properties}
\paragraph*{Properties}
\label{properties}
\begin{description}
\item[{cssRules: A css::CSSRuleList of all CSS rules contained within the}] \leavevmode 
media block.

\item[{media: of type stylesheets::MediaList, (DOM readonly)}] \leavevmode 
A list of media types for this rule of type MediaList.

\item[{inherited from CSSRule}] \leavevmode 
cssText

\end{description}


%___________________________________________________________________________

\hypertarget{cssutils-only}{}
\pdfbookmark[4]{cssutils only}{cssutils-only}
\subparagraph*{cssutils only}
\label{cssutils-only}
\begin{description}
\item[{atkeyword:}] \leavevmode 
the literal keyword used

\end{description}


%___________________________________________________________________________

\hypertarget{format}{}
\pdfbookmark[3]{Format}{format}
\paragraph*{Format}
\label{format}
\begin{description}
\item[{media}] \leavevmode 
: MEDIA{\_}SYM S* medium {[} COMMA S* medium {]}* LBRACE S* ruleset* '{\}}' S*;

\end{description}

%%%%%%%%%%%%%%%%%%%%%%%%%%%%%%%%%%%%%%%%%%%%%%%%%%%%%%%%%%%%%%%%%%%%%%%%%%%
%%                                Methods                                %%
%%%%%%%%%%%%%%%%%%%%%%%%%%%%%%%%%%%%%%%%%%%%%%%%%%%%%%%%%%%%%%%%%%%%%%%%%%%

  \subsubsection{Methods}

    \vspace{0.5ex}

\hspace{.8\funcindent}\begin{boxedminipage}{\funcwidth}

    \raggedright \textbf{\_\_init\_\_}(\textit{self}, \textit{mediaText}={\tt \texttt{'}\texttt{all}\texttt{'}}, \textit{parentRule}={\tt None}, \textit{parentStyleSheet}={\tt None}, \textit{readonly}={\tt False})

    \vspace{-1.5ex}

    \rule{\textwidth}{0.5\fboxrule}
\setlength{\parskip}{2ex}

constructor
\setlength{\parskip}{1ex}
      Overrides: object.\_\_init\_\_

    \end{boxedminipage}

    \label{cssutils:css:cssmediarule:CSSMediaRule:__iter__}
    \index{cssutils \textit{(package)}!cssutils.css \textit{(package)}!cssutils.css.cssmediarule \textit{(module)}!cssutils.css.cssmediarule.CSSMediaRule \textit{(class)}!cssutils.css.cssmediarule.CSSMediaRule.\_\_iter\_\_ \textit{(method)}}

    \vspace{0.5ex}

\hspace{.8\funcindent}\begin{boxedminipage}{\funcwidth}

    \raggedright \textbf{\_\_iter\_\_}(\textit{self})

    \vspace{-1.5ex}

    \rule{\textwidth}{0.5\fboxrule}
\setlength{\parskip}{2ex}

generator which iterates over cssRules.
\setlength{\parskip}{1ex}
    \end{boxedminipage}

    \label{cssutils:css:cssmediarule:CSSMediaRule:deleteRule}
    \index{cssutils \textit{(package)}!cssutils.css \textit{(package)}!cssutils.css.cssmediarule \textit{(module)}!cssutils.css.cssmediarule.CSSMediaRule \textit{(class)}!cssutils.css.cssmediarule.CSSMediaRule.deleteRule \textit{(method)}}

    \vspace{0.5ex}

\hspace{.8\funcindent}\begin{boxedminipage}{\funcwidth}

    \raggedright \textbf{deleteRule}(\textit{self}, \textit{index})

    \vspace{-1.5ex}

    \rule{\textwidth}{0.5\fboxrule}
\setlength{\parskip}{2ex}
\begin{description}
\item[{index}] \leavevmode 
within the media block's rule collection of the rule to remove.

\end{description}

Used to delete a rule from the media block.

DOMExceptions
\begin{itemize}
\item {} 
INDEX{\_}SIZE{\_}ERR: (self)
Raised if the specified index does not correspond to a rule in
the media rule list.

\item {} 
NO{\_}MODIFICATION{\_}ALLOWED{\_}ERR: (self)
Raised if this media rule is readonly.

\end{itemize}
\setlength{\parskip}{1ex}
    \end{boxedminipage}

    \label{cssutils:css:cssmediarule:CSSMediaRule:add}
    \index{cssutils \textit{(package)}!cssutils.css \textit{(package)}!cssutils.css.cssmediarule \textit{(module)}!cssutils.css.cssmediarule.CSSMediaRule \textit{(class)}!cssutils.css.cssmediarule.CSSMediaRule.add \textit{(method)}}

    \vspace{0.5ex}

\hspace{.8\funcindent}\begin{boxedminipage}{\funcwidth}

    \raggedright \textbf{add}(\textit{self}, \textit{rule})

    \vspace{-1.5ex}

    \rule{\textwidth}{0.5\fboxrule}
\setlength{\parskip}{2ex}

Adds rule to end of this mediarule. Same as \texttt{.insertRule(rule)}.
\setlength{\parskip}{1ex}
    \end{boxedminipage}

    \label{cssutils:css:cssmediarule:CSSMediaRule:insertRule}
    \index{cssutils \textit{(package)}!cssutils.css \textit{(package)}!cssutils.css.cssmediarule \textit{(module)}!cssutils.css.cssmediarule.CSSMediaRule \textit{(class)}!cssutils.css.cssmediarule.CSSMediaRule.insertRule \textit{(method)}}

    \vspace{0.5ex}

\hspace{.8\funcindent}\begin{boxedminipage}{\funcwidth}

    \raggedright \textbf{insertRule}(\textit{self}, \textit{rule}, \textit{index}={\tt None})

    \vspace{-1.5ex}

    \rule{\textwidth}{0.5\fboxrule}
\setlength{\parskip}{2ex}
\begin{description}
\item[{rule}] \leavevmode 
The parsable text representing the rule. For rule sets this
contains both the selector and the style declaration. For
at-rules, this specifies both the at-identifier and the rule
content.

cssutils also allows rule to be a valid \textbf{CSSRule} object

\item[{index}] \leavevmode 
within the media block's rule collection of the rule before
which to insert the specified rule. If the specified index is
equal to the length of the media blocks's rule collection, the
rule will be added to the end of the media block.
If index is not given or None rule will be appended to rule
list.

\end{description}

Used to insert a new rule into the media block.

DOMException on setting
\begin{itemize}
\item {} 
HIERARCHY{\_}REQUEST{\_}ERR:
(no use case yet as no @charset or @import allowed))
Raised if the rule cannot be inserted at the specified index,
e.g., if an @import rule is inserted after a standard rule set
or other at-rule.

\item {} 
INDEX{\_}SIZE{\_}ERR: (self)
Raised if the specified index is not a valid insertion point.

\item {} 
NO{\_}MODIFICATION{\_}ALLOWED{\_}ERR: (self)
Raised if this media rule is readonly.

\item {} 
SYNTAX{\_}ERR: (CSSStyleRule)
Raised if the specified rule has a syntax error and is
unparsable.

\end{itemize}

returns the index within the media block's rule collection of the
newly inserted rule.
\setlength{\parskip}{1ex}
    \end{boxedminipage}

    \vspace{0.5ex}

\hspace{.8\funcindent}\begin{boxedminipage}{\funcwidth}

    \raggedright \textbf{\_\_repr\_\_}(\textit{self})

\setlength{\parskip}{2ex}
    repr(x)

\setlength{\parskip}{1ex}
      Overrides: object.\_\_repr\_\_ 	extit{(inherited documentation)}

    \end{boxedminipage}

    \vspace{0.5ex}

\hspace{.8\funcindent}\begin{boxedminipage}{\funcwidth}

    \raggedright \textbf{\_\_str\_\_}(\textit{self})

\setlength{\parskip}{2ex}
    str(x)

\setlength{\parskip}{1ex}
      Overrides: object.\_\_str\_\_ 	extit{(inherited documentation)}

    \end{boxedminipage}


\large{\textbf{\textit{Inherited from object}}}

\begin{quote}
\_\_delattr\_\_(), \_\_getattribute\_\_(), \_\_hash\_\_(), \_\_new\_\_(), \_\_reduce\_\_(), \_\_reduce\_ex\_\_(), \_\_setattr\_\_()
\end{quote}

%%%%%%%%%%%%%%%%%%%%%%%%%%%%%%%%%%%%%%%%%%%%%%%%%%%%%%%%%%%%%%%%%%%%%%%%%%%
%%                              Properties                               %%
%%%%%%%%%%%%%%%%%%%%%%%%%%%%%%%%%%%%%%%%%%%%%%%%%%%%%%%%%%%%%%%%%%%%%%%%%%%

  \subsubsection{Properties}

    \vspace{-1cm}
\hspace{\varindent}\begin{longtable}{|p{\varnamewidth}|p{\vardescrwidth}|l}
\cline{1-2}
\cline{1-2} \centering \textbf{Name} & \centering \textbf{Description}& \\
\cline{1-2}
\endhead\cline{1-2}\multicolumn{3}{r}{\small\textit{continued on next page}}\\\endfoot\cline{1-2}
\endlastfoot\raggedright c\-s\-s\-T\-e\-x\-t\- & \raggedright (DOM attribute) The parsable textual representation.&\\
\cline{1-2}
\raggedright m\-e\-d\-i\-a\- & \raggedright (DOM readonly) A list of media types for this rule of type            MediaList&\\
\cline{1-2}
\multicolumn{2}{|l|}{\textit{Inherited from cssutils.css.cssrule.CSSRule \textit{(Section \ref{cssutils:css:cssrule:CSSRule})}}}\\
\multicolumn{2}{|p{\varwidth}|}{\raggedright parentRule, parentStyleSheet, typeString}\\
\cline{1-2}
\multicolumn{2}{|l|}{\textit{Inherited from object}}\\
\multicolumn{2}{|p{\varwidth}|}{\raggedright \_\_class\_\_}\\
\cline{1-2}
\end{longtable}


%%%%%%%%%%%%%%%%%%%%%%%%%%%%%%%%%%%%%%%%%%%%%%%%%%%%%%%%%%%%%%%%%%%%%%%%%%%
%%                            Class Variables                            %%
%%%%%%%%%%%%%%%%%%%%%%%%%%%%%%%%%%%%%%%%%%%%%%%%%%%%%%%%%%%%%%%%%%%%%%%%%%%

  \subsubsection{Class Variables}

    \vspace{-1cm}
\hspace{\varindent}\begin{longtable}{|p{\varnamewidth}|p{\vardescrwidth}|l}
\cline{1-2}
\cline{1-2} \centering \textbf{Name} & \centering \textbf{Description}& \\
\cline{1-2}
\endhead\cline{1-2}\multicolumn{3}{r}{\small\textit{continued on next page}}\\\endfoot\cline{1-2}
\endlastfoot\raggedright t\-y\-p\-e\- & \raggedright The type of this rule, as defined by a CSSRule type constant.
Overwritten in derived classes.

The expectation is that binding-specific casting methods can be used to
cast down from an instance of the CSSRule interface to the specific
derived interface implied by the type.
(Casting not for this Python implementation I guess...)

\textbf{Value:} 
{\tt 4}&\\
\cline{1-2}
\multicolumn{2}{|l|}{\textit{Inherited from cssutils.css.cssrule.CSSRule \textit{(Section \ref{cssutils:css:cssrule:CSSRule})}}}\\
\multicolumn{2}{|p{\varwidth}|}{\raggedright CHARSET\_RULE, COMMENT, FONT\_FACE\_RULE, IMPORT\_RULE, MEDIA\_RULE, NAMESPACE\_RULE, PAGE\_RULE, STYLE\_RULE, UNKNOWN\_RULE}\\
\cline{1-2}
\end{longtable}

    \index{cssutils \textit{(package)}!cssutils.css \textit{(package)}!cssutils.css.cssmediarule \textit{(module)}!cssutils.css.cssmediarule.CSSMediaRule \textit{(class)}|)}

%%%%%%%%%%%%%%%%%%%%%%%%%%%%%%%%%%%%%%%%%%%%%%%%%%%%%%%%%%%%%%%%%%%%%%%%%%%
%%                           Class Description                           %%
%%%%%%%%%%%%%%%%%%%%%%%%%%%%%%%%%%%%%%%%%%%%%%%%%%%%%%%%%%%%%%%%%%%%%%%%%%%

    \index{cssutils \textit{(package)}!cssutils.css \textit{(package)}!cssutils.css.cssnamespacerule \textit{(module)}!cssutils.css.cssnamespacerule.CSSNamespaceRule \textit{(class)}|(}
\subsection{Class CSSNamespaceRule}

    \label{cssutils:css:cssnamespacerule:CSSNamespaceRule}
\begin{tabular}{cccccccccc}
% Line for object, linespec=[False, False, False]
\multicolumn{2}{r}{\settowidth{\BCL}{object}\multirow{2}{\BCL}{object}}
&&
&&
&&
  \\\cline{3-3}
  &&\multicolumn{1}{c|}{}
&&
&&
&&
  \\
% Line for cssutils.util.Base, linespec=[False, False]
\multicolumn{4}{r}{\settowidth{\BCL}{cssutils.util.Base}\multirow{2}{\BCL}{cssutils.util.Base}}
&&
&&
  \\\cline{5-5}
  &&&&\multicolumn{1}{c|}{}
&&
&&
  \\
% Line for cssutils.css.cssrule.CSSRule, linespec=[False]
\multicolumn{6}{r}{\settowidth{\BCL}{cssutils.css.cssrule.CSSRule}\multirow{2}{\BCL}{cssutils.css.cssrule.CSSRule}}
&&
  \\\cline{7-7}
  &&&&&&\multicolumn{1}{c|}{}
&&
  \\
&&&&&&\multicolumn{2}{l}{\textbf{cssutils.css.cssnamespacerule.CSSNamespaceRule}}
\end{tabular}


Represents an @namespace rule within a CSS style sheet.

The @namespace at-rule declares a namespace prefix and associates
it with a given namespace (a string). This namespace prefix can then be
used in namespace-qualified names such as those described in the
Selectors Module {[}SELECT{]} or the Values and Units module {[}CSS3VAL{]}.


%___________________________________________________________________________

\hypertarget{properties}{}
\pdfbookmark[3]{Properties}{properties}
\paragraph*{Properties}
\label{properties}
\begin{description}
\item[{cssText: of type DOMString}] \leavevmode 
The parsable textual representation of this rule

\item[{namespaceURI: of type DOMString}] \leavevmode 
The namespace URI (a simple string!) which is bound to the given
prefix. If no prefix is set (\texttt{CSSNamespaceRule.prefix=='{}'})
the namespace defined by \texttt{namespaceURI} is set as the default
namespace.

\item[{prefix: of type DOMString}] \leavevmode 
The prefix used in the stylesheet for the given
\texttt{CSSNamespaceRule.nsuri}. If prefix is empty namespaceURI sets a
default namespace for the stylesheet.

\end{description}


%___________________________________________________________________________

\hypertarget{cssutils-only}{}
\pdfbookmark[4]{cssutils only}{cssutils-only}
\subparagraph*{cssutils only}
\label{cssutils-only}
\begin{description}
\item[{atkeyword:}] \leavevmode 
the literal keyword used

\end{description}

Inherits properties from CSSRule


%___________________________________________________________________________

\hypertarget{format}{}
\pdfbookmark[3]{Format}{format}
\paragraph*{Format}
\label{format}
\begin{description}
\item[{namespace}] \leavevmode 
: NAMESPACE{\_}SYM S* {[}namespace{\_}prefix S*{]}? {[}STRING{\textbar}URI{]} S* ';' S*
;

\item[{namespace{\_}prefix}] \leavevmode 
: IDENT
;

\end{description}

%%%%%%%%%%%%%%%%%%%%%%%%%%%%%%%%%%%%%%%%%%%%%%%%%%%%%%%%%%%%%%%%%%%%%%%%%%%
%%                                Methods                                %%
%%%%%%%%%%%%%%%%%%%%%%%%%%%%%%%%%%%%%%%%%%%%%%%%%%%%%%%%%%%%%%%%%%%%%%%%%%%

  \subsubsection{Methods}

    \vspace{0.5ex}

\hspace{.8\funcindent}\begin{boxedminipage}{\funcwidth}

    \raggedright \textbf{\_\_init\_\_}(\textit{self}, \textit{namespaceURI}={\tt None}, \textit{prefix}={\tt None}, \textit{cssText}={\tt None}, \textit{parentRule}={\tt None}, \textit{parentStyleSheet}={\tt None}, \textit{readonly}={\tt False})

    \vspace{-1.5ex}

    \rule{\textwidth}{0.5\fboxrule}
\setlength{\parskip}{2ex}

Do not use as positional but as keyword parameters only!

If readonly allows setting of properties in constructor only

format namespace:
\begin{quote}{\ttfamily \raggedright \noindent
:~NAMESPACE{\_}SYM~S*~{[}namespace{\_}prefix~S*{]}?~{[}STRING|URI{]}~S*~';'~S*~\\
;
}\end{quote}
\setlength{\parskip}{1ex}
      \textbf{Parameters}
      \vspace{-1ex}

      \begin{quote}
        \begin{Ventry}{xxxxxxxxxxxxxxxx}

          \item[namespaceURI]


The namespace URI (a simple string!) which is bound to the
given prefix. If no prefix is set
(\texttt{CSSNamespaceRule.prefix=='{}'}) the namespace defined by
namespaceURI is set as the default namespace
          \item[prefix]


The prefix used in the stylesheet for the given
\texttt{CSSNamespaceRule.uri}.
          \item[cssText]


if no namespaceURI is given cssText must be given to set
a namespaceURI as this is readonly later on
          \item[parentStyleSheet]


sheet where this rule belongs to
        \end{Ventry}

      \end{quote}

      Overrides: object.\_\_init\_\_

    \end{boxedminipage}

    \vspace{0.5ex}

\hspace{.8\funcindent}\begin{boxedminipage}{\funcwidth}

    \raggedright \textbf{\_\_repr\_\_}(\textit{self})

\setlength{\parskip}{2ex}
    repr(x)

\setlength{\parskip}{1ex}
      Overrides: object.\_\_repr\_\_ 	extit{(inherited documentation)}

    \end{boxedminipage}

    \vspace{0.5ex}

\hspace{.8\funcindent}\begin{boxedminipage}{\funcwidth}

    \raggedright \textbf{\_\_str\_\_}(\textit{self})

\setlength{\parskip}{2ex}
    str(x)

\setlength{\parskip}{1ex}
      Overrides: object.\_\_str\_\_ 	extit{(inherited documentation)}

    \end{boxedminipage}


\large{\textbf{\textit{Inherited from object}}}

\begin{quote}
\_\_delattr\_\_(), \_\_getattribute\_\_(), \_\_hash\_\_(), \_\_new\_\_(), \_\_reduce\_\_(), \_\_reduce\_ex\_\_(), \_\_setattr\_\_()
\end{quote}

%%%%%%%%%%%%%%%%%%%%%%%%%%%%%%%%%%%%%%%%%%%%%%%%%%%%%%%%%%%%%%%%%%%%%%%%%%%
%%                              Properties                               %%
%%%%%%%%%%%%%%%%%%%%%%%%%%%%%%%%%%%%%%%%%%%%%%%%%%%%%%%%%%%%%%%%%%%%%%%%%%%

  \subsubsection{Properties}

    \vspace{-1cm}
\hspace{\varindent}\begin{longtable}{|p{\varnamewidth}|p{\vardescrwidth}|l}
\cline{1-2}
\cline{1-2} \centering \textbf{Name} & \centering \textbf{Description}& \\
\cline{1-2}
\endhead\cline{1-2}\multicolumn{3}{r}{\small\textit{continued on next page}}\\\endfoot\cline{1-2}
\endlastfoot\raggedright n\-a\-m\-e\-s\-p\-a\-c\-e\-U\-R\-I\- & \raggedright URI (string!) of the defined namespace.&\\
\cline{1-2}
\raggedright p\-r\-e\-f\-i\-x\- & \raggedright Prefix used for the defined namespace.&\\
\cline{1-2}
\raggedright p\-a\-r\-e\-n\-t\-S\-t\-y\-l\-e\-S\-h\-e\-e\-t\- & \raggedright Containing CSSStyleSheet.&\\
\cline{1-2}
\raggedright c\-s\-s\-T\-e\-x\-t\- & \raggedright (DOM attribute) The parsable textual representation.&\\
\cline{1-2}
\multicolumn{2}{|l|}{\textit{Inherited from cssutils.css.cssrule.CSSRule \textit{(Section \ref{cssutils:css:cssrule:CSSRule})}}}\\
\multicolumn{2}{|p{\varwidth}|}{\raggedright parentRule, typeString}\\
\cline{1-2}
\multicolumn{2}{|l|}{\textit{Inherited from object}}\\
\multicolumn{2}{|p{\varwidth}|}{\raggedright \_\_class\_\_}\\
\cline{1-2}
\end{longtable}


%%%%%%%%%%%%%%%%%%%%%%%%%%%%%%%%%%%%%%%%%%%%%%%%%%%%%%%%%%%%%%%%%%%%%%%%%%%
%%                            Class Variables                            %%
%%%%%%%%%%%%%%%%%%%%%%%%%%%%%%%%%%%%%%%%%%%%%%%%%%%%%%%%%%%%%%%%%%%%%%%%%%%

  \subsubsection{Class Variables}

    \vspace{-1cm}
\hspace{\varindent}\begin{longtable}{|p{\varnamewidth}|p{\vardescrwidth}|l}
\cline{1-2}
\cline{1-2} \centering \textbf{Name} & \centering \textbf{Description}& \\
\cline{1-2}
\endhead\cline{1-2}\multicolumn{3}{r}{\small\textit{continued on next page}}\\\endfoot\cline{1-2}
\endlastfoot\raggedright t\-y\-p\-e\- & \raggedright The type of this rule, as defined by a CSSRule type constant.
Overwritten in derived classes.

The expectation is that binding-specific casting methods can be used to
cast down from an instance of the CSSRule interface to the specific
derived interface implied by the type.
(Casting not for this Python implementation I guess...)

\textbf{Value:} 
{\tt 7}&\\
\cline{1-2}
\multicolumn{2}{|l|}{\textit{Inherited from cssutils.css.cssrule.CSSRule \textit{(Section \ref{cssutils:css:cssrule:CSSRule})}}}\\
\multicolumn{2}{|p{\varwidth}|}{\raggedright CHARSET\_RULE, COMMENT, FONT\_FACE\_RULE, IMPORT\_RULE, MEDIA\_RULE, NAMESPACE\_RULE, PAGE\_RULE, STYLE\_RULE, UNKNOWN\_RULE}\\
\cline{1-2}
\end{longtable}

    \index{cssutils \textit{(package)}!cssutils.css \textit{(package)}!cssutils.css.cssnamespacerule \textit{(module)}!cssutils.css.cssnamespacerule.CSSNamespaceRule \textit{(class)}|)}

%%%%%%%%%%%%%%%%%%%%%%%%%%%%%%%%%%%%%%%%%%%%%%%%%%%%%%%%%%%%%%%%%%%%%%%%%%%
%%                           Class Description                           %%
%%%%%%%%%%%%%%%%%%%%%%%%%%%%%%%%%%%%%%%%%%%%%%%%%%%%%%%%%%%%%%%%%%%%%%%%%%%

    \index{cssutils \textit{(package)}!cssutils.css \textit{(package)}!cssutils.css.csspagerule \textit{(module)}!cssutils.css.csspagerule.CSSPageRule \textit{(class)}|(}
\subsection{Class CSSPageRule}

    \label{cssutils:css:csspagerule:CSSPageRule}
\begin{tabular}{cccccccccc}
% Line for object, linespec=[False, False, False]
\multicolumn{2}{r}{\settowidth{\BCL}{object}\multirow{2}{\BCL}{object}}
&&
&&
&&
  \\\cline{3-3}
  &&\multicolumn{1}{c|}{}
&&
&&
&&
  \\
% Line for cssutils.util.Base, linespec=[False, False]
\multicolumn{4}{r}{\settowidth{\BCL}{cssutils.util.Base}\multirow{2}{\BCL}{cssutils.util.Base}}
&&
&&
  \\\cline{5-5}
  &&&&\multicolumn{1}{c|}{}
&&
&&
  \\
% Line for cssutils.css.cssrule.CSSRule, linespec=[False]
\multicolumn{6}{r}{\settowidth{\BCL}{cssutils.css.cssrule.CSSRule}\multirow{2}{\BCL}{cssutils.css.cssrule.CSSRule}}
&&
  \\\cline{7-7}
  &&&&&&\multicolumn{1}{c|}{}
&&
  \\
&&&&&&\multicolumn{2}{l}{\textbf{cssutils.css.csspagerule.CSSPageRule}}
\end{tabular}


The CSSPageRule interface represents a @page rule within a CSS style
sheet. The @page rule is used to specify the dimensions, orientation,
margins, etc. of a page box for paged media.


%___________________________________________________________________________

\hypertarget{properties}{}
\pdfbookmark[3]{Properties}{properties}
\paragraph*{Properties}
\label{properties}
\begin{description}
\item[{cssText: of type DOMString}] \leavevmode 
The parsable textual representation of this rule

\item[{selectorText: of type DOMString}] \leavevmode 
The parsable textual representation of the page selector for the rule.

\item[{style: of type CSSStyleDeclaration}] \leavevmode 
The declaration-block of this rule.

\end{description}


%___________________________________________________________________________

\hypertarget{cssutils-only}{}
\pdfbookmark[4]{cssutils only}{cssutils-only}
\subparagraph*{cssutils only}
\label{cssutils-only}
\begin{description}
\item[{atkeyword:}] \leavevmode 
the literal keyword used

\end{description}

Inherits properties from CSSRule


%___________________________________________________________________________

\hypertarget{format}{}
\pdfbookmark[3]{Format}{format}
\paragraph*{Format}
\label{format}
\begin{quote}{\ttfamily \raggedright \noindent
page~\\
~~:~PAGE{\_}SYM~S*~pseudo{\_}page?~S*~\\
~~~~LBRACE~S*~declaration~{[}~';'~S*~declaration~{]}*~'{\}}'~S*~\\
~~;~\\
pseudo{\_}page~\\
~~:~':'~IDENT~{\#}~:first,~:left,~:right~in~CSS~2.1~\\
~~;
}\end{quote}

%%%%%%%%%%%%%%%%%%%%%%%%%%%%%%%%%%%%%%%%%%%%%%%%%%%%%%%%%%%%%%%%%%%%%%%%%%%
%%                                Methods                                %%
%%%%%%%%%%%%%%%%%%%%%%%%%%%%%%%%%%%%%%%%%%%%%%%%%%%%%%%%%%%%%%%%%%%%%%%%%%%

  \subsubsection{Methods}

    \vspace{0.5ex}

\hspace{.8\funcindent}\begin{boxedminipage}{\funcwidth}

    \raggedright \textbf{\_\_init\_\_}(\textit{self}, \textit{selectorText}={\tt None}, \textit{style}={\tt None}, \textit{parentRule}={\tt None}, \textit{parentStyleSheet}={\tt None}, \textit{readonly}={\tt False})

    \vspace{-1.5ex}

    \rule{\textwidth}{0.5\fboxrule}
\setlength{\parskip}{2ex}

if readonly allows setting of properties in constructor only
\begin{description}
\item[{selectorText}] \leavevmode 
type string

\item[{style}] \leavevmode 
CSSStyleDeclaration for this CSSStyleRule

\end{description}
\setlength{\parskip}{1ex}
      Overrides: object.\_\_init\_\_

    \end{boxedminipage}

    \vspace{0.5ex}

\hspace{.8\funcindent}\begin{boxedminipage}{\funcwidth}

    \raggedright \textbf{\_\_repr\_\_}(\textit{self})

\setlength{\parskip}{2ex}
    repr(x)

\setlength{\parskip}{1ex}
      Overrides: object.\_\_repr\_\_ 	extit{(inherited documentation)}

    \end{boxedminipage}

    \vspace{0.5ex}

\hspace{.8\funcindent}\begin{boxedminipage}{\funcwidth}

    \raggedright \textbf{\_\_str\_\_}(\textit{self})

\setlength{\parskip}{2ex}
    str(x)

\setlength{\parskip}{1ex}
      Overrides: object.\_\_str\_\_ 	extit{(inherited documentation)}

    \end{boxedminipage}


\large{\textbf{\textit{Inherited from object}}}

\begin{quote}
\_\_delattr\_\_(), \_\_getattribute\_\_(), \_\_hash\_\_(), \_\_new\_\_(), \_\_reduce\_\_(), \_\_reduce\_ex\_\_(), \_\_setattr\_\_()
\end{quote}

%%%%%%%%%%%%%%%%%%%%%%%%%%%%%%%%%%%%%%%%%%%%%%%%%%%%%%%%%%%%%%%%%%%%%%%%%%%
%%                              Properties                               %%
%%%%%%%%%%%%%%%%%%%%%%%%%%%%%%%%%%%%%%%%%%%%%%%%%%%%%%%%%%%%%%%%%%%%%%%%%%%

  \subsubsection{Properties}

    \vspace{-1cm}
\hspace{\varindent}\begin{longtable}{|p{\varnamewidth}|p{\vardescrwidth}|l}
\cline{1-2}
\cline{1-2} \centering \textbf{Name} & \centering \textbf{Description}& \\
\cline{1-2}
\endhead\cline{1-2}\multicolumn{3}{r}{\small\textit{continued on next page}}\\\endfoot\cline{1-2}
\endlastfoot\raggedright c\-s\-s\-T\-e\-x\-t\- & \raggedright (DOM) The parsable textual representation of the rule.&\\
\cline{1-2}
\raggedright s\-e\-l\-e\-c\-t\-o\-r\-T\-e\-x\-t\- & \raggedright (DOM) The parsable textual representation of the page selector for the rule.&\\
\cline{1-2}
\raggedright s\-t\-y\-l\-e\- & \raggedright (DOM) The declaration-block of this rule set.&\\
\cline{1-2}
\multicolumn{2}{|l|}{\textit{Inherited from cssutils.css.cssrule.CSSRule \textit{(Section \ref{cssutils:css:cssrule:CSSRule})}}}\\
\multicolumn{2}{|p{\varwidth}|}{\raggedright parentRule, parentStyleSheet, typeString}\\
\cline{1-2}
\multicolumn{2}{|l|}{\textit{Inherited from object}}\\
\multicolumn{2}{|p{\varwidth}|}{\raggedright \_\_class\_\_}\\
\cline{1-2}
\end{longtable}


%%%%%%%%%%%%%%%%%%%%%%%%%%%%%%%%%%%%%%%%%%%%%%%%%%%%%%%%%%%%%%%%%%%%%%%%%%%
%%                            Class Variables                            %%
%%%%%%%%%%%%%%%%%%%%%%%%%%%%%%%%%%%%%%%%%%%%%%%%%%%%%%%%%%%%%%%%%%%%%%%%%%%

  \subsubsection{Class Variables}

    \vspace{-1cm}
\hspace{\varindent}\begin{longtable}{|p{\varnamewidth}|p{\vardescrwidth}|l}
\cline{1-2}
\cline{1-2} \centering \textbf{Name} & \centering \textbf{Description}& \\
\cline{1-2}
\endhead\cline{1-2}\multicolumn{3}{r}{\small\textit{continued on next page}}\\\endfoot\cline{1-2}
\endlastfoot\raggedright t\-y\-p\-e\- & \raggedright The type of this rule, as defined by a CSSRule type constant.
Overwritten in derived classes.

The expectation is that binding-specific casting methods can be used to
cast down from an instance of the CSSRule interface to the specific
derived interface implied by the type.
(Casting not for this Python implementation I guess...)

\textbf{Value:} 
{\tt 6}&\\
\cline{1-2}
\multicolumn{2}{|l|}{\textit{Inherited from cssutils.css.cssrule.CSSRule \textit{(Section \ref{cssutils:css:cssrule:CSSRule})}}}\\
\multicolumn{2}{|p{\varwidth}|}{\raggedright CHARSET\_RULE, COMMENT, FONT\_FACE\_RULE, IMPORT\_RULE, MEDIA\_RULE, NAMESPACE\_RULE, PAGE\_RULE, STYLE\_RULE, UNKNOWN\_RULE}\\
\cline{1-2}
\end{longtable}

    \index{cssutils \textit{(package)}!cssutils.css \textit{(package)}!cssutils.css.csspagerule \textit{(module)}!cssutils.css.csspagerule.CSSPageRule \textit{(class)}|)}

%%%%%%%%%%%%%%%%%%%%%%%%%%%%%%%%%%%%%%%%%%%%%%%%%%%%%%%%%%%%%%%%%%%%%%%%%%%
%%                           Class Description                           %%
%%%%%%%%%%%%%%%%%%%%%%%%%%%%%%%%%%%%%%%%%%%%%%%%%%%%%%%%%%%%%%%%%%%%%%%%%%%

    \index{cssutils \textit{(package)}!cssutils.css \textit{(package)}!cssutils.css.cssstylerule \textit{(module)}!cssutils.css.cssstylerule.CSSStyleRule \textit{(class)}|(}
\subsection{Class CSSStyleRule}

    \label{cssutils:css:cssstylerule:CSSStyleRule}
\begin{tabular}{cccccccccc}
% Line for object, linespec=[False, False, False]
\multicolumn{2}{r}{\settowidth{\BCL}{object}\multirow{2}{\BCL}{object}}
&&
&&
&&
  \\\cline{3-3}
  &&\multicolumn{1}{c|}{}
&&
&&
&&
  \\
% Line for cssutils.util.Base, linespec=[False, False]
\multicolumn{4}{r}{\settowidth{\BCL}{cssutils.util.Base}\multirow{2}{\BCL}{cssutils.util.Base}}
&&
&&
  \\\cline{5-5}
  &&&&\multicolumn{1}{c|}{}
&&
&&
  \\
% Line for cssutils.css.cssrule.CSSRule, linespec=[False]
\multicolumn{6}{r}{\settowidth{\BCL}{cssutils.css.cssrule.CSSRule}\multirow{2}{\BCL}{cssutils.css.cssrule.CSSRule}}
&&
  \\\cline{7-7}
  &&&&&&\multicolumn{1}{c|}{}
&&
  \\
&&&&&&\multicolumn{2}{l}{\textbf{cssutils.css.cssstylerule.CSSStyleRule}}
\end{tabular}


The CSSStyleRule object represents a ruleset specified (if any) in a CSS
style sheet. It provides access to a declaration block as well as to the
associated group of selectors.


%___________________________________________________________________________

\hypertarget{properties}{}
\pdfbookmark[3]{Properties}{properties}
\paragraph*{Properties}
\label{properties}
\begin{description}
\item[{selectorList: of type SelectorList (cssutils only)}] \leavevmode 
A list of all Selector elements for the rule set.

\item[{selectorText: of type DOMString}] \leavevmode 
The textual representation of the selector for the rule set. The
implementation may have stripped out insignificant whitespace while
parsing the selector.

\item[{style: of type CSSStyleDeclaration, (DOM)}] \leavevmode 
The declaration-block of this rule set.

\item[{type}] \leavevmode 
the type of this rule, constant cssutils.CSSRule.STYLE{\_}RULE

\item[{inherited properties:}] \leavevmode \begin{itemize}
\item {} 
cssText

\item {} 
parentRule

\item {} 
parentStyleSheet

\end{itemize}

\end{description}


%___________________________________________________________________________

\hypertarget{format}{}
\pdfbookmark[3]{Format}{format}
\paragraph*{Format}
\label{format}

ruleset:
\begin{quote}{\ttfamily \raggedright \noindent
:~selector~{[}~COMMA~S*~selector~{]}*~\\
LBRACE~S*~declaration~{[}~';'~S*~declaration~{]}*~'{\}}'~S*~\\
;
}\end{quote}

%%%%%%%%%%%%%%%%%%%%%%%%%%%%%%%%%%%%%%%%%%%%%%%%%%%%%%%%%%%%%%%%%%%%%%%%%%%
%%                                Methods                                %%
%%%%%%%%%%%%%%%%%%%%%%%%%%%%%%%%%%%%%%%%%%%%%%%%%%%%%%%%%%%%%%%%%%%%%%%%%%%

  \subsubsection{Methods}

    \vspace{0.5ex}

\hspace{.8\funcindent}\begin{boxedminipage}{\funcwidth}

    \raggedright \textbf{\_\_init\_\_}(\textit{self}, \textit{selectorText}={\tt None}, \textit{style}={\tt None}, \textit{parentRule}={\tt None}, \textit{parentStyleSheet}={\tt None}, \textit{readonly}={\tt False})

    \vspace{-1.5ex}

    \rule{\textwidth}{0.5\fboxrule}
\setlength{\parskip}{2ex}
    x.\_\_init\_\_(...) initializes x; see x.\_\_class\_\_.\_\_doc\_\_ for 
    signature

\setlength{\parskip}{1ex}
      \textbf{Parameters}
      \vspace{-1ex}

      \begin{quote}
        \begin{Ventry}{xxxxxxxxxxxx}

          \item[selectorText]


string parsed into selectorList
          \item[style]


string parsed into CSSStyleDeclaration for this CSSStyleRule
          \item[readonly]


if True allows setting of properties in constructor only
        \end{Ventry}

      \end{quote}

      Overrides: object.\_\_init\_\_

    \end{boxedminipage}

    \vspace{0.5ex}

\hspace{.8\funcindent}\begin{boxedminipage}{\funcwidth}

    \raggedright \textbf{\_\_repr\_\_}(\textit{self})

\setlength{\parskip}{2ex}
    repr(x)

\setlength{\parskip}{1ex}
      Overrides: object.\_\_repr\_\_ 	extit{(inherited documentation)}

    \end{boxedminipage}

    \vspace{0.5ex}

\hspace{.8\funcindent}\begin{boxedminipage}{\funcwidth}

    \raggedright \textbf{\_\_str\_\_}(\textit{self})

\setlength{\parskip}{2ex}
    str(x)

\setlength{\parskip}{1ex}
      Overrides: object.\_\_str\_\_ 	extit{(inherited documentation)}

    \end{boxedminipage}


\large{\textbf{\textit{Inherited from object}}}

\begin{quote}
\_\_delattr\_\_(), \_\_getattribute\_\_(), \_\_hash\_\_(), \_\_new\_\_(), \_\_reduce\_\_(), \_\_reduce\_ex\_\_(), \_\_setattr\_\_()
\end{quote}

%%%%%%%%%%%%%%%%%%%%%%%%%%%%%%%%%%%%%%%%%%%%%%%%%%%%%%%%%%%%%%%%%%%%%%%%%%%
%%                              Properties                               %%
%%%%%%%%%%%%%%%%%%%%%%%%%%%%%%%%%%%%%%%%%%%%%%%%%%%%%%%%%%%%%%%%%%%%%%%%%%%

  \subsubsection{Properties}

    \vspace{-1cm}
\hspace{\varindent}\begin{longtable}{|p{\varnamewidth}|p{\vardescrwidth}|l}
\cline{1-2}
\cline{1-2} \centering \textbf{Name} & \centering \textbf{Description}& \\
\cline{1-2}
\endhead\cline{1-2}\multicolumn{3}{r}{\small\textit{continued on next page}}\\\endfoot\cline{1-2}
\endlastfoot\raggedright c\-s\-s\-T\-e\-x\-t\- & \raggedright (DOM) The parsable textual representation of the rule.&\\
\cline{1-2}
\raggedright s\-e\-l\-e\-c\-t\-o\-r\-L\-i\-s\-t\- & \raggedright The SelectorList of this rule.&\\
\cline{1-2}
\raggedright s\-e\-l\-e\-c\-t\-o\-r\-T\-e\-x\-t\- & \raggedright (DOM) The textual representation of the selector for the
rule set.&\\
\cline{1-2}
\raggedright s\-t\-y\-l\-e\- & \raggedright (DOM) The declaration-block of this rule set.&\\
\cline{1-2}
\multicolumn{2}{|l|}{\textit{Inherited from cssutils.css.cssrule.CSSRule \textit{(Section \ref{cssutils:css:cssrule:CSSRule})}}}\\
\multicolumn{2}{|p{\varwidth}|}{\raggedright parentRule, parentStyleSheet, typeString}\\
\cline{1-2}
\multicolumn{2}{|l|}{\textit{Inherited from object}}\\
\multicolumn{2}{|p{\varwidth}|}{\raggedright \_\_class\_\_}\\
\cline{1-2}
\end{longtable}


%%%%%%%%%%%%%%%%%%%%%%%%%%%%%%%%%%%%%%%%%%%%%%%%%%%%%%%%%%%%%%%%%%%%%%%%%%%
%%                            Class Variables                            %%
%%%%%%%%%%%%%%%%%%%%%%%%%%%%%%%%%%%%%%%%%%%%%%%%%%%%%%%%%%%%%%%%%%%%%%%%%%%

  \subsubsection{Class Variables}

    \vspace{-1cm}
\hspace{\varindent}\begin{longtable}{|p{\varnamewidth}|p{\vardescrwidth}|l}
\cline{1-2}
\cline{1-2} \centering \textbf{Name} & \centering \textbf{Description}& \\
\cline{1-2}
\endhead\cline{1-2}\multicolumn{3}{r}{\small\textit{continued on next page}}\\\endfoot\cline{1-2}
\endlastfoot\raggedright t\-y\-p\-e\- & \raggedright The type of this rule, as defined by a CSSRule type constant.
Overwritten in derived classes.

The expectation is that binding-specific casting methods can be used to
cast down from an instance of the CSSRule interface to the specific
derived interface implied by the type.
(Casting not for this Python implementation I guess...)

\textbf{Value:} 
{\tt 1}&\\
\cline{1-2}
\multicolumn{2}{|l|}{\textit{Inherited from cssutils.css.cssrule.CSSRule \textit{(Section \ref{cssutils:css:cssrule:CSSRule})}}}\\
\multicolumn{2}{|p{\varwidth}|}{\raggedright CHARSET\_RULE, COMMENT, FONT\_FACE\_RULE, IMPORT\_RULE, MEDIA\_RULE, NAMESPACE\_RULE, PAGE\_RULE, STYLE\_RULE, UNKNOWN\_RULE}\\
\cline{1-2}
\end{longtable}

    \index{cssutils \textit{(package)}!cssutils.css \textit{(package)}!cssutils.css.cssstylerule \textit{(module)}!cssutils.css.cssstylerule.CSSStyleRule \textit{(class)}|)}

%%%%%%%%%%%%%%%%%%%%%%%%%%%%%%%%%%%%%%%%%%%%%%%%%%%%%%%%%%%%%%%%%%%%%%%%%%%
%%                           Class Description                           %%
%%%%%%%%%%%%%%%%%%%%%%%%%%%%%%%%%%%%%%%%%%%%%%%%%%%%%%%%%%%%%%%%%%%%%%%%%%%

    \index{cssutils \textit{(package)}!cssutils.css \textit{(package)}!cssutils.css.cssunknownrule \textit{(module)}!cssutils.css.cssunknownrule.CSSUnknownRule \textit{(class)}|(}
\subsection{Class CSSUnknownRule}

    \label{cssutils:css:cssunknownrule:CSSUnknownRule}
\begin{tabular}{cccccccccc}
% Line for object, linespec=[False, False, False]
\multicolumn{2}{r}{\settowidth{\BCL}{object}\multirow{2}{\BCL}{object}}
&&
&&
&&
  \\\cline{3-3}
  &&\multicolumn{1}{c|}{}
&&
&&
&&
  \\
% Line for cssutils.util.Base, linespec=[False, False]
\multicolumn{4}{r}{\settowidth{\BCL}{cssutils.util.Base}\multirow{2}{\BCL}{cssutils.util.Base}}
&&
&&
  \\\cline{5-5}
  &&&&\multicolumn{1}{c|}{}
&&
&&
  \\
% Line for cssutils.css.cssrule.CSSRule, linespec=[False]
\multicolumn{6}{r}{\settowidth{\BCL}{cssutils.css.cssrule.CSSRule}\multirow{2}{\BCL}{cssutils.css.cssrule.CSSRule}}
&&
  \\\cline{7-7}
  &&&&&&\multicolumn{1}{c|}{}
&&
  \\
&&&&&&\multicolumn{2}{l}{\textbf{cssutils.css.cssunknownrule.CSSUnknownRule}}
\end{tabular}


represents an at-rule not supported by this user agent.


%___________________________________________________________________________

\hypertarget{properties}{}
\pdfbookmark[3]{Properties}{properties}
\paragraph*{Properties}
\label{properties}
\begin{description}
\item[{inherited from CSSRule}] \leavevmode \begin{itemize}
\item {} 
cssText

\item {} 
type

\end{itemize}

\end{description}


%___________________________________________________________________________

\hypertarget{cssutils-only}{}
\pdfbookmark[4]{cssutils only}{cssutils-only}
\subparagraph*{cssutils only}
\label{cssutils-only}
\begin{description}
\item[{atkeyword:}] \leavevmode 
the literal keyword used

\item[{seq: a list (cssutils)}] \leavevmode 
All parts of this rule excluding @KEYWORD but including CSSComments

\end{description}


%___________________________________________________________________________

\hypertarget{format}{}
\pdfbookmark[3]{Format}{format}
\paragraph*{Format}
\label{format}
\begin{description}
\item[{unknownrule:}] \leavevmode 
@xxx until ';' or block {\{}...{\}}

\end{description}

%%%%%%%%%%%%%%%%%%%%%%%%%%%%%%%%%%%%%%%%%%%%%%%%%%%%%%%%%%%%%%%%%%%%%%%%%%%
%%                                Methods                                %%
%%%%%%%%%%%%%%%%%%%%%%%%%%%%%%%%%%%%%%%%%%%%%%%%%%%%%%%%%%%%%%%%%%%%%%%%%%%

  \subsubsection{Methods}

    \vspace{0.5ex}

\hspace{.8\funcindent}\begin{boxedminipage}{\funcwidth}

    \raggedright \textbf{\_\_init\_\_}(\textit{self}, \textit{cssText}={\tt \texttt{u'}\texttt{}\texttt{'}}, \textit{parentRule}={\tt None}, \textit{parentStyleSheet}={\tt None}, \textit{readonly}={\tt False})

    \vspace{-1.5ex}

    \rule{\textwidth}{0.5\fboxrule}
\setlength{\parskip}{2ex}
\begin{description}
\item[{cssText}] \leavevmode 
of type string

\end{description}
\setlength{\parskip}{1ex}
      Overrides: object.\_\_init\_\_

    \end{boxedminipage}

    \vspace{0.5ex}

\hspace{.8\funcindent}\begin{boxedminipage}{\funcwidth}

    \raggedright \textbf{\_\_repr\_\_}(\textit{self})

\setlength{\parskip}{2ex}
    repr(x)

\setlength{\parskip}{1ex}
      Overrides: object.\_\_repr\_\_ 	extit{(inherited documentation)}

    \end{boxedminipage}

    \vspace{0.5ex}

\hspace{.8\funcindent}\begin{boxedminipage}{\funcwidth}

    \raggedright \textbf{\_\_str\_\_}(\textit{self})

\setlength{\parskip}{2ex}
    str(x)

\setlength{\parskip}{1ex}
      Overrides: object.\_\_str\_\_ 	extit{(inherited documentation)}

    \end{boxedminipage}


\large{\textbf{\textit{Inherited from object}}}

\begin{quote}
\_\_delattr\_\_(), \_\_getattribute\_\_(), \_\_hash\_\_(), \_\_new\_\_(), \_\_reduce\_\_(), \_\_reduce\_ex\_\_(), \_\_setattr\_\_()
\end{quote}

%%%%%%%%%%%%%%%%%%%%%%%%%%%%%%%%%%%%%%%%%%%%%%%%%%%%%%%%%%%%%%%%%%%%%%%%%%%
%%                              Properties                               %%
%%%%%%%%%%%%%%%%%%%%%%%%%%%%%%%%%%%%%%%%%%%%%%%%%%%%%%%%%%%%%%%%%%%%%%%%%%%

  \subsubsection{Properties}

    \vspace{-1cm}
\hspace{\varindent}\begin{longtable}{|p{\varnamewidth}|p{\vardescrwidth}|l}
\cline{1-2}
\cline{1-2} \centering \textbf{Name} & \centering \textbf{Description}& \\
\cline{1-2}
\endhead\cline{1-2}\multicolumn{3}{r}{\small\textit{continued on next page}}\\\endfoot\cline{1-2}
\endlastfoot\raggedright c\-s\-s\-T\-e\-x\-t\- & \raggedright (DOM) The parsable textual representation.&\\
\cline{1-2}
\multicolumn{2}{|l|}{\textit{Inherited from cssutils.css.cssrule.CSSRule \textit{(Section \ref{cssutils:css:cssrule:CSSRule})}}}\\
\multicolumn{2}{|p{\varwidth}|}{\raggedright parentRule, parentStyleSheet, typeString}\\
\cline{1-2}
\multicolumn{2}{|l|}{\textit{Inherited from object}}\\
\multicolumn{2}{|p{\varwidth}|}{\raggedright \_\_class\_\_}\\
\cline{1-2}
\end{longtable}


%%%%%%%%%%%%%%%%%%%%%%%%%%%%%%%%%%%%%%%%%%%%%%%%%%%%%%%%%%%%%%%%%%%%%%%%%%%
%%                            Class Variables                            %%
%%%%%%%%%%%%%%%%%%%%%%%%%%%%%%%%%%%%%%%%%%%%%%%%%%%%%%%%%%%%%%%%%%%%%%%%%%%

  \subsubsection{Class Variables}

    \vspace{-1cm}
\hspace{\varindent}\begin{longtable}{|p{\varnamewidth}|p{\vardescrwidth}|l}
\cline{1-2}
\cline{1-2} \centering \textbf{Name} & \centering \textbf{Description}& \\
\cline{1-2}
\endhead\cline{1-2}\multicolumn{3}{r}{\small\textit{continued on next page}}\\\endfoot\cline{1-2}
\endlastfoot\raggedright t\-y\-p\-e\- & \raggedright The type of this rule, as defined by a CSSRule type constant.
Overwritten in derived classes.

The expectation is that binding-specific casting methods can be used to
cast down from an instance of the CSSRule interface to the specific
derived interface implied by the type.
(Casting not for this Python implementation I guess...)

\textbf{Value:} 
{\tt 0}&\\
\cline{1-2}
\multicolumn{2}{|l|}{\textit{Inherited from cssutils.css.cssrule.CSSRule \textit{(Section \ref{cssutils:css:cssrule:CSSRule})}}}\\
\multicolumn{2}{|p{\varwidth}|}{\raggedright CHARSET\_RULE, COMMENT, FONT\_FACE\_RULE, IMPORT\_RULE, MEDIA\_RULE, NAMESPACE\_RULE, PAGE\_RULE, STYLE\_RULE, UNKNOWN\_RULE}\\
\cline{1-2}
\end{longtable}

    \index{cssutils \textit{(package)}!cssutils.css \textit{(package)}!cssutils.css.cssunknownrule \textit{(module)}!cssutils.css.cssunknownrule.CSSUnknownRule \textit{(class)}|)}

%%%%%%%%%%%%%%%%%%%%%%%%%%%%%%%%%%%%%%%%%%%%%%%%%%%%%%%%%%%%%%%%%%%%%%%%%%%
%%                           Class Description                           %%
%%%%%%%%%%%%%%%%%%%%%%%%%%%%%%%%%%%%%%%%%%%%%%%%%%%%%%%%%%%%%%%%%%%%%%%%%%%

    \index{cssutils \textit{(package)}!cssutils.css \textit{(package)}!cssutils.css.selector \textit{(module)}!cssutils.css.selector.Selector \textit{(class)}|(}
\subsection{Class Selector}

    \label{cssutils:css:selector:Selector}
\begin{tabular}{cccccccccc}
% Line for object, linespec=[False, False, False]
\multicolumn{2}{r}{\settowidth{\BCL}{object}\multirow{2}{\BCL}{object}}
&&
&&
&&
  \\\cline{3-3}
  &&\multicolumn{1}{c|}{}
&&
&&
&&
  \\
% Line for cssutils.util.Base, linespec=[False, False]
\multicolumn{4}{r}{\settowidth{\BCL}{cssutils.util.Base}\multirow{2}{\BCL}{cssutils.util.Base}}
&&
&&
  \\\cline{5-5}
  &&&&\multicolumn{1}{c|}{}
&&
&&
  \\
% Line for cssutils.util.Base2, linespec=[False]
\multicolumn{6}{r}{\settowidth{\BCL}{cssutils.util.Base2}\multirow{2}{\BCL}{cssutils.util.Base2}}
&&
  \\\cline{7-7}
  &&&&&&\multicolumn{1}{c|}{}
&&
  \\
&&&&&&\multicolumn{2}{l}{\textbf{cssutils.css.selector.Selector}}
\end{tabular}


(cssutils) a single selector in a SelectorList of a CSSStyleRule


%___________________________________________________________________________

\hypertarget{properties}{}
\pdfbookmark[3]{Properties}{properties}
\paragraph*{Properties}
\label{properties}
\begin{description}
\item[{element}] \leavevmode 
Effective element target of this selector

\item[{parentList: of type SelectorList, readonly}] \leavevmode 
The SelectorList that contains this selector or None if this
Selector is not attached to a SelectorList.

\item[{selectorText}] \leavevmode 
textual representation of this Selector

\item[{seq}] \leavevmode 
sequence of Selector parts including comments

\item[{specificity (READONLY)}] \leavevmode 
tuple of (a, b, c, d) where:
\begin{description}
\item[{a}] \leavevmode 
presence of style in document, always 0 if not used on a document

\item[{b}] \leavevmode 
number of ID selectors

\item[{c}] \leavevmode 
number of .class selectors

\item[{d}] \leavevmode 
number of Element (type) selectors

\end{description}

\item[{wellformed}] \leavevmode 
if this selector is wellformed regarding the Selector spec

\end{description}


%___________________________________________________________________________

\hypertarget{format}{}
\pdfbookmark[3]{Format}{format}
\paragraph*{Format}
\label{format}
\begin{quote}{\ttfamily \raggedright \noindent
{\#}~implemented~in~SelectorList~\\
selectors{\_}group~\\
~~:~selector~{[}~COMMA~S*~selector~{]}*~\\
~~;~\\
~\\
selector~\\
~~:~simple{\_}selector{\_}sequence~{[}~combinator~simple{\_}selector{\_}sequence~{]}*~\\
~~;~\\
~\\
combinator~\\
~~/*~combinators~can~be~surrounded~by~white~space~*/~\\
~~:~PLUS~S*~|~GREATER~S*~|~TILDE~S*~|~S+~\\
~~;~\\
~\\
simple{\_}selector{\_}sequence~\\
~~:~{[}~type{\_}selector~|~universal~{]}~\\
~~~~{[}~HASH~|~class~|~attrib~|~pseudo~|~negation~{]}*~\\
~~|~{[}~HASH~|~class~|~attrib~|~pseudo~|~negation~{]}+~\\
~~;~\\
~\\
type{\_}selector~\\
~~:~{[}~namespace{\_}prefix~{]}?~element{\_}name~\\
~~;~\\
~\\
namespace{\_}prefix~\\
~~:~{[}~IDENT~|~'*'~{]}?~'|'~\\
~~;~\\
~\\
element{\_}name~\\
~~:~IDENT~\\
~~;~\\
~\\
universal~\\
~~:~{[}~namespace{\_}prefix~{]}?~'*'~\\
~~;~\\
~\\
class~\\
~~:~'.'~IDENT~\\
~~;~\\
~\\
attrib~\\
~~:~'{[}'~S*~{[}~namespace{\_}prefix~{]}?~IDENT~S*~\\
~~~~~~~~{[}~{[}~PREFIXMATCH~|~\\
~~~~~~~~~~~~SUFFIXMATCH~|~\\
~~~~~~~~~~~~SUBSTRINGMATCH~|~\\
~~~~~~~~~~~~'='~|~\\
~~~~~~~~~~~~INCLUDES~|~\\
~~~~~~~~~~~~DASHMATCH~{]}~S*~{[}~IDENT~|~STRING~{]}~S*~\\
~~~~~~~~{]}?~'{]}'~\\
~~;~\\
~\\
pseudo~\\
~~/*~'::'~starts~a~pseudo-element,~':'~a~pseudo-class~*/~\\
~~/*~Exceptions:~:first-line,~:first-letter,~:before~and~:after.~*/~\\
~~/*~Note~that~pseudo-elements~are~restricted~to~one~per~selector~and~*/~\\
~~/*~occur~only~in~the~last~simple{\_}selector{\_}sequence.~*/~\\
~~:~':'~':'?~{[}~IDENT~|~functional{\_}pseudo~{]}~\\
~~;~\\
~\\
functional{\_}pseudo~\\
~~:~FUNCTION~S*~expression~')'~\\
~~;~\\
~\\
expression~\\
~~/*~In~CSS3,~the~expressions~are~identifiers,~strings,~*/~\\
~~/*~or~of~the~form~"an+b"~*/~\\
~~:~{[}~{[}~PLUS~|~'-'~|~DIMENSION~|~NUMBER~|~STRING~|~IDENT~{]}~S*~{]}+~\\
~~;~\\
~\\
negation~\\
~~:~NOT~S*~negation{\_}arg~S*~')'~\\
~~;~\\
~\\
negation{\_}arg~\\
~~:~type{\_}selector~|~universal~|~HASH~|~class~|~attrib~|~pseudo~\\
~~;
}\end{quote}

%%%%%%%%%%%%%%%%%%%%%%%%%%%%%%%%%%%%%%%%%%%%%%%%%%%%%%%%%%%%%%%%%%%%%%%%%%%
%%                                Methods                                %%
%%%%%%%%%%%%%%%%%%%%%%%%%%%%%%%%%%%%%%%%%%%%%%%%%%%%%%%%%%%%%%%%%%%%%%%%%%%

  \subsubsection{Methods}

    \vspace{0.5ex}

\hspace{.8\funcindent}\begin{boxedminipage}{\funcwidth}

    \raggedright \textbf{\_\_init\_\_}(\textit{self}, \textit{selectorText}={\tt None}, \textit{parentList}={\tt None}, \textit{readonly}={\tt False})

    \vspace{-1.5ex}

    \rule{\textwidth}{0.5\fboxrule}
\setlength{\parskip}{2ex}
    x.\_\_init\_\_(...) initializes x; see x.\_\_class\_\_.\_\_doc\_\_ for 
    signature

\setlength{\parskip}{1ex}
      \textbf{Parameters}
      \vspace{-1ex}

      \begin{quote}
        \begin{Ventry}{xxxxxxxxxxxx}

          \item[selectorText]


initial value of this selector
          \item[parentList]


a SelectorList
          \item[readonly]


default to False
        \end{Ventry}

      \end{quote}

      Overrides: object.\_\_init\_\_

    \end{boxedminipage}

    \vspace{0.5ex}

\hspace{.8\funcindent}\begin{boxedminipage}{\funcwidth}

    \raggedright \textbf{\_\_repr\_\_}(\textit{self})

\setlength{\parskip}{2ex}
    repr(x)

\setlength{\parskip}{1ex}
      Overrides: object.\_\_repr\_\_ 	extit{(inherited documentation)}

    \end{boxedminipage}

    \vspace{0.5ex}

\hspace{.8\funcindent}\begin{boxedminipage}{\funcwidth}

    \raggedright \textbf{\_\_str\_\_}(\textit{self})

\setlength{\parskip}{2ex}
    str(x)

\setlength{\parskip}{1ex}
      Overrides: object.\_\_str\_\_ 	extit{(inherited documentation)}

    \end{boxedminipage}


\large{\textbf{\textit{Inherited from object}}}

\begin{quote}
\_\_delattr\_\_(), \_\_getattribute\_\_(), \_\_hash\_\_(), \_\_new\_\_(), \_\_reduce\_\_(), \_\_reduce\_ex\_\_(), \_\_setattr\_\_()
\end{quote}

%%%%%%%%%%%%%%%%%%%%%%%%%%%%%%%%%%%%%%%%%%%%%%%%%%%%%%%%%%%%%%%%%%%%%%%%%%%
%%                              Properties                               %%
%%%%%%%%%%%%%%%%%%%%%%%%%%%%%%%%%%%%%%%%%%%%%%%%%%%%%%%%%%%%%%%%%%%%%%%%%%%

  \subsubsection{Properties}

    \vspace{-1cm}
\hspace{\varindent}\begin{longtable}{|p{\varnamewidth}|p{\vardescrwidth}|l}
\cline{1-2}
\cline{1-2} \centering \textbf{Name} & \centering \textbf{Description}& \\
\cline{1-2}
\endhead\cline{1-2}\multicolumn{3}{r}{\small\textit{continued on next page}}\\\endfoot\cline{1-2}
\endlastfoot\raggedright e\-l\-e\-m\-e\-n\-t\- & \raggedright Effective element target of this selector.&\\
\cline{1-2}
\raggedright p\-a\-r\-e\-n\-t\-L\-i\-s\-t\- & \raggedright (DOM) The SelectorList that contains this Selector or        None if this Selector is not attached to a SelectorList.&\\
\cline{1-2}
\raggedright s\-e\-l\-e\-c\-t\-o\-r\-T\-e\-x\-t\- & \raggedright (DOM) The parsable textual representation of the selector.&\\
\cline{1-2}
\raggedright s\-p\-e\-c\-i\-f\-i\-c\-i\-t\-y\- & \raggedright Specificity of this selector (READONLY).&\\
\cline{1-2}
\multicolumn{2}{|l|}{\textit{Inherited from cssutils.util.Base2}}\\
\multicolumn{2}{|p{\varwidth}|}{\raggedright seq}\\
\cline{1-2}
\multicolumn{2}{|l|}{\textit{Inherited from object}}\\
\multicolumn{2}{|p{\varwidth}|}{\raggedright \_\_class\_\_}\\
\cline{1-2}
\end{longtable}

    \index{cssutils \textit{(package)}!cssutils.css \textit{(package)}!cssutils.css.selector \textit{(module)}!cssutils.css.selector.Selector \textit{(class)}|)}

%%%%%%%%%%%%%%%%%%%%%%%%%%%%%%%%%%%%%%%%%%%%%%%%%%%%%%%%%%%%%%%%%%%%%%%%%%%
%%                           Class Description                           %%
%%%%%%%%%%%%%%%%%%%%%%%%%%%%%%%%%%%%%%%%%%%%%%%%%%%%%%%%%%%%%%%%%%%%%%%%%%%

    \index{cssutils \textit{(package)}!cssutils.css \textit{(package)}!cssutils.css.selectorlist \textit{(module)}!cssutils.css.selectorlist.SelectorList \textit{(class)}|(}
\subsection{Class SelectorList}

    \label{cssutils:css:selectorlist:SelectorList}
\begin{tabular}{cccccccc}
% Line for object, linespec=[False, False]
\multicolumn{2}{r}{\settowidth{\BCL}{object}\multirow{2}{\BCL}{object}}
&&
&&
  \\\cline{3-3}
  &&\multicolumn{1}{c|}{}
&&
&&
  \\
% Line for cssutils.util.Base, linespec=[False]
\multicolumn{4}{r}{\settowidth{\BCL}{cssutils.util.Base}\multirow{2}{\BCL}{cssutils.util.Base}}
&&
  \\\cline{5-5}
  &&&&\multicolumn{1}{c|}{}
&&
  \\
% Line for object, linespec=[False, True]
\multicolumn{2}{r}{\settowidth{\BCL}{object}\multirow{2}{\BCL}{object}}
&&
&&\multicolumn{1}{|c}{}
  \\\cline{3-3}
  &&\multicolumn{1}{c|}{}
&&
&\multicolumn{1}{|c}{}&
  \\
% Line for cssutils.util.ListSeq, linespec=[True]
\multicolumn{4}{r}{\settowidth{\BCL}{cssutils.util.ListSeq}\multirow{2}{\BCL}{cssutils.util.ListSeq}}
&&\multicolumn{1}{|c}{}
  \\\cline{5-5}
  &&&&\multicolumn{1}{c|}{}
&\multicolumn{1}{|c}{}&
  \\
&&&&\multicolumn{2}{l}{\textbf{cssutils.css.selectorlist.SelectorList}}
\end{tabular}


(cssutils) a list of Selectors of a CSSStyleRule


%___________________________________________________________________________

\hypertarget{properties}{}
\pdfbookmark[3]{Properties}{properties}
\paragraph*{Properties}
\label{properties}
\begin{description}
\item[{length: of type unsigned long, readonly}] \leavevmode 
The number of Selector elements in the list.

\item[{parentRule: of type CSSRule, readonly}] \leavevmode 
The CSS rule that contains this selector list or None if this
list is not attached to a CSSRule.

\item[{selectorText: of type DOMString}] \leavevmode 
The textual representation of the selector for the rule set. The
implementation may have stripped out insignificant whitespace while
parsing the selector.

\item[{seq: (internal use!)}] \leavevmode 
A list of Selector objects

\item[{wellformed}] \leavevmode 
if this selectorlist is wellformed regarding the Selector spec

\end{description}

%%%%%%%%%%%%%%%%%%%%%%%%%%%%%%%%%%%%%%%%%%%%%%%%%%%%%%%%%%%%%%%%%%%%%%%%%%%
%%                                Methods                                %%
%%%%%%%%%%%%%%%%%%%%%%%%%%%%%%%%%%%%%%%%%%%%%%%%%%%%%%%%%%%%%%%%%%%%%%%%%%%

  \subsubsection{Methods}

    \vspace{0.5ex}

\hspace{.8\funcindent}\begin{boxedminipage}{\funcwidth}

    \raggedright \textbf{\_\_init\_\_}(\textit{self}, \textit{selectorText}={\tt None}, \textit{parentRule}={\tt None}, \textit{readonly}={\tt False})

    \vspace{-1.5ex}

    \rule{\textwidth}{0.5\fboxrule}
\setlength{\parskip}{2ex}

initializes SelectorList with optional selectorText
\setlength{\parskip}{1ex}
      \textbf{Parameters}
      \vspace{-1ex}

      \begin{quote}
        \begin{Ventry}{xxxxxxxxxxxx}

          \item[selectorText]


parsable list of Selectors
          \item[parentRule]


the parent CSSRule if available
        \end{Ventry}

      \end{quote}

      Overrides: object.\_\_init\_\_

    \end{boxedminipage}

    \vspace{0.5ex}

\hspace{.8\funcindent}\begin{boxedminipage}{\funcwidth}

    \raggedright \textbf{\_\_setitem\_\_}(\textit{self}, \textit{index}, \textit{newSelector})

    \vspace{-1.5ex}

    \rule{\textwidth}{0.5\fboxrule}
\setlength{\parskip}{2ex}

overwrites ListSeq.{\_}{\_}setitem{\_}{\_}

Any duplicate Selectors are \textbf{not} removed.
\setlength{\parskip}{1ex}
      Overrides: cssutils.util.ListSeq.\_\_setitem\_\_

    \end{boxedminipage}

    \vspace{0.5ex}

\hspace{.8\funcindent}\begin{boxedminipage}{\funcwidth}

    \raggedright \textbf{append}(\textit{self}, \textit{newSelector})

    \vspace{-1.5ex}

    \rule{\textwidth}{0.5\fboxrule}
\setlength{\parskip}{2ex}

overwrites ListSeq.append
\setlength{\parskip}{1ex}
      Overrides: cssutils.util.ListSeq.append

    \end{boxedminipage}

    \label{cssutils:css:selectorlist:SelectorList:appendSelector}
    \index{cssutils \textit{(package)}!cssutils.css \textit{(package)}!cssutils.css.selectorlist \textit{(module)}!cssutils.css.selectorlist.SelectorList \textit{(class)}!cssutils.css.selectorlist.SelectorList.appendSelector \textit{(method)}}

    \vspace{0.5ex}

\hspace{.8\funcindent}\begin{boxedminipage}{\funcwidth}

    \raggedright \textbf{appendSelector}(\textit{self}, \textit{newSelector})

    \vspace{-1.5ex}

    \rule{\textwidth}{0.5\fboxrule}
\setlength{\parskip}{2ex}

Append newSelector (a string will be converted to a new
Selector).
\setlength{\parskip}{1ex}
      \textbf{Parameters}
      \vspace{-1ex}

      \begin{quote}
        \begin{Ventry}{xxxxxxxxxxx}

          \item[newSelector]


comma-separated list of selectors or a tuple of
(selectorText, dict-of-namespaces)
        \end{Ventry}

      \end{quote}

      \textbf{Return Value}
    \vspace{-1ex}

      \begin{quote}

New Selector or None if newSelector is not wellformed.
      \end{quote}

      \textbf{Raises}
    \vspace{-1ex}

      \begin{quote}
        \begin{description}

          \item[\texttt{NAMESPACE\_ERR}]


(self)
Raised if the specified selector uses an unknown namespace
prefix.
          \item[\texttt{SYNTAX\_ERR}]


(self)
Raised if the specified CSS string value has a syntax error
and is unparsable.
          \item[\texttt{NO\_MODIFICATION\_ALLOWED\_ERR}]


(self)
Raised if this rule is readonly.
        \end{description}

      \end{quote}

    \end{boxedminipage}

    \vspace{0.5ex}

\hspace{.8\funcindent}\begin{boxedminipage}{\funcwidth}

    \raggedright \textbf{\_\_repr\_\_}(\textit{self})

\setlength{\parskip}{2ex}
    repr(x)

\setlength{\parskip}{1ex}
      Overrides: object.\_\_repr\_\_ 	extit{(inherited documentation)}

    \end{boxedminipage}

    \vspace{0.5ex}

\hspace{.8\funcindent}\begin{boxedminipage}{\funcwidth}

    \raggedright \textbf{\_\_str\_\_}(\textit{self})

\setlength{\parskip}{2ex}
    str(x)

\setlength{\parskip}{1ex}
      Overrides: object.\_\_str\_\_ 	extit{(inherited documentation)}

    \end{boxedminipage}


\large{\textbf{\textit{Inherited from cssutils.util.ListSeq}}}

\begin{quote}
\_\_contains\_\_(), \_\_delitem\_\_(), \_\_getitem\_\_(), \_\_iter\_\_(), \_\_len\_\_()
\end{quote}

\large{\textbf{\textit{Inherited from object}}}

\begin{quote}
\_\_delattr\_\_(), \_\_getattribute\_\_(), \_\_hash\_\_(), \_\_new\_\_(), \_\_reduce\_\_(), \_\_reduce\_ex\_\_(), \_\_setattr\_\_()
\end{quote}

%%%%%%%%%%%%%%%%%%%%%%%%%%%%%%%%%%%%%%%%%%%%%%%%%%%%%%%%%%%%%%%%%%%%%%%%%%%
%%                              Properties                               %%
%%%%%%%%%%%%%%%%%%%%%%%%%%%%%%%%%%%%%%%%%%%%%%%%%%%%%%%%%%%%%%%%%%%%%%%%%%%

  \subsubsection{Properties}

    \vspace{-1cm}
\hspace{\varindent}\begin{longtable}{|p{\varnamewidth}|p{\vardescrwidth}|l}
\cline{1-2}
\cline{1-2} \centering \textbf{Name} & \centering \textbf{Description}& \\
\cline{1-2}
\endhead\cline{1-2}\multicolumn{3}{r}{\small\textit{continued on next page}}\\\endfoot\cline{1-2}
\endlastfoot\raggedright l\-e\-n\-g\-t\-h\- & \raggedright The number of Selector elements in the list.&\\
\cline{1-2}
\raggedright p\-a\-r\-e\-n\-t\-R\-u\-l\-e\- & \raggedright (DOM) The CSS rule that contains this SelectorList or        None if this SelectorList is not attached to a CSSRule.&\\
\cline{1-2}
\raggedright s\-e\-l\-e\-c\-t\-o\-r\-T\-e\-x\-t\- & \raggedright (cssutils) The textual representation of the selector for
a rule set.&\\
\cline{1-2}
\multicolumn{2}{|l|}{\textit{Inherited from object}}\\
\multicolumn{2}{|p{\varwidth}|}{\raggedright \_\_class\_\_}\\
\cline{1-2}
\end{longtable}

    \index{cssutils \textit{(package)}!cssutils.css \textit{(package)}!cssutils.css.selectorlist \textit{(module)}!cssutils.css.selectorlist.SelectorList \textit{(class)}|)}

%%%%%%%%%%%%%%%%%%%%%%%%%%%%%%%%%%%%%%%%%%%%%%%%%%%%%%%%%%%%%%%%%%%%%%%%%%%
%%                           Class Description                           %%
%%%%%%%%%%%%%%%%%%%%%%%%%%%%%%%%%%%%%%%%%%%%%%%%%%%%%%%%%%%%%%%%%%%%%%%%%%%

    \index{cssutils \textit{(package)}!cssutils.css \textit{(package)}!cssutils.css.property \textit{(module)}!cssutils.css.property.Property \textit{(class)}|(}
\subsection{Class Property}

    \label{cssutils:css:property:Property}
\begin{tabular}{cccccccc}
% Line for object, linespec=[False, False]
\multicolumn{2}{r}{\settowidth{\BCL}{object}\multirow{2}{\BCL}{object}}
&&
&&
  \\\cline{3-3}
  &&\multicolumn{1}{c|}{}
&&
&&
  \\
% Line for cssutils.util.Base, linespec=[False]
\multicolumn{4}{r}{\settowidth{\BCL}{cssutils.util.Base}\multirow{2}{\BCL}{cssutils.util.Base}}
&&
  \\\cline{5-5}
  &&&&\multicolumn{1}{c|}{}
&&
  \\
&&&&\multicolumn{2}{l}{\textbf{cssutils.css.property.Property}}
\end{tabular}


(cssutils) a CSS property in a StyleDeclaration of a CSSStyleRule


%___________________________________________________________________________

\hypertarget{properties}{}
\pdfbookmark[3]{Properties}{properties}
\paragraph*{Properties}
\label{properties}
\begin{description}
\item[{cssText}] \leavevmode 
a parsable textual representation of this property

\item[{name}] \leavevmode 
normalized name of the property, e.g. ``color'' when name is ``color''
(since 0.9.5)

\item[{literalname (since 0.9.5)}] \leavevmode 
original name of the property in the source CSS which is not normalized
e.g. ``COLor''

\item[{cssValue}] \leavevmode 
the relevant CSSValue instance for this property

\item[{value}] \leavevmode 
the string value of the property, same as cssValue.cssText

\item[{priority}] \leavevmode 
of the property (currently only u``important'' or None)

\item[{literalpriority}] \leavevmode 
original priority of the property in the source CSS which is not
normalized e.g. ``IMportant''

\item[{seqs}] \leavevmode 
combination of a list for seq of name, a CSSValue object, and
a list for seq of  priority (empty or {[}!important{]} currently)

\item[{valid}] \leavevmode 
if this Property is valid

\item[{wellformed}] \leavevmode 
if this Property is syntactically ok

\item[{DEPRECATED normalname (since 0.9.5)}] \leavevmode 
normalized name of the property, e.g. ``color'' when name is ``color''

\end{description}


%___________________________________________________________________________

\hypertarget{format}{}
\pdfbookmark[3]{Format}{format}
\paragraph*{Format}
\label{format}
\begin{quote}{\ttfamily \raggedright \noindent
property~=~name~\\
~~:~IDENT~S*~\\
~~;~\\
~\\
expr~=~value~\\
~~:~term~{[}~operator~term~{]}*~\\
~~;~\\
term~\\
~~:~unary{\_}operator?~\\
~~~~{[}~NUMBER~S*~|~PERCENTAGE~S*~|~LENGTH~S*~|~EMS~S*~|~EXS~S*~|~ANGLE~S*~|~\\
~~~~~~TIME~S*~|~FREQ~S*~|~function~{]}~\\
~~|~STRING~S*~|~IDENT~S*~|~URI~S*~|~hexcolor~\\
~~;~\\
function~\\
~~:~FUNCTION~S*~expr~')'~S*~\\
~~;~\\
/*~\\
~*~There~is~a~constraint~on~the~color~that~it~must~\\
~*~have~either~3~or~6~hex-digits~(i.e.,~{[}0-9a-fA-F{]})~\\
~*~after~the~"{\#}";~e.g.,~"{\#}000"~is~OK,~but~"{\#}abcd"~is~not.~\\
~*/~\\
hexcolor~\\
~~:~HASH~S*~\\
~~;~\\
~\\
prio~\\
~~:~IMPORTANT{\_}SYM~S*~\\
~~;
}\end{quote}

%%%%%%%%%%%%%%%%%%%%%%%%%%%%%%%%%%%%%%%%%%%%%%%%%%%%%%%%%%%%%%%%%%%%%%%%%%%
%%                                Methods                                %%
%%%%%%%%%%%%%%%%%%%%%%%%%%%%%%%%%%%%%%%%%%%%%%%%%%%%%%%%%%%%%%%%%%%%%%%%%%%

  \subsubsection{Methods}

    \vspace{0.5ex}

\hspace{.8\funcindent}\begin{boxedminipage}{\funcwidth}

    \raggedright \textbf{\_\_init\_\_}(\textit{self}, \textit{name}={\tt None}, \textit{value}={\tt None}, \textit{priority}={\tt \texttt{u'}\texttt{}\texttt{'}}, \textit{\_mediaQuery}={\tt False})

    \vspace{-1.5ex}

    \rule{\textwidth}{0.5\fboxrule}
\setlength{\parskip}{2ex}

inits property
\begin{description}
\item[{name}] \leavevmode 
a property name string (will be normalized)

\item[{value}] \leavevmode 
a property value string

\item[{priority}] \leavevmode 
an optional priority string which currently must be u'',
u'!important' or u'important'

\item[{{\_}mediaQuery boolean}] \leavevmode 
if True value is optional as used by MediaQuery objects

\end{description}
\setlength{\parskip}{1ex}
      Overrides: object.\_\_init\_\_

    \end{boxedminipage}

    \vspace{0.5ex}

\hspace{.8\funcindent}\begin{boxedminipage}{\funcwidth}

    \raggedright \textbf{\_\_repr\_\_}(\textit{self})

\setlength{\parskip}{2ex}
    repr(x)

\setlength{\parskip}{1ex}
      Overrides: object.\_\_repr\_\_ 	extit{(inherited documentation)}

    \end{boxedminipage}

    \vspace{0.5ex}

\hspace{.8\funcindent}\begin{boxedminipage}{\funcwidth}

    \raggedright \textbf{\_\_str\_\_}(\textit{self})

\setlength{\parskip}{2ex}
    str(x)

\setlength{\parskip}{1ex}
      Overrides: object.\_\_str\_\_ 	extit{(inherited documentation)}

    \end{boxedminipage}


\large{\textbf{\textit{Inherited from object}}}

\begin{quote}
\_\_delattr\_\_(), \_\_getattribute\_\_(), \_\_hash\_\_(), \_\_new\_\_(), \_\_reduce\_\_(), \_\_reduce\_ex\_\_(), \_\_setattr\_\_()
\end{quote}

%%%%%%%%%%%%%%%%%%%%%%%%%%%%%%%%%%%%%%%%%%%%%%%%%%%%%%%%%%%%%%%%%%%%%%%%%%%
%%                              Properties                               %%
%%%%%%%%%%%%%%%%%%%%%%%%%%%%%%%%%%%%%%%%%%%%%%%%%%%%%%%%%%%%%%%%%%%%%%%%%%%

  \subsubsection{Properties}

    \vspace{-1cm}
\hspace{\varindent}\begin{longtable}{|p{\varnamewidth}|p{\vardescrwidth}|l}
\cline{1-2}
\cline{1-2} \centering \textbf{Name} & \centering \textbf{Description}& \\
\cline{1-2}
\endhead\cline{1-2}\multicolumn{3}{r}{\small\textit{continued on next page}}\\\endfoot\cline{1-2}
\endlastfoot\raggedright c\-s\-s\-T\-e\-x\-t\- & \raggedright A parsable textual representation.&\\
\cline{1-2}
\raggedright n\-a\-m\-e\- & \raggedright Name of this property&\\
\cline{1-2}
\raggedright l\-i\-t\-e\-r\-a\-l\-n\-a\-m\-e\- & \raggedright Readonly literal (not normalized) name of this property&\\
\cline{1-2}
\raggedright c\-s\-s\-V\-a\-l\-u\-e\- & \raggedright (cssutils) CSSValue object of this property&\\
\cline{1-2}
\raggedright v\-a\-l\-u\-e\- & \raggedright The textual value of this Properties cssValue.&\\
\cline{1-2}
\raggedright p\-r\-i\-o\-r\-i\-t\-y\- & \raggedright (cssutils) Priority of this property&\\
\cline{1-2}
\raggedright l\-i\-t\-e\-r\-a\-l\-p\-r\-i\-o\-r\-i\-t\-y\- & \raggedright Readonly literal (not normalized) priority of this property&\\
\cline{1-2}
\raggedright n\-o\-r\-m\-a\-l\-n\-a\-m\-e\- & \raggedright DEPRECATED since 0.9.5, use name instead&\\
\cline{1-2}
\multicolumn{2}{|l|}{\textit{Inherited from object}}\\
\multicolumn{2}{|p{\varwidth}|}{\raggedright \_\_class\_\_}\\
\cline{1-2}
\end{longtable}

    \index{cssutils \textit{(package)}!cssutils.css \textit{(package)}!cssutils.css.property \textit{(module)}!cssutils.css.property.Property \textit{(class)}|)}

%%%%%%%%%%%%%%%%%%%%%%%%%%%%%%%%%%%%%%%%%%%%%%%%%%%%%%%%%%%%%%%%%%%%%%%%%%%
%%                           Class Description                           %%
%%%%%%%%%%%%%%%%%%%%%%%%%%%%%%%%%%%%%%%%%%%%%%%%%%%%%%%%%%%%%%%%%%%%%%%%%%%

    \index{cssutils \textit{(package)}!cssutils.css \textit{(package)}!cssutils.css.cssstyledeclaration \textit{(module)}!cssutils.css.cssstyledeclaration.CSSStyleDeclaration \textit{(class)}|(}
\subsection{Class CSSStyleDeclaration}

    \label{cssutils:css:cssstyledeclaration:CSSStyleDeclaration}
\begin{tabular}{cccccccc}
% Line for object, linespec=[False, False]
\multicolumn{2}{r}{\settowidth{\BCL}{object}\multirow{2}{\BCL}{object}}
&&
&&
  \\\cline{3-3}
  &&\multicolumn{1}{c|}{}
&&
&&
  \\
% Line for cssutils.css.cssproperties.CSS2Properties, linespec=[False]
\multicolumn{4}{r}{\settowidth{\BCL}{cssutils.css.cssproperties.CSS2Properties}\multirow{2}{\BCL}{cssutils.css.cssproperties.CSS2Properties}}
&&
  \\\cline{5-5}
  &&&&\multicolumn{1}{c|}{}
&&
  \\
% Line for object, linespec=[False, True]
\multicolumn{2}{r}{\settowidth{\BCL}{object}\multirow{2}{\BCL}{object}}
&&
&&\multicolumn{1}{|c}{}
  \\\cline{3-3}
  &&\multicolumn{1}{c|}{}
&&
&\multicolumn{1}{|c}{}&
  \\
% Line for cssutils.util.Base, linespec=[True]
\multicolumn{4}{r}{\settowidth{\BCL}{cssutils.util.Base}\multirow{2}{\BCL}{cssutils.util.Base}}
&&\multicolumn{1}{|c}{}
  \\\cline{5-5}
  &&&&\multicolumn{1}{c|}{}
&\multicolumn{1}{|c}{}&
  \\
&&&&\multicolumn{2}{l}{\textbf{cssutils.css.cssstyledeclaration.CSSStyleDeclaration}}
\end{tabular}


The CSSStyleDeclaration class represents a single CSS declaration
block. This class may be used to determine the style properties
currently set in a block or to set style properties explicitly
within the block.

While an implementation may not recognize all CSS properties within
a CSS declaration block, it is expected to provide access to all
specified properties in the style sheet through the
CSSStyleDeclaration interface.
Furthermore, implementations that support a specific level of CSS
should correctly handle CSS shorthand properties for that level. For
a further discussion of shorthand properties, see the CSS2Properties
interface.

Additionally the CSS2Properties interface is implemented.


%___________________________________________________________________________

\hypertarget{properties}{}
\pdfbookmark[3]{Properties}{properties}
\paragraph*{Properties}
\label{properties}
\begin{description}
\item[{cssText}] \leavevmode 
The parsable textual representation of the declaration block
(excluding the surrounding curly braces). Setting this attribute
will result in the parsing of the new value and resetting of the
properties in the declaration block. It also allows the insertion
of additional properties and their values into the block.

\item[{length: of type unsigned long, readonly}] \leavevmode 
The number of properties that have been explicitly set in this
declaration block. The range of valid indices is 0 to length-1
inclusive.

\item[{parentRule: of type CSSRule, readonly}] \leavevmode 
The CSS rule that contains this declaration block or None if this
CSSStyleDeclaration is not attached to a CSSRule.

\item[{seq: a list (cssutils)}] \leavevmode 
All parts of this style declaration including CSSComments

\item[{valid}] \leavevmode 
if this declaration is valid, currently to CSS 2.1 (?)

\item[{wellformed}] \leavevmode 
if this declaration is syntactically ok

\item[{{\$}css2propertyname}] \leavevmode 
All properties defined in the CSS2Properties class are available
as direct properties of CSSStyleDeclaration with their respective
DOM name, so e.g. \texttt{fontStyle} for property 'font-style'.

These may be used as:
\begin{quote}{\ttfamily \raggedright \noindent
>{}>{}>~style~=~CSSStyleDeclaration(cssText='color:~red')~\\
>{}>{}>~style.color~=~'green'~\\
>{}>{}>~print~style.color~\\
green~\\
>{}>{}>~del~style.color~\\
>{}>{}>~print~style.color~{\#}~print~empty~string
}\end{quote}

\end{description}


%___________________________________________________________________________

\hypertarget{format}{}
\pdfbookmark[3]{Format}{format}
\paragraph*{Format}
\label{format}

{[}Property: Value Priority?;{]}* {[}Property: Value Priority?{]}?

%%%%%%%%%%%%%%%%%%%%%%%%%%%%%%%%%%%%%%%%%%%%%%%%%%%%%%%%%%%%%%%%%%%%%%%%%%%
%%                                Methods                                %%
%%%%%%%%%%%%%%%%%%%%%%%%%%%%%%%%%%%%%%%%%%%%%%%%%%%%%%%%%%%%%%%%%%%%%%%%%%%

  \subsubsection{Methods}

    \vspace{0.5ex}

\hspace{.8\funcindent}\begin{boxedminipage}{\funcwidth}

    \raggedright \textbf{\_\_init\_\_}(\textit{self}, \textit{cssText}={\tt \texttt{u'}\texttt{}\texttt{'}}, \textit{parentRule}={\tt None}, \textit{readonly}={\tt False})

    \vspace{-1.5ex}

    \rule{\textwidth}{0.5\fboxrule}
\setlength{\parskip}{2ex}
\begin{description}
\item[{cssText}] \leavevmode 
Shortcut, sets CSSStyleDeclaration.cssText

\item[{parentRule}] \leavevmode 
The CSS rule that contains this declaration block or
None if this CSSStyleDeclaration is not attached to a CSSRule.

\item[{readonly}] \leavevmode 
defaults to False

\end{description}
\setlength{\parskip}{1ex}
      Overrides: object.\_\_init\_\_

    \end{boxedminipage}

    \label{cssutils:css:cssstyledeclaration:CSSStyleDeclaration:__contains__}
    \index{cssutils \textit{(package)}!cssutils.css \textit{(package)}!cssutils.css.cssstyledeclaration \textit{(module)}!cssutils.css.cssstyledeclaration.CSSStyleDeclaration \textit{(class)}!cssutils.css.cssstyledeclaration.CSSStyleDeclaration.\_\_contains\_\_ \textit{(method)}}

    \vspace{0.5ex}

\hspace{.8\funcindent}\begin{boxedminipage}{\funcwidth}

    \raggedright \textbf{\_\_contains\_\_}(\textit{self}, \textit{nameOrProperty})

    \vspace{-1.5ex}

    \rule{\textwidth}{0.5\fboxrule}
\setlength{\parskip}{2ex}

checks if a property (or a property with given name is in style
\begin{description}
\item[{name}] \leavevmode 
a string or Property, uses normalized name and not literalname

\end{description}
\setlength{\parskip}{1ex}
    \end{boxedminipage}

    \label{cssutils:css:cssstyledeclaration:CSSStyleDeclaration:__iter__}
    \index{cssutils \textit{(package)}!cssutils.css \textit{(package)}!cssutils.css.cssstyledeclaration \textit{(module)}!cssutils.css.cssstyledeclaration.CSSStyleDeclaration \textit{(class)}!cssutils.css.cssstyledeclaration.CSSStyleDeclaration.\_\_iter\_\_ \textit{(method)}}

    \vspace{0.5ex}

\hspace{.8\funcindent}\begin{boxedminipage}{\funcwidth}

    \raggedright \textbf{\_\_iter\_\_}(\textit{self})

    \vspace{-1.5ex}

    \rule{\textwidth}{0.5\fboxrule}
\setlength{\parskip}{2ex}

iterator of set Property objects with different normalized names.
\setlength{\parskip}{1ex}
    \end{boxedminipage}

    \vspace{0.5ex}

\hspace{.8\funcindent}\begin{boxedminipage}{\funcwidth}

    \raggedright \textbf{\_\_setattr\_\_}(\textit{self}, \textit{n}, \textit{v})

    \vspace{-1.5ex}

    \rule{\textwidth}{0.5\fboxrule}
\setlength{\parskip}{2ex}

Prevent setting of unknown properties on CSSStyleDeclaration
which would not work anyway. For these
\texttt{CSSStyleDeclaration.setProperty} MUST be called explicitly!
\begin{description}
\item[{TODO:}] \leavevmode 
implementation of known is not really nice, any alternative?

\end{description}
\setlength{\parskip}{1ex}
      Overrides: object.\_\_setattr\_\_

    \end{boxedminipage}

    \label{cssutils:css:cssstyledeclaration:CSSStyleDeclaration:getCssText}
    \index{cssutils \textit{(package)}!cssutils.css \textit{(package)}!cssutils.css.cssstyledeclaration \textit{(module)}!cssutils.css.cssstyledeclaration.CSSStyleDeclaration \textit{(class)}!cssutils.css.cssstyledeclaration.CSSStyleDeclaration.getCssText \textit{(method)}}

    \vspace{0.5ex}

\hspace{.8\funcindent}\begin{boxedminipage}{\funcwidth}

    \raggedright \textbf{getCssText}(\textit{self}, \textit{separator}={\tt None})

    \vspace{-1.5ex}

    \rule{\textwidth}{0.5\fboxrule}
\setlength{\parskip}{2ex}

returns serialized property cssText, each property separated by
given \texttt{separator} which may e.g. be u'' to be able to use
cssText directly in an HTML style attribute. ``;'' is always part of
each property (except the last one) and can \textbf{not} be set with
separator!
\setlength{\parskip}{1ex}
    \end{boxedminipage}

    \label{cssutils:css:cssstyledeclaration:CSSStyleDeclaration:getProperties}
    \index{cssutils \textit{(package)}!cssutils.css \textit{(package)}!cssutils.css.cssstyledeclaration \textit{(module)}!cssutils.css.cssstyledeclaration.CSSStyleDeclaration \textit{(class)}!cssutils.css.cssstyledeclaration.CSSStyleDeclaration.getProperties \textit{(method)}}

    \vspace{0.5ex}

\hspace{.8\funcindent}\begin{boxedminipage}{\funcwidth}

    \raggedright \textbf{getProperties}(\textit{self}, \textit{name}={\tt None}, \textit{all}={\tt False})

    \vspace{-1.5ex}

    \rule{\textwidth}{0.5\fboxrule}
\setlength{\parskip}{2ex}

Returns a list of Property objects set in this declaration.
\begin{description}
\item[{name}] \leavevmode 
optional name of properties which are requested (a filter).
Only properties with this \textbf{always normalized} name are returned.

\item[{all=False}] \leavevmode 
if False (DEFAULT) only the effective properties (the ones set
last) are returned. If name is given a list with only one property
is returned.

if True all properties including properties set multiple times with
different values or priorities for different UAs are returned.
The order of the properties is fully kept as in the original
stylesheet.

\end{description}
\setlength{\parskip}{1ex}
    \end{boxedminipage}

    \label{cssutils:css:cssstyledeclaration:CSSStyleDeclaration:getProperty}
    \index{cssutils \textit{(package)}!cssutils.css \textit{(package)}!cssutils.css.cssstyledeclaration \textit{(module)}!cssutils.css.cssstyledeclaration.CSSStyleDeclaration \textit{(class)}!cssutils.css.cssstyledeclaration.CSSStyleDeclaration.getProperty \textit{(method)}}

    \vspace{0.5ex}

\hspace{.8\funcindent}\begin{boxedminipage}{\funcwidth}

    \raggedright \textbf{getProperty}(\textit{self}, \textit{name}, \textit{normalize}={\tt True})

    \vspace{-1.5ex}

    \rule{\textwidth}{0.5\fboxrule}
\setlength{\parskip}{2ex}

Returns the effective Property object.
\begin{description}
\item[{name}] \leavevmode 
of the CSS property, always lowercase (even if not normalized)

\item[{normalize}] \leavevmode 
if True (DEFAULT) name will be normalized (lowercase, no simple
escapes) so ``color'', ``COLOR'' or ``Color'' will all be equivalent

If False may return \textbf{NOT} the effective value but the effective
for the unnormalized name.

\end{description}
\setlength{\parskip}{1ex}
    \end{boxedminipage}

    \label{cssutils:css:cssstyledeclaration:CSSStyleDeclaration:getPropertyCSSValue}
    \index{cssutils \textit{(package)}!cssutils.css \textit{(package)}!cssutils.css.cssstyledeclaration \textit{(module)}!cssutils.css.cssstyledeclaration.CSSStyleDeclaration \textit{(class)}!cssutils.css.cssstyledeclaration.CSSStyleDeclaration.getPropertyCSSValue \textit{(method)}}

    \vspace{0.5ex}

\hspace{.8\funcindent}\begin{boxedminipage}{\funcwidth}

    \raggedright \textbf{getPropertyCSSValue}(\textit{self}, \textit{name}, \textit{normalize}={\tt True})

    \vspace{-1.5ex}

    \rule{\textwidth}{0.5\fboxrule}
\setlength{\parskip}{2ex}

Returns CSSValue, the value of the effective property if it has been
explicitly set for this declaration block.
\begin{description}
\item[{name}] \leavevmode 
of the CSS property, always lowercase (even if not normalized)

\item[{normalize}] \leavevmode 
if True (DEFAULT) name will be normalized (lowercase, no simple
escapes) so ``color'', ``COLOR'' or ``Color'' will all be equivalent

If False may return \textbf{NOT} the effective value but the effective
for the unnormalized name.

\end{description}

(DOM)
Used to retrieve the object representation of the value of a CSS
property if it has been explicitly set within this declaration
block. Returns None if the property has not been set.

(This method returns None if the property is a shorthand
property. Shorthand property values can only be accessed and
modified as strings, using the getPropertyValue and setProperty
methods.)

\textbf{cssutils currently always returns a CSSValue if the property is
set.}
\begin{description}
\item[{for more on shorthand properties see}] \leavevmode 
\href{http://www.dustindiaz.com/css-shorthand/}{http://www.dustindiaz.com/css-shorthand/}

\end{description}
\setlength{\parskip}{1ex}
    \end{boxedminipage}

    \label{cssutils:css:cssstyledeclaration:CSSStyleDeclaration:getPropertyValue}
    \index{cssutils \textit{(package)}!cssutils.css \textit{(package)}!cssutils.css.cssstyledeclaration \textit{(module)}!cssutils.css.cssstyledeclaration.CSSStyleDeclaration \textit{(class)}!cssutils.css.cssstyledeclaration.CSSStyleDeclaration.getPropertyValue \textit{(method)}}

    \vspace{0.5ex}

\hspace{.8\funcindent}\begin{boxedminipage}{\funcwidth}

    \raggedright \textbf{getPropertyValue}(\textit{self}, \textit{name}, \textit{normalize}={\tt True})

    \vspace{-1.5ex}

    \rule{\textwidth}{0.5\fboxrule}
\setlength{\parskip}{2ex}

Returns the value of the effective property if it has been explicitly
set for this declaration block. Returns the empty string if the
property has not been set.
\begin{description}
\item[{name}] \leavevmode 
of the CSS property, always lowercase (even if not normalized)

\item[{normalize}] \leavevmode 
if True (DEFAULT) name will be normalized (lowercase, no simple
escapes) so ``color'', ``COLOR'' or ``Color'' will all be equivalent

If False may return \textbf{NOT} the effective value but the effective
for the unnormalized name.

\end{description}
\setlength{\parskip}{1ex}
    \end{boxedminipage}

    \label{cssutils:css:cssstyledeclaration:CSSStyleDeclaration:getPropertyPriority}
    \index{cssutils \textit{(package)}!cssutils.css \textit{(package)}!cssutils.css.cssstyledeclaration \textit{(module)}!cssutils.css.cssstyledeclaration.CSSStyleDeclaration \textit{(class)}!cssutils.css.cssstyledeclaration.CSSStyleDeclaration.getPropertyPriority \textit{(method)}}

    \vspace{0.5ex}

\hspace{.8\funcindent}\begin{boxedminipage}{\funcwidth}

    \raggedright \textbf{getPropertyPriority}(\textit{self}, \textit{name}, \textit{normalize}={\tt True})

    \vspace{-1.5ex}

    \rule{\textwidth}{0.5\fboxrule}
\setlength{\parskip}{2ex}

Returns the priority of the effective CSS property (e.g. the
``important'' qualifier) if the property has been explicitly set in
this declaration block. The empty string if none exists.
\begin{description}
\item[{name}] \leavevmode 
of the CSS property, always lowercase (even if not normalized)

\item[{normalize}] \leavevmode 
if True (DEFAULT) name will be normalized (lowercase, no simple
escapes) so ``color'', ``COLOR'' or ``Color'' will all be equivalent

If False may return \textbf{NOT} the effective value but the effective
for the unnormalized name.

\end{description}
\setlength{\parskip}{1ex}
    \end{boxedminipage}

    \label{cssutils:css:cssstyledeclaration:CSSStyleDeclaration:removeProperty}
    \index{cssutils \textit{(package)}!cssutils.css \textit{(package)}!cssutils.css.cssstyledeclaration \textit{(module)}!cssutils.css.cssstyledeclaration.CSSStyleDeclaration \textit{(class)}!cssutils.css.cssstyledeclaration.CSSStyleDeclaration.removeProperty \textit{(method)}}

    \vspace{0.5ex}

\hspace{.8\funcindent}\begin{boxedminipage}{\funcwidth}

    \raggedright \textbf{removeProperty}(\textit{self}, \textit{name}, \textit{normalize}={\tt True})

    \vspace{-1.5ex}

    \rule{\textwidth}{0.5\fboxrule}
\setlength{\parskip}{2ex}

(DOM)
Used to remove a CSS property if it has been explicitly set within
this declaration block.

Returns the value of the property if it has been explicitly set for
this declaration block. Returns the empty string if the property
has not been set or the property name does not correspond to a
known CSS property
\begin{description}
\item[{name}] \leavevmode 
of the CSS property

\item[{normalize}] \leavevmode 
if True (DEFAULT) name will be normalized (lowercase, no simple
escapes) so ``color'', ``COLOR'' or ``Color'' will all be equivalent.
The effective Property value is returned and \emph{all} Properties
with \texttt{Property.name == name} are removed.

If False may return \textbf{NOT} the effective value but the effective
for the unnormalized \texttt{name} only. Also only the Properties with
the literal name \texttt{name} are removed.

\end{description}

raises DOMException
\begin{itemize}
\item {} 
NO{\_}MODIFICATION{\_}ALLOWED{\_}ERR: (self)
Raised if this declaration is readonly or the property is
readonly.

\end{itemize}
\setlength{\parskip}{1ex}
    \end{boxedminipage}

    \label{cssutils:css:cssstyledeclaration:CSSStyleDeclaration:setProperty}
    \index{cssutils \textit{(package)}!cssutils.css \textit{(package)}!cssutils.css.cssstyledeclaration \textit{(module)}!cssutils.css.cssstyledeclaration.CSSStyleDeclaration \textit{(class)}!cssutils.css.cssstyledeclaration.CSSStyleDeclaration.setProperty \textit{(method)}}

    \vspace{0.5ex}

\hspace{.8\funcindent}\begin{boxedminipage}{\funcwidth}

    \raggedright \textbf{setProperty}(\textit{self}, \textit{name}, \textit{value}={\tt None}, \textit{priority}={\tt \texttt{u'}\texttt{}\texttt{'}}, \textit{normalize}={\tt True})

    \vspace{-1.5ex}

    \rule{\textwidth}{0.5\fboxrule}
\setlength{\parskip}{2ex}

(DOM)
Used to set a property value and priority within this declaration
block.
\begin{description}
\item[{name}] \leavevmode 
of the CSS property to set (in W3C DOM the parameter is called
``propertyName''), always lowercase (even if not normalized)

If a property with this name is present it will be reset

cssutils also allowed name to be a Property object, all other
parameter are ignored in this case

\item[{value}] \leavevmode 
the new value of the property, omit if name is already a Property

\item[{priority}] \leavevmode 
the optional priority of the property (e.g. ``important'')

\item[{normalize}] \leavevmode 
if True (DEFAULT) name will be normalized (lowercase, no simple
escapes) so ``color'', ``COLOR'' or ``Color'' will all be equivalent

\end{description}

DOMException on setting
\begin{itemize}
\item {} 
SYNTAX{\_}ERR: (self)
Raised if the specified value has a syntax error and is
unparsable.

\item {} 
NO{\_}MODIFICATION{\_}ALLOWED{\_}ERR: (self)
Raised if this declaration is readonly or the property is
readonly.

\end{itemize}
\setlength{\parskip}{1ex}
    \end{boxedminipage}

    \label{cssutils:css:cssstyledeclaration:CSSStyleDeclaration:item}
    \index{cssutils \textit{(package)}!cssutils.css \textit{(package)}!cssutils.css.cssstyledeclaration \textit{(module)}!cssutils.css.cssstyledeclaration.CSSStyleDeclaration \textit{(class)}!cssutils.css.cssstyledeclaration.CSSStyleDeclaration.item \textit{(method)}}

    \vspace{0.5ex}

\hspace{.8\funcindent}\begin{boxedminipage}{\funcwidth}

    \raggedright \textbf{item}(\textit{self}, \textit{index})

    \vspace{-1.5ex}

    \rule{\textwidth}{0.5\fboxrule}
\setlength{\parskip}{2ex}

(DOM)
Used to retrieve the properties that have been explicitly set in
this declaration block. The order of the properties retrieved using
this method does not have to be the order in which they were set.
This method can be used to iterate over all properties in this
declaration block.
\begin{description}
\item[{index}] \leavevmode 
of the property to retrieve, negative values behave like
negative indexes on Python lists, so -1 is the last element

\end{description}

returns the name of the property at this ordinal position. The
empty string if no property exists at this position.

ATTENTION:
Only properties with a different name are counted. If two
properties with the same name are present in this declaration
only the effective one is included.

\texttt{item()} and \texttt{length} work on the same set here.
\setlength{\parskip}{1ex}
    \end{boxedminipage}

    \vspace{0.5ex}

\hspace{.8\funcindent}\begin{boxedminipage}{\funcwidth}

    \raggedright \textbf{\_\_repr\_\_}(\textit{self})

\setlength{\parskip}{2ex}
    repr(x)

\setlength{\parskip}{1ex}
      Overrides: object.\_\_repr\_\_ 	extit{(inherited documentation)}

    \end{boxedminipage}

    \vspace{0.5ex}

\hspace{.8\funcindent}\begin{boxedminipage}{\funcwidth}

    \raggedright \textbf{\_\_str\_\_}(\textit{self})

\setlength{\parskip}{2ex}
    str(x)

\setlength{\parskip}{1ex}
      Overrides: object.\_\_str\_\_ 	extit{(inherited documentation)}

    \end{boxedminipage}


\large{\textbf{\textit{Inherited from object}}}

\begin{quote}
\_\_delattr\_\_(), \_\_getattribute\_\_(), \_\_hash\_\_(), \_\_new\_\_(), \_\_reduce\_\_(), \_\_reduce\_ex\_\_()
\end{quote}

%%%%%%%%%%%%%%%%%%%%%%%%%%%%%%%%%%%%%%%%%%%%%%%%%%%%%%%%%%%%%%%%%%%%%%%%%%%
%%                              Properties                               %%
%%%%%%%%%%%%%%%%%%%%%%%%%%%%%%%%%%%%%%%%%%%%%%%%%%%%%%%%%%%%%%%%%%%%%%%%%%%

  \subsubsection{Properties}

    \vspace{-1cm}
\hspace{\varindent}\begin{longtable}{|p{\varnamewidth}|p{\vardescrwidth}|l}
\cline{1-2}
\cline{1-2} \centering \textbf{Name} & \centering \textbf{Description}& \\
\cline{1-2}
\endhead\cline{1-2}\multicolumn{3}{r}{\small\textit{continued on next page}}\\\endfoot\cline{1-2}
\endlastfoot\raggedright c\-s\-s\-T\-e\-x\-t\- & \raggedright (DOM) A parsable textual representation of the declaration        block excluding the surrounding curly braces.&\\
\cline{1-2}
\raggedright p\-a\-r\-e\-n\-t\-R\-u\-l\-e\- & \raggedright (DOM) The CSS rule that contains this declaration block or        None if this CSSStyleDeclaration is not attached to a CSSRule.&\\
\cline{1-2}
\raggedright l\-e\-n\-g\-t\-h\- & \raggedright (DOM) The number of distinct properties that have been explicitly        in this declaration block. The range of valid indices is 0 to        length-1 inclusive. These are properties with a different \texttt{name}        only. \texttt{item()} and \texttt{length} work on the same set here.&\\
\cline{1-2}
\multicolumn{2}{|l|}{\textit{Inherited from cssutils.css.cssproperties.CSS2Properties \textit{(Section \ref{cssutils:css:cssproperties:CSS2Properties})}}}\\
\multicolumn{2}{|p{\varwidth}|}{\raggedright azimuth, background, backgroundAttachment, backgroundColor, backgroundImage, backgroundPosition, backgroundRepeat, border, borderBottom, borderBottomColor, borderBottomStyle, borderBottomWidth, borderCollapse, borderColor, borderLeft, borderLeftColor, borderLeftStyle, borderRight, borderRightColor, borderRightStyle, borderRightWidth, borderSpacing, borderStyle, borderTop, borderTopColor, borderTopStyle, borderTopWidth, borderWidth, bottom, captionSide, clear, clip, color, content, counterIncrement, counterReset, cue, cueAfter, cueBefore, cursor, direction, display, elevation, emptyCells, float, font, fontFamily, fontSize, fontStyle, fontVariant, fontWeight, height, left, letterSpacing, lineHeight, listStyle, listStyleImage, listStylePosition, listStyleType, margin, marginBottom, marginLeft, marginRight, marginTop, maxHeight, maxWidth, minHeight, minWidth, orphans, outline, outlineColor, outlineStyle, outlineWidth, overflow, padding, paddingBottom, paddingLeft, paddingRight, paddingTop, pageBreakAfter, pageBreakBefore, pageBreakInside, pause, pauseAfter, pauseBefore, pitch, pitchRange, playDuring, position, quotes, richness, right, speak, speakHeader, speakNumeral, speakPunctuation, speechRate, stress, tableLayout, textAlign, textDecoration, textIndent, textTransform, top, unicodeBidi, verticalAlign, visibility, voiceFamily, volume, whiteSpace, widows, width, wordSpacing, zIndex}\\
\cline{1-2}
\multicolumn{2}{|l|}{\textit{Inherited from object}}\\
\multicolumn{2}{|p{\varwidth}|}{\raggedright \_\_class\_\_}\\
\cline{1-2}
\end{longtable}

    \index{cssutils \textit{(package)}!cssutils.css \textit{(package)}!cssutils.css.cssstyledeclaration \textit{(module)}!cssutils.css.cssstyledeclaration.CSSStyleDeclaration \textit{(class)}|)}

%%%%%%%%%%%%%%%%%%%%%%%%%%%%%%%%%%%%%%%%%%%%%%%%%%%%%%%%%%%%%%%%%%%%%%%%%%%
%%                           Class Description                           %%
%%%%%%%%%%%%%%%%%%%%%%%%%%%%%%%%%%%%%%%%%%%%%%%%%%%%%%%%%%%%%%%%%%%%%%%%%%%

    \index{cssutils \textit{(package)}!cssutils.css \textit{(package)}!cssutils.css.cssvalue \textit{(module)}!cssutils.css.cssvalue.CSSValueList \textit{(class)}|(}
\subsection{Class CSSValueList}

    \label{cssutils:css:cssvalue:CSSValueList}
\begin{tabular}{cccccccccc}
% Line for object, linespec=[False, False, False]
\multicolumn{2}{r}{\settowidth{\BCL}{object}\multirow{2}{\BCL}{object}}
&&
&&
&&
  \\\cline{3-3}
  &&\multicolumn{1}{c|}{}
&&
&&
&&
  \\
% Line for cssutils.util.Base, linespec=[False, False]
\multicolumn{4}{r}{\settowidth{\BCL}{cssutils.util.Base}\multirow{2}{\BCL}{cssutils.util.Base}}
&&
&&
  \\\cline{5-5}
  &&&&\multicolumn{1}{c|}{}
&&
&&
  \\
% Line for cssutils.css.cssvalue.CSSValue, linespec=[False]
\multicolumn{6}{r}{\settowidth{\BCL}{cssutils.css.cssvalue.CSSValue}\multirow{2}{\BCL}{cssutils.css.cssvalue.CSSValue}}
&&
  \\\cline{7-7}
  &&&&&&\multicolumn{1}{c|}{}
&&
  \\
&&&&&&\multicolumn{2}{l}{\textbf{cssutils.css.cssvalue.CSSValueList}}
\end{tabular}


The CSSValueList interface provides the abstraction of an ordered
collection of CSS values.

Some properties allow an empty list into their syntax. In that case,
these properties take the none identifier. So, an empty list means
that the property has the value none.

The items in the CSSValueList are accessible via an integral index,
starting from 0.

%%%%%%%%%%%%%%%%%%%%%%%%%%%%%%%%%%%%%%%%%%%%%%%%%%%%%%%%%%%%%%%%%%%%%%%%%%%
%%                                Methods                                %%
%%%%%%%%%%%%%%%%%%%%%%%%%%%%%%%%%%%%%%%%%%%%%%%%%%%%%%%%%%%%%%%%%%%%%%%%%%%

  \subsubsection{Methods}

    \vspace{0.5ex}

\hspace{.8\funcindent}\begin{boxedminipage}{\funcwidth}

    \raggedright \textbf{\_\_init\_\_}(\textit{self}, \textit{cssText}={\tt None}, \textit{readonly}={\tt False}, \textit{\_propertyName}={\tt None})

    \vspace{-1.5ex}

    \rule{\textwidth}{0.5\fboxrule}
\setlength{\parskip}{2ex}

inits a new CSSValueList
\setlength{\parskip}{1ex}
      Overrides: object.\_\_init\_\_

    \end{boxedminipage}

    \label{cssutils:css:cssvalue:CSSValueList:item}
    \index{cssutils \textit{(package)}!cssutils.css \textit{(package)}!cssutils.css.cssvalue \textit{(module)}!cssutils.css.cssvalue.CSSValueList \textit{(class)}!cssutils.css.cssvalue.CSSValueList.item \textit{(method)}}

    \vspace{0.5ex}

\hspace{.8\funcindent}\begin{boxedminipage}{\funcwidth}

    \raggedright \textbf{item}(\textit{self}, \textit{index})

    \vspace{-1.5ex}

    \rule{\textwidth}{0.5\fboxrule}
\setlength{\parskip}{2ex}

(DOM method) Used to retrieve a CSSValue by ordinal index. The
order in this collection represents the order of the values in the
CSS style property. If index is greater than or equal to the number
of values in the list, this returns None.
\setlength{\parskip}{1ex}
    \end{boxedminipage}

    \label{cssutils:css:cssvalue:CSSValueList:__iter__}
    \index{cssutils \textit{(package)}!cssutils.css \textit{(package)}!cssutils.css.cssvalue \textit{(module)}!cssutils.css.cssvalue.CSSValueList \textit{(class)}!cssutils.css.cssvalue.CSSValueList.\_\_iter\_\_ \textit{(method)}}

    \vspace{0.5ex}

\hspace{.8\funcindent}\begin{boxedminipage}{\funcwidth}

    \raggedright \textbf{\_\_iter\_\_}(\textit{self})

    \vspace{-1.5ex}

    \rule{\textwidth}{0.5\fboxrule}
\setlength{\parskip}{2ex}

CSSValueList is iterable
\setlength{\parskip}{1ex}
    \end{boxedminipage}

    \label{cssutils:css:cssvalue:CSSValueList:__str_}
    \index{cssutils \textit{(package)}!cssutils.css \textit{(package)}!cssutils.css.cssvalue \textit{(module)}!cssutils.css.cssvalue.CSSValueList \textit{(class)}!cssutils.css.cssvalue.CSSValueList.\_\_str\_ \textit{(method)}}

    \vspace{0.5ex}

\hspace{.8\funcindent}\begin{boxedminipage}{\funcwidth}

    \raggedright \textbf{\_\_str\_}(\textit{self})

\setlength{\parskip}{2ex}
\setlength{\parskip}{1ex}
    \end{boxedminipage}


\large{\textbf{\textit{Inherited from cssutils.css.cssvalue.CSSValue\textit{(Section \ref{cssutils:css:cssvalue:CSSValue})}}}}

\begin{quote}
\_\_repr\_\_(), \_\_str\_\_()
\end{quote}

\large{\textbf{\textit{Inherited from object}}}

\begin{quote}
\_\_delattr\_\_(), \_\_getattribute\_\_(), \_\_hash\_\_(), \_\_new\_\_(), \_\_reduce\_\_(), \_\_reduce\_ex\_\_(), \_\_setattr\_\_()
\end{quote}

%%%%%%%%%%%%%%%%%%%%%%%%%%%%%%%%%%%%%%%%%%%%%%%%%%%%%%%%%%%%%%%%%%%%%%%%%%%
%%                              Properties                               %%
%%%%%%%%%%%%%%%%%%%%%%%%%%%%%%%%%%%%%%%%%%%%%%%%%%%%%%%%%%%%%%%%%%%%%%%%%%%

  \subsubsection{Properties}

    \vspace{-1cm}
\hspace{\varindent}\begin{longtable}{|p{\varnamewidth}|p{\vardescrwidth}|l}
\cline{1-2}
\cline{1-2} \centering \textbf{Name} & \centering \textbf{Description}& \\
\cline{1-2}
\endhead\cline{1-2}\multicolumn{3}{r}{\small\textit{continued on next page}}\\\endfoot\cline{1-2}
\endlastfoot\raggedright l\-e\-n\-g\-t\-h\- & \raggedright (DOM attribute) The number of CSSValues in the list.&\\
\cline{1-2}
\multicolumn{2}{|l|}{\textit{Inherited from cssutils.css.cssvalue.CSSValue \textit{(Section \ref{cssutils:css:cssvalue:CSSValue})}}}\\
\multicolumn{2}{|p{\varwidth}|}{\raggedright cssText, cssValueTypeString}\\
\cline{1-2}
\multicolumn{2}{|l|}{\textit{Inherited from object}}\\
\multicolumn{2}{|p{\varwidth}|}{\raggedright \_\_class\_\_}\\
\cline{1-2}
\end{longtable}


%%%%%%%%%%%%%%%%%%%%%%%%%%%%%%%%%%%%%%%%%%%%%%%%%%%%%%%%%%%%%%%%%%%%%%%%%%%
%%                            Class Variables                            %%
%%%%%%%%%%%%%%%%%%%%%%%%%%%%%%%%%%%%%%%%%%%%%%%%%%%%%%%%%%%%%%%%%%%%%%%%%%%

  \subsubsection{Class Variables}

    \vspace{-1cm}
\hspace{\varindent}\begin{longtable}{|p{\varnamewidth}|p{\vardescrwidth}|l}
\cline{1-2}
\cline{1-2} \centering \textbf{Name} & \centering \textbf{Description}& \\
\cline{1-2}
\endhead\cline{1-2}\multicolumn{3}{r}{\small\textit{continued on next page}}\\\endfoot\cline{1-2}
\endlastfoot\raggedright c\-s\-s\-V\-a\-l\-u\-e\-T\-y\-p\-e\- & \raggedright \textbf{Value:} 
{\tt 2}&\\
\cline{1-2}
\multicolumn{2}{|l|}{\textit{Inherited from cssutils.css.cssvalue.CSSValue \textit{(Section \ref{cssutils:css:cssvalue:CSSValue})}}}\\
\multicolumn{2}{|p{\varwidth}|}{\raggedright CSS\_CUSTOM, CSS\_INHERIT, CSS\_PRIMITIVE\_VALUE, CSS\_VALUE\_LIST}\\
\cline{1-2}
\end{longtable}

    \index{cssutils \textit{(package)}!cssutils.css \textit{(package)}!cssutils.css.cssvalue \textit{(module)}!cssutils.css.cssvalue.CSSValueList \textit{(class)}|)}

%%%%%%%%%%%%%%%%%%%%%%%%%%%%%%%%%%%%%%%%%%%%%%%%%%%%%%%%%%%%%%%%%%%%%%%%%%%
%%                           Class Description                           %%
%%%%%%%%%%%%%%%%%%%%%%%%%%%%%%%%%%%%%%%%%%%%%%%%%%%%%%%%%%%%%%%%%%%%%%%%%%%

    \index{cssutils \textit{(package)}!cssutils.css \textit{(package)}!cssutils.css.cssvalue \textit{(module)}!cssutils.css.cssvalue.CSSValue \textit{(class)}|(}
\subsection{Class CSSValue}

    \label{cssutils:css:cssvalue:CSSValue}
\begin{tabular}{cccccccc}
% Line for object, linespec=[False, False]
\multicolumn{2}{r}{\settowidth{\BCL}{object}\multirow{2}{\BCL}{object}}
&&
&&
  \\\cline{3-3}
  &&\multicolumn{1}{c|}{}
&&
&&
  \\
% Line for cssutils.util.Base, linespec=[False]
\multicolumn{4}{r}{\settowidth{\BCL}{cssutils.util.Base}\multirow{2}{\BCL}{cssutils.util.Base}}
&&
  \\\cline{5-5}
  &&&&\multicolumn{1}{c|}{}
&&
  \\
&&&&\multicolumn{2}{l}{\textbf{cssutils.css.cssvalue.CSSValue}}
\end{tabular}

\textbf{Known Subclasses:}
cssutils.css.cssvalue.CSSPrimitiveValue,
    cssutils.css.cssvalue.CSSValueList


The CSSValue interface represents a simple or a complex value.
A CSSValue object only occurs in a context of a CSS property


%___________________________________________________________________________

\hypertarget{properties}{}
\pdfbookmark[3]{Properties}{properties}
\paragraph*{Properties}
\label{properties}
\begin{description}
\item[{cssText}] \leavevmode 
A string representation of the current value.

\item[{cssValueType}] \leavevmode 
A (readonly) code defining the type of the value.

\item[{seq: a list (cssutils)}] \leavevmode 
All parts of this style declaration including CSSComments

\item[{valid: boolean}] \leavevmode 
if the value is valid at all, False for e.g. color: {\#}1

\item[{wellformed}] \leavevmode 
if this Property is syntactically ok

\item[{{\_}value (INTERNAL!)}] \leavevmode 
value without any comments, used to validate

\end{description}

%%%%%%%%%%%%%%%%%%%%%%%%%%%%%%%%%%%%%%%%%%%%%%%%%%%%%%%%%%%%%%%%%%%%%%%%%%%
%%                                Methods                                %%
%%%%%%%%%%%%%%%%%%%%%%%%%%%%%%%%%%%%%%%%%%%%%%%%%%%%%%%%%%%%%%%%%%%%%%%%%%%

  \subsubsection{Methods}

    \vspace{0.5ex}

\hspace{.8\funcindent}\begin{boxedminipage}{\funcwidth}

    \raggedright \textbf{\_\_init\_\_}(\textit{self}, \textit{cssText}={\tt None}, \textit{readonly}={\tt False}, \textit{\_propertyName}={\tt None})

    \vspace{-1.5ex}

    \rule{\textwidth}{0.5\fboxrule}
\setlength{\parskip}{2ex}

inits a new CSS Value
\begin{description}
\item[{cssText}] \leavevmode 
the parsable cssText of the value

\item[{readonly}] \leavevmode 
defaults to False

\item[{property}] \leavevmode 
used to validate this value in the context of a property

\end{description}
\setlength{\parskip}{1ex}
      Overrides: object.\_\_init\_\_

    \end{boxedminipage}

    \vspace{0.5ex}

\hspace{.8\funcindent}\begin{boxedminipage}{\funcwidth}

    \raggedright \textbf{\_\_repr\_\_}(\textit{self})

\setlength{\parskip}{2ex}
    repr(x)

\setlength{\parskip}{1ex}
      Overrides: object.\_\_repr\_\_ 	extit{(inherited documentation)}

    \end{boxedminipage}

    \vspace{0.5ex}

\hspace{.8\funcindent}\begin{boxedminipage}{\funcwidth}

    \raggedright \textbf{\_\_str\_\_}(\textit{self})

\setlength{\parskip}{2ex}
    str(x)

\setlength{\parskip}{1ex}
      Overrides: object.\_\_str\_\_ 	extit{(inherited documentation)}

    \end{boxedminipage}


\large{\textbf{\textit{Inherited from object}}}

\begin{quote}
\_\_delattr\_\_(), \_\_getattribute\_\_(), \_\_hash\_\_(), \_\_new\_\_(), \_\_reduce\_\_(), \_\_reduce\_ex\_\_(), \_\_setattr\_\_()
\end{quote}

%%%%%%%%%%%%%%%%%%%%%%%%%%%%%%%%%%%%%%%%%%%%%%%%%%%%%%%%%%%%%%%%%%%%%%%%%%%
%%                              Properties                               %%
%%%%%%%%%%%%%%%%%%%%%%%%%%%%%%%%%%%%%%%%%%%%%%%%%%%%%%%%%%%%%%%%%%%%%%%%%%%

  \subsubsection{Properties}

    \vspace{-1cm}
\hspace{\varindent}\begin{longtable}{|p{\varnamewidth}|p{\vardescrwidth}|l}
\cline{1-2}
\cline{1-2} \centering \textbf{Name} & \centering \textbf{Description}& \\
\cline{1-2}
\endhead\cline{1-2}\multicolumn{3}{r}{\small\textit{continued on next page}}\\\endfoot\cline{1-2}
\endlastfoot\raggedright c\-s\-s\-T\-e\-x\-t\- & \raggedright A string representation of the current value.&\\
\cline{1-2}
\raggedright c\-s\-s\-V\-a\-l\-u\-e\-T\-y\-p\-e\- & \raggedright A (readonly) code defining the type of the value as defined above.&\\
\cline{1-2}
\raggedright c\-s\-s\-V\-a\-l\-u\-e\-T\-y\-p\-e\-S\-t\-r\-i\-n\-g\- & \raggedright cssutils: Name of cssValueType of this CSSValue (readonly).&\\
\cline{1-2}
\multicolumn{2}{|l|}{\textit{Inherited from object}}\\
\multicolumn{2}{|p{\varwidth}|}{\raggedright \_\_class\_\_}\\
\cline{1-2}
\end{longtable}


%%%%%%%%%%%%%%%%%%%%%%%%%%%%%%%%%%%%%%%%%%%%%%%%%%%%%%%%%%%%%%%%%%%%%%%%%%%
%%                            Class Variables                            %%
%%%%%%%%%%%%%%%%%%%%%%%%%%%%%%%%%%%%%%%%%%%%%%%%%%%%%%%%%%%%%%%%%%%%%%%%%%%

  \subsubsection{Class Variables}

    \vspace{-1cm}
\hspace{\varindent}\begin{longtable}{|p{\varnamewidth}|p{\vardescrwidth}|l}
\cline{1-2}
\cline{1-2} \centering \textbf{Name} & \centering \textbf{Description}& \\
\cline{1-2}
\endhead\cline{1-2}\multicolumn{3}{r}{\small\textit{continued on next page}}\\\endfoot\cline{1-2}
\endlastfoot\raggedright C\-S\-S\-\_\-I\-N\-H\-E\-R\-I\-T\- & \raggedright The value is inherited and the cssText contains ``inherit''.

\textbf{Value:} 
{\tt 0}&\\
\cline{1-2}
\raggedright C\-S\-S\-\_\-P\-R\-I\-M\-I\-T\-I\-V\-E\-\_\-V\-A\-L\-U\-E\- & \raggedright The value is a primitive value and an instance of the
CSSPrimitiveValue interface can be obtained by using binding-specific
casting methods on this instance of the CSSValue interface.

\textbf{Value:} 
{\tt 1}&\\
\cline{1-2}
\raggedright C\-S\-S\-\_\-V\-A\-L\-U\-E\-\_\-L\-I\-S\-T\- & \raggedright The value is a CSSValue list and an instance of the CSSValueList
interface can be obtained by using binding-specific casting
methods on this instance of the CSSValue interface.

\textbf{Value:} 
{\tt 2}&\\
\cline{1-2}
\raggedright C\-S\-S\-\_\-C\-U\-S\-T\-O\-M\- & \raggedright The value is a custom value.

\textbf{Value:} 
{\tt 3}&\\
\cline{1-2}
\end{longtable}

    \index{cssutils \textit{(package)}!cssutils.css \textit{(package)}!cssutils.css.cssvalue \textit{(module)}!cssutils.css.cssvalue.CSSValue \textit{(class)}|)}

%%%%%%%%%%%%%%%%%%%%%%%%%%%%%%%%%%%%%%%%%%%%%%%%%%%%%%%%%%%%%%%%%%%%%%%%%%%
%%                           Class Description                           %%
%%%%%%%%%%%%%%%%%%%%%%%%%%%%%%%%%%%%%%%%%%%%%%%%%%%%%%%%%%%%%%%%%%%%%%%%%%%

    \index{cssutils \textit{(package)}!cssutils.css \textit{(package)}!cssutils.css.cssvalue \textit{(module)}!cssutils.css.cssvalue.CSSPrimitiveValue \textit{(class)}|(}
\subsection{Class CSSPrimitiveValue}

    \label{cssutils:css:cssvalue:CSSPrimitiveValue}
\begin{tabular}{cccccccccc}
% Line for object, linespec=[False, False, False]
\multicolumn{2}{r}{\settowidth{\BCL}{object}\multirow{2}{\BCL}{object}}
&&
&&
&&
  \\\cline{3-3}
  &&\multicolumn{1}{c|}{}
&&
&&
&&
  \\
% Line for cssutils.util.Base, linespec=[False, False]
\multicolumn{4}{r}{\settowidth{\BCL}{cssutils.util.Base}\multirow{2}{\BCL}{cssutils.util.Base}}
&&
&&
  \\\cline{5-5}
  &&&&\multicolumn{1}{c|}{}
&&
&&
  \\
% Line for cssutils.css.cssvalue.CSSValue, linespec=[False]
\multicolumn{6}{r}{\settowidth{\BCL}{cssutils.css.cssvalue.CSSValue}\multirow{2}{\BCL}{cssutils.css.cssvalue.CSSValue}}
&&
  \\\cline{7-7}
  &&&&&&\multicolumn{1}{c|}{}
&&
  \\
&&&&&&\multicolumn{2}{l}{\textbf{cssutils.css.cssvalue.CSSPrimitiveValue}}
\end{tabular}


represents a single CSS Value.  May be used to determine the value of a
specific style property currently set in a block or to set a specific
style property explicitly within the block. Might be obtained from the
getPropertyCSSValue method of CSSStyleDeclaration.

Conversions are allowed between absolute values (from millimeters to
centimeters, from degrees to radians, and so on) but not between
relative values. (For example, a pixel value cannot be converted to a
centimeter value.) Percentage values can't be converted since they are
relative to the parent value (or another property value). There is one
exception for color percentage values: since a color percentage value
is relative to the range 0-255, a color percentage value can be
converted to a number; (see also the RGBColor interface).

%%%%%%%%%%%%%%%%%%%%%%%%%%%%%%%%%%%%%%%%%%%%%%%%%%%%%%%%%%%%%%%%%%%%%%%%%%%
%%                                Methods                                %%
%%%%%%%%%%%%%%%%%%%%%%%%%%%%%%%%%%%%%%%%%%%%%%%%%%%%%%%%%%%%%%%%%%%%%%%%%%%

  \subsubsection{Methods}

    \vspace{0.5ex}

\hspace{.8\funcindent}\begin{boxedminipage}{\funcwidth}

    \raggedright \textbf{\_\_init\_\_}(\textit{self}, \textit{cssText}={\tt None}, \textit{readonly}={\tt False}, \textit{\_propertyName}={\tt None})

    \vspace{-1.5ex}

    \rule{\textwidth}{0.5\fboxrule}
\setlength{\parskip}{2ex}

see CSSPrimitiveValue.{\_}{\_}init{\_}{\_}()
\setlength{\parskip}{1ex}
      Overrides: object.\_\_init\_\_

    \end{boxedminipage}

    \label{cssutils:css:cssvalue:CSSPrimitiveValue:getFloatValue}
    \index{cssutils \textit{(package)}!cssutils.css \textit{(package)}!cssutils.css.cssvalue \textit{(module)}!cssutils.css.cssvalue.CSSPrimitiveValue \textit{(class)}!cssutils.css.cssvalue.CSSPrimitiveValue.getFloatValue \textit{(method)}}

    \vspace{0.5ex}

\hspace{.8\funcindent}\begin{boxedminipage}{\funcwidth}

    \raggedright \textbf{getFloatValue}(\textit{self}, \textit{unitType})

    \vspace{-1.5ex}

    \rule{\textwidth}{0.5\fboxrule}
\setlength{\parskip}{2ex}

(DOM method) This method is used to get a float value in a
specified unit. If this CSS value doesn't contain a float value
or can't be converted into the specified unit, a DOMException
is raised.
\begin{description}
\item[{unitType}] \leavevmode 
to get the float value. The unit code can only be a float unit type
(i.e. CSS{\_}NUMBER, CSS{\_}PERCENTAGE, CSS{\_}EMS, CSS{\_}EXS, CSS{\_}PX, CSS{\_}CM,
CSS{\_}MM, CSS{\_}IN, CSS{\_}PT, CSS{\_}PC, CSS{\_}DEG, CSS{\_}RAD, CSS{\_}GRAD, CSS{\_}MS,
CSS{\_}S, CSS{\_}HZ, CSS{\_}KHZ, CSS{\_}DIMENSION).

\end{description}

returns not necessarily a float but some cases just an int
e.g. if the value is \texttt{1px} it return \texttt{1} and \textbf{not} \texttt{1.0}

conversions might return strange values like 1.000000000001
\setlength{\parskip}{1ex}
    \end{boxedminipage}

    \label{cssutils:css:cssvalue:CSSPrimitiveValue:setFloatValue}
    \index{cssutils \textit{(package)}!cssutils.css \textit{(package)}!cssutils.css.cssvalue \textit{(module)}!cssutils.css.cssvalue.CSSPrimitiveValue \textit{(class)}!cssutils.css.cssvalue.CSSPrimitiveValue.setFloatValue \textit{(method)}}

    \vspace{0.5ex}

\hspace{.8\funcindent}\begin{boxedminipage}{\funcwidth}

    \raggedright \textbf{setFloatValue}(\textit{self}, \textit{unitType}, \textit{floatValue})

    \vspace{-1.5ex}

    \rule{\textwidth}{0.5\fboxrule}
\setlength{\parskip}{2ex}

(DOM method) A method to set the float value with a specified unit.
If the property attached with this value can not accept the
specified unit or the float value, the value will be unchanged and
a DOMException will be raised.
\begin{description}
\item[{unitType}] \leavevmode 
a unit code as defined above. The unit code can only be a float
unit type

\item[{floatValue}] \leavevmode 
the new float value which does not have to be a float value but
may simple be an int e.g. if setting:
\begin{quote}{\ttfamily \raggedright \noindent
setFloatValue(CSS{\_}PX,~1)
}\end{quote}

\item[{raises DOMException}] \leavevmode \begin{itemize}
\item {} \begin{description}
\item[{INVALID{\_}ACCESS{\_}ERR: Raised if the attached property doesn't}] \leavevmode 
support the float value or the unit type.

\end{description}

\item {} 
NO{\_}MODIFICATION{\_}ALLOWED{\_}ERR: Raised if this property is readonly.

\end{itemize}

\end{description}
\setlength{\parskip}{1ex}
    \end{boxedminipage}

    \label{cssutils:css:cssvalue:CSSPrimitiveValue:getStringValue}
    \index{cssutils \textit{(package)}!cssutils.css \textit{(package)}!cssutils.css.cssvalue \textit{(module)}!cssutils.css.cssvalue.CSSPrimitiveValue \textit{(class)}!cssutils.css.cssvalue.CSSPrimitiveValue.getStringValue \textit{(method)}}

    \vspace{0.5ex}

\hspace{.8\funcindent}\begin{boxedminipage}{\funcwidth}

    \raggedright \textbf{getStringValue}(\textit{self})

    \vspace{-1.5ex}

    \rule{\textwidth}{0.5\fboxrule}
\setlength{\parskip}{2ex}

(DOM method) This method is used to get the string value. If the
CSS value doesn't contain a string value, a DOMException is raised.

Some properties (like 'font-family' or 'voice-family')
convert a whitespace separated list of idents to a string.

Only the actual value is returned so e.g. all the following return the
actual value \texttt{a}: url(a), attr(a), ``a'', 'a'
\setlength{\parskip}{1ex}
    \end{boxedminipage}

    \label{cssutils:css:cssvalue:CSSPrimitiveValue:setStringValue}
    \index{cssutils \textit{(package)}!cssutils.css \textit{(package)}!cssutils.css.cssvalue \textit{(module)}!cssutils.css.cssvalue.CSSPrimitiveValue \textit{(class)}!cssutils.css.cssvalue.CSSPrimitiveValue.setStringValue \textit{(method)}}

    \vspace{0.5ex}

\hspace{.8\funcindent}\begin{boxedminipage}{\funcwidth}

    \raggedright \textbf{setStringValue}(\textit{self}, \textit{stringType}, \textit{stringValue})

    \vspace{-1.5ex}

    \rule{\textwidth}{0.5\fboxrule}
\setlength{\parskip}{2ex}

(DOM method) A method to set the string value with the specified
unit. If the property attached to this value can't accept the
specified unit or the string value, the value will be unchanged and
a DOMException will be raised.
\begin{description}
\item[{stringType}] \leavevmode 
a string code as defined above. The string code can only be a
string unit type (i.e. CSS{\_}STRING, CSS{\_}URI, CSS{\_}IDENT, and
CSS{\_}ATTR).

\item[{stringValue}] \leavevmode 
the new string value
Only the actual value is expected so for (CSS{\_}URI, ``a'') the
new value will be \texttt{url(a)}. For (CSS{\_}STRING, ``'a''')
the new value will be \texttt{"{\textbackslash}'a{\textbackslash}'"} as the surrounding \texttt{'} are
not part of the string value

\item[{raises}] \leavevmode 
DOMException
\begin{itemize}
\item {} 
INVALID{\_}ACCESS{\_}ERR: Raised if the CSS value doesn't contain a
string value or if the string value can't be converted into
the specified unit.

\item {} 
NO{\_}MODIFICATION{\_}ALLOWED{\_}ERR: Raised if this property is readonly.

\end{itemize}

\end{description}
\setlength{\parskip}{1ex}
    \end{boxedminipage}

    \label{cssutils:css:cssvalue:CSSPrimitiveValue:getCounterValue}
    \index{cssutils \textit{(package)}!cssutils.css \textit{(package)}!cssutils.css.cssvalue \textit{(module)}!cssutils.css.cssvalue.CSSPrimitiveValue \textit{(class)}!cssutils.css.cssvalue.CSSPrimitiveValue.getCounterValue \textit{(method)}}

    \vspace{0.5ex}

\hspace{.8\funcindent}\begin{boxedminipage}{\funcwidth}

    \raggedright \textbf{getCounterValue}(\textit{self})

    \vspace{-1.5ex}

    \rule{\textwidth}{0.5\fboxrule}
\setlength{\parskip}{2ex}

(DOM method) This method is used to get the Counter value. If
this CSS value doesn't contain a counter value, a DOMException
is raised. Modification to the corresponding style property
can be achieved using the Counter interface.
\setlength{\parskip}{1ex}
    \end{boxedminipage}

    \label{cssutils:css:cssvalue:CSSPrimitiveValue:getRGBColorValue}
    \index{cssutils \textit{(package)}!cssutils.css \textit{(package)}!cssutils.css.cssvalue \textit{(module)}!cssutils.css.cssvalue.CSSPrimitiveValue \textit{(class)}!cssutils.css.cssvalue.CSSPrimitiveValue.getRGBColorValue \textit{(method)}}

    \vspace{0.5ex}

\hspace{.8\funcindent}\begin{boxedminipage}{\funcwidth}

    \raggedright \textbf{getRGBColorValue}(\textit{self})

    \vspace{-1.5ex}

    \rule{\textwidth}{0.5\fboxrule}
\setlength{\parskip}{2ex}

(DOM method) This method is used to get the RGB color. If this
CSS value doesn't contain a RGB color value, a DOMException
is raised. Modification to the corresponding style property
can be achieved using the RGBColor interface.
\setlength{\parskip}{1ex}
    \end{boxedminipage}

    \label{cssutils:css:cssvalue:CSSPrimitiveValue:getRectValue}
    \index{cssutils \textit{(package)}!cssutils.css \textit{(package)}!cssutils.css.cssvalue \textit{(module)}!cssutils.css.cssvalue.CSSPrimitiveValue \textit{(class)}!cssutils.css.cssvalue.CSSPrimitiveValue.getRectValue \textit{(method)}}

    \vspace{0.5ex}

\hspace{.8\funcindent}\begin{boxedminipage}{\funcwidth}

    \raggedright \textbf{getRectValue}(\textit{self})

    \vspace{-1.5ex}

    \rule{\textwidth}{0.5\fboxrule}
\setlength{\parskip}{2ex}

(DOM method) This method is used to get the Rect value. If this CSS
value doesn't contain a rect value, a DOMException is raised.
Modification to the corresponding style property can be achieved
using the Rect interface.
\setlength{\parskip}{1ex}
    \end{boxedminipage}

    \vspace{0.5ex}

\hspace{.8\funcindent}\begin{boxedminipage}{\funcwidth}

    \raggedright \textbf{\_\_str\_\_}(\textit{self})

\setlength{\parskip}{2ex}
    str(x)

\setlength{\parskip}{1ex}
      Overrides: object.\_\_str\_\_ 	extit{(inherited documentation)}

    \end{boxedminipage}


\large{\textbf{\textit{Inherited from cssutils.css.cssvalue.CSSValue\textit{(Section \ref{cssutils:css:cssvalue:CSSValue})}}}}

\begin{quote}
\_\_repr\_\_()
\end{quote}

\large{\textbf{\textit{Inherited from object}}}

\begin{quote}
\_\_delattr\_\_(), \_\_getattribute\_\_(), \_\_hash\_\_(), \_\_new\_\_(), \_\_reduce\_\_(), \_\_reduce\_ex\_\_(), \_\_setattr\_\_()
\end{quote}

%%%%%%%%%%%%%%%%%%%%%%%%%%%%%%%%%%%%%%%%%%%%%%%%%%%%%%%%%%%%%%%%%%%%%%%%%%%
%%                              Properties                               %%
%%%%%%%%%%%%%%%%%%%%%%%%%%%%%%%%%%%%%%%%%%%%%%%%%%%%%%%%%%%%%%%%%%%%%%%%%%%

  \subsubsection{Properties}

    \vspace{-1cm}
\hspace{\varindent}\begin{longtable}{|p{\varnamewidth}|p{\vardescrwidth}|l}
\cline{1-2}
\cline{1-2} \centering \textbf{Name} & \centering \textbf{Description}& \\
\cline{1-2}
\endhead\cline{1-2}\multicolumn{3}{r}{\small\textit{continued on next page}}\\\endfoot\cline{1-2}
\endlastfoot\raggedright p\-r\-i\-m\-i\-t\-i\-v\-e\-T\-y\-p\-e\- & \raggedright READONLY: The type of the value as defined by the constants specified above.&\\
\cline{1-2}
\raggedright p\-r\-i\-m\-i\-t\-i\-v\-e\-T\-y\-p\-e\-S\-t\-r\-i\-n\-g\- & \raggedright Name of primitive type of this value.&\\
\cline{1-2}
\multicolumn{2}{|l|}{\textit{Inherited from cssutils.css.cssvalue.CSSValue \textit{(Section \ref{cssutils:css:cssvalue:CSSValue})}}}\\
\multicolumn{2}{|p{\varwidth}|}{\raggedright cssText, cssValueTypeString}\\
\cline{1-2}
\multicolumn{2}{|l|}{\textit{Inherited from object}}\\
\multicolumn{2}{|p{\varwidth}|}{\raggedright \_\_class\_\_}\\
\cline{1-2}
\end{longtable}


%%%%%%%%%%%%%%%%%%%%%%%%%%%%%%%%%%%%%%%%%%%%%%%%%%%%%%%%%%%%%%%%%%%%%%%%%%%
%%                            Class Variables                            %%
%%%%%%%%%%%%%%%%%%%%%%%%%%%%%%%%%%%%%%%%%%%%%%%%%%%%%%%%%%%%%%%%%%%%%%%%%%%

  \subsubsection{Class Variables}

    \vspace{-1cm}
\hspace{\varindent}\begin{longtable}{|p{\varnamewidth}|p{\vardescrwidth}|l}
\cline{1-2}
\cline{1-2} \centering \textbf{Name} & \centering \textbf{Description}& \\
\cline{1-2}
\endhead\cline{1-2}\multicolumn{3}{r}{\small\textit{continued on next page}}\\\endfoot\cline{1-2}
\endlastfoot\raggedright c\-s\-s\-V\-a\-l\-u\-e\-T\-y\-p\-e\- & \raggedright \textbf{Value:} 
{\tt 1}&\\
\cline{1-2}
\raggedright C\-S\-S\-\_\-U\-N\-K\-N\-O\-W\-N\- & \raggedright \textbf{Value:} 
{\tt 0}&\\
\cline{1-2}
\raggedright C\-S\-S\-\_\-N\-U\-M\-B\-E\-R\- & \raggedright \textbf{Value:} 
{\tt 1}&\\
\cline{1-2}
\raggedright C\-S\-S\-\_\-P\-E\-R\-C\-E\-N\-T\-A\-G\-E\- & \raggedright \textbf{Value:} 
{\tt 2}&\\
\cline{1-2}
\raggedright C\-S\-S\-\_\-E\-M\-S\- & \raggedright \textbf{Value:} 
{\tt 3}&\\
\cline{1-2}
\raggedright C\-S\-S\-\_\-E\-X\-S\- & \raggedright \textbf{Value:} 
{\tt 4}&\\
\cline{1-2}
\raggedright C\-S\-S\-\_\-P\-X\- & \raggedright \textbf{Value:} 
{\tt 5}&\\
\cline{1-2}
\raggedright C\-S\-S\-\_\-C\-M\- & \raggedright \textbf{Value:} 
{\tt 6}&\\
\cline{1-2}
\raggedright C\-S\-S\-\_\-M\-M\- & \raggedright \textbf{Value:} 
{\tt 7}&\\
\cline{1-2}
\raggedright C\-S\-S\-\_\-I\-N\- & \raggedright \textbf{Value:} 
{\tt 8}&\\
\cline{1-2}
\raggedright C\-S\-S\-\_\-P\-T\- & \raggedright \textbf{Value:} 
{\tt 9}&\\
\cline{1-2}
\raggedright C\-S\-S\-\_\-P\-C\- & \raggedright \textbf{Value:} 
{\tt 10}&\\
\cline{1-2}
\raggedright C\-S\-S\-\_\-D\-E\-G\- & \raggedright \textbf{Value:} 
{\tt 11}&\\
\cline{1-2}
\raggedright C\-S\-S\-\_\-R\-A\-D\- & \raggedright \textbf{Value:} 
{\tt 12}&\\
\cline{1-2}
\raggedright C\-S\-S\-\_\-G\-R\-A\-D\- & \raggedright \textbf{Value:} 
{\tt 13}&\\
\cline{1-2}
\raggedright C\-S\-S\-\_\-M\-S\- & \raggedright \textbf{Value:} 
{\tt 14}&\\
\cline{1-2}
\raggedright C\-S\-S\-\_\-S\- & \raggedright \textbf{Value:} 
{\tt 15}&\\
\cline{1-2}
\raggedright C\-S\-S\-\_\-H\-Z\- & \raggedright \textbf{Value:} 
{\tt 16}&\\
\cline{1-2}
\raggedright C\-S\-S\-\_\-K\-H\-Z\- & \raggedright \textbf{Value:} 
{\tt 17}&\\
\cline{1-2}
\raggedright C\-S\-S\-\_\-D\-I\-M\-E\-N\-S\-I\-O\-N\- & \raggedright \textbf{Value:} 
{\tt 18}&\\
\cline{1-2}
\raggedright C\-S\-S\-\_\-S\-T\-R\-I\-N\-G\- & \raggedright \textbf{Value:} 
{\tt 19}&\\
\cline{1-2}
\raggedright C\-S\-S\-\_\-U\-R\-I\- & \raggedright \textbf{Value:} 
{\tt 20}&\\
\cline{1-2}
\raggedright C\-S\-S\-\_\-I\-D\-E\-N\-T\- & \raggedright \textbf{Value:} 
{\tt 21}&\\
\cline{1-2}
\raggedright C\-S\-S\-\_\-A\-T\-T\-R\- & \raggedright \textbf{Value:} 
{\tt 22}&\\
\cline{1-2}
\raggedright C\-S\-S\-\_\-C\-O\-U\-N\-T\-E\-R\- & \raggedright \textbf{Value:} 
{\tt 23}&\\
\cline{1-2}
\raggedright C\-S\-S\-\_\-R\-E\-C\-T\- & \raggedright \textbf{Value:} 
{\tt 24}&\\
\cline{1-2}
\raggedright C\-S\-S\-\_\-R\-G\-B\-C\-O\-L\-O\-R\- & \raggedright \textbf{Value:} 
{\tt 25}&\\
\cline{1-2}
\raggedright C\-S\-S\-\_\-R\-G\-B\-A\-C\-O\-L\-O\-R\- & \raggedright \textbf{Value:} 
{\tt 26}&\\
\cline{1-2}
\multicolumn{2}{|l|}{\textit{Inherited from cssutils.css.cssvalue.CSSValue \textit{(Section \ref{cssutils:css:cssvalue:CSSValue})}}}\\
\multicolumn{2}{|p{\varwidth}|}{\raggedright CSS\_CUSTOM, CSS\_INHERIT, CSS\_PRIMITIVE\_VALUE, CSS\_VALUE\_LIST}\\
\cline{1-2}
\end{longtable}

    \index{cssutils \textit{(package)}!cssutils.css \textit{(package)}!cssutils.css.cssvalue \textit{(module)}!cssutils.css.cssvalue.CSSPrimitiveValue \textit{(class)}|)}
    \index{cssutils \textit{(package)}!cssutils.css \textit{(package)}|)}
