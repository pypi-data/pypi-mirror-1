%
% API Documentation for cssutils
% Module cssutils.css.cssvalue
%
% Generated by epydoc 3.0.1
% [Fri Feb 01 19:05:21 2008]
%

%%%%%%%%%%%%%%%%%%%%%%%%%%%%%%%%%%%%%%%%%%%%%%%%%%%%%%%%%%%%%%%%%%%%%%%%%%%
%%                          Module Description                           %%
%%%%%%%%%%%%%%%%%%%%%%%%%%%%%%%%%%%%%%%%%%%%%%%%%%%%%%%%%%%%%%%%%%%%%%%%%%%

    \index{cssutils \textit{(package)}!cssutils.css \textit{(package)}!cssutils.css.cssvalue \textit{(module)}|(}
\section{Module cssutils.css.cssvalue}

    \label{cssutils:css:cssvalue}

CSSValue related classes
\begin{itemize}
\item {} 
CSSValue implements DOM Level 2 CSS CSSValue

\item {} 
CSSPrimitiveValue implements DOM Level 2 CSS CSSPrimitiveValue

\item {} 
CSSValueList implements DOM Level 2 CSS CSSValueList

\end{itemize}
\textbf{Version:} \$LastChangedRevision: 837 \$



\textbf{Date:} \$LastChangedDate: 2008-01-13 00:06:00 +0100 (So, 13 Jan 2008) \$



\textbf{Author:} \$LastChangedBy: cthedot \$




%%%%%%%%%%%%%%%%%%%%%%%%%%%%%%%%%%%%%%%%%%%%%%%%%%%%%%%%%%%%%%%%%%%%%%%%%%%
%%                           Class Description                           %%
%%%%%%%%%%%%%%%%%%%%%%%%%%%%%%%%%%%%%%%%%%%%%%%%%%%%%%%%%%%%%%%%%%%%%%%%%%%

    \index{cssutils \textit{(package)}!cssutils.css \textit{(package)}!cssutils.css.cssvalue \textit{(module)}!cssutils.css.cssvalue.CSSValue \textit{(class)}|(}
\subsection{Class CSSValue}

    \label{cssutils:css:cssvalue:CSSValue}
\begin{tabular}{cccccccc}
% Line for object, linespec=[False, False]
\multicolumn{2}{r}{\settowidth{\BCL}{object}\multirow{2}{\BCL}{object}}
&&
&&
  \\\cline{3-3}
  &&\multicolumn{1}{c|}{}
&&
&&
  \\
% Line for cssutils.util.Base, linespec=[False]
\multicolumn{4}{r}{\settowidth{\BCL}{cssutils.util.Base}\multirow{2}{\BCL}{cssutils.util.Base}}
&&
  \\\cline{5-5}
  &&&&\multicolumn{1}{c|}{}
&&
  \\
&&&&\multicolumn{2}{l}{\textbf{cssutils.css.cssvalue.CSSValue}}
\end{tabular}

\textbf{Known Subclasses:}
cssutils.css.cssvalue.CSSPrimitiveValue,
    cssutils.css.cssvalue.CSSValueList


The CSSValue interface represents a simple or a complex value.
A CSSValue object only occurs in a context of a CSS property


%___________________________________________________________________________

\hypertarget{properties}{}
\pdfbookmark[3]{Properties}{properties}
\paragraph*{Properties}
\label{properties}
\begin{description}
\item[{cssText}] \leavevmode 
A string representation of the current value.

\item[{cssValueType}] \leavevmode 
A (readonly) code defining the type of the value.

\item[{seq: a list (cssutils)}] \leavevmode 
All parts of this style declaration including CSSComments

\item[{valid: boolean}] \leavevmode 
if the value is valid at all, False for e.g. color: {\#}1

\item[{wellformed}] \leavevmode 
if this Property is syntactically ok

\item[{{\_}value (INTERNAL!)}] \leavevmode 
value without any comments, used to validate

\end{description}

%%%%%%%%%%%%%%%%%%%%%%%%%%%%%%%%%%%%%%%%%%%%%%%%%%%%%%%%%%%%%%%%%%%%%%%%%%%
%%                                Methods                                %%
%%%%%%%%%%%%%%%%%%%%%%%%%%%%%%%%%%%%%%%%%%%%%%%%%%%%%%%%%%%%%%%%%%%%%%%%%%%

  \subsubsection{Methods}

    \vspace{0.5ex}

\hspace{.8\funcindent}\begin{boxedminipage}{\funcwidth}

    \raggedright \textbf{\_\_init\_\_}(\textit{self}, \textit{cssText}={\tt None}, \textit{readonly}={\tt False}, \textit{\_propertyName}={\tt None})

    \vspace{-1.5ex}

    \rule{\textwidth}{0.5\fboxrule}
\setlength{\parskip}{2ex}

inits a new CSS Value
\begin{description}
\item[{cssText}] \leavevmode 
the parsable cssText of the value

\item[{readonly}] \leavevmode 
defaults to False

\item[{property}] \leavevmode 
used to validate this value in the context of a property

\end{description}
\setlength{\parskip}{1ex}
      Overrides: object.\_\_init\_\_

    \end{boxedminipage}

    \vspace{0.5ex}

\hspace{.8\funcindent}\begin{boxedminipage}{\funcwidth}

    \raggedright \textbf{\_\_repr\_\_}(\textit{self})

\setlength{\parskip}{2ex}
    repr(x)

\setlength{\parskip}{1ex}
      Overrides: object.\_\_repr\_\_ 	extit{(inherited documentation)}

    \end{boxedminipage}

    \vspace{0.5ex}

\hspace{.8\funcindent}\begin{boxedminipage}{\funcwidth}

    \raggedright \textbf{\_\_str\_\_}(\textit{self})

\setlength{\parskip}{2ex}
    str(x)

\setlength{\parskip}{1ex}
      Overrides: object.\_\_str\_\_ 	extit{(inherited documentation)}

    \end{boxedminipage}


\large{\textbf{\textit{Inherited from object}}}

\begin{quote}
\_\_delattr\_\_(), \_\_getattribute\_\_(), \_\_hash\_\_(), \_\_new\_\_(), \_\_reduce\_\_(), \_\_reduce\_ex\_\_(), \_\_setattr\_\_()
\end{quote}

%%%%%%%%%%%%%%%%%%%%%%%%%%%%%%%%%%%%%%%%%%%%%%%%%%%%%%%%%%%%%%%%%%%%%%%%%%%
%%                              Properties                               %%
%%%%%%%%%%%%%%%%%%%%%%%%%%%%%%%%%%%%%%%%%%%%%%%%%%%%%%%%%%%%%%%%%%%%%%%%%%%

  \subsubsection{Properties}

    \vspace{-1cm}
\hspace{\varindent}\begin{longtable}{|p{\varnamewidth}|p{\vardescrwidth}|l}
\cline{1-2}
\cline{1-2} \centering \textbf{Name} & \centering \textbf{Description}& \\
\cline{1-2}
\endhead\cline{1-2}\multicolumn{3}{r}{\small\textit{continued on next page}}\\\endfoot\cline{1-2}
\endlastfoot\raggedright c\-s\-s\-T\-e\-x\-t\- & \raggedright A string representation of the current value.&\\
\cline{1-2}
\raggedright c\-s\-s\-V\-a\-l\-u\-e\-T\-y\-p\-e\- & \raggedright A (readonly) code defining the type of the value as defined above.&\\
\cline{1-2}
\raggedright c\-s\-s\-V\-a\-l\-u\-e\-T\-y\-p\-e\-S\-t\-r\-i\-n\-g\- & \raggedright cssutils: Name of cssValueType of this CSSValue (readonly).&\\
\cline{1-2}
\multicolumn{2}{|l|}{\textit{Inherited from object}}\\
\multicolumn{2}{|p{\varwidth}|}{\raggedright \_\_class\_\_}\\
\cline{1-2}
\end{longtable}


%%%%%%%%%%%%%%%%%%%%%%%%%%%%%%%%%%%%%%%%%%%%%%%%%%%%%%%%%%%%%%%%%%%%%%%%%%%
%%                            Class Variables                            %%
%%%%%%%%%%%%%%%%%%%%%%%%%%%%%%%%%%%%%%%%%%%%%%%%%%%%%%%%%%%%%%%%%%%%%%%%%%%

  \subsubsection{Class Variables}

    \vspace{-1cm}
\hspace{\varindent}\begin{longtable}{|p{\varnamewidth}|p{\vardescrwidth}|l}
\cline{1-2}
\cline{1-2} \centering \textbf{Name} & \centering \textbf{Description}& \\
\cline{1-2}
\endhead\cline{1-2}\multicolumn{3}{r}{\small\textit{continued on next page}}\\\endfoot\cline{1-2}
\endlastfoot\raggedright C\-S\-S\-\_\-I\-N\-H\-E\-R\-I\-T\- & \raggedright The value is inherited and the cssText contains ``inherit''.

\textbf{Value:} 
{\tt 0}&\\
\cline{1-2}
\raggedright C\-S\-S\-\_\-P\-R\-I\-M\-I\-T\-I\-V\-E\-\_\-V\-A\-L\-U\-E\- & \raggedright The value is a primitive value and an instance of the
CSSPrimitiveValue interface can be obtained by using binding-specific
casting methods on this instance of the CSSValue interface.

\textbf{Value:} 
{\tt 1}&\\
\cline{1-2}
\raggedright C\-S\-S\-\_\-V\-A\-L\-U\-E\-\_\-L\-I\-S\-T\- & \raggedright The value is a CSSValue list and an instance of the CSSValueList
interface can be obtained by using binding-specific casting
methods on this instance of the CSSValue interface.

\textbf{Value:} 
{\tt 2}&\\
\cline{1-2}
\raggedright C\-S\-S\-\_\-C\-U\-S\-T\-O\-M\- & \raggedright The value is a custom value.

\textbf{Value:} 
{\tt 3}&\\
\cline{1-2}
\end{longtable}

    \index{cssutils \textit{(package)}!cssutils.css \textit{(package)}!cssutils.css.cssvalue \textit{(module)}!cssutils.css.cssvalue.CSSValue \textit{(class)}|)}

%%%%%%%%%%%%%%%%%%%%%%%%%%%%%%%%%%%%%%%%%%%%%%%%%%%%%%%%%%%%%%%%%%%%%%%%%%%
%%                           Class Description                           %%
%%%%%%%%%%%%%%%%%%%%%%%%%%%%%%%%%%%%%%%%%%%%%%%%%%%%%%%%%%%%%%%%%%%%%%%%%%%

    \index{cssutils \textit{(package)}!cssutils.css \textit{(package)}!cssutils.css.cssvalue \textit{(module)}!cssutils.css.cssvalue.CSSPrimitiveValue \textit{(class)}|(}
\subsection{Class CSSPrimitiveValue}

    \label{cssutils:css:cssvalue:CSSPrimitiveValue}
\begin{tabular}{cccccccccc}
% Line for object, linespec=[False, False, False]
\multicolumn{2}{r}{\settowidth{\BCL}{object}\multirow{2}{\BCL}{object}}
&&
&&
&&
  \\\cline{3-3}
  &&\multicolumn{1}{c|}{}
&&
&&
&&
  \\
% Line for cssutils.util.Base, linespec=[False, False]
\multicolumn{4}{r}{\settowidth{\BCL}{cssutils.util.Base}\multirow{2}{\BCL}{cssutils.util.Base}}
&&
&&
  \\\cline{5-5}
  &&&&\multicolumn{1}{c|}{}
&&
&&
  \\
% Line for cssutils.css.cssvalue.CSSValue, linespec=[False]
\multicolumn{6}{r}{\settowidth{\BCL}{cssutils.css.cssvalue.CSSValue}\multirow{2}{\BCL}{cssutils.css.cssvalue.CSSValue}}
&&
  \\\cline{7-7}
  &&&&&&\multicolumn{1}{c|}{}
&&
  \\
&&&&&&\multicolumn{2}{l}{\textbf{cssutils.css.cssvalue.CSSPrimitiveValue}}
\end{tabular}


represents a single CSS Value.  May be used to determine the value of a
specific style property currently set in a block or to set a specific
style property explicitly within the block. Might be obtained from the
getPropertyCSSValue method of CSSStyleDeclaration.

Conversions are allowed between absolute values (from millimeters to
centimeters, from degrees to radians, and so on) but not between
relative values. (For example, a pixel value cannot be converted to a
centimeter value.) Percentage values can't be converted since they are
relative to the parent value (or another property value). There is one
exception for color percentage values: since a color percentage value
is relative to the range 0-255, a color percentage value can be
converted to a number; (see also the RGBColor interface).

%%%%%%%%%%%%%%%%%%%%%%%%%%%%%%%%%%%%%%%%%%%%%%%%%%%%%%%%%%%%%%%%%%%%%%%%%%%
%%                                Methods                                %%
%%%%%%%%%%%%%%%%%%%%%%%%%%%%%%%%%%%%%%%%%%%%%%%%%%%%%%%%%%%%%%%%%%%%%%%%%%%

  \subsubsection{Methods}

    \vspace{0.5ex}

\hspace{.8\funcindent}\begin{boxedminipage}{\funcwidth}

    \raggedright \textbf{\_\_init\_\_}(\textit{self}, \textit{cssText}={\tt None}, \textit{readonly}={\tt False}, \textit{\_propertyName}={\tt None})

    \vspace{-1.5ex}

    \rule{\textwidth}{0.5\fboxrule}
\setlength{\parskip}{2ex}

see CSSPrimitiveValue.{\_}{\_}init{\_}{\_}()
\setlength{\parskip}{1ex}
      Overrides: object.\_\_init\_\_

    \end{boxedminipage}

    \label{cssutils:css:cssvalue:CSSPrimitiveValue:getFloatValue}
    \index{cssutils \textit{(package)}!cssutils.css \textit{(package)}!cssutils.css.cssvalue \textit{(module)}!cssutils.css.cssvalue.CSSPrimitiveValue \textit{(class)}!cssutils.css.cssvalue.CSSPrimitiveValue.getFloatValue \textit{(method)}}

    \vspace{0.5ex}

\hspace{.8\funcindent}\begin{boxedminipage}{\funcwidth}

    \raggedright \textbf{getFloatValue}(\textit{self}, \textit{unitType})

    \vspace{-1.5ex}

    \rule{\textwidth}{0.5\fboxrule}
\setlength{\parskip}{2ex}

(DOM method) This method is used to get a float value in a
specified unit. If this CSS value doesn't contain a float value
or can't be converted into the specified unit, a DOMException
is raised.
\begin{description}
\item[{unitType}] \leavevmode 
to get the float value. The unit code can only be a float unit type
(i.e. CSS{\_}NUMBER, CSS{\_}PERCENTAGE, CSS{\_}EMS, CSS{\_}EXS, CSS{\_}PX, CSS{\_}CM,
CSS{\_}MM, CSS{\_}IN, CSS{\_}PT, CSS{\_}PC, CSS{\_}DEG, CSS{\_}RAD, CSS{\_}GRAD, CSS{\_}MS,
CSS{\_}S, CSS{\_}HZ, CSS{\_}KHZ, CSS{\_}DIMENSION).

\end{description}

returns not necessarily a float but some cases just an int
e.g. if the value is \texttt{1px} it return \texttt{1} and \textbf{not} \texttt{1.0}

conversions might return strange values like 1.000000000001
\setlength{\parskip}{1ex}
    \end{boxedminipage}

    \label{cssutils:css:cssvalue:CSSPrimitiveValue:setFloatValue}
    \index{cssutils \textit{(package)}!cssutils.css \textit{(package)}!cssutils.css.cssvalue \textit{(module)}!cssutils.css.cssvalue.CSSPrimitiveValue \textit{(class)}!cssutils.css.cssvalue.CSSPrimitiveValue.setFloatValue \textit{(method)}}

    \vspace{0.5ex}

\hspace{.8\funcindent}\begin{boxedminipage}{\funcwidth}

    \raggedright \textbf{setFloatValue}(\textit{self}, \textit{unitType}, \textit{floatValue})

    \vspace{-1.5ex}

    \rule{\textwidth}{0.5\fboxrule}
\setlength{\parskip}{2ex}

(DOM method) A method to set the float value with a specified unit.
If the property attached with this value can not accept the
specified unit or the float value, the value will be unchanged and
a DOMException will be raised.
\begin{description}
\item[{unitType}] \leavevmode 
a unit code as defined above. The unit code can only be a float
unit type

\item[{floatValue}] \leavevmode 
the new float value which does not have to be a float value but
may simple be an int e.g. if setting:
\begin{quote}{\ttfamily \raggedright \noindent
setFloatValue(CSS{\_}PX,~1)
}\end{quote}

\item[{raises DOMException}] \leavevmode \begin{itemize}
\item {} \begin{description}
\item[{INVALID{\_}ACCESS{\_}ERR: Raised if the attached property doesn't}] \leavevmode 
support the float value or the unit type.

\end{description}

\item {} 
NO{\_}MODIFICATION{\_}ALLOWED{\_}ERR: Raised if this property is readonly.

\end{itemize}

\end{description}
\setlength{\parskip}{1ex}
    \end{boxedminipage}

    \label{cssutils:css:cssvalue:CSSPrimitiveValue:getStringValue}
    \index{cssutils \textit{(package)}!cssutils.css \textit{(package)}!cssutils.css.cssvalue \textit{(module)}!cssutils.css.cssvalue.CSSPrimitiveValue \textit{(class)}!cssutils.css.cssvalue.CSSPrimitiveValue.getStringValue \textit{(method)}}

    \vspace{0.5ex}

\hspace{.8\funcindent}\begin{boxedminipage}{\funcwidth}

    \raggedright \textbf{getStringValue}(\textit{self})

    \vspace{-1.5ex}

    \rule{\textwidth}{0.5\fboxrule}
\setlength{\parskip}{2ex}

(DOM method) This method is used to get the string value. If the
CSS value doesn't contain a string value, a DOMException is raised.

Some properties (like 'font-family' or 'voice-family')
convert a whitespace separated list of idents to a string.

Only the actual value is returned so e.g. all the following return the
actual value \texttt{a}: url(a), attr(a), ``a'', 'a'
\setlength{\parskip}{1ex}
    \end{boxedminipage}

    \label{cssutils:css:cssvalue:CSSPrimitiveValue:setStringValue}
    \index{cssutils \textit{(package)}!cssutils.css \textit{(package)}!cssutils.css.cssvalue \textit{(module)}!cssutils.css.cssvalue.CSSPrimitiveValue \textit{(class)}!cssutils.css.cssvalue.CSSPrimitiveValue.setStringValue \textit{(method)}}

    \vspace{0.5ex}

\hspace{.8\funcindent}\begin{boxedminipage}{\funcwidth}

    \raggedright \textbf{setStringValue}(\textit{self}, \textit{stringType}, \textit{stringValue})

    \vspace{-1.5ex}

    \rule{\textwidth}{0.5\fboxrule}
\setlength{\parskip}{2ex}

(DOM method) A method to set the string value with the specified
unit. If the property attached to this value can't accept the
specified unit or the string value, the value will be unchanged and
a DOMException will be raised.
\begin{description}
\item[{stringType}] \leavevmode 
a string code as defined above. The string code can only be a
string unit type (i.e. CSS{\_}STRING, CSS{\_}URI, CSS{\_}IDENT, and
CSS{\_}ATTR).

\item[{stringValue}] \leavevmode 
the new string value
Only the actual value is expected so for (CSS{\_}URI, ``a'') the
new value will be \texttt{url(a)}. For (CSS{\_}STRING, ``'a''')
the new value will be \texttt{"{\textbackslash}'a{\textbackslash}'"} as the surrounding \texttt{'} are
not part of the string value

\item[{raises}] \leavevmode 
DOMException
\begin{itemize}
\item {} 
INVALID{\_}ACCESS{\_}ERR: Raised if the CSS value doesn't contain a
string value or if the string value can't be converted into
the specified unit.

\item {} 
NO{\_}MODIFICATION{\_}ALLOWED{\_}ERR: Raised if this property is readonly.

\end{itemize}

\end{description}
\setlength{\parskip}{1ex}
    \end{boxedminipage}

    \label{cssutils:css:cssvalue:CSSPrimitiveValue:getCounterValue}
    \index{cssutils \textit{(package)}!cssutils.css \textit{(package)}!cssutils.css.cssvalue \textit{(module)}!cssutils.css.cssvalue.CSSPrimitiveValue \textit{(class)}!cssutils.css.cssvalue.CSSPrimitiveValue.getCounterValue \textit{(method)}}

    \vspace{0.5ex}

\hspace{.8\funcindent}\begin{boxedminipage}{\funcwidth}

    \raggedright \textbf{getCounterValue}(\textit{self})

    \vspace{-1.5ex}

    \rule{\textwidth}{0.5\fboxrule}
\setlength{\parskip}{2ex}

(DOM method) This method is used to get the Counter value. If
this CSS value doesn't contain a counter value, a DOMException
is raised. Modification to the corresponding style property
can be achieved using the Counter interface.
\setlength{\parskip}{1ex}
    \end{boxedminipage}

    \label{cssutils:css:cssvalue:CSSPrimitiveValue:getRGBColorValue}
    \index{cssutils \textit{(package)}!cssutils.css \textit{(package)}!cssutils.css.cssvalue \textit{(module)}!cssutils.css.cssvalue.CSSPrimitiveValue \textit{(class)}!cssutils.css.cssvalue.CSSPrimitiveValue.getRGBColorValue \textit{(method)}}

    \vspace{0.5ex}

\hspace{.8\funcindent}\begin{boxedminipage}{\funcwidth}

    \raggedright \textbf{getRGBColorValue}(\textit{self})

    \vspace{-1.5ex}

    \rule{\textwidth}{0.5\fboxrule}
\setlength{\parskip}{2ex}

(DOM method) This method is used to get the RGB color. If this
CSS value doesn't contain a RGB color value, a DOMException
is raised. Modification to the corresponding style property
can be achieved using the RGBColor interface.
\setlength{\parskip}{1ex}
    \end{boxedminipage}

    \label{cssutils:css:cssvalue:CSSPrimitiveValue:getRectValue}
    \index{cssutils \textit{(package)}!cssutils.css \textit{(package)}!cssutils.css.cssvalue \textit{(module)}!cssutils.css.cssvalue.CSSPrimitiveValue \textit{(class)}!cssutils.css.cssvalue.CSSPrimitiveValue.getRectValue \textit{(method)}}

    \vspace{0.5ex}

\hspace{.8\funcindent}\begin{boxedminipage}{\funcwidth}

    \raggedright \textbf{getRectValue}(\textit{self})

    \vspace{-1.5ex}

    \rule{\textwidth}{0.5\fboxrule}
\setlength{\parskip}{2ex}

(DOM method) This method is used to get the Rect value. If this CSS
value doesn't contain a rect value, a DOMException is raised.
Modification to the corresponding style property can be achieved
using the Rect interface.
\setlength{\parskip}{1ex}
    \end{boxedminipage}

    \vspace{0.5ex}

\hspace{.8\funcindent}\begin{boxedminipage}{\funcwidth}

    \raggedright \textbf{\_\_str\_\_}(\textit{self})

\setlength{\parskip}{2ex}
    str(x)

\setlength{\parskip}{1ex}
      Overrides: object.\_\_str\_\_ 	extit{(inherited documentation)}

    \end{boxedminipage}


\large{\textbf{\textit{Inherited from cssutils.css.cssvalue.CSSValue\textit{(Section \ref{cssutils:css:cssvalue:CSSValue})}}}}

\begin{quote}
\_\_repr\_\_()
\end{quote}

\large{\textbf{\textit{Inherited from object}}}

\begin{quote}
\_\_delattr\_\_(), \_\_getattribute\_\_(), \_\_hash\_\_(), \_\_new\_\_(), \_\_reduce\_\_(), \_\_reduce\_ex\_\_(), \_\_setattr\_\_()
\end{quote}

%%%%%%%%%%%%%%%%%%%%%%%%%%%%%%%%%%%%%%%%%%%%%%%%%%%%%%%%%%%%%%%%%%%%%%%%%%%
%%                              Properties                               %%
%%%%%%%%%%%%%%%%%%%%%%%%%%%%%%%%%%%%%%%%%%%%%%%%%%%%%%%%%%%%%%%%%%%%%%%%%%%

  \subsubsection{Properties}

    \vspace{-1cm}
\hspace{\varindent}\begin{longtable}{|p{\varnamewidth}|p{\vardescrwidth}|l}
\cline{1-2}
\cline{1-2} \centering \textbf{Name} & \centering \textbf{Description}& \\
\cline{1-2}
\endhead\cline{1-2}\multicolumn{3}{r}{\small\textit{continued on next page}}\\\endfoot\cline{1-2}
\endlastfoot\raggedright p\-r\-i\-m\-i\-t\-i\-v\-e\-T\-y\-p\-e\- & \raggedright READONLY: The type of the value as defined by the constants specified above.&\\
\cline{1-2}
\raggedright p\-r\-i\-m\-i\-t\-i\-v\-e\-T\-y\-p\-e\-S\-t\-r\-i\-n\-g\- & \raggedright Name of primitive type of this value.&\\
\cline{1-2}
\multicolumn{2}{|l|}{\textit{Inherited from cssutils.css.cssvalue.CSSValue \textit{(Section \ref{cssutils:css:cssvalue:CSSValue})}}}\\
\multicolumn{2}{|p{\varwidth}|}{\raggedright cssText, cssValueTypeString}\\
\cline{1-2}
\multicolumn{2}{|l|}{\textit{Inherited from object}}\\
\multicolumn{2}{|p{\varwidth}|}{\raggedright \_\_class\_\_}\\
\cline{1-2}
\end{longtable}


%%%%%%%%%%%%%%%%%%%%%%%%%%%%%%%%%%%%%%%%%%%%%%%%%%%%%%%%%%%%%%%%%%%%%%%%%%%
%%                            Class Variables                            %%
%%%%%%%%%%%%%%%%%%%%%%%%%%%%%%%%%%%%%%%%%%%%%%%%%%%%%%%%%%%%%%%%%%%%%%%%%%%

  \subsubsection{Class Variables}

    \vspace{-1cm}
\hspace{\varindent}\begin{longtable}{|p{\varnamewidth}|p{\vardescrwidth}|l}
\cline{1-2}
\cline{1-2} \centering \textbf{Name} & \centering \textbf{Description}& \\
\cline{1-2}
\endhead\cline{1-2}\multicolumn{3}{r}{\small\textit{continued on next page}}\\\endfoot\cline{1-2}
\endlastfoot\raggedright c\-s\-s\-V\-a\-l\-u\-e\-T\-y\-p\-e\- & \raggedright \textbf{Value:} 
{\tt 1}&\\
\cline{1-2}
\raggedright C\-S\-S\-\_\-U\-N\-K\-N\-O\-W\-N\- & \raggedright \textbf{Value:} 
{\tt 0}&\\
\cline{1-2}
\raggedright C\-S\-S\-\_\-N\-U\-M\-B\-E\-R\- & \raggedright \textbf{Value:} 
{\tt 1}&\\
\cline{1-2}
\raggedright C\-S\-S\-\_\-P\-E\-R\-C\-E\-N\-T\-A\-G\-E\- & \raggedright \textbf{Value:} 
{\tt 2}&\\
\cline{1-2}
\raggedright C\-S\-S\-\_\-E\-M\-S\- & \raggedright \textbf{Value:} 
{\tt 3}&\\
\cline{1-2}
\raggedright C\-S\-S\-\_\-E\-X\-S\- & \raggedright \textbf{Value:} 
{\tt 4}&\\
\cline{1-2}
\raggedright C\-S\-S\-\_\-P\-X\- & \raggedright \textbf{Value:} 
{\tt 5}&\\
\cline{1-2}
\raggedright C\-S\-S\-\_\-C\-M\- & \raggedright \textbf{Value:} 
{\tt 6}&\\
\cline{1-2}
\raggedright C\-S\-S\-\_\-M\-M\- & \raggedright \textbf{Value:} 
{\tt 7}&\\
\cline{1-2}
\raggedright C\-S\-S\-\_\-I\-N\- & \raggedright \textbf{Value:} 
{\tt 8}&\\
\cline{1-2}
\raggedright C\-S\-S\-\_\-P\-T\- & \raggedright \textbf{Value:} 
{\tt 9}&\\
\cline{1-2}
\raggedright C\-S\-S\-\_\-P\-C\- & \raggedright \textbf{Value:} 
{\tt 10}&\\
\cline{1-2}
\raggedright C\-S\-S\-\_\-D\-E\-G\- & \raggedright \textbf{Value:} 
{\tt 11}&\\
\cline{1-2}
\raggedright C\-S\-S\-\_\-R\-A\-D\- & \raggedright \textbf{Value:} 
{\tt 12}&\\
\cline{1-2}
\raggedright C\-S\-S\-\_\-G\-R\-A\-D\- & \raggedright \textbf{Value:} 
{\tt 13}&\\
\cline{1-2}
\raggedright C\-S\-S\-\_\-M\-S\- & \raggedright \textbf{Value:} 
{\tt 14}&\\
\cline{1-2}
\raggedright C\-S\-S\-\_\-S\- & \raggedright \textbf{Value:} 
{\tt 15}&\\
\cline{1-2}
\raggedright C\-S\-S\-\_\-H\-Z\- & \raggedright \textbf{Value:} 
{\tt 16}&\\
\cline{1-2}
\raggedright C\-S\-S\-\_\-K\-H\-Z\- & \raggedright \textbf{Value:} 
{\tt 17}&\\
\cline{1-2}
\raggedright C\-S\-S\-\_\-D\-I\-M\-E\-N\-S\-I\-O\-N\- & \raggedright \textbf{Value:} 
{\tt 18}&\\
\cline{1-2}
\raggedright C\-S\-S\-\_\-S\-T\-R\-I\-N\-G\- & \raggedright \textbf{Value:} 
{\tt 19}&\\
\cline{1-2}
\raggedright C\-S\-S\-\_\-U\-R\-I\- & \raggedright \textbf{Value:} 
{\tt 20}&\\
\cline{1-2}
\raggedright C\-S\-S\-\_\-I\-D\-E\-N\-T\- & \raggedright \textbf{Value:} 
{\tt 21}&\\
\cline{1-2}
\raggedright C\-S\-S\-\_\-A\-T\-T\-R\- & \raggedright \textbf{Value:} 
{\tt 22}&\\
\cline{1-2}
\raggedright C\-S\-S\-\_\-C\-O\-U\-N\-T\-E\-R\- & \raggedright \textbf{Value:} 
{\tt 23}&\\
\cline{1-2}
\raggedright C\-S\-S\-\_\-R\-E\-C\-T\- & \raggedright \textbf{Value:} 
{\tt 24}&\\
\cline{1-2}
\raggedright C\-S\-S\-\_\-R\-G\-B\-C\-O\-L\-O\-R\- & \raggedright \textbf{Value:} 
{\tt 25}&\\
\cline{1-2}
\raggedright C\-S\-S\-\_\-R\-G\-B\-A\-C\-O\-L\-O\-R\- & \raggedright \textbf{Value:} 
{\tt 26}&\\
\cline{1-2}
\multicolumn{2}{|l|}{\textit{Inherited from cssutils.css.cssvalue.CSSValue \textit{(Section \ref{cssutils:css:cssvalue:CSSValue})}}}\\
\multicolumn{2}{|p{\varwidth}|}{\raggedright CSS\_CUSTOM, CSS\_INHERIT, CSS\_PRIMITIVE\_VALUE, CSS\_VALUE\_LIST}\\
\cline{1-2}
\end{longtable}

    \index{cssutils \textit{(package)}!cssutils.css \textit{(package)}!cssutils.css.cssvalue \textit{(module)}!cssutils.css.cssvalue.CSSPrimitiveValue \textit{(class)}|)}

%%%%%%%%%%%%%%%%%%%%%%%%%%%%%%%%%%%%%%%%%%%%%%%%%%%%%%%%%%%%%%%%%%%%%%%%%%%
%%                           Class Description                           %%
%%%%%%%%%%%%%%%%%%%%%%%%%%%%%%%%%%%%%%%%%%%%%%%%%%%%%%%%%%%%%%%%%%%%%%%%%%%

    \index{cssutils \textit{(package)}!cssutils.css \textit{(package)}!cssutils.css.cssvalue \textit{(module)}!cssutils.css.cssvalue.CSSValueList \textit{(class)}|(}
\subsection{Class CSSValueList}

    \label{cssutils:css:cssvalue:CSSValueList}
\begin{tabular}{cccccccccc}
% Line for object, linespec=[False, False, False]
\multicolumn{2}{r}{\settowidth{\BCL}{object}\multirow{2}{\BCL}{object}}
&&
&&
&&
  \\\cline{3-3}
  &&\multicolumn{1}{c|}{}
&&
&&
&&
  \\
% Line for cssutils.util.Base, linespec=[False, False]
\multicolumn{4}{r}{\settowidth{\BCL}{cssutils.util.Base}\multirow{2}{\BCL}{cssutils.util.Base}}
&&
&&
  \\\cline{5-5}
  &&&&\multicolumn{1}{c|}{}
&&
&&
  \\
% Line for cssutils.css.cssvalue.CSSValue, linespec=[False]
\multicolumn{6}{r}{\settowidth{\BCL}{cssutils.css.cssvalue.CSSValue}\multirow{2}{\BCL}{cssutils.css.cssvalue.CSSValue}}
&&
  \\\cline{7-7}
  &&&&&&\multicolumn{1}{c|}{}
&&
  \\
&&&&&&\multicolumn{2}{l}{\textbf{cssutils.css.cssvalue.CSSValueList}}
\end{tabular}


The CSSValueList interface provides the abstraction of an ordered
collection of CSS values.

Some properties allow an empty list into their syntax. In that case,
these properties take the none identifier. So, an empty list means
that the property has the value none.

The items in the CSSValueList are accessible via an integral index,
starting from 0.

%%%%%%%%%%%%%%%%%%%%%%%%%%%%%%%%%%%%%%%%%%%%%%%%%%%%%%%%%%%%%%%%%%%%%%%%%%%
%%                                Methods                                %%
%%%%%%%%%%%%%%%%%%%%%%%%%%%%%%%%%%%%%%%%%%%%%%%%%%%%%%%%%%%%%%%%%%%%%%%%%%%

  \subsubsection{Methods}

    \vspace{0.5ex}

\hspace{.8\funcindent}\begin{boxedminipage}{\funcwidth}

    \raggedright \textbf{\_\_init\_\_}(\textit{self}, \textit{cssText}={\tt None}, \textit{readonly}={\tt False}, \textit{\_propertyName}={\tt None})

    \vspace{-1.5ex}

    \rule{\textwidth}{0.5\fboxrule}
\setlength{\parskip}{2ex}

inits a new CSSValueList
\setlength{\parskip}{1ex}
      Overrides: object.\_\_init\_\_

    \end{boxedminipage}

    \label{cssutils:css:cssvalue:CSSValueList:item}
    \index{cssutils \textit{(package)}!cssutils.css \textit{(package)}!cssutils.css.cssvalue \textit{(module)}!cssutils.css.cssvalue.CSSValueList \textit{(class)}!cssutils.css.cssvalue.CSSValueList.item \textit{(method)}}

    \vspace{0.5ex}

\hspace{.8\funcindent}\begin{boxedminipage}{\funcwidth}

    \raggedright \textbf{item}(\textit{self}, \textit{index})

    \vspace{-1.5ex}

    \rule{\textwidth}{0.5\fboxrule}
\setlength{\parskip}{2ex}

(DOM method) Used to retrieve a CSSValue by ordinal index. The
order in this collection represents the order of the values in the
CSS style property. If index is greater than or equal to the number
of values in the list, this returns None.
\setlength{\parskip}{1ex}
    \end{boxedminipage}

    \label{cssutils:css:cssvalue:CSSValueList:__iter__}
    \index{cssutils \textit{(package)}!cssutils.css \textit{(package)}!cssutils.css.cssvalue \textit{(module)}!cssutils.css.cssvalue.CSSValueList \textit{(class)}!cssutils.css.cssvalue.CSSValueList.\_\_iter\_\_ \textit{(method)}}

    \vspace{0.5ex}

\hspace{.8\funcindent}\begin{boxedminipage}{\funcwidth}

    \raggedright \textbf{\_\_iter\_\_}(\textit{self})

    \vspace{-1.5ex}

    \rule{\textwidth}{0.5\fboxrule}
\setlength{\parskip}{2ex}

CSSValueList is iterable
\setlength{\parskip}{1ex}
    \end{boxedminipage}

    \label{cssutils:css:cssvalue:CSSValueList:__str_}
    \index{cssutils \textit{(package)}!cssutils.css \textit{(package)}!cssutils.css.cssvalue \textit{(module)}!cssutils.css.cssvalue.CSSValueList \textit{(class)}!cssutils.css.cssvalue.CSSValueList.\_\_str\_ \textit{(method)}}

    \vspace{0.5ex}

\hspace{.8\funcindent}\begin{boxedminipage}{\funcwidth}

    \raggedright \textbf{\_\_str\_}(\textit{self})

\setlength{\parskip}{2ex}
\setlength{\parskip}{1ex}
    \end{boxedminipage}


\large{\textbf{\textit{Inherited from cssutils.css.cssvalue.CSSValue\textit{(Section \ref{cssutils:css:cssvalue:CSSValue})}}}}

\begin{quote}
\_\_repr\_\_(), \_\_str\_\_()
\end{quote}

\large{\textbf{\textit{Inherited from object}}}

\begin{quote}
\_\_delattr\_\_(), \_\_getattribute\_\_(), \_\_hash\_\_(), \_\_new\_\_(), \_\_reduce\_\_(), \_\_reduce\_ex\_\_(), \_\_setattr\_\_()
\end{quote}

%%%%%%%%%%%%%%%%%%%%%%%%%%%%%%%%%%%%%%%%%%%%%%%%%%%%%%%%%%%%%%%%%%%%%%%%%%%
%%                              Properties                               %%
%%%%%%%%%%%%%%%%%%%%%%%%%%%%%%%%%%%%%%%%%%%%%%%%%%%%%%%%%%%%%%%%%%%%%%%%%%%

  \subsubsection{Properties}

    \vspace{-1cm}
\hspace{\varindent}\begin{longtable}{|p{\varnamewidth}|p{\vardescrwidth}|l}
\cline{1-2}
\cline{1-2} \centering \textbf{Name} & \centering \textbf{Description}& \\
\cline{1-2}
\endhead\cline{1-2}\multicolumn{3}{r}{\small\textit{continued on next page}}\\\endfoot\cline{1-2}
\endlastfoot\raggedright l\-e\-n\-g\-t\-h\- & \raggedright (DOM attribute) The number of CSSValues in the list.&\\
\cline{1-2}
\multicolumn{2}{|l|}{\textit{Inherited from cssutils.css.cssvalue.CSSValue \textit{(Section \ref{cssutils:css:cssvalue:CSSValue})}}}\\
\multicolumn{2}{|p{\varwidth}|}{\raggedright cssText, cssValueTypeString}\\
\cline{1-2}
\multicolumn{2}{|l|}{\textit{Inherited from object}}\\
\multicolumn{2}{|p{\varwidth}|}{\raggedright \_\_class\_\_}\\
\cline{1-2}
\end{longtable}


%%%%%%%%%%%%%%%%%%%%%%%%%%%%%%%%%%%%%%%%%%%%%%%%%%%%%%%%%%%%%%%%%%%%%%%%%%%
%%                            Class Variables                            %%
%%%%%%%%%%%%%%%%%%%%%%%%%%%%%%%%%%%%%%%%%%%%%%%%%%%%%%%%%%%%%%%%%%%%%%%%%%%

  \subsubsection{Class Variables}

    \vspace{-1cm}
\hspace{\varindent}\begin{longtable}{|p{\varnamewidth}|p{\vardescrwidth}|l}
\cline{1-2}
\cline{1-2} \centering \textbf{Name} & \centering \textbf{Description}& \\
\cline{1-2}
\endhead\cline{1-2}\multicolumn{3}{r}{\small\textit{continued on next page}}\\\endfoot\cline{1-2}
\endlastfoot\raggedright c\-s\-s\-V\-a\-l\-u\-e\-T\-y\-p\-e\- & \raggedright \textbf{Value:} 
{\tt 2}&\\
\cline{1-2}
\multicolumn{2}{|l|}{\textit{Inherited from cssutils.css.cssvalue.CSSValue \textit{(Section \ref{cssutils:css:cssvalue:CSSValue})}}}\\
\multicolumn{2}{|p{\varwidth}|}{\raggedright CSS\_CUSTOM, CSS\_INHERIT, CSS\_PRIMITIVE\_VALUE, CSS\_VALUE\_LIST}\\
\cline{1-2}
\end{longtable}

    \index{cssutils \textit{(package)}!cssutils.css \textit{(package)}!cssutils.css.cssvalue \textit{(module)}!cssutils.css.cssvalue.CSSValueList \textit{(class)}|)}
    \index{cssutils \textit{(package)}!cssutils.css \textit{(package)}!cssutils.css.cssvalue \textit{(module)}|)}
