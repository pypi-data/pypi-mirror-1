%
% API Documentation for cssutils
% Module cssutils.codec
%
% Generated by epydoc 3.0.1
% [Fri Feb 01 19:05:21 2008]
%

%%%%%%%%%%%%%%%%%%%%%%%%%%%%%%%%%%%%%%%%%%%%%%%%%%%%%%%%%%%%%%%%%%%%%%%%%%%
%%                          Module Description                           %%
%%%%%%%%%%%%%%%%%%%%%%%%%%%%%%%%%%%%%%%%%%%%%%%%%%%%%%%%%%%%%%%%%%%%%%%%%%%

    \index{cssutils \textit{(package)}!cssutils.codec \textit{(module)}|(}
\section{Module cssutils.codec}

    \label{cssutils:codec}

Python codec for CSS.
\textbf{Version:} \$LastChangedRevision: 645 \$



\textbf{Date:} \$LastChangedDate: 2007-11-05 14:52:02 +0100 (Mo, 05 Nov 2007) \$



\textbf{Author:} \$LastChangedBy: doerwalter \$




%%%%%%%%%%%%%%%%%%%%%%%%%%%%%%%%%%%%%%%%%%%%%%%%%%%%%%%%%%%%%%%%%%%%%%%%%%%
%%                               Functions                               %%
%%%%%%%%%%%%%%%%%%%%%%%%%%%%%%%%%%%%%%%%%%%%%%%%%%%%%%%%%%%%%%%%%%%%%%%%%%%

  \subsection{Functions}

    \label{cssutils:codec:decode}
    \index{cssutils \textit{(package)}!cssutils.codec \textit{(module)}!cssutils.codec.decode \textit{(function)}}

    \vspace{0.5ex}

\hspace{.8\funcindent}\begin{boxedminipage}{\funcwidth}

    \raggedright \textbf{decode}(\textit{input}, \textit{errors}={\tt \texttt{'}\texttt{strict}\texttt{'}}, \textit{encoding}={\tt None})

\setlength{\parskip}{2ex}
\setlength{\parskip}{1ex}
    \end{boxedminipage}

    \label{cssutils:codec:encode}
    \index{cssutils \textit{(package)}!cssutils.codec \textit{(module)}!cssutils.codec.encode \textit{(function)}}

    \vspace{0.5ex}

\hspace{.8\funcindent}\begin{boxedminipage}{\funcwidth}

    \raggedright \textbf{encode}(\textit{input}, \textit{errors}={\tt \texttt{'}\texttt{strict}\texttt{'}}, \textit{encoding}={\tt None})

\setlength{\parskip}{2ex}
\setlength{\parskip}{1ex}
    \end{boxedminipage}

    \label{cssutils:codec:utf8sig_encode}
    \index{cssutils \textit{(package)}!cssutils.codec \textit{(module)}!cssutils.codec.utf8sig\_encode \textit{(function)}}

    \vspace{0.5ex}

\hspace{.8\funcindent}\begin{boxedminipage}{\funcwidth}

    \raggedright \textbf{utf8sig\_encode}(\textit{input}, \textit{errors}={\tt 'strict'})

\setlength{\parskip}{2ex}
\setlength{\parskip}{1ex}
    \end{boxedminipage}

    \label{cssutils:codec:utf8sig_decode}
    \index{cssutils \textit{(package)}!cssutils.codec \textit{(module)}!cssutils.codec.utf8sig\_decode \textit{(function)}}

    \vspace{0.5ex}

\hspace{.8\funcindent}\begin{boxedminipage}{\funcwidth}

    \raggedright \textbf{utf8sig\_decode}(\textit{input}, \textit{errors}={\tt 'strict'})

\setlength{\parskip}{2ex}
\setlength{\parskip}{1ex}
    \end{boxedminipage}

    \label{cssutils:codec:search_function}
    \index{cssutils \textit{(package)}!cssutils.codec \textit{(module)}!cssutils.codec.search\_function \textit{(function)}}

    \vspace{0.5ex}

\hspace{.8\funcindent}\begin{boxedminipage}{\funcwidth}

    \raggedright \textbf{search\_function}(\textit{name})

\setlength{\parskip}{2ex}
\setlength{\parskip}{1ex}
    \end{boxedminipage}

    \label{cssutils:codec:cssescape}
    \index{cssutils \textit{(package)}!cssutils.codec \textit{(module)}!cssutils.codec.cssescape \textit{(function)}}

    \vspace{0.5ex}

\hspace{.8\funcindent}\begin{boxedminipage}{\funcwidth}

    \raggedright \textbf{cssescape}(\textit{exc})

\setlength{\parskip}{2ex}
\setlength{\parskip}{1ex}
    \end{boxedminipage}


%%%%%%%%%%%%%%%%%%%%%%%%%%%%%%%%%%%%%%%%%%%%%%%%%%%%%%%%%%%%%%%%%%%%%%%%%%%
%%                           Class Description                           %%
%%%%%%%%%%%%%%%%%%%%%%%%%%%%%%%%%%%%%%%%%%%%%%%%%%%%%%%%%%%%%%%%%%%%%%%%%%%

    \index{cssutils \textit{(package)}!cssutils.codec \textit{(module)}!cssutils.codec.IncrementalDecoder \textit{(class)}|(}
\subsection{Class IncrementalDecoder}

    \label{cssutils:codec:IncrementalDecoder}
\begin{tabular}{cccccccc}
% Line for object, linespec=[False, False]
\multicolumn{2}{r}{\settowidth{\BCL}{object}\multirow{2}{\BCL}{object}}
&&
&&
  \\\cline{3-3}
  &&\multicolumn{1}{c|}{}
&&
&&
  \\
% Line for codecs.IncrementalDecoder, linespec=[False]
\multicolumn{4}{r}{\settowidth{\BCL}{codecs.IncrementalDecoder}\multirow{2}{\BCL}{codecs.IncrementalDecoder}}
&&
  \\\cline{5-5}
  &&&&\multicolumn{1}{c|}{}
&&
  \\
&&&&\multicolumn{2}{l}{\textbf{cssutils.codec.IncrementalDecoder}}
\end{tabular}


%%%%%%%%%%%%%%%%%%%%%%%%%%%%%%%%%%%%%%%%%%%%%%%%%%%%%%%%%%%%%%%%%%%%%%%%%%%
%%                                Methods                                %%
%%%%%%%%%%%%%%%%%%%%%%%%%%%%%%%%%%%%%%%%%%%%%%%%%%%%%%%%%%%%%%%%%%%%%%%%%%%

  \subsubsection{Methods}

    \vspace{0.5ex}

\hspace{.8\funcindent}\begin{boxedminipage}{\funcwidth}

    \raggedright \textbf{\_\_init\_\_}(\textit{self}, \textit{errors}={\tt \texttt{'}\texttt{strict}\texttt{'}}, \textit{encoding}={\tt None})

\setlength{\parskip}{2ex}
    Creates a IncrementalDecoder instance.

    The IncrementalDecoder may use different error handling schemes by 
    providing the errors keyword argument. See the module docstring for a 
    list of possible values.

\setlength{\parskip}{1ex}
      Overrides: object.\_\_init\_\_ 	extit{(inherited documentation)}

    \end{boxedminipage}

    \label{cssutils:codec:IncrementalDecoder:iterdecode}
    \index{cssutils \textit{(package)}!cssutils.codec \textit{(module)}!cssutils.codec.IncrementalDecoder \textit{(class)}!cssutils.codec.IncrementalDecoder.iterdecode \textit{(method)}}

    \vspace{0.5ex}

\hspace{.8\funcindent}\begin{boxedminipage}{\funcwidth}

    \raggedright \textbf{iterdecode}(\textit{self}, \textit{input})

\setlength{\parskip}{2ex}
\setlength{\parskip}{1ex}
    \end{boxedminipage}

    \vspace{0.5ex}

\hspace{.8\funcindent}\begin{boxedminipage}{\funcwidth}

    \raggedright \textbf{decode}(\textit{self}, \textit{input}, \textit{final}={\tt False})

\setlength{\parskip}{2ex}
    Decodes input and returns the resulting object.

\setlength{\parskip}{1ex}
      Overrides: codecs.IncrementalDecoder.decode 	extit{(inherited documentation)}

    \end{boxedminipage}

    \vspace{0.5ex}

\hspace{.8\funcindent}\begin{boxedminipage}{\funcwidth}

    \raggedright \textbf{reset}(\textit{self})

\setlength{\parskip}{2ex}
    Resets the decoder to the initial state.

\setlength{\parskip}{1ex}
      Overrides: codecs.IncrementalDecoder.reset 	extit{(inherited documentation)}

    \end{boxedminipage}

    \label{cssutils:codec:IncrementalDecoder:getstate}
    \index{cssutils \textit{(package)}!cssutils.codec \textit{(module)}!cssutils.codec.IncrementalDecoder \textit{(class)}!cssutils.codec.IncrementalDecoder.getstate \textit{(method)}}

    \vspace{0.5ex}

\hspace{.8\funcindent}\begin{boxedminipage}{\funcwidth}

    \raggedright \textbf{getstate}(\textit{self})

\setlength{\parskip}{2ex}
\setlength{\parskip}{1ex}
    \end{boxedminipage}

    \label{cssutils:codec:IncrementalDecoder:setstate}
    \index{cssutils \textit{(package)}!cssutils.codec \textit{(module)}!cssutils.codec.IncrementalDecoder \textit{(class)}!cssutils.codec.IncrementalDecoder.setstate \textit{(method)}}

    \vspace{0.5ex}

\hspace{.8\funcindent}\begin{boxedminipage}{\funcwidth}

    \raggedright \textbf{setstate}(\textit{self}, \textit{state})

\setlength{\parskip}{2ex}
\setlength{\parskip}{1ex}
    \end{boxedminipage}


\large{\textbf{\textit{Inherited from object}}}

\begin{quote}
\_\_delattr\_\_(), \_\_getattribute\_\_(), \_\_hash\_\_(), \_\_new\_\_(), \_\_reduce\_\_(), \_\_reduce\_ex\_\_(), \_\_repr\_\_(), \_\_setattr\_\_(), \_\_str\_\_()
\end{quote}

%%%%%%%%%%%%%%%%%%%%%%%%%%%%%%%%%%%%%%%%%%%%%%%%%%%%%%%%%%%%%%%%%%%%%%%%%%%
%%                              Properties                               %%
%%%%%%%%%%%%%%%%%%%%%%%%%%%%%%%%%%%%%%%%%%%%%%%%%%%%%%%%%%%%%%%%%%%%%%%%%%%

  \subsubsection{Properties}

    \vspace{-1cm}
\hspace{\varindent}\begin{longtable}{|p{\varnamewidth}|p{\vardescrwidth}|l}
\cline{1-2}
\cline{1-2} \centering \textbf{Name} & \centering \textbf{Description}& \\
\cline{1-2}
\endhead\cline{1-2}\multicolumn{3}{r}{\small\textit{continued on next page}}\\\endfoot\cline{1-2}
\endlastfoot\raggedright e\-r\-r\-o\-r\-s\- & &\\
\cline{1-2}
\multicolumn{2}{|l|}{\textit{Inherited from object}}\\
\multicolumn{2}{|p{\varwidth}|}{\raggedright \_\_class\_\_}\\
\cline{1-2}
\end{longtable}

    \index{cssutils \textit{(package)}!cssutils.codec \textit{(module)}!cssutils.codec.IncrementalDecoder \textit{(class)}|)}

%%%%%%%%%%%%%%%%%%%%%%%%%%%%%%%%%%%%%%%%%%%%%%%%%%%%%%%%%%%%%%%%%%%%%%%%%%%
%%                           Class Description                           %%
%%%%%%%%%%%%%%%%%%%%%%%%%%%%%%%%%%%%%%%%%%%%%%%%%%%%%%%%%%%%%%%%%%%%%%%%%%%

    \index{cssutils \textit{(package)}!cssutils.codec \textit{(module)}!cssutils.codec.IncrementalEncoder \textit{(class)}|(}
\subsection{Class IncrementalEncoder}

    \label{cssutils:codec:IncrementalEncoder}
\begin{tabular}{cccccccc}
% Line for object, linespec=[False, False]
\multicolumn{2}{r}{\settowidth{\BCL}{object}\multirow{2}{\BCL}{object}}
&&
&&
  \\\cline{3-3}
  &&\multicolumn{1}{c|}{}
&&
&&
  \\
% Line for codecs.IncrementalEncoder, linespec=[False]
\multicolumn{4}{r}{\settowidth{\BCL}{codecs.IncrementalEncoder}\multirow{2}{\BCL}{codecs.IncrementalEncoder}}
&&
  \\\cline{5-5}
  &&&&\multicolumn{1}{c|}{}
&&
  \\
&&&&\multicolumn{2}{l}{\textbf{cssutils.codec.IncrementalEncoder}}
\end{tabular}


%%%%%%%%%%%%%%%%%%%%%%%%%%%%%%%%%%%%%%%%%%%%%%%%%%%%%%%%%%%%%%%%%%%%%%%%%%%
%%                                Methods                                %%
%%%%%%%%%%%%%%%%%%%%%%%%%%%%%%%%%%%%%%%%%%%%%%%%%%%%%%%%%%%%%%%%%%%%%%%%%%%

  \subsubsection{Methods}

    \vspace{0.5ex}

\hspace{.8\funcindent}\begin{boxedminipage}{\funcwidth}

    \raggedright \textbf{\_\_init\_\_}(\textit{self}, \textit{errors}={\tt \texttt{'}\texttt{strict}\texttt{'}}, \textit{encoding}={\tt None})

\setlength{\parskip}{2ex}
    Creates an IncrementalEncoder instance.

    The IncrementalEncoder may use different error handling schemes by 
    providing the errors keyword argument. See the module docstring for a 
    list of possible values.

\setlength{\parskip}{1ex}
      Overrides: object.\_\_init\_\_ 	extit{(inherited documentation)}

    \end{boxedminipage}

    \label{cssutils:codec:IncrementalEncoder:iterencode}
    \index{cssutils \textit{(package)}!cssutils.codec \textit{(module)}!cssutils.codec.IncrementalEncoder \textit{(class)}!cssutils.codec.IncrementalEncoder.iterencode \textit{(method)}}

    \vspace{0.5ex}

\hspace{.8\funcindent}\begin{boxedminipage}{\funcwidth}

    \raggedright \textbf{iterencode}(\textit{self}, \textit{input})

\setlength{\parskip}{2ex}
\setlength{\parskip}{1ex}
    \end{boxedminipage}

    \vspace{0.5ex}

\hspace{.8\funcindent}\begin{boxedminipage}{\funcwidth}

    \raggedright \textbf{encode}(\textit{self}, \textit{input}, \textit{final}={\tt False})

\setlength{\parskip}{2ex}
    Encodes input and returns the resulting object.

\setlength{\parskip}{1ex}
      Overrides: codecs.IncrementalEncoder.encode 	extit{(inherited documentation)}

    \end{boxedminipage}

    \vspace{0.5ex}

\hspace{.8\funcindent}\begin{boxedminipage}{\funcwidth}

    \raggedright \textbf{reset}(\textit{self})

\setlength{\parskip}{2ex}
    Resets the encoder to the initial state.

\setlength{\parskip}{1ex}
      Overrides: codecs.IncrementalEncoder.reset 	extit{(inherited documentation)}

    \end{boxedminipage}

    \label{cssutils:codec:IncrementalEncoder:getstate}
    \index{cssutils \textit{(package)}!cssutils.codec \textit{(module)}!cssutils.codec.IncrementalEncoder \textit{(class)}!cssutils.codec.IncrementalEncoder.getstate \textit{(method)}}

    \vspace{0.5ex}

\hspace{.8\funcindent}\begin{boxedminipage}{\funcwidth}

    \raggedright \textbf{getstate}(\textit{self})

\setlength{\parskip}{2ex}
\setlength{\parskip}{1ex}
    \end{boxedminipage}

    \label{cssutils:codec:IncrementalEncoder:setstate}
    \index{cssutils \textit{(package)}!cssutils.codec \textit{(module)}!cssutils.codec.IncrementalEncoder \textit{(class)}!cssutils.codec.IncrementalEncoder.setstate \textit{(method)}}

    \vspace{0.5ex}

\hspace{.8\funcindent}\begin{boxedminipage}{\funcwidth}

    \raggedright \textbf{setstate}(\textit{self}, \textit{state})

\setlength{\parskip}{2ex}
\setlength{\parskip}{1ex}
    \end{boxedminipage}


\large{\textbf{\textit{Inherited from object}}}

\begin{quote}
\_\_delattr\_\_(), \_\_getattribute\_\_(), \_\_hash\_\_(), \_\_new\_\_(), \_\_reduce\_\_(), \_\_reduce\_ex\_\_(), \_\_repr\_\_(), \_\_setattr\_\_(), \_\_str\_\_()
\end{quote}

%%%%%%%%%%%%%%%%%%%%%%%%%%%%%%%%%%%%%%%%%%%%%%%%%%%%%%%%%%%%%%%%%%%%%%%%%%%
%%                              Properties                               %%
%%%%%%%%%%%%%%%%%%%%%%%%%%%%%%%%%%%%%%%%%%%%%%%%%%%%%%%%%%%%%%%%%%%%%%%%%%%

  \subsubsection{Properties}

    \vspace{-1cm}
\hspace{\varindent}\begin{longtable}{|p{\varnamewidth}|p{\vardescrwidth}|l}
\cline{1-2}
\cline{1-2} \centering \textbf{Name} & \centering \textbf{Description}& \\
\cline{1-2}
\endhead\cline{1-2}\multicolumn{3}{r}{\small\textit{continued on next page}}\\\endfoot\cline{1-2}
\endlastfoot\raggedright e\-r\-r\-o\-r\-s\- & &\\
\cline{1-2}
\multicolumn{2}{|l|}{\textit{Inherited from object}}\\
\multicolumn{2}{|p{\varwidth}|}{\raggedright \_\_class\_\_}\\
\cline{1-2}
\end{longtable}

    \index{cssutils \textit{(package)}!cssutils.codec \textit{(module)}!cssutils.codec.IncrementalEncoder \textit{(class)}|)}

%%%%%%%%%%%%%%%%%%%%%%%%%%%%%%%%%%%%%%%%%%%%%%%%%%%%%%%%%%%%%%%%%%%%%%%%%%%
%%                           Class Description                           %%
%%%%%%%%%%%%%%%%%%%%%%%%%%%%%%%%%%%%%%%%%%%%%%%%%%%%%%%%%%%%%%%%%%%%%%%%%%%

    \index{cssutils \textit{(package)}!cssutils.codec \textit{(module)}!cssutils.codec.StreamWriter \textit{(class)}|(}
\subsection{Class StreamWriter}

    \label{cssutils:codec:StreamWriter}
\begin{tabular}{cccccccc}
% Line for codecs.Codec, linespec=[False, False]
\multicolumn{2}{r}{\settowidth{\BCL}{codecs.Codec}\multirow{2}{\BCL}{codecs.Codec}}
&&
&&
  \\\cline{3-3}
  &&\multicolumn{1}{c|}{}
&&
&&
  \\
% Line for codecs.StreamWriter, linespec=[False]
\multicolumn{4}{r}{\settowidth{\BCL}{codecs.StreamWriter}\multirow{2}{\BCL}{codecs.StreamWriter}}
&&
  \\\cline{5-5}
  &&&&\multicolumn{1}{c|}{}
&&
  \\
&&&&\multicolumn{2}{l}{\textbf{cssutils.codec.StreamWriter}}
\end{tabular}


%%%%%%%%%%%%%%%%%%%%%%%%%%%%%%%%%%%%%%%%%%%%%%%%%%%%%%%%%%%%%%%%%%%%%%%%%%%
%%                                Methods                                %%
%%%%%%%%%%%%%%%%%%%%%%%%%%%%%%%%%%%%%%%%%%%%%%%%%%%%%%%%%%%%%%%%%%%%%%%%%%%

  \subsubsection{Methods}

    \vspace{0.5ex}

\hspace{.8\funcindent}\begin{boxedminipage}{\funcwidth}

    \raggedright \textbf{\_\_init\_\_}(\textit{self}, \textit{stream}, \textit{errors}={\tt \texttt{'}\texttt{strict}\texttt{'}}, \textit{encoding}={\tt None}, \textit{header}={\tt False})

\setlength{\parskip}{2ex}
\begin{alltt}
Creates a StreamWriter instance.

stream must be a file-like object open for writing
(binary) data.

The StreamWriter may use different error handling
schemes by providing the errors keyword argument. These
parameters are predefined:

 'strict' - raise a ValueError (or a subclass)
 'ignore' - ignore the character and continue with the next
 'replace'- replace with a suitable replacement character
 'xmlcharrefreplace' - Replace with the appropriate XML
                       character reference.
 'backslashreplace'  - Replace with backslashed escape
                       sequences (only for encoding).

The set of allowed parameter values can be extended via
register\_error.
\end{alltt}

\setlength{\parskip}{1ex}
      Overrides: codecs.StreamWriter.\_\_init\_\_ 	extit{(inherited documentation)}

    \end{boxedminipage}

    \vspace{0.5ex}

\hspace{.8\funcindent}\begin{boxedminipage}{\funcwidth}

    \raggedright \textbf{encode}(\textit{self}, \textit{input}, \textit{errors}={\tt \texttt{'}\texttt{strict}\texttt{'}})

\setlength{\parskip}{2ex}
    Encodes the object input and returns a tuple (output object, length 
    consumed).

    errors defines the error handling to apply. It defaults to 'strict' 
    handling.

    The method may not store state in the Codec instance. Use StreamCodec 
    for codecs which have to keep state in order to make encoding/decoding 
    efficient.

    The encoder must be able to handle zero length input and return an 
    empty object of the output object type in this situation.

\setlength{\parskip}{1ex}
      Overrides: codecs.Codec.encode 	extit{(inherited documentation)}

    \end{boxedminipage}


\large{\textbf{\textit{Inherited from codecs.StreamWriter}}}

\begin{quote}
\_\_enter\_\_(), \_\_exit\_\_(), \_\_getattr\_\_(), reset(), write(), writelines()
\end{quote}

\large{\textbf{\textit{Inherited from codecs.Codec}}}

\begin{quote}
decode()
\end{quote}

%%%%%%%%%%%%%%%%%%%%%%%%%%%%%%%%%%%%%%%%%%%%%%%%%%%%%%%%%%%%%%%%%%%%%%%%%%%
%%                              Properties                               %%
%%%%%%%%%%%%%%%%%%%%%%%%%%%%%%%%%%%%%%%%%%%%%%%%%%%%%%%%%%%%%%%%%%%%%%%%%%%

  \subsubsection{Properties}

    \vspace{-1cm}
\hspace{\varindent}\begin{longtable}{|p{\varnamewidth}|p{\vardescrwidth}|l}
\cline{1-2}
\cline{1-2} \centering \textbf{Name} & \centering \textbf{Description}& \\
\cline{1-2}
\endhead\cline{1-2}\multicolumn{3}{r}{\small\textit{continued on next page}}\\\endfoot\cline{1-2}
\endlastfoot\raggedright e\-r\-r\-o\-r\-s\- & &\\
\cline{1-2}
\end{longtable}

    \index{cssutils \textit{(package)}!cssutils.codec \textit{(module)}!cssutils.codec.StreamWriter \textit{(class)}|)}

%%%%%%%%%%%%%%%%%%%%%%%%%%%%%%%%%%%%%%%%%%%%%%%%%%%%%%%%%%%%%%%%%%%%%%%%%%%
%%                           Class Description                           %%
%%%%%%%%%%%%%%%%%%%%%%%%%%%%%%%%%%%%%%%%%%%%%%%%%%%%%%%%%%%%%%%%%%%%%%%%%%%

    \index{cssutils \textit{(package)}!cssutils.codec \textit{(module)}!cssutils.codec.StreamReader \textit{(class)}|(}
\subsection{Class StreamReader}

    \label{cssutils:codec:StreamReader}
\begin{tabular}{cccccccc}
% Line for codecs.Codec, linespec=[False, False]
\multicolumn{2}{r}{\settowidth{\BCL}{codecs.Codec}\multirow{2}{\BCL}{codecs.Codec}}
&&
&&
  \\\cline{3-3}
  &&\multicolumn{1}{c|}{}
&&
&&
  \\
% Line for codecs.StreamReader, linespec=[False]
\multicolumn{4}{r}{\settowidth{\BCL}{codecs.StreamReader}\multirow{2}{\BCL}{codecs.StreamReader}}
&&
  \\\cline{5-5}
  &&&&\multicolumn{1}{c|}{}
&&
  \\
&&&&\multicolumn{2}{l}{\textbf{cssutils.codec.StreamReader}}
\end{tabular}


%%%%%%%%%%%%%%%%%%%%%%%%%%%%%%%%%%%%%%%%%%%%%%%%%%%%%%%%%%%%%%%%%%%%%%%%%%%
%%                                Methods                                %%
%%%%%%%%%%%%%%%%%%%%%%%%%%%%%%%%%%%%%%%%%%%%%%%%%%%%%%%%%%%%%%%%%%%%%%%%%%%

  \subsubsection{Methods}

    \vspace{0.5ex}

\hspace{.8\funcindent}\begin{boxedminipage}{\funcwidth}

    \raggedright \textbf{\_\_init\_\_}(\textit{self}, \textit{stream}, \textit{errors}={\tt \texttt{'}\texttt{strict}\texttt{'}}, \textit{encoding}={\tt None})

\setlength{\parskip}{2ex}
\begin{alltt}
Creates a StreamReader instance.

stream must be a file-like object open for reading
(binary) data.

The StreamReader may use different error handling
schemes by providing the errors keyword argument. These
parameters are predefined:

 'strict' - raise a ValueError (or a subclass)
 'ignore' - ignore the character and continue with the next
 'replace'- replace with a suitable replacement character;

The set of allowed parameter values can be extended via
register\_error.
\end{alltt}

\setlength{\parskip}{1ex}
      Overrides: codecs.StreamReader.\_\_init\_\_ 	extit{(inherited documentation)}

    \end{boxedminipage}

    \vspace{0.5ex}

\hspace{.8\funcindent}\begin{boxedminipage}{\funcwidth}

    \raggedright \textbf{decode}(\textit{self}, \textit{input}, \textit{errors}={\tt \texttt{'}\texttt{strict}\texttt{'}})

\setlength{\parskip}{2ex}
    Decodes the object input and returns a tuple (output object, length 
    consumed).

    input must be an object which provides the bf\_getreadbuf buffer slot. 
    Python strings, buffer objects and memory mapped files are examples of 
    objects providing this slot.

    errors defines the error handling to apply. It defaults to 'strict' 
    handling.

    The method may not store state in the Codec instance. Use StreamCodec 
    for codecs which have to keep state in order to make encoding/decoding 
    efficient.

    The decoder must be able to handle zero length input and return an 
    empty object of the output object type in this situation.

\setlength{\parskip}{1ex}
      Overrides: codecs.Codec.decode 	extit{(inherited documentation)}

    \end{boxedminipage}


\large{\textbf{\textit{Inherited from codecs.StreamReader}}}

\begin{quote}
\_\_enter\_\_(), \_\_exit\_\_(), \_\_getattr\_\_(), \_\_iter\_\_(), next(), read(), readline(), readlines(), reset(), seek()
\end{quote}

\large{\textbf{\textit{Inherited from codecs.Codec}}}

\begin{quote}
encode()
\end{quote}

%%%%%%%%%%%%%%%%%%%%%%%%%%%%%%%%%%%%%%%%%%%%%%%%%%%%%%%%%%%%%%%%%%%%%%%%%%%
%%                              Properties                               %%
%%%%%%%%%%%%%%%%%%%%%%%%%%%%%%%%%%%%%%%%%%%%%%%%%%%%%%%%%%%%%%%%%%%%%%%%%%%

  \subsubsection{Properties}

    \vspace{-1cm}
\hspace{\varindent}\begin{longtable}{|p{\varnamewidth}|p{\vardescrwidth}|l}
\cline{1-2}
\cline{1-2} \centering \textbf{Name} & \centering \textbf{Description}& \\
\cline{1-2}
\endhead\cline{1-2}\multicolumn{3}{r}{\small\textit{continued on next page}}\\\endfoot\cline{1-2}
\endlastfoot\raggedright e\-r\-r\-o\-r\-s\- & &\\
\cline{1-2}
\end{longtable}

    \index{cssutils \textit{(package)}!cssutils.codec \textit{(module)}!cssutils.codec.StreamReader \textit{(class)}|)}

%%%%%%%%%%%%%%%%%%%%%%%%%%%%%%%%%%%%%%%%%%%%%%%%%%%%%%%%%%%%%%%%%%%%%%%%%%%
%%                           Class Description                           %%
%%%%%%%%%%%%%%%%%%%%%%%%%%%%%%%%%%%%%%%%%%%%%%%%%%%%%%%%%%%%%%%%%%%%%%%%%%%

    \index{cssutils \textit{(package)}!cssutils.codec \textit{(module)}!cssutils.codec.UTF8SigStreamWriter \textit{(class)}|(}
\subsection{Class UTF8SigStreamWriter}

    \label{cssutils:codec:UTF8SigStreamWriter}
\begin{tabular}{cccccccc}
% Line for codecs.Codec, linespec=[False, False]
\multicolumn{2}{r}{\settowidth{\BCL}{codecs.Codec}\multirow{2}{\BCL}{codecs.Codec}}
&&
&&
  \\\cline{3-3}
  &&\multicolumn{1}{c|}{}
&&
&&
  \\
% Line for codecs.StreamWriter, linespec=[False]
\multicolumn{4}{r}{\settowidth{\BCL}{codecs.StreamWriter}\multirow{2}{\BCL}{codecs.StreamWriter}}
&&
  \\\cline{5-5}
  &&&&\multicolumn{1}{c|}{}
&&
  \\
&&&&\multicolumn{2}{l}{\textbf{cssutils.codec.UTF8SigStreamWriter}}
\end{tabular}


%%%%%%%%%%%%%%%%%%%%%%%%%%%%%%%%%%%%%%%%%%%%%%%%%%%%%%%%%%%%%%%%%%%%%%%%%%%
%%                                Methods                                %%
%%%%%%%%%%%%%%%%%%%%%%%%%%%%%%%%%%%%%%%%%%%%%%%%%%%%%%%%%%%%%%%%%%%%%%%%%%%

  \subsubsection{Methods}

    \vspace{0.5ex}

\hspace{.8\funcindent}\begin{boxedminipage}{\funcwidth}

    \raggedright \textbf{reset}(\textit{self})

\setlength{\parskip}{2ex}
    Flushes and resets the codec buffers used for keeping state.

    Calling this method should ensure that the data on the output is put 
    into a clean state, that allows appending of new fresh data without 
    having to rescan the whole stream to recover state.

\setlength{\parskip}{1ex}
      Overrides: codecs.StreamWriter.reset 	extit{(inherited documentation)}

    \end{boxedminipage}

    \vspace{0.5ex}

\hspace{.8\funcindent}\begin{boxedminipage}{\funcwidth}

    \raggedright \textbf{encode}(\textit{self}, \textit{input}, \textit{errors}={\tt 'strict'})

\setlength{\parskip}{2ex}
    Encodes the object input and returns a tuple (output object, length 
    consumed).

    errors defines the error handling to apply. It defaults to 'strict' 
    handling.

    The method may not store state in the Codec instance. Use StreamCodec 
    for codecs which have to keep state in order to make encoding/decoding 
    efficient.

    The encoder must be able to handle zero length input and return an 
    empty object of the output object type in this situation.

\setlength{\parskip}{1ex}
      Overrides: codecs.Codec.encode 	extit{(inherited documentation)}

    \end{boxedminipage}


\large{\textbf{\textit{Inherited from codecs.StreamWriter}}}

\begin{quote}
\_\_enter\_\_(), \_\_exit\_\_(), \_\_getattr\_\_(), \_\_init\_\_(), write(), writelines()
\end{quote}

\large{\textbf{\textit{Inherited from codecs.Codec}}}

\begin{quote}
decode()
\end{quote}
    \index{cssutils \textit{(package)}!cssutils.codec \textit{(module)}!cssutils.codec.UTF8SigStreamWriter \textit{(class)}|)}

%%%%%%%%%%%%%%%%%%%%%%%%%%%%%%%%%%%%%%%%%%%%%%%%%%%%%%%%%%%%%%%%%%%%%%%%%%%
%%                           Class Description                           %%
%%%%%%%%%%%%%%%%%%%%%%%%%%%%%%%%%%%%%%%%%%%%%%%%%%%%%%%%%%%%%%%%%%%%%%%%%%%

    \index{cssutils \textit{(package)}!cssutils.codec \textit{(module)}!cssutils.codec.UTF8SigStreamReader \textit{(class)}|(}
\subsection{Class UTF8SigStreamReader}

    \label{cssutils:codec:UTF8SigStreamReader}
\begin{tabular}{cccccccc}
% Line for codecs.Codec, linespec=[False, False]
\multicolumn{2}{r}{\settowidth{\BCL}{codecs.Codec}\multirow{2}{\BCL}{codecs.Codec}}
&&
&&
  \\\cline{3-3}
  &&\multicolumn{1}{c|}{}
&&
&&
  \\
% Line for codecs.StreamReader, linespec=[False]
\multicolumn{4}{r}{\settowidth{\BCL}{codecs.StreamReader}\multirow{2}{\BCL}{codecs.StreamReader}}
&&
  \\\cline{5-5}
  &&&&\multicolumn{1}{c|}{}
&&
  \\
&&&&\multicolumn{2}{l}{\textbf{cssutils.codec.UTF8SigStreamReader}}
\end{tabular}


%%%%%%%%%%%%%%%%%%%%%%%%%%%%%%%%%%%%%%%%%%%%%%%%%%%%%%%%%%%%%%%%%%%%%%%%%%%
%%                                Methods                                %%
%%%%%%%%%%%%%%%%%%%%%%%%%%%%%%%%%%%%%%%%%%%%%%%%%%%%%%%%%%%%%%%%%%%%%%%%%%%

  \subsubsection{Methods}

    \vspace{0.5ex}

\hspace{.8\funcindent}\begin{boxedminipage}{\funcwidth}

    \raggedright \textbf{reset}(\textit{self})

\setlength{\parskip}{2ex}
    Resets the codec buffers used for keeping state.

    Note that no stream repositioning should take place. This method is 
    primarily intended to be able to recover from decoding errors.

\setlength{\parskip}{1ex}
      Overrides: codecs.StreamReader.reset 	extit{(inherited documentation)}

    \end{boxedminipage}

    \vspace{0.5ex}

\hspace{.8\funcindent}\begin{boxedminipage}{\funcwidth}

    \raggedright \textbf{decode}(\textit{self}, \textit{input}, \textit{errors}={\tt 'strict'})

\setlength{\parskip}{2ex}
    Decodes the object input and returns a tuple (output object, length 
    consumed).

    input must be an object which provides the bf\_getreadbuf buffer slot. 
    Python strings, buffer objects and memory mapped files are examples of 
    objects providing this slot.

    errors defines the error handling to apply. It defaults to 'strict' 
    handling.

    The method may not store state in the Codec instance. Use StreamCodec 
    for codecs which have to keep state in order to make encoding/decoding 
    efficient.

    The decoder must be able to handle zero length input and return an 
    empty object of the output object type in this situation.

\setlength{\parskip}{1ex}
      Overrides: codecs.Codec.decode 	extit{(inherited documentation)}

    \end{boxedminipage}


\large{\textbf{\textit{Inherited from codecs.StreamReader}}}

\begin{quote}
\_\_enter\_\_(), \_\_exit\_\_(), \_\_getattr\_\_(), \_\_init\_\_(), \_\_iter\_\_(), next(), read(), readline(), readlines(), seek()
\end{quote}

\large{\textbf{\textit{Inherited from codecs.Codec}}}

\begin{quote}
encode()
\end{quote}
    \index{cssutils \textit{(package)}!cssutils.codec \textit{(module)}!cssutils.codec.UTF8SigStreamReader \textit{(class)}|)}
    \index{cssutils \textit{(package)}!cssutils.codec \textit{(module)}|)}
