%
% API Documentation for cssutils
% Package cssutils.stylesheets
%
% Generated by epydoc 3.0.1
% [Fri Feb 01 19:05:22 2008]
%

%%%%%%%%%%%%%%%%%%%%%%%%%%%%%%%%%%%%%%%%%%%%%%%%%%%%%%%%%%%%%%%%%%%%%%%%%%%
%%                          Module Description                           %%
%%%%%%%%%%%%%%%%%%%%%%%%%%%%%%%%%%%%%%%%%%%%%%%%%%%%%%%%%%%%%%%%%%%%%%%%%%%

    \index{cssutils \textit{(package)}!cssutils.stylesheets \textit{(package)}|(}
\section{Package cssutils.stylesheets}

    \label{cssutils:stylesheets}

Document Object Model Level 2 Style Sheets
\href{http://www.w3.org/TR/2000/PR-DOM-Level-2-Style-20000927/stylesheets.html}{http://www.w3.org/TR/2000/PR-DOM-Level-2-Style-20000927/stylesheets.html}
\begin{description}
\item[{currently implemented:}] \leavevmode \begin{itemize}
\item {} 
MediaList

\item {} 
MediaQuery (\href{http://www.w3.org/TR/css3-mediaqueries/}{http://www.w3.org/TR/css3-mediaqueries/})

\item {} 
StyleSheet

\item {} 
StyleSheetList

\end{itemize}

\end{description}
\textbf{Version:} \$LastChangedRevision: 883 \$



\textbf{Date:} \$LastChangedDate: 2008-01-19 22:58:42 +0100 (Sa, 19 Jan 2008) \$



\textbf{Author:} \$LastChangedBy: cthedot \$




%%%%%%%%%%%%%%%%%%%%%%%%%%%%%%%%%%%%%%%%%%%%%%%%%%%%%%%%%%%%%%%%%%%%%%%%%%%
%%                                Modules                                %%
%%%%%%%%%%%%%%%%%%%%%%%%%%%%%%%%%%%%%%%%%%%%%%%%%%%%%%%%%%%%%%%%%%%%%%%%%%%

\subsection{Modules}

\begin{itemize}
\setlength{\parskip}{0ex}
\item \textbf{medialist}: 
MediaList implements DOM Level 2 Style Sheets MediaList.


  \textit{(Section \ref{cssutils:stylesheets:medialist}, p.~\pageref{cssutils:stylesheets:medialist})}

\item \textbf{mediaquery}: 
MediaQuery, see \href{http://www.w3.org/TR/css3-mediaqueries/}{http://www.w3.org/TR/css3-mediaqueries/}


  \textit{(Section \ref{cssutils:stylesheets:mediaquery}, p.~\pageref{cssutils:stylesheets:mediaquery})}

\item \textbf{stylesheet}: 
StyleSheet implements DOM Level 2 Style Sheets StyleSheet.


  \textit{(Section \ref{cssutils:stylesheets:stylesheet}, p.~\pageref{cssutils:stylesheets:stylesheet})}

\item \textbf{stylesheetlist}: 
StyleSheetList implements DOM Level 2 Style Sheets StyleSheetList.


  \textit{(Section \ref{cssutils:stylesheets:stylesheetlist}, p.~\pageref{cssutils:stylesheets:stylesheetlist})}

\end{itemize}


%%%%%%%%%%%%%%%%%%%%%%%%%%%%%%%%%%%%%%%%%%%%%%%%%%%%%%%%%%%%%%%%%%%%%%%%%%%
%%                           Class Description                           %%
%%%%%%%%%%%%%%%%%%%%%%%%%%%%%%%%%%%%%%%%%%%%%%%%%%%%%%%%%%%%%%%%%%%%%%%%%%%

    \index{cssutils \textit{(package)}!cssutils.stylesheets \textit{(package)}!cssutils.stylesheets.medialist \textit{(module)}!cssutils.stylesheets.medialist.MediaList \textit{(class)}|(}
\subsection{Class MediaList}

    \label{cssutils:stylesheets:medialist:MediaList}
\begin{tabular}{cccccccc}
% Line for object, linespec=[False, False]
\multicolumn{2}{r}{\settowidth{\BCL}{object}\multirow{2}{\BCL}{object}}
&&
&&
  \\\cline{3-3}
  &&\multicolumn{1}{c|}{}
&&
&&
  \\
% Line for cssutils.util.Base, linespec=[False]
\multicolumn{4}{r}{\settowidth{\BCL}{cssutils.util.Base}\multirow{2}{\BCL}{cssutils.util.Base}}
&&
  \\\cline{5-5}
  &&&&\multicolumn{1}{c|}{}
&&
  \\
% Line for object, linespec=[False, True]
\multicolumn{2}{r}{\settowidth{\BCL}{object}\multirow{2}{\BCL}{object}}
&&
&&\multicolumn{1}{|c}{}
  \\\cline{3-3}
  &&\multicolumn{1}{c|}{}
&&
&\multicolumn{1}{|c}{}&
  \\
% Line for cssutils.util.ListSeq, linespec=[True]
\multicolumn{4}{r}{\settowidth{\BCL}{cssutils.util.ListSeq}\multirow{2}{\BCL}{cssutils.util.ListSeq}}
&&\multicolumn{1}{|c}{}
  \\\cline{5-5}
  &&&&\multicolumn{1}{c|}{}
&\multicolumn{1}{|c}{}&
  \\
&&&&\multicolumn{2}{l}{\textbf{cssutils.stylesheets.medialist.MediaList}}
\end{tabular}


Provides the abstraction of an ordered collection of media,
without defining or constraining how this collection is
implemented.

A media is always an instance of MediaQuery.

An empty list is the same as a list that contains the medium ``all''.


%___________________________________________________________________________

\hypertarget{properties}{}
\pdfbookmark[3]{Properties}{properties}
\paragraph*{Properties}
\label{properties}
\begin{description}
\item[{length:}] \leavevmode 
The number of MediaQuery objects in the list.

\item[{mediaText: of type DOMString}] \leavevmode 
The parsable textual representation of this MediaList

\item[{self: a list (cssutils)}] \leavevmode 
All MediaQueries in this MediaList

\item[{valid:}] \leavevmode 
if this list is valid

\end{description}


%___________________________________________________________________________

\hypertarget{format}{}
\pdfbookmark[3]{Format}{format}
\paragraph*{Format}
\label{format}
\begin{quote}{\ttfamily \raggedright \noindent
medium~{[}~COMMA~S*~medium~{]}*
}\end{quote}

New:
\begin{quote}{\ttfamily \raggedright \noindent
<media{\_}query>~{[},~<media{\_}query>~{]}*
}\end{quote}

%%%%%%%%%%%%%%%%%%%%%%%%%%%%%%%%%%%%%%%%%%%%%%%%%%%%%%%%%%%%%%%%%%%%%%%%%%%
%%                                Methods                                %%
%%%%%%%%%%%%%%%%%%%%%%%%%%%%%%%%%%%%%%%%%%%%%%%%%%%%%%%%%%%%%%%%%%%%%%%%%%%

  \subsubsection{Methods}

    \vspace{0.5ex}

\hspace{.8\funcindent}\begin{boxedminipage}{\funcwidth}

    \raggedright \textbf{\_\_init\_\_}(\textit{self}, \textit{mediaText}={\tt None}, \textit{readonly}={\tt False})

    \vspace{-1.5ex}

    \rule{\textwidth}{0.5\fboxrule}
\setlength{\parskip}{2ex}
\begin{description}
\item[{mediaText}] \leavevmode 
unicodestring of parsable comma separared media
or a list of media

\end{description}
\setlength{\parskip}{1ex}
      Overrides: object.\_\_init\_\_

    \end{boxedminipage}

    \vspace{0.5ex}

\hspace{.8\funcindent}\begin{boxedminipage}{\funcwidth}

    \raggedright \textbf{\_\_setitem\_\_}(\textit{self}, \textit{index}, \textit{newMedium})

    \vspace{-1.5ex}

    \rule{\textwidth}{0.5\fboxrule}
\setlength{\parskip}{2ex}

overwrites ListSeq.{\_}{\_}setitem{\_}{\_}

Any duplicate items are \textbf{not} removed.
\setlength{\parskip}{1ex}
      Overrides: cssutils.util.ListSeq.\_\_setitem\_\_

    \end{boxedminipage}

    \label{cssutils:stylesheets:medialist:MediaList:appendMedium}
    \index{cssutils \textit{(package)}!cssutils.stylesheets \textit{(package)}!cssutils.stylesheets.medialist \textit{(module)}!cssutils.stylesheets.medialist.MediaList \textit{(class)}!cssutils.stylesheets.medialist.MediaList.appendMedium \textit{(method)}}

    \vspace{0.5ex}

\hspace{.8\funcindent}\begin{boxedminipage}{\funcwidth}

    \raggedright \textbf{appendMedium}(\textit{self}, \textit{newMedium})

    \vspace{-1.5ex}

    \rule{\textwidth}{0.5\fboxrule}
\setlength{\parskip}{2ex}

(DOM)
Adds the medium newMedium to the end of the list. If the newMedium
is already used, it is first removed.
\begin{description}
\item[{newMedium}] \leavevmode 
a string or a MediaQuery object

\end{description}

returns if newMedium is valid

DOMException
\begin{itemize}
\item {} 
INVALID{\_}CHARACTER{\_}ERR: (self)
If the medium contains characters that are invalid in the
underlying style language.

\item {} 
NO{\_}MODIFICATION{\_}ALLOWED{\_}ERR: (self)
Raised if this list is readonly.

\end{itemize}
\setlength{\parskip}{1ex}
    \end{boxedminipage}

    \vspace{0.5ex}

\hspace{.8\funcindent}\begin{boxedminipage}{\funcwidth}

    \raggedright \textbf{append}(\textit{self}, \textit{newMedium})

    \vspace{-1.5ex}

    \rule{\textwidth}{0.5\fboxrule}
\setlength{\parskip}{2ex}

overwrites ListSeq.append
\setlength{\parskip}{1ex}
      Overrides: cssutils.util.ListSeq.append

    \end{boxedminipage}

    \label{cssutils:stylesheets:medialist:MediaList:deleteMedium}
    \index{cssutils \textit{(package)}!cssutils.stylesheets \textit{(package)}!cssutils.stylesheets.medialist \textit{(module)}!cssutils.stylesheets.medialist.MediaList \textit{(class)}!cssutils.stylesheets.medialist.MediaList.deleteMedium \textit{(method)}}

    \vspace{0.5ex}

\hspace{.8\funcindent}\begin{boxedminipage}{\funcwidth}

    \raggedright \textbf{deleteMedium}(\textit{self}, \textit{oldMedium})

    \vspace{-1.5ex}

    \rule{\textwidth}{0.5\fboxrule}
\setlength{\parskip}{2ex}

(DOM)
Deletes the medium indicated by oldMedium from the list.

DOMException
\begin{itemize}
\item {} 
NO{\_}MODIFICATION{\_}ALLOWED{\_}ERR: (self)
Raised if this list is readonly.

\item {} 
NOT{\_}FOUND{\_}ERR: (self)
Raised if oldMedium is not in the list.

\end{itemize}
\setlength{\parskip}{1ex}
    \end{boxedminipage}

    \label{cssutils:stylesheets:medialist:MediaList:item}
    \index{cssutils \textit{(package)}!cssutils.stylesheets \textit{(package)}!cssutils.stylesheets.medialist \textit{(module)}!cssutils.stylesheets.medialist.MediaList \textit{(class)}!cssutils.stylesheets.medialist.MediaList.item \textit{(method)}}

    \vspace{0.5ex}

\hspace{.8\funcindent}\begin{boxedminipage}{\funcwidth}

    \raggedright \textbf{item}(\textit{self}, \textit{index})

    \vspace{-1.5ex}

    \rule{\textwidth}{0.5\fboxrule}
\setlength{\parskip}{2ex}

(DOM)
Returns the mediaType of the index'th element in the list.
If index is greater than or equal to the number of media in the
list, returns None.
\setlength{\parskip}{1ex}
    \end{boxedminipage}

    \vspace{0.5ex}

\hspace{.8\funcindent}\begin{boxedminipage}{\funcwidth}

    \raggedright \textbf{\_\_repr\_\_}(\textit{self})

\setlength{\parskip}{2ex}
    repr(x)

\setlength{\parskip}{1ex}
      Overrides: object.\_\_repr\_\_ 	extit{(inherited documentation)}

    \end{boxedminipage}

    \vspace{0.5ex}

\hspace{.8\funcindent}\begin{boxedminipage}{\funcwidth}

    \raggedright \textbf{\_\_str\_\_}(\textit{self})

\setlength{\parskip}{2ex}
    str(x)

\setlength{\parskip}{1ex}
      Overrides: object.\_\_str\_\_ 	extit{(inherited documentation)}

    \end{boxedminipage}


\large{\textbf{\textit{Inherited from cssutils.util.ListSeq}}}

\begin{quote}
\_\_contains\_\_(), \_\_delitem\_\_(), \_\_getitem\_\_(), \_\_iter\_\_(), \_\_len\_\_()
\end{quote}

\large{\textbf{\textit{Inherited from object}}}

\begin{quote}
\_\_delattr\_\_(), \_\_getattribute\_\_(), \_\_hash\_\_(), \_\_new\_\_(), \_\_reduce\_\_(), \_\_reduce\_ex\_\_(), \_\_setattr\_\_()
\end{quote}

%%%%%%%%%%%%%%%%%%%%%%%%%%%%%%%%%%%%%%%%%%%%%%%%%%%%%%%%%%%%%%%%%%%%%%%%%%%
%%                              Properties                               %%
%%%%%%%%%%%%%%%%%%%%%%%%%%%%%%%%%%%%%%%%%%%%%%%%%%%%%%%%%%%%%%%%%%%%%%%%%%%

  \subsubsection{Properties}

    \vspace{-1cm}
\hspace{\varindent}\begin{longtable}{|p{\varnamewidth}|p{\vardescrwidth}|l}
\cline{1-2}
\cline{1-2} \centering \textbf{Name} & \centering \textbf{Description}& \\
\cline{1-2}
\endhead\cline{1-2}\multicolumn{3}{r}{\small\textit{continued on next page}}\\\endfoot\cline{1-2}
\endlastfoot\raggedright l\-e\-n\-g\-t\-h\- & \raggedright (DOM readonly) The number of media in the list.&\\
\cline{1-2}
\raggedright m\-e\-d\-i\-a\-T\-e\-x\-t\- & \raggedright (DOM) The parsable textual representation of the media list.
This is a comma-separated list of media.&\\
\cline{1-2}
\multicolumn{2}{|l|}{\textit{Inherited from object}}\\
\multicolumn{2}{|p{\varwidth}|}{\raggedright \_\_class\_\_}\\
\cline{1-2}
\end{longtable}

    \index{cssutils \textit{(package)}!cssutils.stylesheets \textit{(package)}!cssutils.stylesheets.medialist \textit{(module)}!cssutils.stylesheets.medialist.MediaList \textit{(class)}|)}

%%%%%%%%%%%%%%%%%%%%%%%%%%%%%%%%%%%%%%%%%%%%%%%%%%%%%%%%%%%%%%%%%%%%%%%%%%%
%%                           Class Description                           %%
%%%%%%%%%%%%%%%%%%%%%%%%%%%%%%%%%%%%%%%%%%%%%%%%%%%%%%%%%%%%%%%%%%%%%%%%%%%

    \index{cssutils \textit{(package)}!cssutils.stylesheets \textit{(package)}!cssutils.stylesheets.mediaquery \textit{(module)}!cssutils.stylesheets.mediaquery.MediaQuery \textit{(class)}|(}
\subsection{Class MediaQuery}

    \label{cssutils:stylesheets:mediaquery:MediaQuery}
\begin{tabular}{cccccccc}
% Line for object, linespec=[False, False]
\multicolumn{2}{r}{\settowidth{\BCL}{object}\multirow{2}{\BCL}{object}}
&&
&&
  \\\cline{3-3}
  &&\multicolumn{1}{c|}{}
&&
&&
  \\
% Line for cssutils.util.Base, linespec=[False]
\multicolumn{4}{r}{\settowidth{\BCL}{cssutils.util.Base}\multirow{2}{\BCL}{cssutils.util.Base}}
&&
  \\\cline{5-5}
  &&&&\multicolumn{1}{c|}{}
&&
  \\
&&&&\multicolumn{2}{l}{\textbf{cssutils.stylesheets.mediaquery.MediaQuery}}
\end{tabular}


A Media Query consists of a media type and one or more
expressions involving media features.


%___________________________________________________________________________

\hypertarget{properties}{}
\pdfbookmark[3]{Properties}{properties}
\paragraph*{Properties}
\label{properties}
\begin{description}
\item[{mediaText: of type DOMString}] \leavevmode 
The parsable textual representation of this MediaQuery

\item[{mediaType: of type DOMString}] \leavevmode 
one of MEDIA{\_}TYPES like e.g. 'print'

\item[{seq: a list (cssutils)}] \leavevmode 
All parts of this MediaQuery including CSSComments

\item[{valid:}] \leavevmode 
if this query is valid

\end{description}


%___________________________________________________________________________

\hypertarget{format}{}
\pdfbookmark[3]{Format}{format}
\paragraph*{Format}
\label{format}
\begin{quote}{\ttfamily \raggedright \noindent
media{\_}query:~{[}{[}only~|~not{]}?~<media{\_}type>~{[}~and~<expression>~{]}*{]}~\\
~~|~<expression>~{[}~and~<expression>~{]}*~\\
expression:~(~<media{\_}feature>~{[}:~<value>{]}?~)~\\
media{\_}type:~all~|~aural~|~braille~|~handheld~|~print~|~\\
~~projection~|~screen~|~tty~|~tv~|~embossed~\\
media{\_}feature:~width~|~min-width~|~max-width~\\
~~|~height~|~min-height~|~max-height~\\
~~|~device-width~|~min-device-width~|~max-device-width~\\
~~|~device-height~|~min-device-height~|~max-device-height~\\
~~|~device-aspect-ratio~|~min-device-aspect-ratio~|~max-device-aspect-ratio~\\
~~|~color~|~min-color~|~max-color~\\
~~|~color-index~|~min-color-index~|~max-color-index~\\
~~|~monochrome~|~min-monochrome~|~max-monochrome~\\
~~|~resolution~|~min-resolution~|~max-resolution~\\
~~|~scan~|~grid
}\end{quote}

%%%%%%%%%%%%%%%%%%%%%%%%%%%%%%%%%%%%%%%%%%%%%%%%%%%%%%%%%%%%%%%%%%%%%%%%%%%
%%                                Methods                                %%
%%%%%%%%%%%%%%%%%%%%%%%%%%%%%%%%%%%%%%%%%%%%%%%%%%%%%%%%%%%%%%%%%%%%%%%%%%%

  \subsubsection{Methods}

    \vspace{0.5ex}

\hspace{.8\funcindent}\begin{boxedminipage}{\funcwidth}

    \raggedright \textbf{\_\_init\_\_}(\textit{self}, \textit{mediaText}={\tt None}, \textit{readonly}={\tt False})

    \vspace{-1.5ex}

    \rule{\textwidth}{0.5\fboxrule}
\setlength{\parskip}{2ex}
\begin{description}
\item[{mediaText}] \leavevmode 
unicodestring of parsable media

\end{description}
\setlength{\parskip}{1ex}
      Overrides: object.\_\_init\_\_

    \end{boxedminipage}

    \vspace{0.5ex}

\hspace{.8\funcindent}\begin{boxedminipage}{\funcwidth}

    \raggedright \textbf{\_\_repr\_\_}(\textit{self})

\setlength{\parskip}{2ex}
    repr(x)

\setlength{\parskip}{1ex}
      Overrides: object.\_\_repr\_\_ 	extit{(inherited documentation)}

    \end{boxedminipage}

    \vspace{0.5ex}

\hspace{.8\funcindent}\begin{boxedminipage}{\funcwidth}

    \raggedright \textbf{\_\_str\_\_}(\textit{self})

\setlength{\parskip}{2ex}
    str(x)

\setlength{\parskip}{1ex}
      Overrides: object.\_\_str\_\_ 	extit{(inherited documentation)}

    \end{boxedminipage}


\large{\textbf{\textit{Inherited from object}}}

\begin{quote}
\_\_delattr\_\_(), \_\_getattribute\_\_(), \_\_hash\_\_(), \_\_new\_\_(), \_\_reduce\_\_(), \_\_reduce\_ex\_\_(), \_\_setattr\_\_()
\end{quote}

%%%%%%%%%%%%%%%%%%%%%%%%%%%%%%%%%%%%%%%%%%%%%%%%%%%%%%%%%%%%%%%%%%%%%%%%%%%
%%                              Properties                               %%
%%%%%%%%%%%%%%%%%%%%%%%%%%%%%%%%%%%%%%%%%%%%%%%%%%%%%%%%%%%%%%%%%%%%%%%%%%%

  \subsubsection{Properties}

    \vspace{-1cm}
\hspace{\varindent}\begin{longtable}{|p{\varnamewidth}|p{\vardescrwidth}|l}
\cline{1-2}
\cline{1-2} \centering \textbf{Name} & \centering \textbf{Description}& \\
\cline{1-2}
\endhead\cline{1-2}\multicolumn{3}{r}{\small\textit{continued on next page}}\\\endfoot\cline{1-2}
\endlastfoot\raggedright m\-e\-d\-i\-a\-T\-e\-x\-t\- & \raggedright (DOM) The parsable textual representation of the media list.
This is a comma-separated list of media.&\\
\cline{1-2}
\raggedright m\-e\-d\-i\-a\-T\-y\-p\-e\- & \raggedright (DOM) media type (one of MediaQuery.MEDIA{\_}TYPES) of this MediaQuery.&\\
\cline{1-2}
\multicolumn{2}{|l|}{\textit{Inherited from object}}\\
\multicolumn{2}{|p{\varwidth}|}{\raggedright \_\_class\_\_}\\
\cline{1-2}
\end{longtable}


%%%%%%%%%%%%%%%%%%%%%%%%%%%%%%%%%%%%%%%%%%%%%%%%%%%%%%%%%%%%%%%%%%%%%%%%%%%
%%                            Class Variables                            %%
%%%%%%%%%%%%%%%%%%%%%%%%%%%%%%%%%%%%%%%%%%%%%%%%%%%%%%%%%%%%%%%%%%%%%%%%%%%

  \subsubsection{Class Variables}

    \vspace{-1cm}
\hspace{\varindent}\begin{longtable}{|p{\varnamewidth}|p{\vardescrwidth}|l}
\cline{1-2}
\cline{1-2} \centering \textbf{Name} & \centering \textbf{Description}& \\
\cline{1-2}
\endhead\cline{1-2}\multicolumn{3}{r}{\small\textit{continued on next page}}\\\endfoot\cline{1-2}
\endlastfoot\raggedright M\-E\-D\-I\-A\-\_\-T\-Y\-P\-E\-S\- & \raggedright \textbf{Value:} 
{\tt \texttt{[}\texttt{u'}\texttt{all}\texttt{'}\texttt{, }\texttt{u'}\texttt{aural}\texttt{'}\texttt{, }\texttt{u'}\texttt{braille}\texttt{'}\texttt{, }\texttt{u'}\texttt{embossed}\texttt{'}\texttt{, }\texttt{u'}\texttt{handheld}\texttt{'}\texttt{, }\texttt{...}}&\\
\cline{1-2}
\end{longtable}

    \index{cssutils \textit{(package)}!cssutils.stylesheets \textit{(package)}!cssutils.stylesheets.mediaquery \textit{(module)}!cssutils.stylesheets.mediaquery.MediaQuery \textit{(class)}|)}

%%%%%%%%%%%%%%%%%%%%%%%%%%%%%%%%%%%%%%%%%%%%%%%%%%%%%%%%%%%%%%%%%%%%%%%%%%%
%%                           Class Description                           %%
%%%%%%%%%%%%%%%%%%%%%%%%%%%%%%%%%%%%%%%%%%%%%%%%%%%%%%%%%%%%%%%%%%%%%%%%%%%

    \index{cssutils \textit{(package)}!cssutils.stylesheets \textit{(package)}!cssutils.stylesheets.stylesheet \textit{(module)}!cssutils.stylesheets.stylesheet.StyleSheet \textit{(class)}|(}
\subsection{Class StyleSheet}

    \label{cssutils:stylesheets:stylesheet:StyleSheet}
\begin{tabular}{cccccccc}
% Line for object, linespec=[False, False]
\multicolumn{2}{r}{\settowidth{\BCL}{object}\multirow{2}{\BCL}{object}}
&&
&&
  \\\cline{3-3}
  &&\multicolumn{1}{c|}{}
&&
&&
  \\
% Line for cssutils.util.Base, linespec=[False]
\multicolumn{4}{r}{\settowidth{\BCL}{cssutils.util.Base}\multirow{2}{\BCL}{cssutils.util.Base}}
&&
  \\\cline{5-5}
  &&&&\multicolumn{1}{c|}{}
&&
  \\
&&&&\multicolumn{2}{l}{\textbf{cssutils.stylesheets.stylesheet.StyleSheet}}
\end{tabular}

\textbf{Known Subclasses:} cssutils.css.cssstylesheet.CSSStyleSheet


The StyleSheet interface is the abstract base interface
for any type of style sheet. It represents a single style
sheet associated with a structured document.

In HTML, the StyleSheet interface represents either an
external style sheet, included via the HTML LINK element,
or an inline STYLE element (-ch: also an @import stylesheet?).

In XML, this interface represents
an external style sheet, included via a style sheet
processing instruction.

%%%%%%%%%%%%%%%%%%%%%%%%%%%%%%%%%%%%%%%%%%%%%%%%%%%%%%%%%%%%%%%%%%%%%%%%%%%
%%                                Methods                                %%
%%%%%%%%%%%%%%%%%%%%%%%%%%%%%%%%%%%%%%%%%%%%%%%%%%%%%%%%%%%%%%%%%%%%%%%%%%%

  \subsubsection{Methods}

    \vspace{0.5ex}

\hspace{.8\funcindent}\begin{boxedminipage}{\funcwidth}

    \raggedright \textbf{\_\_init\_\_}(\textit{self}, \textit{type}={\tt \texttt{'}\texttt{text/css}\texttt{'}}, \textit{href}={\tt None}, \textit{media}={\tt None}, \textit{title}={\tt \texttt{u'}\texttt{}\texttt{'}}, \textit{disabled}={\tt None}, \textit{ownerNode}={\tt None}, \textit{parentStyleSheet}={\tt None})

    \vspace{-1.5ex}

    \rule{\textwidth}{0.5\fboxrule}
\setlength{\parskip}{2ex}
\begin{description}
\item[{type: readonly}] \leavevmode 
This specifies the style sheet language for this
style sheet. The style sheet language is specified
as a content type (e.g. ``text/css''). The content
type is often specified in the ownerNode. Also see
the type attribute definition for the LINK element
in HTML 4.0, and the type pseudo-attribute for the
XML style sheet processing instruction.

\item[{href: readonly}] \leavevmode 
If the style sheet is a linked style sheet, the value
of this attribute is its location. For inline style
sheets, the value of this attribute is None. See the
href attribute definition for the LINK element in HTML
4.0, and the href pseudo-attribute for the XML style
sheet processing instruction.

\item[{media: of type MediaList, readonly}] \leavevmode 
The intended destination media for style information.
The media is often specified in the ownerNode. If no
media has been specified, the MediaList will be empty.
See the media attribute definition for the LINK element
in HTML 4.0, and the media pseudo-attribute for the XML
style sheet processing instruction. Modifying the media
list may cause a change to the attribute disabled.

\item[{title: readonly}] \leavevmode 
The advisory title. The title is often specified in
the ownerNode. See the title attribute definition for
the LINK element in HTML 4.0, and the title
pseudo-attribute for the XML style sheet processing
instruction.

\item[{disabled: False if the style sheet is applied to the}] \leavevmode 
document. True if it is not. Modifying this attribute
may cause a new resolution of style for the document.
A stylesheet only applies if both an appropriate medium
definition is present and the disabled attribute is False.
So, if the media doesn't apply to the current user agent,
the disabled attribute is ignored.

\item[{ownerNode: of type Node, readonly}] \leavevmode 
The node that associates this style sheet with the
document. For HTML, this may be the corresponding LINK
or STYLE element. For XML, it may be the linking
processing instruction. For style sheets that are
included by other style sheets, the value of this
attribute is None.

\item[{parentStyleSheet: of type StyleSheet, readonly}] \leavevmode 
For style sheet languages that support the concept
of style sheet inclusion, this attribute represents
the including style sheet, if one exists. If the style
sheet is a top-level style sheet, or the style sheet
language does not support inclusion, the value of this
attribute is None.

\end{description}
\setlength{\parskip}{1ex}
      Overrides: object.\_\_init\_\_

    \end{boxedminipage}


\large{\textbf{\textit{Inherited from object}}}

\begin{quote}
\_\_delattr\_\_(), \_\_getattribute\_\_(), \_\_hash\_\_(), \_\_new\_\_(), \_\_reduce\_\_(), \_\_reduce\_ex\_\_(), \_\_repr\_\_(), \_\_setattr\_\_(), \_\_str\_\_()
\end{quote}

%%%%%%%%%%%%%%%%%%%%%%%%%%%%%%%%%%%%%%%%%%%%%%%%%%%%%%%%%%%%%%%%%%%%%%%%%%%
%%                              Properties                               %%
%%%%%%%%%%%%%%%%%%%%%%%%%%%%%%%%%%%%%%%%%%%%%%%%%%%%%%%%%%%%%%%%%%%%%%%%%%%

  \subsubsection{Properties}

    \vspace{-1cm}
\hspace{\varindent}\begin{longtable}{|p{\varnamewidth}|p{\vardescrwidth}|l}
\cline{1-2}
\cline{1-2} \centering \textbf{Name} & \centering \textbf{Description}& \\
\cline{1-2}
\endhead\cline{1-2}\multicolumn{3}{r}{\small\textit{continued on next page}}\\\endfoot\cline{1-2}
\endlastfoot\raggedright p\-a\-r\-e\-n\-t\-S\-t\-y\-l\-e\-S\-h\-e\-e\-t\- & &\\
\cline{1-2}
\multicolumn{2}{|l|}{\textit{Inherited from object}}\\
\multicolumn{2}{|p{\varwidth}|}{\raggedright \_\_class\_\_}\\
\cline{1-2}
\end{longtable}

    \index{cssutils \textit{(package)}!cssutils.stylesheets \textit{(package)}!cssutils.stylesheets.stylesheet \textit{(module)}!cssutils.stylesheets.stylesheet.StyleSheet \textit{(class)}|)}

%%%%%%%%%%%%%%%%%%%%%%%%%%%%%%%%%%%%%%%%%%%%%%%%%%%%%%%%%%%%%%%%%%%%%%%%%%%
%%                           Class Description                           %%
%%%%%%%%%%%%%%%%%%%%%%%%%%%%%%%%%%%%%%%%%%%%%%%%%%%%%%%%%%%%%%%%%%%%%%%%%%%

    \index{cssutils \textit{(package)}!cssutils.stylesheets \textit{(package)}!cssutils.stylesheets.stylesheetlist \textit{(module)}!cssutils.stylesheets.stylesheetlist.StyleSheetList \textit{(class)}|(}
\subsection{Class StyleSheetList}

    \label{cssutils:stylesheets:stylesheetlist:StyleSheetList}
\begin{tabular}{cccccccc}
% Line for object, linespec=[False, False]
\multicolumn{2}{r}{\settowidth{\BCL}{object}\multirow{2}{\BCL}{object}}
&&
&&
  \\\cline{3-3}
  &&\multicolumn{1}{c|}{}
&&
&&
  \\
% Line for list, linespec=[False]
\multicolumn{4}{r}{\settowidth{\BCL}{list}\multirow{2}{\BCL}{list}}
&&
  \\\cline{5-5}
  &&&&\multicolumn{1}{c|}{}
&&
  \\
&&&&\multicolumn{2}{l}{\textbf{cssutils.stylesheets.stylesheetlist.StyleSheetList}}
\end{tabular}


Interface StyleSheetList (introduced in DOM Level 2)

The StyleSheetList interface provides the abstraction of an ordered
collection of style sheets.

The items in the StyleSheetList are accessible via an integral index,
starting from 0.

This Python implementation is based on a standard Python list so e.g.
allows \texttt{examplelist{[}index{]}} usage.

%%%%%%%%%%%%%%%%%%%%%%%%%%%%%%%%%%%%%%%%%%%%%%%%%%%%%%%%%%%%%%%%%%%%%%%%%%%
%%                                Methods                                %%
%%%%%%%%%%%%%%%%%%%%%%%%%%%%%%%%%%%%%%%%%%%%%%%%%%%%%%%%%%%%%%%%%%%%%%%%%%%

  \subsubsection{Methods}

    \label{cssutils:stylesheets:stylesheetlist:StyleSheetList:item}
    \index{cssutils \textit{(package)}!cssutils.stylesheets \textit{(package)}!cssutils.stylesheets.stylesheetlist \textit{(module)}!cssutils.stylesheets.stylesheetlist.StyleSheetList \textit{(class)}!cssutils.stylesheets.stylesheetlist.StyleSheetList.item \textit{(method)}}

    \vspace{0.5ex}

\hspace{.8\funcindent}\begin{boxedminipage}{\funcwidth}

    \raggedright \textbf{item}(\textit{self}, \textit{index})

    \vspace{-1.5ex}

    \rule{\textwidth}{0.5\fboxrule}
\setlength{\parskip}{2ex}

Used to retrieve a style sheet by ordinal index. If index is
greater than or equal to the number of style sheets in the list,
this returns None.
\setlength{\parskip}{1ex}
    \end{boxedminipage}


\large{\textbf{\textit{Inherited from list}}}

\begin{quote}
\_\_add\_\_(), \_\_contains\_\_(), \_\_delitem\_\_(), \_\_delslice\_\_(), \_\_eq\_\_(), \_\_ge\_\_(), \_\_getattribute\_\_(), \_\_getitem\_\_(), \_\_getslice\_\_(), \_\_gt\_\_(), \_\_hash\_\_(), \_\_iadd\_\_(), \_\_imul\_\_(), \_\_init\_\_(), \_\_iter\_\_(), \_\_le\_\_(), \_\_len\_\_(), \_\_lt\_\_(), \_\_mul\_\_(), \_\_ne\_\_(), \_\_new\_\_(), \_\_repr\_\_(), \_\_reversed\_\_(), \_\_rmul\_\_(), \_\_setitem\_\_(), \_\_setslice\_\_(), append(), count(), extend(), index(), insert(), pop(), remove(), reverse(), sort()
\end{quote}

\large{\textbf{\textit{Inherited from object}}}

\begin{quote}
\_\_delattr\_\_(), \_\_reduce\_\_(), \_\_reduce\_ex\_\_(), \_\_setattr\_\_(), \_\_str\_\_()
\end{quote}

%%%%%%%%%%%%%%%%%%%%%%%%%%%%%%%%%%%%%%%%%%%%%%%%%%%%%%%%%%%%%%%%%%%%%%%%%%%
%%                              Properties                               %%
%%%%%%%%%%%%%%%%%%%%%%%%%%%%%%%%%%%%%%%%%%%%%%%%%%%%%%%%%%%%%%%%%%%%%%%%%%%

  \subsubsection{Properties}

    \vspace{-1cm}
\hspace{\varindent}\begin{longtable}{|p{\varnamewidth}|p{\vardescrwidth}|l}
\cline{1-2}
\cline{1-2} \centering \textbf{Name} & \centering \textbf{Description}& \\
\cline{1-2}
\endhead\cline{1-2}\multicolumn{3}{r}{\small\textit{continued on next page}}\\\endfoot\cline{1-2}
\endlastfoot\raggedright l\-e\-n\-g\-t\-h\- & \raggedright The number of StyleSheets in the list. The range of valid
child stylesheet indices is 0 to length-1 inclusive.&\\
\cline{1-2}
\multicolumn{2}{|l|}{\textit{Inherited from object}}\\
\multicolumn{2}{|p{\varwidth}|}{\raggedright \_\_class\_\_}\\
\cline{1-2}
\end{longtable}

    \index{cssutils \textit{(package)}!cssutils.stylesheets \textit{(package)}!cssutils.stylesheets.stylesheetlist \textit{(module)}!cssutils.stylesheets.stylesheetlist.StyleSheetList \textit{(class)}|)}
    \index{cssutils \textit{(package)}!cssutils.stylesheets \textit{(package)}|)}
