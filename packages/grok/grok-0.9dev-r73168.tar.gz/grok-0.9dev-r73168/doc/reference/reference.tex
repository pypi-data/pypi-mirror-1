% Complete documentation on the extended LaTeX markup used for Python
% documentation is available in ``Documenting Python'', which is part
% of the standard documentation for Python.  It may be found online
% at:
%
%     http://www.python.org/doc/current/doc/doc.html

\documentclass{manual}
\RequirePackage[latin9]{inputenc}
\usepackage{graphicx}

\title{grok reference}

% Please at least include a long-lived email address;
% the rest is at your discretion.
\authoraddress{
    The grok team\\
    Email: <grok-dev@zope.org>
}

\date{\today}   % update before release!
                % Use an explicit date so that reformatting
                % doesn't cause a new date to be used.  Setting
                % the date to \today can be used during draft
                % stages to make it easier to handle versions.

\release{unreleased}      % release version; this is used to define the
                          % \version macro

\makeindex          % tell \index to actually write the .idx file

\begin{document}

\maketitle

``Grok means to understand so thoroughly that the observer becomes a part of the
observed --- merge, blend, intermarry, lose identity in group experience.
It means almost everything that we mean by religion, philosophy, and
science --- it means as little to us (because we are from Earth) as color
means to a blind man.'' -- Robert A. Heinlein, Stranger in a Strange Land

\begin{abstract}
This is the grok reference documentation. It is organized by the Python
artefacts that implement the concepts.

Grok makes Zope 3 concepts more accessible for application developers. This
reference is
not intended as introductory material for those concepts. Please refer to the 
original Zope 3 documentation and the grok tutorial for introductory material.
\end{abstract}


\tableofcontents

\chapter{Core}

The \module{grok} module defines a few functions to interact with grok itself.


\section{\function{grok.grok} -- Grok a package or module}

    \begin{funcdesc}{grok}{dotted_name}

    Grokking a package or module activates the contained components (like
    models, views, adapters, templates, etc.) and registers them with Zope 3's
    component architecture.

    The \var{dotted_name} must specify either a Python module or package
    that is available from the current PYTHONPATH.

    Grokking a module:

    \begin{enumerate}

        \item Scan the module for known components: models, adapters,
              utilities, views, traversers, templates and subscribers.

        \item Check whether a directory with file system templates
              exists (\file{<modulename>_templates}).  If it exists,
              load the file system templates into the template
              registry for this module.

        \item Determine the module context. 

        \item Register all components with the Zope 3 component architecture.

        \item Initialize schemata for registered models

    \end{enumerate}

    Grokking a package:

    \begin{enumerate}
        \item Grok the package as a module.

        \item Check for a static resource directory (\file{static})
          and register it if it exists.

        \item Recursively grok all sub-modules and sub-packages.

    \end{enumerate}

    \end{funcdesc}



\chapter{Components}

The \module{grok} module defines a set of components that provide basic Zope 3
functionality in a convenient way.

\section{\class{grok.Adapter}}

  Implementation, configuration, and registration of Zope 3 adapters.

  \begin{classdesc*}{grok.Adapter}
    Base class to define an adapter. Adapters are automatically registered when
    a module is "grokked".

    \begin{memberdesc}{context}
      The adapted object.
    \end{memberdesc}

  \begin{bf}Directives:\end{bf}

  \begin{itemize}
    \item[\function{grok.context(context_obj_or_interface)}] Maybe required.
    Identifies the type of objects or interface for the adaptation.

    If Grok can determine a context for adaptation from the module, this
    directive can be omitted. If the automatically determined context is not
    correct, or if no context can be derived from the module the directive is
    required.

    \item[\function{grok.implements(*interfaces)}] Required. Identifies the
    interface(s) the adapter implements.

    \item[\function{grok.name(name)}] Optional. Identifies the name used for
    the adapter registration. If ommitted, no name will be used.

    When a name is used for the adapter registration, the adapter can only be
    retrieved by explicitely using its name.

    \item[\function{grok.provides(name)}] Maybe required. If the adapter
    implements more than one interface, \function{grok.provides} is required to
    disambiguate for what interface the adapter will be registered.
  \end{itemize}
  \end{classdesc*}

  \begin{bf}Example:\end{bf}

  \begin{verbatim}
import grok
from zope import interface

class Cave(grok.Model):
    pass

class IHome(interface.Interface):
    pass

class Home(grok.Adapter):
    grok.implements(IHome)

home = IHome(cave)
  \end{verbatim}

  \begin{bf}Example 2:\end{bf}

  \begin{verbatim}
import grok
from zope import interface

class Cave(grok.Model):
    pass

class IHome(interface.Interface):
    pass

class Home(grok.Adapter):
    grok.implements(IHome)
    grok.name('home')

from zope.component import getAdapter
home = getAdapter(cave, IHome, name='home')
  \end{verbatim}

\section{\class{grok.AddForm}}

\section{\class{grok.Annotation}}

\section{\class{grok.Application}}

\section{grok.ClassGrokker}

\section{\class{grok.Container}}

  \begin{classdesc*}{grok.Container}
    Mixin base class to define a container object. The container implements the
    zope.app.container.interfaces.IContainer interface using a BTree, providing
    reasonable performance for large collections of objects.
  \end{classdesc*}

\section{\class{grok.DisplayForm}}

\section{\class{grok.EditForm}}

\section{\class{grok.Form}}

\section{\class{grok.GlobalUtility}}

  \begin{classdesc*}{grok.GlobalUtility}
    Base class to define a globally registered utility. Global utilities are
    automatically registered when a module is "grokked".

  \begin{bf}Directives:\end{bf}

  \begin{itemize}
    \item[\function{grok.implements(*interfaces)}] Required. Identifies the
    interfaces(s) the utility implements.

    \item[\function{grok.name(name)}] Optional. Identifies the name used for
    the adapter registration. If ommitted, no name will be used.

    When a name is used for the global utility registration, the global utility
    can only be retrieved by explicitely using its name.

    \item[\function{grok.provides(name)}] Maybe required. If the global utility
    implements more than one interface, \function{grok.provides} is required to
    disambiguate for what interface the global utility will be registered.
  \end{itemize}
  \end{classdesc*}

\section{\class{grok.Indexes}}

\section{grok.InstanceGrokker}

\section{\class{grok.JSON}}

\section{\class{grok.LocalUtility}}

  \begin{classdesc*}{grok.LocalUtility}
    Base class to define a utility that will be registered local to a
    \class{grok.Site} or \class{grok.Application} object by using the
    \function{grok.local_utility} directive.

  \begin{bf}Directives:\end{bf}

  \begin{itemize}
    \item[\function{grok.implements(*interfaces)}] Optional. Identifies the
    interfaces(s) the utility implements.

    \item[\function{grok.name(name)}] Optional. Identifies the name used for
    the adapter registration. If ommitted, no name will be used.

    When a name is used for the local utility registration, the local utility
    can only be retrieved by explicitely using its name.

    \item[\function{grok.provides(name)}] Maybe required. If the local utility
    implements more than one interface or if the implemented interface cannot
    be determined, \function{grok.provides} is required to disambiguate for
    what interface the local utility will be registered.
  \end{itemize}
  \end{classdesc*}

  \begin{seealso}
  Local utilities need to be registered in the context of \class{grok.Site} or
  \class{grok.Application} using the \function{grok.local_utility} directive.
  \end{seealso}

\section{\class{grok.Model}}

  Base class to define an application "content" or model object. Model objects
  provide persistence and containment.

\section{grok.ModuleGrokker}

\section{\class{grok.MultiAdapter}}

  \begin{classdesc*}{grok.MultiAdapter}
    Base class to define a multi adapter. MultiAdapters are automatically
    registered when a module is "grokked".

  \begin{bf}Directives:\end{bf}

  \begin{itemize}
    \item[\function{grok.adapts(*objects_or_interfaces)}] Required. Identifies
    the combination of types of objects or interfaces for the adaptation.

    \item[\function{grok.implements(*interfaces)}] Required. Identifies the
    interfaces(s) the adapter implements.

    \item[\function{grok.name(name)}] Optional. Identifies the name used for
    the adapter registration. If ommitted, no name will be used.

    When a name is used for the adapter registration, the adapter can only
    be retrieved by explicitely using its name.

    \item[\function{grok.provides(name)}] Maybe required. If the adapter
    implements more than one interface, \function{grok.provides} is required to
    disambiguate for what interface the adapter will be registered.
  \end{itemize}
  \end{classdesc*}

  \begin{bf}Example:\end{bf}

  \begin{verbatim}
import grok
from zope import interface

class Fireplace(grok.Model):
    pass

class Cave(grok.Model):
    pass

class IHome(interface.Interface):
    pass

class Home(grok.MultiAdapter):
    grok.adapts(Cave, Fireplace)
    grok.implements(IHome)

    def __init__(self, cave, fireplace):
        self.cave = cave
        self.fireplace = fireplace

home = IHome(cave, fireplace)
  \end{verbatim}

\section{grok.PageTemplate}

\section{grok.PageTemplateFile}

\section{\class{grok.Site}}

  Base class to define an site object. Site objects provide persistence and
  containment.

\section{\class{grok.Traverser}}

\section{\class{grok.View}}

\section{\class{grok.XMLRPC}}


\chapter{Directives}

The \module{grok} module defines a set of directives that allow you to
configure and register your components. Most directives assume a default, based
on the environment of a module. (For example, a view will be automatically
associated with a model if the association can be made unambigously.) 

If no default can be assumed for a value, grok will explicitly tell you what is
missing and how you can provide a default or explicit assignment for the value
in question.

    \section{\function{grok.implements}}

    \section{\function{grok.context}}

    \section{\function{grok.name}}

    Used to associate a component with a name. Typically this directive is
    optional. The default behaviour when no name is given depends on the
    component.

    \section{\function{grok.template}}

    \section{\function{grok.templatedir}}

    \section{\function{grok.resourcedir --- XXX Not implemented yet}}

Resource directories are used to embed static resources like HTML-,
JavaScript-, CSS- and other files in your application. 

XXX insert directive description here (first: define the name, second: describe
the default behaviour if the directive isn't given)

A resource directory is created when a package contains a directory
with the name \file{static}.  All files from this directory become
accessible from a browser under the URL
\file{http://<servername>/++resource++<packagename>/<filename>}.

\begin{bf}Example:\end{bf} The package \module{a.b.c} is grokked and contains a
directory \file{static} which contains the file \file{example.css}. The
stylesheet will be available via
\file{http://<servername>/++resource++a.b.c/example.css}.

\begin{notice}A package can never have both a \file{static} directory
  and a Python module with the name \file{static.py} at the same
  time. grok will remind you of this conflict when grokking a package
  by displaying an error message.

\end{notice}

\subsection{Linking to resources from templates}

    grok provides a convenient way to calculate the URLs to static resource
    using the keyword \keyword{static} in page templates:

    \begin{verbatim}
    <link rel="stylesheet" tal:attributes="href static/example.css" type="text/css">
    \end{verbatim}

    The keyword \keyword{static} will be replaced by the reference to the
    resource directory for the package in which the template was registered.



\chapter{Decorators}

grok uses a few decorators to register functions or methods for specific
functionality.

    \section{\function{grok.subscribe} -- Register a function as a subscriber
    for an event}


        \begin{funcdesc}{subscribe}{*classes_or_interfaces}

        Declare that the decorated function subscribes to an event or a
        combination of objects and events and register it.
        

        Applicable on module-level for functions. Requires at least one class
        or interface as argument.

        (Similar to Zope 3's \function{subscriber} decorator, but automatically
        performs the registration of the component.)
        \end{funcdesc}


    \section{grok.action}



\chapter{Events}

grok provides convenient access to a set of often-used events from Zope 3.
Those events include object and containment events. All events are available as
interface and implemented class.

    \section{grok.IObjectCreatedEvent}

    \section{grok.IObjectModifiedEvent}

    \section{grok.IObjectCopiedEvent}

    \section{grok.IObjectAddedEvent}

    \section{grok.IObjectMovedEvent}

    \section{grok.IObjectRemovedEvent}

    \section{grok.IContainerModifiedEvent}



\chapter{Exceptions}

grok tries to inform you about errors early and with as much guidance as
possible. grok can detect some errors already while importing a module, which
will lead to the \class{GrokImportError}.  Other errors require more context
and can only be detected while executing the \function{grok} function.

    \section{\class{grok.GrokImportError} -- errors while importing a module}

    This exception is raised if a grok-specific problem was found while
    importing a module of your application. \class{GrokImportError} means there
    was a problem in how you are using a part of grok. The error message tries
    to be as informative as possible tell you why something went wrong and how
    you can fix it.

    \class{GrokImportError} is a subclass of Python's \class{ImportError}.

    Examples of situations in which a GrokImportError occurs:

    \begin{itemize}
        \item Using a directive in the wrong context (e.g. grok.templatedir on
        class-level instead of module-level.)
        \item Using a decorator with wrong arguments (e.g. grok.subscribe
        without any argument)
        \item \ldots
    \end{itemize}

    \section{\class{grok.GrokError} -- errors while grokking a module}

    This exception is raised if an error occurs while grokking a module.

    Typically a \class{GrokError} will be raised if one of your modules uses a
    feature of grok that requires some sort of unambigous context to establish
    a reasonable default.

    For example, the \class{grok.View} requires exactly one model to be defined
    locally in the module to assume a default module to be associated with.
    Having no model defined, or more than one model, will lead to an error
    because the context is either underspecified or ambigous.

    The error message of a \class{GrokError} will include the reason for the
    error, the place in your code that triggered the error, and a hint, to help
    you fix the error.

    \begin{classdesc}{GrokError}{Exception}
        \begin{memberdesc}{component}
            The component that was grokked and triggered the error.
        \end{memberdesc}
    \end{classdesc}



\end{document}
