%
% API Documentation for biblio.webquery api
% Module biblio.webquery.basewebquery
%
% Generated by epydoc 3.0.1
% [Wed May  6 19:44:01 2009]
%

%%%%%%%%%%%%%%%%%%%%%%%%%%%%%%%%%%%%%%%%%%%%%%%%%%%%%%%%%%%%%%%%%%%%%%%%%%%
%%                          Module Description                           %%
%%%%%%%%%%%%%%%%%%%%%%%%%%%%%%%%%%%%%%%%%%%%%%%%%%%%%%%%%%%%%%%%%%%%%%%%%%%

    \index{biblio \textit{(package)}!biblio.webquery \textit{(package)}!biblio.webquery.basewebquery \textit{(module)}|(}
\section{Module biblio.webquery.basewebquery}

    \label{biblio:webquery:basewebquery}

A base class for querying webservices.

%%%%%%%%%%%%%%%%%%%%%%%%%%%%%%%%%%%%%%%%%%%%%%%%%%%%%%%%%%%%%%%%%%%%%%%%%%%
%%                           Class Description                           %%
%%%%%%%%%%%%%%%%%%%%%%%%%%%%%%%%%%%%%%%%%%%%%%%%%%%%%%%%%%%%%%%%%%%%%%%%%%%

    \index{biblio \textit{(package)}!biblio.webquery \textit{(package)}!biblio.webquery.basewebquery \textit{(module)}!biblio.webquery.basewebquery.BaseWebquery \textit{(class)}|(}
\subsection{Class BaseWebquery}

    \label{biblio:webquery:basewebquery:BaseWebquery}
\begin{tabular}{cccccccc}
% Line for object, linespec=[False, False]
\multicolumn{2}{r}{\settowidth{\BCL}{object}\multirow{2}{\BCL}{object}}
&&
&&
  \\\cline{3-3}
  &&\multicolumn{1}{c|}{}
&&
&&
  \\
% Line for biblio.webquery.impl.ReprObj, linespec=[False]
\multicolumn{4}{r}{\settowidth{\BCL}{biblio.webquery.impl.ReprObj}\multirow{2}{\BCL}{biblio.webquery.impl.ReprObj}}
&&
  \\\cline{5-5}
  &&&&\multicolumn{1}{c|}{}
&&
  \\
&&&&\multicolumn{2}{l}{\textbf{biblio.webquery.basewebquery.BaseWebquery}}
\end{tabular}

\textbf{Known Subclasses:}
biblio.webquery.basewebquery.BaseKeyedWebQuery,
    biblio.webquery.xisbn.XisbnQuery,
    biblio.webquery.worldcat.WorldcatQuery,
    biblio.webquery.loc.LocQuery


A base class for querying webservices.

This serves as a foundation for other web-query classes, centralising a
small amount of functionality and providing a common interface.

%%%%%%%%%%%%%%%%%%%%%%%%%%%%%%%%%%%%%%%%%%%%%%%%%%%%%%%%%%%%%%%%%%%%%%%%%%%
%%                                Methods                                %%
%%%%%%%%%%%%%%%%%%%%%%%%%%%%%%%%%%%%%%%%%%%%%%%%%%%%%%%%%%%%%%%%%%%%%%%%%%%

  \subsubsection{Methods}

    \vspace{0.5ex}

\hspace{.8\funcindent}\begin{boxedminipage}{\funcwidth}

    \raggedright \textbf{\_\_init\_\_}(\textit{self}, \textit{root\_url}, \textit{timeout}={\tt 5.0}, \textit{limits}={\tt \texttt{[}\texttt{]}})

    \vspace{-1.5ex}

    \rule{\textwidth}{0.5\fboxrule}
\setlength{\parskip}{2ex}

Ctor, allowing the setting of the webservice, timeout and limits on use.
\setlength{\parskip}{1ex}
      \textbf{Parameters}
      \vspace{-1ex}

      \begin{quote}
        \begin{Ventry}{xxxxxxxx}

          \item[root\_url]


The url to be used as the basis for all requests to this service.
It should be the common ``stem'' that does not vary for any request.
            {\it (type=string)}

          \item[timeout]


How many seconds to wait for a response.
            {\it (type=int or float)}

          \item[limits]


A list of QueryThrottles to impose upon the use of this webservice.
            {\it (type=iterable)}

        \end{Ventry}

      \end{quote}

      Overrides: object.\_\_init\_\_

    \end{boxedminipage}

    \label{biblio:webquery:basewebquery:BaseWebquery:request}
    \index{biblio \textit{(package)}!biblio.webquery \textit{(package)}!biblio.webquery.basewebquery \textit{(module)}!biblio.webquery.basewebquery.BaseWebquery \textit{(class)}!biblio.webquery.basewebquery.BaseWebquery.request \textit{(method)}}

    \vspace{0.5ex}

\hspace{.8\funcindent}\begin{boxedminipage}{\funcwidth}

    \raggedright \textbf{request}(\textit{self}, \textit{sub\_url})

    \vspace{-1.5ex}

    \rule{\textwidth}{0.5\fboxrule}
\setlength{\parskip}{2ex}

Send a request to the webservice and return the response.

This is the low-level calls that checks any throttling, send the request
and actually fetches the response data. For consistency, all service
access should be placed through here.
\setlength{\parskip}{1ex}
      \textbf{Parameters}
      \vspace{-1ex}

      \begin{quote}
        \begin{Ventry}{xxxxxxx}

          \item[sub\_url]


This will be added to the root url set in the c'tor and used as the
actual url that is requested.
            {\it (type=string)}

        \end{Ventry}

      \end{quote}

      \textbf{Return Value}
    \vspace{-1ex}

      \begin{quote}

The data in the webservice response.
      \end{quote}

    \end{boxedminipage}


\large{\textbf{\textit{Inherited from biblio.webquery.impl.ReprObj\textit{(Section \ref{biblio:webquery:impl:ReprObj})}}}}

\begin{quote}
\_\_repr\_\_(), \_\_str\_\_(), \_\_unicode\_\_()
\end{quote}

\large{\textbf{\textit{Inherited from object}}}

\begin{quote}
\_\_delattr\_\_(), \_\_getattribute\_\_(), \_\_hash\_\_(), \_\_new\_\_(), \_\_reduce\_\_(), \_\_reduce\_ex\_\_(), \_\_setattr\_\_()
\end{quote}

%%%%%%%%%%%%%%%%%%%%%%%%%%%%%%%%%%%%%%%%%%%%%%%%%%%%%%%%%%%%%%%%%%%%%%%%%%%
%%                              Properties                               %%
%%%%%%%%%%%%%%%%%%%%%%%%%%%%%%%%%%%%%%%%%%%%%%%%%%%%%%%%%%%%%%%%%%%%%%%%%%%

  \subsubsection{Properties}

    \vspace{-1cm}
\hspace{\varindent}\begin{longtable}{|p{\varnamewidth}|p{\vardescrwidth}|l}
\cline{1-2}
\cline{1-2} \centering \textbf{Name} & \centering \textbf{Description}& \\
\cline{1-2}
\endhead\cline{1-2}\multicolumn{3}{r}{\small\textit{continued on next page}}\\\endfoot\cline{1-2}
\endlastfoot\multicolumn{2}{|l|}{\textit{Inherited from object}}\\
\multicolumn{2}{|p{\varwidth}|}{\raggedright \_\_class\_\_}\\
\cline{1-2}
\end{longtable}

    \index{biblio \textit{(package)}!biblio.webquery \textit{(package)}!biblio.webquery.basewebquery \textit{(module)}!biblio.webquery.basewebquery.BaseWebquery \textit{(class)}|)}

%%%%%%%%%%%%%%%%%%%%%%%%%%%%%%%%%%%%%%%%%%%%%%%%%%%%%%%%%%%%%%%%%%%%%%%%%%%
%%                           Class Description                           %%
%%%%%%%%%%%%%%%%%%%%%%%%%%%%%%%%%%%%%%%%%%%%%%%%%%%%%%%%%%%%%%%%%%%%%%%%%%%

    \index{biblio \textit{(package)}!biblio.webquery \textit{(package)}!biblio.webquery.basewebquery \textit{(module)}!biblio.webquery.basewebquery.BaseKeyedWebQuery \textit{(class)}|(}
\subsection{Class BaseKeyedWebQuery}

    \label{biblio:webquery:basewebquery:BaseKeyedWebQuery}
\begin{tabular}{cccccccccc}
% Line for object, linespec=[False, False, False]
\multicolumn{2}{r}{\settowidth{\BCL}{object}\multirow{2}{\BCL}{object}}
&&
&&
&&
  \\\cline{3-3}
  &&\multicolumn{1}{c|}{}
&&
&&
&&
  \\
% Line for biblio.webquery.impl.ReprObj, linespec=[False, False]
\multicolumn{4}{r}{\settowidth{\BCL}{biblio.webquery.impl.ReprObj}\multirow{2}{\BCL}{biblio.webquery.impl.ReprObj}}
&&
&&
  \\\cline{5-5}
  &&&&\multicolumn{1}{c|}{}
&&
&&
  \\
% Line for biblio.webquery.basewebquery.BaseWebquery, linespec=[False]
\multicolumn{6}{r}{\settowidth{\BCL}{biblio.webquery.basewebquery.BaseWebquery}\multirow{2}{\BCL}{biblio.webquery.basewebquery.BaseWebquery}}
&&
  \\\cline{7-7}
  &&&&&&\multicolumn{1}{c|}{}
&&
  \\
&&&&&&\multicolumn{2}{l}{\textbf{biblio.webquery.basewebquery.BaseKeyedWebQuery}}
\end{tabular}

\textbf{Known Subclasses:} biblio.webquery.isbndb.IsbndbQuery


A Webquery that requires an access key.

%%%%%%%%%%%%%%%%%%%%%%%%%%%%%%%%%%%%%%%%%%%%%%%%%%%%%%%%%%%%%%%%%%%%%%%%%%%
%%                                Methods                                %%
%%%%%%%%%%%%%%%%%%%%%%%%%%%%%%%%%%%%%%%%%%%%%%%%%%%%%%%%%%%%%%%%%%%%%%%%%%%

  \subsubsection{Methods}

    \vspace{0.5ex}

\hspace{.8\funcindent}\begin{boxedminipage}{\funcwidth}

    \raggedright \textbf{\_\_init\_\_}(\textit{self}, \textit{root\_url}, \textit{key}, \textit{timeout}={\tt 5.0}, \textit{limits}={\tt \texttt{[}\texttt{]}})

    \vspace{-1.5ex}

    \rule{\textwidth}{0.5\fboxrule}
\setlength{\parskip}{2ex}

Ctor, allowing the setting of a webservice access key.
\setlength{\parskip}{1ex}
      \textbf{Parameters}
      \vspace{-1ex}

      \begin{quote}
        \begin{Ventry}{xxxxxxxx}

          \item[root\_url]


See \texttt{BaseWebquery}. Either this or the sub{\_}url passed to \texttt{request}
must include a keyword formatting for the access key, i.e.
\texttt{{\%}(key)s}.
          \item[key]


The access or PAI key to be passed to the webservice for access.
            {\it (type=string)}

          \item[timeout]


See \texttt{BaseWebquery}.
          \item[limits]


See \texttt{BaseWebquery}.
        \end{Ventry}

      \end{quote}

      Overrides: object.\_\_init\_\_

    \end{boxedminipage}


\large{\textbf{\textit{Inherited from biblio.webquery.basewebquery.BaseWebquery\textit{(Section \ref{biblio:webquery:basewebquery:BaseWebquery})}}}}

\begin{quote}
request()
\end{quote}

\large{\textbf{\textit{Inherited from biblio.webquery.impl.ReprObj\textit{(Section \ref{biblio:webquery:impl:ReprObj})}}}}

\begin{quote}
\_\_repr\_\_(), \_\_str\_\_(), \_\_unicode\_\_()
\end{quote}

\large{\textbf{\textit{Inherited from object}}}

\begin{quote}
\_\_delattr\_\_(), \_\_getattribute\_\_(), \_\_hash\_\_(), \_\_new\_\_(), \_\_reduce\_\_(), \_\_reduce\_ex\_\_(), \_\_setattr\_\_()
\end{quote}

%%%%%%%%%%%%%%%%%%%%%%%%%%%%%%%%%%%%%%%%%%%%%%%%%%%%%%%%%%%%%%%%%%%%%%%%%%%
%%                              Properties                               %%
%%%%%%%%%%%%%%%%%%%%%%%%%%%%%%%%%%%%%%%%%%%%%%%%%%%%%%%%%%%%%%%%%%%%%%%%%%%

  \subsubsection{Properties}

    \vspace{-1cm}
\hspace{\varindent}\begin{longtable}{|p{\varnamewidth}|p{\vardescrwidth}|l}
\cline{1-2}
\cline{1-2} \centering \textbf{Name} & \centering \textbf{Description}& \\
\cline{1-2}
\endhead\cline{1-2}\multicolumn{3}{r}{\small\textit{continued on next page}}\\\endfoot\cline{1-2}
\endlastfoot\multicolumn{2}{|l|}{\textit{Inherited from object}}\\
\multicolumn{2}{|p{\varwidth}|}{\raggedright \_\_class\_\_}\\
\cline{1-2}
\end{longtable}

    \index{biblio \textit{(package)}!biblio.webquery \textit{(package)}!biblio.webquery.basewebquery \textit{(module)}!biblio.webquery.basewebquery.BaseKeyedWebQuery \textit{(class)}|)}
    \index{biblio \textit{(package)}!biblio.webquery \textit{(package)}!biblio.webquery.basewebquery \textit{(module)}|)}
